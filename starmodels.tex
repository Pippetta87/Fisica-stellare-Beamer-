\section{Equazioni struttura stellare}\linkdest{equilibrium}

\begin{wordonframe}{da fare: kippenhahn wiegert}
\begin{itemize}
\item strutture autogravitanti 1-62' (43). EQuilibrio idrostatico, vento stellare, stabilit\'a e pulsazioni
\item metodi numerici 77'-84(44-48)
\item esistenza e unicit\'a 85'-99'(48-56)
\item Properties of stellar matter: ideal gas with radiation, ionization, degenerate electron gas, equazione di stato, opacit\'a 102'-144'(57-78)
\item produzione energia reazioni nucleari:  146'-172'(79-92)
\item politrope 174'-190' (93-102)
\end{itemize}
\end{wordonframe}

\subsection{Struttura di equilibrio}\linkdest{stellarstructure}

\begin{frame}{Equazioni struttura di equilibrio}
Determino la struttura solare integrando numericamente le equazioni fondamentali della struttura stellare
\begin{subequations}\label{subeqn:stellarstructure}
\begin{align*}
&\TDy{r}{m}=4\pi r^2\rho\\
&\TDy{r}{P}=-\frac{Gm(r)\rho(r)}{r^2}\\
&\TDy{r}{T}=\nabla\frac{T}{p}\TDy{r}{p}\\
&\TDy{r}{L}=4\pi r^2[\rho(\epsilon-\epsilon_{\nu})-\rho\TDof{t}u+\frac{P}{\rho}\TDy{t}{\rho}]
\end{align*}

\begin{equation*}
\PDy{t}{n_i}+\frac{1}{r^2}\PDof{r}(r^2n_iv_i)=\Dcvar{\PDy{t}{n_i}}{Nucl}\label{eq:difffusionchange}
\end{equation*}
\end{subequations}
con $v_i$ velocit\'a di diffusione della specie i. Ottengo il profilo radiale delle grandezze $\{P,m,T,L,X_i\}$, note la metallicit\'a iniziale Z, l'equazione di stato $P(\rho,T,X_i)$, l'opacit\'a $\kappa(P,T,X_i)$, il rate di produzione di energia nucleare per grammo $\epsilon(P,T,X_i)$.
\end{frame}

\begin{frame}{Energia gravitazionale}

\end{frame}

\begin{frame}{Trasporto radiativo e convettivo: \'e valida approx idrostatica}

\end{frame}

\subsection{Metodi per integrazione equazioni di struttura e raccordo con modelli dell'atmosfera stellare}\linkdest{nummod}

\begin{frame}{Atmosfera e subatmosfera}

\end{frame}

\begin{frame}{Metodo del fitting}

\end{frame}

\begin{frame}{Metodo di Henyey}

\end{frame}

\subsection{Relazioni approssimate per grandezze stellari fondamentali}\linkdest{omrel}

\begin{frame}{Teorema del viriale}

\end{frame}

\begin{frame}{Relazione massa, densit\'a, temperatura/ massa, peso molecolare, opacit\'a}

\end{frame}

\section{Trasporto}\linkdest{transport}

\subsection{Trasporto radiativo}\linkdest{trarad}

\begin{frame}{Trasporto radiativo}
media di rosseland
\end{frame}

\begin{frame}{Opacit\'a radiativa (analitico)}
Scattering elettronico, processi ff, fotoionizzazione; opacit\'a atmosfera: ione H-
\end{frame}

\subsection{Conduzione}\linkdest{tracond}

\begin{frame}{Diffusione per conduzione}
stima opacit\'a per conduzione gas degenere NR
\end{frame}

\subsection{Convezione}\linkdest{traconv}

\begin{frame}{Convective regions and temperature gradient}
\begin{block}{Criterio di \sch e Ledoux: regioni convettive}
\begin{align*}
&\nrad{}>\nad+\frac{\phi}{\delta}\nmu{}\\
&\nrad{}>\nad
\end{align*}
\end{block}
\begin{block}{T gradient: Mixing length}
Gas with negligible radiation pressure
\begin{align*}
&F=\frac{L}{4\pi r^2}=F_{rad}+F_{con}=-\frac{4acT^3}{3\kappa\rho}\TDy{r}{T}|_{amb}+\frac{1}{2}\rho vc_p[\TDy{r}{T}|_{Ad}-\TDy{r}{T}|_{amb}]\Lambda\\
&v^2=\frac{1}{8}g\frac{\Lambda}{H_P}Q(\nabla-\nad),\ Q=1-\TDly{T}{\mu}\\
&F_{con}=\frac{1}{2}\rho vc_PT\frac{\lambda}{H_P}(\nabla-\nad{})
\end{align*}
$\nabla\to\nad{}$ in interni stellari convettivi
\end{block}
\end{frame}

\begin{frame}{Mixing length per esterni stellari}
%Calcolo altezza scala di pressione, flusso convettivo e gradiente ambientale in funzione della velocit\'a media degli elementi
Le stelle con massa $M\leq1.1\msun{}$ hanno una regione radiativa interna mentre la parte esterna \'e convettiva
\end{frame}

\begin{wordonframe}{Forza di archimede}
Una regione stellare \'e convettivamente stabile se una perturbazione di densit\'a infinitesima non cresce ad ampiezza finita.
\begin{equation*}\label{eq:buoyancyEOM}
\rho\PtwoDy{t}{(\Delta r)}=-g\Delta\rho=-g[\Dcvar{\TDy{r}{\rho}}{e}-\Dcvar{\TDy{r}{\rho}}{amb}]\Delta r
\end{equation*}

La forza di Archimede ha verso opposta alla perturbazione se
\begin{equation*}\label{eq:Acriterion}
[\Dcvar{\TDy{r}{\rho}}{e}-\Dcvar{\TDy{r}{\rho}}{amb}]>0
\end{equation*}
\end{wordonframe}

\begin{wordonframe}{EOS e $\nad{}$}

Considero un'equazione di stato generica $\rho(P,T,\mu)$ e definita tramite:
\begin{align*}
&\frac{d\rho}{\rho}=\alpha\frac{dP}{P}-\delta\frac{dT}{T}+\phi\frac{d\mu}{\mu}\\
&P=\frac{\rho\gasconstant{}T}{\mu}\quad\Rightarrow\quad\alpha=\delta=\phi=1
\end{align*}

Definisco le lunghezze caratteristiche per variazione di densit\'a e pressione:
\begin{equation*}
\densityscale{}=-\frac{dr}{d\ln{\rho}},\ H_P=-\frac{dr}{d\ln{P}}
\end{equation*}
e i gradienti termici per il blob, l'ambiente e il gradiente di composizione chimica ambientale
\begin{equation*}
\nabla=\Dcvar{\TDly{P}{T}}{amb},\ \nabla_e=\Dcvar{\TDly{P}{T}}{e},\ \nmu{}=\Dcvar{\TDly{P}{\mu}}{amb}
\end{equation*}
\end{wordonframe}

\begin{wordonframe}{EOS e EOM}
Riscrivo l'equazione del moto utilizzando l'equazione di stato per scrivere la differenza di densit\'a in termini dei gradienti termici e di composizione chimica; inoltre supponendo il moto dell'elemento in equilibrio di pressione con l'ambiente e assumendo $\nmu{}_{blob}\approx0$ risulta:
\begin{equation*}
\PtwoDy{t}{(\Delta r)}=-g\frac{\delta}{H_P}[\nabla_e-\nabla-\frac{\phi}{\delta}\nmu{}]\Delta r
\end{equation*}

\end{wordonframe}

\begin{wordonframe}{Criterio di \sch/Ledoux}

Infine per ricavare il criterio di stabilit\'a per convezione suppongo  il moto del blob adiabatico:
\begin{equation*}
dq=c_P\,dT-\frac{\delta}{\rho}\,dP
\end{equation*}
da cui risulta:
\begin{equation*}
\nabla_e=\nabla_{ad}=\frac{P\delta}{T\rho c_P}
\end{equation*}
cio\'e una regione solare \'e stabile per convezione se
\begin{equation*}
\nrad{}<\nad+\frac{\phi}{\delta}\nmu{}\label{eq:ledoux}
\end{equation*}
dove ho usato $\nabla_{amb}=\nrad{}$, cio\'e il gradiente che si ha nel caso l'energia sia trasportata dai fotoni.
\end{wordonframe}

\begin{wordonframe}{Stabilit\'a convettiva e frequenza di \bv{}}
Gradiente adiabatico: riscrivo prima legge della termodinamica come $dq=c_P\,dT-\frac{\delta}{\rho}\,dP$

Introduco la frequenza di \bv{}:
\begin{equation*}
N^2=g(\frac{1}{\Gamma_1P}\TDy{r}{P}-\frac{1}{\rho}\TDy{r}{\rho})=g(\frac{1}{\densityscale{}}-\frac{g}{c_s^2})\label{eq:bvfs}
\end{equation*}
$N^2$ rappresenta la massima frequenza sotto cui pu\'o oscillare una particella di fluido sottoposta a onde di gravit\'a mantenendo l'equilibrio di pressione con l'ambiente.

\begin{equation*}
\PtwoDy{t}{(\Delta r)}=-N^2\Delta r
\end{equation*}
che descrive un comportamento oscillatorio per $N^2>0$.

\end{wordonframe}

\subsection{Teoria della mixing-length.}

\begin{wordonframe}{Convezione in esterni stellari}

In presenza di convezione il flusso di energia verso l'esterno ha una componente radiativa, determinata dal gradiente di temperatura, e una componente dominante convettiva 
\begin{equation*}\label{eq:radconvflux}
F=F_{con}+F_{rad}=\frac{\lsun{}}{4\pi r^2}
\end{equation*}

Una maggiore efficienza del trasporto convettivo di energia si riflette in una minore differenza tra il gradiente di temperature adiabatico ed effettivo.

\begin{figure}[!h]
%   \includegraphics[ width=0.99\textwidth,keepaspectratio]{proportionflux}
%   \subcaption{Profilo radiale (profondit\'a in \si{\kilo\meter}) del flusso convettivo $F_c$ rispetto al flusso totale $F$, della super-adiabaticit\'a $\nabla-\nad{}$ e regioni di ionizzazione idrogeno e $\cel{He}{4}{}{}$. Da \cite{christensen1997effects}.}\label{fluxproportion}

%\includegraphics[keepaspectratio,width=0.9\textwidth]{specificheatnablaa}
%\subcaption{Profilo radiale di $c_P$ e $\nabla_a$: si ha cambiamento di comportamento nelle regioni di ionizzazione parziale di idrogeno ed elio. Da \cite{stix91sun}.}\label{specificheatnablaa}
\end{figure}

Per determinare il gradiente di temperatura effettivo $\nabla$ uso la teoria della mixing-length:
si considera l'eccesso di calore trasportato dai blob di gas nel moto convettivo $c_P\Delta T$ rispetto all'ambiente, il cui cammino libero medio \'e la mixing-length $l_m=\alpha H_P$, che da luogo al flusso di energia
\begin{equation*}
F_{con}=\exv{\rho vc_P\Delta T}\label{eq:convectiveflux}
\end{equation*}
dove $\exv{}$ indica una media opportuna sulla sfera di raggio r. Determino il valor medio della differenza di temperatura prendendo come valore caratteristico dello spostamento del blob $\Delta r\approx\frac{l_m}{2}$:
%, considerando moti in entrambi i versi,
\begin{equation*}
\frac{\Delta T}{T}\approx\frac{1}{T}\PDy{r}{(\Delta T)}\frac{l_m}{2}=(\nabla-\nabla_e)\frac{l_m}{2}\frac{1}{H_P}\label{eq:blobambdiff}
\end{equation*}

Assumo il lavoro medio fatto dalla forza di galleggiamento per unit\'a di massa $-g\frac{\Delta\rho}{\rho}$ uguale al valore medio della forza, cio\'e la met\'a di quello alla superficie sferica data, moltiplicato lo spostamento medio $\frac{l_m}{2}$ quindi, assumendo in oltre che in media met\'a del lavoro fatto dalla forza di galleggiamento sia trasformato in energia cinetica del blob si ottiene
\begin{equation*}
v^2=g\delta(\nabla-\nabla_e)\frac{l_m^2}{8H_P}\label{eq:blobvelocity}
\end{equation*}

Infine determino gli scambi radiative del blob: il modulo del flusso radiativo \'e proporzionale al gradiente termico in direzione normale alla superficie del blob
\begin{equation*}
f=\frac{4acT^3}{3\kappa\rho}|\PDy{n}{T}|
\end{equation*}
quindi l'energia scambiata dall'intera superficie S del blob \'e $\lambda=Sf$ che determina, per la prima legge della termodinamica, una variazione di temperatura per unit\'a di tempo:
\begin{equation*}
\PDy{t}{T_e}=-\frac{\lambda}{\rho Vc_P}
\end{equation*}
indicato con $V$ il volume del blob.

La variazione della temperatura del blob per unit\'a distanza percorsa \'e quindi
\begin{equation*}
\Dcvar{\TDy{r}{T}}{e}=\Dcvar{\TDy{r}{T}}{ad}-\frac{\lambda}{\rho Vc_Pv}\label{eq:Tchangelength}
\end{equation*}
e approssimando il gradiente normale alla superficie con $\exv{\Delta T}$ ed usando le definizioni \eqref{eq:nablavitense} si ottiene:
\begin{equation*}
\frac{\nabla_e-\nad{}}{\nabla-\nabla_e}=\frac{6acT^3}{\kappa\rho^2c_Pl_mv}
\end{equation*}
Il gradiente termico ambientale $\nabla$ e del blob $\nabla_e$ sono determinati da \eqref{eq:radconvflux} e \eqref{eq:Tchangelength} inserendo le espressioni per il flusso radiativo \eqref{eq:radiativeflux} e il flusso convettivo \eqref{eq:convectiveflux}.

In figura (\subref{fluxproportion}) si mostrano l'andamento di $\nabla-\nad{}$, il profilo termico e la frazione di flusso totale trasportato dalla convezione; in figure (\subref{specificheatnablaa}) si mostrano il profilo del calore specifico per unit\'a di massa e del gradiente adiabatico.

Le 5 equazioni del flusso convettivo

Le 5 equazioni determinano completamente le variabili $F_{rad}, F_{con}, v, \nabla_e, \nabla$ in funzione di $P,T,l(r),m(r),c_P,\nad{},\nrad{},g$.

Come determino il gradiente effettivo ??

Determino $\nabla-\nabla_e$
cubic equation for $(\nabla-\nabla_e)$

\end{wordonframe}

\subsection{Diffusion}\linkdest{diffusion}

\subsection{Approssimazione politropa}\linkdest{poly}

\begin{frame}{Trasformazioni politropiche}
\begin{block}{Gener. T. adiabatica}
	Il rapporto $\gamma=\frac{c_P}{c_V}$ costante per gas perfetto di sole particelle totalmente ionizzato.
	T. adiabatica:
	\[TV\expy{\gamma-1}=\const,\ PV\expy{\gamma}=\const,\ P\expy{1-\gamma}T\expy{\gamma}=\const\]
$0=dE-\frac{P}{\rho^2}d\rho$
Caso pi\'u generale delle trasformazioni adiabatiche: \keyword{trasformazione politropa} trasformazione quasi-statica in maniera che $c=\TDy{Q}{T}$ (calore specifico) vari in maniera assegnata. (adiabatica: $c=0$, isoterma: $c=\infty$, isometrica: $c=c_V$, ...)
\end{block}

\end{frame}

\begin{frame}{Equazione Lane-Emden}

\end{frame}

\begin{frame}{Strutture isoterme e modello solare}

\end{frame}


\section{Descrizione materia stellare}\linkdest{descstellarmatter}

\subsection{Equazione di stato}\linkdest{eos}

\begin{frame}{Approssimazione gas perfetto monoatomico}

\end{frame}

\begin{frame}{Degenerazione e- ed effetti coulombiani. Equazione di saha}

\end{frame}

\subsection{Reazioni nucleari}\linkdest{nuclearreactions}

\begin{frame}{Sezione d'urto nucleare}
schermaggio elettronico nelle stelle
\end{frame}

\begin{frame}{Catena PP}
dipendenza da T
flusso neutrini
\end{frame}

\begin{frame}{Biciclo CNO}
dipendenza da T
flusso neutrini
Modalit\'a combustione H in He: sequenza principale inferiore/superiore
\end{frame}

\begin{frame}{Combustion He}

\end{frame}

\begin{frame}{Produzione nuclei fino al Fe56}
3$\alpha$: $He4+\alpha\to C12$, $C12+\alpha\to O16$
Produzione neutroni liberi
Fusione $C12$, fotodisintegrzione $Ne20$, fusione $O16$, fotodisintegrzione $Si28$, catture $\alpha$ su nuclei fino a produzione $Fe56$
\end{frame}

\begin{frame}{Cattura neutronica: processi r e s}
picchi r e s nella nella curva universale delle abbondanze
\end{frame}

%! TEX root = main.tex
\section{Equazioni struttura stellare}\linkdest{stellarmodel}

\begin{wordonframe}{da fare: kippenhahn wiegert}
\begin{itemize}
\item strutture autogravitanti 1-62' (43). EQuilibrio idrostatico, vento stellare, stabilit\'a e pulsazioni: ho iniziato da succo
\item metodi numerici 77'-84(44-48): sposto nella parte solve
\item esistenza e unicit\'a 85'-99'(48-56)
\item Properties of stellar matter: ideal gas with radiation, ionization, degenerate electron gas, equazione di stato, opacit\'a 102'-144'(57-78)
\item produzione energia reazioni nucleari:  146'-172'(79-92)
\item politrope 174'-190' (93-102)
\end{itemize}
\end{wordonframe}

\subsection{Struttura di equilibrio}\linkdest{stellarstructure}

\begin{frame}{Equazioni struttura di equilibrio}

\begin{align*}
&\TDy{m}{r}=\frac{1}{4\pi r^2\rho}\\
&\TDy{m}{P}=-\frac{Gm}{4\pi r^4}\overbrace{[-\frac{1}{4\pi r^2}\PtwoDy{t}{r}]}^{\tau_{hyd}=\frac{1}{2}(G\exv{\rho})\expy{-1/2}}\\
&\TDy{m}{T}=-\nabla\frac{T}{p}\frac{Gm}{4\pi r^4}\\
&\TDy{m}{L}=\epsilon-\epsilon_{\nu} \underbrace{-c_P[\TDy{t}{T}-\nad\frac{T}{P}\TDy{t}{P}]}_{-c_P\PDy{t}{T}+\frac{\delta}{\rho}\PDy{t}{P}}\quad(\TDy{t}{q}=\underbrace{\TDy{t}{u}+P\TDof{t}(\frac{1}{\rho})}_{c_P\TDy{t}{T}-\frac{\delta}{\rho}\TDy{t}{P}}=\underbrace{\epsilon-\frac{1}{\rho}\nabla\cdot\vec{F})}_{\epsilon-\PDy{m}{l}}\\
&\TDy{t}{X_s}\frac{1}{A_s}=\sum_{production}\rho^{n_h+n_k-1}n_p\frac{X_h^{n_h}X_k^{n_k}}{A_h^{n_h}A_k^{n_k}}\frac{\exv{\sigma v}_{hk}}{m_H^{n_h+n_k-1}n_h!n_k!}\tag{$n_hh+n_kk\to n_ps$}\\
&-\sum_{distruction}\rho^{n_d+n_j-1}n_d\frac{X_s^{n_d}X_j^{n_j}}{A_s^{n_d}A_j^{n_j}}\frac{\exv{\sigma v}_{sj}}{m_H^{n_d+n_j-1}n_d!n_j!}\tag{$n_jj+n_ds\to n_zz$}
\end{align*}
\end{frame}

\begin{frame}{Conservazione massa e HE}
\begin{columns}[T]
	\begin{column}{0.5\textwidth}
\begin{align*}
&\TDy{m}{r}=\frac{1}{4\pi r^2\rho}\\
&\TDy{m}{P}=-\frac{Gm}{4\pi r^4}\overbrace{[-\frac{1}{4\pi r^2}\PtwoDy{t}{r}]}^{\tau_{hyd}=\frac{1}{2}(G\exv{\rho})\expy{-1/2}}\\
&\tau_{hyd}=\sqrt{\frac{R^3}{GM}}
\end{align*}
\end{column}\begin{column}{0.5\textwidth}
\begin{align*}
&\TDy{r}{m}=4\pi r^2\rho\\
&\TDy{r}{P}=-\frac{Gm}{r^2}\rho\overbrace{[-\rho\PtwoDy{t}{r}]}^{\tau_{hyd}=\frac{1}{2}(G\exv{\rho})\expy{-1/2}}\\
&\tau_{ff}=\sqrt{\frac{R}{g}}\\
&\tau_{exp}=R\sqrt{\frac{\rho}{P}}
\end{align*}
\end{column}
\end{columns}
Red giant: $\tau_H\approx\SI{5}{\day}$ (time for the shock to cross a supergiant star making a SNII, Cepheid); Sun: $\tau_H\approx\SI{38}{\minute}$; WD: $\tau_H\approx\SI{2}{\second}$
\end{frame}

\begin{frame}{Trasporto energia verso la superficie e gradiente termico}
\begin{columns}[T]
	\begin{column}{0.6\textwidth}
		\begin{align*}
		&\TDy{m}{T}=-\nabla\frac{T}{p}\frac{Gm}{4\pi r^4}\\
		&\nabla=\nrad=\frac{3\kappa_R}{16\pi acG}\frac{LP}{mT^4}\tag*{$\nrad\leq\nad$}\\
		&\nabla=\nad+\Delta\nabla\tag*{$\nrad>\nad-\frac{\chi_{\mu}}{\chi_T}\nmu$}
		\end{align*}
	\end{column}\begin{column}{0.4\textwidth}
		\begin{align*}
		&\TDy{r}{T}=\nabla\frac{T}{p}\TDy{r}{p}\\
		&\nabla=\TDly{P}{T}
		\end{align*}
	\end{column}
\end{columns}
\begin{align*}
&dP_{rad}=-dp=-\frac{dF_{Rad}}{c}=-\frac{F_{rad}}{c}\frac{dr}{l}=-\frac{F_{rad}}{c}\kappa_R\rho\,dr=\frac{4}{3}aT^3dT\\
&dP_{rad}(\nu)=-\frac{F_{rad}(\nu)}{c}\kappa_{\nu}\rho\,dr=\frac{4\pi}{3c}\TDy{r}{B_{\nu}(T)}\,dr\\
&B_{\nu}(T)=\frac{2h\nu^3}{c^2}\frac{1}{\exp{\frac{h\nu}{kT}}-1}
\end{align*}
\end{frame}

\begin{frame}{Equazioni struttura stellare: conservazione energia - Luminosit\'a}
\begin{columns}[T]
\begin{column}{0.4\textwidth}
\begin{align*}
&\TDy{m}{L}=\epsilon-\epsilon_{\nu} \underbrace{-c_P[\TDy{t}{T}-\nad\frac{T}{P}\TDy{t}{P}]}_{-c_P\PDy{t}{T}+\frac{\delta}{\rho}\PDy{t}{P}0-T\PDy{t}{s}}\\
&\epsilon_{gr}=-T\PDy{t}{s}=-\TDof{t}u+\frac{P}{\rho^2}\TDy{t}{\rho}\\
&\epsilon_{gr}>0\tag*{contrazione}
\end{align*}
\end{column}\begin{column}{0.6\textwidth}
\begin{align*}
&\TDy{r}{L}=4\pi r^2[\rho(\epsilon-\epsilon_{\nu})-\rho\TDof{t}u+\frac{P}{\rho}\TDy{t}{\rho}]\\
&
\end{align*}
\end{column}
\end{columns}
\end{frame}

\begin{frame}{Chemical evolution: nuclear burning, diffusion and convective mixing}\linkdest{diffusion}\
    \begin{columns}[T]
        \begin{column}{0.3\textwidth}
            $X_i$ frazione in massa dell'elemento s: $\sum_sX_s=1$: $i+j\to k$,$r_{ij}=\frac{N_iN_j}{1+\delta_{ij}}\exv{v\sigma}_{ij}$ is number reaction per unit volume and time, $R_{ij}=r_{ij}/\rho$ is number of reactions per unit time and mass
            \begin{align*}
                &\TDy{t}{X_i}=-A_im_HR_{ij}\\
                &=-\rho \frac{X_iX_j}{m_HA_j}\frac{\exv{v\sigma}_{ij}}{1+\delta_{ij}}
            \end{align*}
        \end{column}
        \begin{column}{0.7\textwidth}
            \begin{itemize}
                \item Cycle
                    \begin{align*}
                        &a+b\to c\\
                        &c+b\to a\\
                        &X_a=X_a^0\exp{-Pt}+X_a^{\infty}(1-\exp{-Pt})
                    \end{align*}
                    If $X_b\gg X_a,X_c$ and $\tau_{eq}=\frac{1}{P}\ll\tau_{dyn}$: $\rho, X_b$ approx const
                \item Fully mixed convective shell ($\tau_{X_i}=\frac{1}{X_i}\dot{X_i}\gg\tau_{conv}$):
                    \begin{align*}
                        &\TDy{t}{\exv{X_i}}=\frac{1}{\Delta m}[\int_{m_1}^{m_2}\TDy{t}{X_i}\,dm+\TDy{t}{m_2}(X_{i2}-\exv{X_i})-\TDy{t}{m_1}(X_{i1}-\exv{X_i})]\\
                        &\Delta m=m_2-m_1
                    \end{align*}
                    $X_{i1},X_{i2}$ abb. on radiative side at inner/outer boundary.
            \end{itemize}
        \end{column}
    \end{columns}
    
\end{frame}

\begin{frame}{Cyclic reactions}
    \begin{columns}[T]
        \begin{column}{0.3\textwidth}
            \begin{align*}
                &a+b\to c\\
                &c+b\to a\\
                &A=\frac{\rho X_bA_a}{m_HA_cA_b}\\
                &B=\frac{\rho X_b}{m_HA_b}\\
                &
            \end{align*}
            c,a secondary, b primary; $\rho, X_b$ approx const ($X_b\gg X_a,X_c$, $\tau_{eq}=\frac{1}{P}\ll\tau_{dyn}$): $A,B$ const; Sum of abb in numbers of a,c is conserved: $\frac{X_a}{A_a}+\frac{X_c}{A_c}=\frac{X_a^0}{A_a}+\frac{X_c^0}{A_c}=K^0$, quindi $X_c=A_cK_0-\frac{A_cX_a}{A_a}$. Equilibrium abundances:
            \begin{align*}
                &X_a^{\infty}=\frac{K^0A_cA\exv{cb}}{P}\\
                &X_c^{\infty}=\frac{K^0A_cB\exv{\sigma v}_{ab}}{P}\\
                &\tau_{eq}\sim \frac{1}{P}
            \end{align*}

        \end{column}
        \begin{column}{0.7\textwidth}
            \begin{align*}
                &\dot{X_a}=\rho[\frac{X_cX_bA_a}{m_HA_cA_b}\exv{\sigma v}_{cb}-\frac{X_aX_b}{m_HA_b}\exv{\sigma v}_{ab}]\\
                &\dot{X_c}=\rho[\frac{X_aX_bA_c}{m_HA_aA_b}\exv{\sigma v}_{ab}-\frac{X_cX_b}{m_HA_b}\exv{\sigma v}_{cb}]\\
                &\Rightarrow\dot{X}_a=X_cA\exv{\sigma v}_{cb}-X_aB\exv{\sigma v}_{ab}\\
                &\dot{X}_a=K^0A_cA\exv{\sigma v}_{cb}-X_a(\frac{A_cA}{A_a}\exv{\sigma v}_{cb}+B\exv{\sigma v}_{ab})\\
                &=-PX_a+K^0A_cA\exv{\sigma v}_{cb}\\
                &X_a=X_a^0\exp{-Pt}+X_a^{\infty}(1-\exp{-Pt})
            \end{align*}
        \end{column}
    \end{columns}
\end{frame}

\begin{frame}{Trasporto radiativo e convettivo: \'e valida approx idrostatica}

\end{frame}

\subsection{Considerazioni energetiche}\linkdest{virialstability}

\begin{frame}{Teorema del viriale}
\begin{columns}[T]
	\begin{column}{0.45\textwidth}
		\begin{align*}
&\Omega=-\int_0^M\frac{Gm(r)}{r}\,dm\\
&\frac{1}{2}\TtwoDy{t}{I}=2E_i+\Omega\tag{T. viriale}\\
&0=\int_M\frac{3P}{\rho}\,dm(r)+\Omega\tag{stationary}\\
&E_i=\frac{1}{\gamma-1}\frac{P}{\rho}
		\end{align*}
	\end{column}\begin{column}{0.55\textwidth}
		\begin{align*}
&W=E_i+\Omega \tag{total E}\\
&\TDy{t}{W}+L=0\tag{E conservation}\\
&L=-\frac{1}{2}\dot{\Omega}=\dot{E}_i\\
&E_T=E_i+\Omega=\frac{3\gamma-4}{3(\gamma-1)}\Omega\\
&\gamma>4/3\tag{stability}
		\end{align*}
	\end{column}
\end{columns}
Nel caso in cui la contrazione gravitazionale sia l'unica fonte di energia per una massa gassosa in equilibrio idrostatico, il suo tempo di evoluzione caratteristico \'e il tempo di \kh{} $\tkh{}=\frac{\Omega}{L}\approx\frac{GM^2}{2RL}$
\end{frame}

\frameinlbftrue
\begin{frame}{Stability}

	\begin{itemize}
		\item 
			Dynamical stability: $P=\int_m^M\frac{mG}{4\pi r^4}\,dm$ - assuming homologous adiabatic compression we have from HE $\frac{P'}{P}=(\frac{R'}{R})^{-4}$ and $\frac{P'}{P}=(\frac{\rho'}{\rho})^{\gamma_{ad}}=(\frac{R'}{R})^{-3\gamma_{ad}}$ - so for $\gamma_{ad}>\frac{4}{3}$: pressure increases more than weight, for $\gamma_{ad}<\frac{4}{3}$ star would collapse
		\end{itemize}
	
\end{frame}
\frameinlbffalse

    \subsection{Andamento qualitativo}
    
    \begin{frame}{Propriet\'a stellari al variare di M? Relazioni M-L?}

    \end{frame}
    
%\begin{frame}{Relazione massa, densit\'a, temperatura/ massa, peso molecolare, opacit\'a}
%\end{frame}
\section{Trasporto}\linkdest{transport}

\subsection{Trasporto radiativo}\linkdest{trarad}

\begin{frame}{Caratteristiche del gas di fotoni}
    \begin{itemize}
        \item Intensity (Macroscopic): energy transported $dE=I(\vec{x},t;\vec{n},\nu)dS\cos{\alpha}d\Omega d\nu dt$, $\alpha$: angolo tra $\vec{n}$ e $d\vec{S}$
        \item Photon number density (Microscopic) $\psi$: $\psi(\vec{x},t;\vec{n},\nu)d\Omega d\nu$ number photons traveling around solid angle centered at $\vec{n}$, number of photons crossing $dS$: $\psi(\vec{n}\cdot d\vec{S}d\Omega d\nu cdt$, $dE=ch\nu\psi dS\cos{\alpha}d\Omega d\nu dt$: $I=\psi ch\nu$
        \item photon distro function $f_R$: $f_R(\vec{x},t;\vec{n},p)d^3p$: number of photons per unit volume with moment $[\vec{p},\vec{p}+dp]$, $d^3p=p^2dpd\Omega=(\frac{h}{c})^3\nu^2d\Omega d\nu$: $I=\frac{h^4\nu^3}{c^2}f_R$
        \item Mean intensity (zeroth moment of radiation field over angles): $J_{\nu}=J(\vec{x},t;\nu)=\invers{(4\pi)}\oint I\,d\Omega$ (\si{\erg\per\squared\cm\per\second\per\hertz\per\ster})
    \end{itemize}
\end{frame}

\begin{frame}{Trasporto radiativo}
\begin{columns}[T]
	\begin{column}{0.5\textwidth}
Diffusion approx: $F_{\nu}=-D_{\nu}\nabla U_{\nu}$.

Radiative transport - $dp$ momentum trasfer from photons to matter
\begin{align*}
&dp=\frac{dF_r}{c}=\frac{F_r}{c}\frac{dr}{\rho\kappa_r}\\
&dP_r=-\frac{F_r}{c}\frac{dr}{l}\\
&\TDy{r}{P_r}=\frac{4}{3}aT^3\TDy{r}{T}\\
&\TDy{r}{P_{r,\nu}}=-\frac{\kappa_{\nu}\rho}{c}F_{\nu}
\end{align*}
Conduzione elettronica
\begin{align*}
&F_e\approx-N_evl\TDy{r}{E}=-kN_evl\TDy{r}{T}
\end{align*}
\end{column}
\begin{column}{0.5\textwidth}
In LTE:
\begin{align*}
&P_{r,\nu}=\frac{4\pi}{3c}B_{\nu}(T)\\
&\frac{1}{\kappa_R}=\frac{\intzi{}\frac{1}{\kappa_{\nu}}\TDy{T}{B_{\nu}(T)}\,d\nu}{\intzi{}\TDy{T}{B_{\nu}(T)}\,d\nu}\\
&\plankfnu{}
\end{align*}
\end{column}
\end{columns}
\end{frame}

\subsection{Conduzione}\linkdest{tracond}

\begin{frame}{Diffusione per conduzione}
stima opacit\'a per conduzione gas degenere NR
\end{frame}

\begin{frame}{Conduzione elettronica}
\begin{columns}[T]
\begin{column}{0.5\textwidth}
	\begin{align*}
	&F_e=-N_evl\TDy{r}{E}=-N_ekvl\TDy{r}{T}\\
	&E_T\approx\frac{3}{2}kT,\ v_T\approx\sqrt{\frac{2E_T}{m}}
	\end{align*}
\end{column}
\begin{column}{0.5\textwidth}
	\Pelectron degenerate are forced in higher momentum state - $P_F=2\pi\hbar(\frac{3}{4\pi g})\expy{1/3}n\expy{1/3}$
\end{column}
\end{columns}
\end{frame}

\subsection{Opacit\'a}\linkdest{kapparad}

\begin{frame}{Opacit\'a radiativa (analitico)}
Scattering elettronico, processi ff, fotoionizzazione; opacit\'a atmosfera: ione H-
\end{frame}

\begin{frame}{Andamento opacit\'a: electron scattering e Kramer's opacity (FF)}
\begin{itemize}
\item Electron scattering $\rho\kappa_{\nu}=n_e\frac{8\pi}{3}(\frac{e^2}{m_ec^2})^2=0.2(1+X)\si{\square\cm\per\gram}$ - $\sigma_T=\SI{0.66e-24}{\square\cm}$ - for $T$ maggiore di few milion K is dominant source - Compton scattering $h\nu|_M\gtrsim0.1 m_ec^2$, $h\nu|_M\approx4.96kT$: Compton scattering reduces opacity about $20\%$ for $T>\SI{e8}{\kelvin}$
\item Kramers opacity: $T<\SI{e7}{\kelvin}$ -when FF, BF dominates: $\kappa_R\propto\rho T\expy{-7/2}$
\end{itemize}
\begin{block}{Free-free: radiation boost free-\Pelectron from lower to higher state}
Fully ionized-$Z_i$ mixture: 
\begin{align*}
&\rho\kappa_{\nu}(ff)=\sum_in_{Z_i}ne\sqrt{\frac{2m_e}{3\pi kT}}[\frac{4\pi Z_i^2e^6}{3m_e^2ch\nu^3}]g_{ff}(\nu)(1-\exp{-h\nu/kT})\\
&=\num{3.8e22}\overbrace{(X+1)}^{\propto n_e}\overbrace{[X+Y+B]}^{\propto\frac{X}{m_u}+\frac{4Y}{4m_u}+\sum_i\frac{X_iZ_i^2}{A_i}}\ [\si{cgs}]
\end{align*}
$\tau_{\Pelectron-int}\propto\frac{1}{v}\propto T\expy{-1/2}$: \Pelectron experience sharp acceleration $\approx\delta$ resulting in constant emission frequency, two body ion-\Pelectron encounter $\propto n_in_e$, \Pelectron have thermal distro $\propto\exp{-\epsilon/kT}$: $\rho j_{\nu}=n_en_iT\expy{-1/2}\exp{-\frac{h\nu}{kT}}$ - dalla legge di Kirchhoff, $j_{\nu}=4\pi\kappa_{\nu}^{abs}B_{\nu}(T)$, we have $\kappa_{\nu}^{abs}\propto\rho T\expy{-1/2}\nu\expy{-3}[1-\expy{-h\nu/kT}]$
\end{block}
\end{frame}

\begin{frame}{Kramer's opacity (BF, BB) and $H^-$ ions opacity}
\begin{block}{BF}
Hydrogenic atom with charge Z and one \Pelectron in bound state n:
\begin{align*}
&\sigma_{\nu}(Z)=n\expy{-5}\frac{8\pi}{3\sqrt{3}}\frac{Z^4m_ee\expy{10}}{c\hbar^3(h\nu)^3}:\ h\nu>\chi=\frac{Ze^2}{2a_Z}\\
&\rho\kappa_{\nu}(bf)=\sum_in_{Z_i}\sigma_{\nu}(Z_i)(1-\exp{-h\nu/kT})\\
&\kappa_R(bf)=\num{3e25}\underbrace{(1-X-Y)}_{n_{\exv{Z}}}\underbrace{(1+X+\frac{3}{4}Y)}_{n_e}\rho T\expy{-7/2}
\end{align*}
\end{block}
\begin{block}{BB: usually neglected in stellar interior}
\end{block}
\begin{block}{$H^-$ opacity $T<\SIrange{e4}{e5}{\kelvin}$}
BB/BF transition for $H^-$ don't follow Kramer as ion abundance in solar photosphere is sensitive to other considerations: law of mass action for $H^-=H+\Pelectron$ is $\frac{n_{H^-}}{n_Hn_e}=K_1(T)$, and sources of free electrons are metals (Na, K) $M^++\Pelectron=M$ so $\frac{n_{M^+}n_e}{n_M}=K_2(T)$, supposing $n_e=n_{M^+}$:
\begin{align*}
&\rho\kappa_{H^-}^{ff}\propto n_{H^-}n_eT\expy{-7/2}=n_Hn_MT\expy{-7/2}K_1(T)K_2(T)
\end{align*}
\end{block}
\end{frame}

\subsection{Convezione}\linkdest{traconv}

\begin{wordonframe}{Forza di archimede}
Una regione stellare \'e convettivamente stabile se una perturbazione di densit\'a infinitesima non cresce ad ampiezza finita.
\begin{equation*}\label{eq:buoyanc yEOM}
\rho\PtwoDy{t}{(\Delta r)}=-g\Delta\rho=-g[\Dcvar{\TDy{r}{\rho}}{e}-\Dcvar{\TDy{r}{\rho}}{amb}]\Delta r
\end{equation*}
La forza di Archimede ha verso opposta alla perturbazione se
\begin{equation*} 
[\Dcvar{\TDy{r}{\rho}}{e}-\Dcvar{\TDy{r}{\rho}}{amb}]>0
\end{equation*}
Riscrivo prima legge della termodinamica come $dq=c_P\,dT-\frac{\delta}{\rho}\,dP$
\begin{equation*} 
N^2=g(\frac{1}{\Gamma_1P}\TDy{r}{P}-\frac{1}{\rho}\TDy{r}{\rho})=g(\frac{1}{\densityscale{}}-\frac{g}{c_s^2})\label{eq:bvfs}
\end{equation*}
$N^2$ rappresenta la massima frequenza sotto cui pu\'o oscillare una particella di fluido sottoposta a onde di gravit\'a mantenendo l'equilibrio di pressione con l'ambiente.
\begin{equation*}
\PtwoDy{t}{(\Delta r)}=-N^2\Delta r
\end{equation*} 
che descrive un comportamento oscillatorio per $N^2>0$
\end{wordonframe}

\begin{frame}{Convective regions and temperature gradient}
\begin{columns}[T]
	\begin{column}{0.6\textwidth}
\begin{block}{Criterio di \sch e Ledoux: regioni convettive}
	\begin{align*}
	&\rho\PtwoDy{t}{(\Delta r)}=-g\Delta\rho=-g[\Dcvar{\TDy{r}{\rho}}{e}-\Dcvar{\TDy{r}{\rho}}{amb}]\Delta r\\
	&\nrad{}>\nad+\frac{\phi}{\delta}\nmu{}=\nad-\frac{\chi_{\mu}}{\chi_T}\TDly{(P)}{(\mu)}\\
	&\nrad{}>\nad\\
	&\frac{d\rho}{\rho}=\alpha\frac{dP}{P}-\delta\frac{dT}{T}+\phi\frac{d\mu}{\mu}\\
	&=-\frac{\chi_T}{\chi_{\rho}}\frac{\Delta T}{T}-\frac{\chi_{\mu}}{\chi_{\rho}}\frac{\Delta\mu(\Lambda)}{\mu}\\
	&\Delta P=0=\chi_{\rho}\frac{\Delta\rho}{\rho}+\chi_T\frac{\Delta T}{T}+\chi_{\mu}\frac{\Delta\mu}{\mu}
	\end{align*}
\end{block}
	\end{column}
	\begin{column}{0.4\textwidth}
\begin{figure}[!ht]
	\includegraphics[trim={0cm 0cm 1cm 0cm},clip, keepaspectratio,width=0.99\textwidth]{convectivestability}\label{fig:convectivestability}
\end{figure}
\end{column}\end{columns}
\end{frame}

\begin{frame}{Mixing length: gradiente ambientale e velocit\'a elementi convettivi}
Le stelle con massa $M\leq1.1\msun{}$ hanno una regione radiativa interna mentre la parte esterna \'e convettiva. Convezione esterni $\nabla>\nad$, $v\approx1-10\si{\kilo\meter\per\second}\approx c_s$- $\nabla\to\nad{}$ in interni stellari convettivi: $\TDy{r}{T}=(1-\frac{1}{\Gamma_2})\frac{T}{P}\TDy{r}{P}$, $v\approx\SI{100}{\meter\per\second}\ll c_s$
	\begin{align*}
&F=\frac{L}{4\pi r^2}=F_{rad}+F_{con}=-\frac{4acT^3}{3\kappa\rho}\TDy{r}{T}|_{amb}+\frac{1}{2}\rho vc_p[\TDy{r}{T}|_{Ad}-\TDy{r}{T}|_{amb}]\Lambda\\
&v^2=-\frac{1}{8}g\frac{\Delta\rho}{\rho}\Lambda=\frac{1}{8}g\frac{\Lambda}{H_P}Q(\nabla-\nad),\ Q=1-\TDly{T}{\mu},\ \Lambda=\alpha H_P\\
&F_{con}^{up}=\frac{1}{2}\rho vc_PT\frac{\lambda}{H_P}(\nabla-\nad{})
\end{align*} 
\begin{columns}[T]
\begin{column}{0.3\textwidth}
\begin{align*}
&f_r&=-g\Delta\rho(r)=0\tag{r}\\
&&\propto\Delta r
\end{align*} 
\end{column}
\begin{column}{0.7\textwidth}
Work done per unit volume moving bubble of $\Delta r$
\begin{align*}
&W(\Delta r)=-g\int_0^{\Delta r}\Delta\rho(\Delta r')d(\Delta r')=-\frac{1}{2}g\Delta\rho(\Delta r)\Delta r\\
&\exv{W(\Delta r)}_{\Delta r}=\frac{1}{4}W(\Lambda)=\frac{1}{2}\rho v^2
\end{align*} 
\end{column}\end{columns}
%Calcolo altezza scala di pressione, flusso convettivo e gradiente ambientale in funzione della velocit\'a media degli elementi
\end{frame}

\begin{wordonframe}{EOS e $\nad{}$}
Considero un'equazione di stato generica $\rho(P,T,\mu)$ e definita tramite:
\begin{align*}
&\frac{d\rho}{\rho}=\alpha\frac{dP}{P}-\delta\frac{dT}{T}+\phi\frac{d\mu}{\mu}\\
&P=\frac{\rho\gasconstant{}T}{\mu}\quad\Rightarrow\quad\alpha=\delta=\phi=1
\end{align*}
Definisco le lunghezze caratteristiche per variazione di densit\'a e pressione:
\begin{equation*}
\densityscale{}=-\frac{dr}{d\ln{\rho}},\ H_P=-\frac{dr}{d\ln{P}}
\end{equation*}
e i gradienti termici per il blob, l'ambiente e il gradiente di composizione chimica ambientale
\begin{equation*}
\nabla=\Dcvar{\TDly{P}{T}}{amb},\ \nabla_e=\Dcvar{\TDly{P}{T}}{e},\ \nmu{}=\Dcvar{\TDly{P}{\mu}}{amb}
\end{equation*}
\end{wordonframe}

\begin{wordonframe}{EOS e EOM}
Riscrivo l'equazione del moto utilizzando l'equazione di stato per scrivere la differenza di densit\'a in termini dei gradienti termici e di composizione chimica; inoltre supponendo il moto dell'elemento in equilibrio di pressione con l'ambiente e assumendo $\nmu{}_{blob}\approx0$ risulta:
\begin{equation*}
\PtwoDy{t}{(\Delta r)}=-g\frac{\delta}{H_P}[\nabla_e-\nabla-\frac{\phi}{\delta}\nmu{}]\Delta r
\end{equation*}
\end{wordonframe}

\begin{wordonframe}{Criterio di \sch/Ledoux}
Infine per ricavare il criterio di stabilit\'a per convezione suppongo  il moto del blob adiabatico:
\begin{equation*}
dq=c_P\,dT-\frac{\delta}{\rho}\,dP
\end{equation*}
da cui risulta:
\begin{equation*}
\nabla_e=\nabla_{ad}=\frac{P\delta}{T\rho c_P}
\end{equation*}
cio\'e una regione solare \'e stabile per convezione se
\begin{equation*}
\nrad{}<\nad+\frac{\phi}{\delta}\nmu{}\label{eq:ledoux}
\end{equation*}
dove ho usato $\nabla_{amb}=\nrad{}$, cio\'e il gradiente che si ha nel caso l'energia sia trasportata dai fotoni.
\end{wordonframe}

\subsection{Teoria della mixing-length.}

\begin{wordonframe}{Convezione in esterni stellari}
    \begin{columns}[T]
        \begin{column}{0.5\textwidth}
            \begin{align*}
                &F=F_{con}+F_{rad}=\frac{\lsun{}}{4\pi r^2}\\
                &F_{con}=\exv{\rho vc_P\Delta T}\label{eq:convectiveflux}\tag{Avg on r-sphere}\\
                &\frac{\Delta T}{T}\approx\frac{1}{T}\PDy{r}{(\Delta T)}\underbrace{\frac{l_m}{2}}_{\Delta r\approx\frac{l_m}{2}}=(\nabla-\nabla_e)\frac{l_m}{2}\frac{1}{H_P}\\
                &-g\frac{\Delta\rho}{\rho}\tag{Force/unit mass}\\
                &v^2=g\delta(\nabla-\nabla_e)\frac{l_m^2}{8H_P}\\
                &f=\frac{4acT^3}{3\kappa\rho}|\PDy{n}{T}|\tag{Rad. exchange}\\
                &\PDy{t}{T_e}=-\frac{\lambda}{\rho Vc_P}\\
                &\Dcvar{\TDy{r}{T}}{e}=\Dcvar{\TDy{r}{T}}{ad}-\frac{\lambda}{\rho Vc_Pv}\label{eq:Tchangelength}\tag{$\Delta T$ per unit length by blob}\\
                &|\PDy{n}{T}|\approx\exv{\Delta T}\\
                &\frac{\nabla_e-\nad{}}{\nabla-\nabla_e}=\frac{6acT^3}{\kappa\rho^2c_Pl_mv}
            \end{align*}
        \end{column}
        \begin{column}{0.45\textwidth}
            Eccesso di calore trasportato dai blob $c_P\Delta T$ rispetto all'ambiente, il cui cammino libero medio \'e la mixing-length $l_m=\alpha H_P$, che da luogo al flusso di energia. Assumo la forza di galleggiamento media sia la met\'a di quella alla superficie r ($\exv{\Delta \rho}\approx\frac{\Delta\rho}{2}$) e lo spostamento medio $\frac{l_m}{2}$: assumo che met\'a del lavoro vada in energia cinetica del blob. Il modulo del flusso radiativo \'e proporzionale al gradiente termico in direzione normale alla superficie del blob quindi l'energia scambiata dall'intera superficie S del blob \'e $\lambda=Sf$ che determina, per la prima legge della termodinamica, dato V volume del blob, una variazione di temperatura per unit\'a di tempo. Le 5 equazioni determinano completamente le variabili $F_{rad}, F_{con}, v, \nabla_e, \nabla$ in funzione di $P,T,l(r),m(r),c_P,\nad{},\nrad{},g$.
%Determino il valor medio della differenza di temperatura prendendo come valore caratteristico dello spostamento del blob $\Delta r\approx\frac{l_m}{2}$ (moti in entrambi i versi)
        \end{column}
    \end{columns}
    %Una maggiore efficienza del trasporto convettivo di energia si riflette in una minore differenza tra il gradiente di temperature adiabatico ed effettivo. Il gradiente termico ambientale $\nabla$ e del blob $\nabla_e$ sono determinati inserendo le espressioni per il flusso radiativo e il flusso convettivo in.
cubic equation for $(\nabla-\nabla_e)$
%\begin{figure}[!h]
%   \includegraphics[ width=0.99\textwidth,keepaspectratio]{proportionflux}
%   \subcaption{Profilo radiale (profondit\'a in \si{\kilo\meter}) del flusso convettivo $F_c$ rispetto al flusso totale $F$, della super-adiabaticit\'a $\nabla-\nad{}$ e regioni di ionizzazione idrogeno e $\cel{He}{4}{}{}$. Da \cite{christensen1997effects}.}\label{fluxproportion}
%\includegraphics[keepaspectratio,width=0.9\textwidth]{specificheatnablaa}
%\subcaption{Profilo radiale di $c_P$ e $\nabla_a$: si ha cambiamento di comportamento nelle regioni di ionizzazione parziale di idrogeno ed elio. Da \cite{stix91sun}.}\label{specificheatnablaa}
%\end{figure}
\end{wordonframe}

\begin{wordonframe}{Solution for ML variables}
    \begin{columns}[T]
        \begin{column}{0.5\textwidth}
            \begin{align*}
                &U=\frac{3acT^3}{c_P\rho^2\kappa l_m^2}\sqrt{\frac{8H_P}{g\delta}}\propto\alpha\expy{-2}H_P\expy{-\frac{3}{2}}\\
                &W=\nrad{}-\nad{}\\
                &\Rightarrow\nabla_e-\nad{}=\nabla-\nad{}-(\nabla-\nabla_e)=2U\sqrt{\nabla-\nabla_e}\\
                &(\nabla-\nabla_e)\expy{\frac{3}{2}}=\frac{8}{9}U(\nrad{}-\nabla)\\
                &(\xi-U)^3+\frac{8U}{9}(\xi^2-U^2-W)=0
            \end{align*}
        \end{column}
        \begin{column}{0.4\textwidth}
            Replacing expression for $v^2$ change temperature per unit length converted in grad form we get third eq. Using explicit formula for $F_{con}$ and $F_{con}=\frac{4acG}{3}\frac{T^4m}{\kappa Pr^2}(\nrad{}-\nabla)$ we get fourth eq. We have reduced yhe five equation to two for $\nabla, \nabla_e$, and defining  $\xi$ as positive root of $\xi^2=\nabla-\nad{}+U^2$ and looking at third eq we see that it's a quadratic eq for $\sqrt{\nabla-\nabla_e}$ with sol. $\sqrt{\nabla-\nabla_e}=-U+\xi$ and inserting this in the fourth eq we arrive at cubic for $\xi$ resol for any $U$ and $W$ params.
        \end{column}
    \end{columns}
    
\end{wordonframe}
%\section{From micro-physics to thermodynamics: Equazione di stato, opacit\'a}\linkdest{micro2macro}%to be throw into stellarplasmamodel

%\begin{multicols}{2}%https://newbedev.com/how-to-explicitly-split-long-toc-in-beamer
%   \tableofcontents[currentsection]{cherryframes}
%\end{multicols}

%\begin{multicols}{2}%https://newbedev.com/how-to-explicitly-split-long-toc-in-beamer
%   \tableofcontents[currentsection]{cherryframes}
%\end{multicols}
\section{Energy production}\linkdest{luminositysourcessinks}

\subsection{Work against gravity}\linkdest{epsilong}

\begin{frame}{Work done in expansion/contraction: }

\end{frame}

\subsection{Fusione nucleare}\linkdest{epsilonn}

\begin{frame}{Sezione d'urto fusion nucleare}
$E$ l'energia cinetica nel centro di massa dei nuclei: $\sigma(E)=\pi\lambdabar^2*P_0(E)*S(E)$
prodotto della sezione d'urto geometrica (nel riferimento del CM: $\sigma\approx\sum_{l=0}^{\frac{R}{\lambdabar}}(2l+1)\pi\lambdabar^2=\pi(R+\lambdabar)^2$
per energie tipiche degli interni stellari approx onda S), della probabilit\'a di attraversamento della barriera coulombiana e del fattore astrofisico. La lunghezza d'onda di de Broglie relativa delle particelle descrive l'indeterminazione sulla posizione nell'urto di due particelle con momento relativo p $\lambdabar=\frac{\hbar}{p}=\frac{\hbar}{\sqrt{2mE}}$.

L'energia potenziale dovuta all'interazione di due nuclei $Z_1$ e $Z_2$ a distanza r contiene un contributo delle altre cariche presenti nel plasma $U=\frac{Z_1Z_2e^2}{r}+U_s(r_{12})$
l'energia potenziale non schermata e contributo della nuvola elettronica: $U_s$ aumenta la probabilit\'a di attraversamento della barriera coulombiana. Fattore moltiplicativo: $f=\exp{-\midfrac{U_0}{KT}}$ dove $U_0=U_s(0)$ poich\'e $r\ll r_D$ e considerando solo la correzione al fattore di penetrazione ($E_G\gg U_0$).

Per determinare $U_0$ considero l'energia potenziale di $Z_1$ e $Z_2$ a distanza $r$
\begin{equation*}
U=Z_2e\int_{\infty}^r\PDy{r_1}{\phi_1}\,dr_1=\frac{Z_1Z_2e^2\exp{-\midfrac{r}{r_D}}}{r},\ U_s=U-\frac{Z_1Z_2e^2}{r}\approx\frac{Z_1Z_2e^2}{r_D}
\end{equation*}
\end{frame}

\begin{frame}{Energia prodotta in reazioni di fusione???}
$S(E)$ descrive l'interazione a livello nucleare: debolmente dipendente dall'energia in assenza di risonanze.
La probabilit\'a di attraversamento della barriera coulombiana: $P_0(E)=\exp{-2\pi\eta},\ \eta=\sqrt{\frac{m}{2}}\frac{Z_1Z_2e^2}{\hbar E\expy{\frac{1}{2}}}$
Per i nuclei di carica $Z_1$, $Z_2$ e m massa ridotta: $\sigma(E)=\frac{S(E)}{E}\exp{-2\pi\eta}$
Il rate per coppia di particelle
\begin{equation*}
\exv{\sigma v}=\num{1.3005e-15}[\frac{Z_1Z_2}{AT_6^2}]\expy{\frac{1}{3}}fS_{eff}\exp{-\tau}\si{\cubic\cm\per\second},\ \tau=\frac{3E_G}{kT}\approx\num{42.487}(Z_1^2Z_2^2AT_6\expy{-1})\expy{\frac{1}{3}}
\end{equation*}
$S_{eff}$ \'e il risultato dell'espansione dell'integrando per $\invers{\tau}\ll1$ ed estrapolato a $E_G$ a partire dal valore $S(0)$ determinato dalla fisica nucleare.
La funzione $\epsilon(\rho,T,X_i)$ \'e determinata dalla somma di tutti i contributi
\begin{equation*}
\epsilon_{ij}=Q_{ij}\frac{n_in_j}{\rho(1+\delta_{ij})}\lambda_{ij}=\frac{1}{1+\delta_{ij}}Q_{ij}\frac{\rho N_A^2X_jX_k}{{A_iA_j}}\exv{\sigma v}_{ij}\label{eq:energyrate}
\end{equation*}
dove $Q_{ij}$ \'e l'energia liberata per reazione tra nucleo di specie i e j e $\exv{\sigma v}_{ij}$ \'e il rate di reazione per coppia di particelle, mediata su MB- distro
$f(E)dE\propto\frac{E\expy{\frac{1}{2}}}{(kT)\expy{\frac{3}{2}}}\exp{-\frac{E}{kT}}\,dE$:
$S(E)\exp{-\frac{E}{kT}-\frac{b}{\sqrt{E}}}$
ha forma approssimativamente gaussiana il cui massimo $E_G$, energia pi\'u probabile di reazione, e FWHM sono: $E_G=\SI{5.665}{\kilo\ev} A\expy{\frac{1}{3}}T_7\expy{\frac{2}{3}}$ $\Delta E=4.249\si{\kilo\ev}W\expy{\frac{1}{6}}T_7\expy{\frac{5}{6}}$
posto $W=Z_i^2Z_j^2A=Z_i^2Z_j^2\frac{A_iA_j}{A_i+A_j}$.
%\begin{equation*}
%\exv{\sigma v}\propto b\expy{1/3}T\expy{-2/3}\exp{-\frac{b\expy{2/3}}{t\expy{1/3}}}
%\end{equation*}
\end{frame}

\subsection{Catena PP}\linkdest{epsilonpp}

\begin{frame}{PP1: twice $\Pproton(\Pproton,\Pnue\APelectron)d(p,\gamma)^3He$ and $^3He(^3He,2^1H)^4He$}
\begin{columns}[T]
	\begin{column}{0.55\textwidth}
\begin{itemize}
	\item $\tau_p(d)\ll[\tau_{^3He}(d)]_e\ll[\tau_d(d)]_e$: $\frac{\dot{D}}{H}=\frac{H}{2}\exv{\sigma v}_{pp}-H(\frac{D}{H})\exv{\sigma v}_{dp}$ - \keyword{D evolution} $(\frac{D}{H})=\frac{\exv{\sigma v}_{pp}}{2\exv{\sigma v}_{dp}}$, evolution: $(\frac{D}{H})_t=(\frac{D}{H})_e-[(\frac{D}{H})_e-(\frac{D}{H})_0]\exp{-\frac{t}{\tau_p(d)}}$
	\item \keyword{$^3He$ evolution} - $\frac{\dot{^3He}}{H}=\frac{H}{2}\exv{\sigma v}_{pp}-H(\frac{^3He}{H})^2\exv{\sigma v}_{^3He^3He}$ - 
\scalebox{0.8}{	\[\frac{^3He}{H}|_t=0+\sqrt{\frac{\exv{\sigma v}_{pp}}{\exv{\sigma v}_{^3He^3He}}}\tanh(t\sqrt{\frac{H}{2}\exv{\sigma v}_{PP}H\exv{\sigma v}_{^3He^3He}})\]}
\item Produzione energia - $\epsilon_{PPI}^e(T_0=\SI{15}{\mega\ev})(\frac{T}{T_0})\expy{3.9}$
\begin{align*}
&\epsilon_{PPI}=\frac{(6.936-0.265)\si{\mega\ev}H^2\exv{\sigma v}_{pp}}{2\rho}\\
&+\frac{\SI{12.861}{\mega\ev}H^2\exv{\sigma v}_{^3He^3He}}{2\rho}(\frac{^3He}{H})^2\\
&\epsilon_{PPI}^e(T)=6.551N_A\exv{\sigma v}_{PP}(\frac{X_H}{m_H})^2\rho N_A\frac{\si{\mega\ev}}{\si{\second\gram}}
\end{align*}
\end{itemize}
	\end{column}
	\begin{column}{0.45\textwidth}
	\begin{figure}[!ht]
\includegraphics[trim={0cm 0cm 1cm 0cm},clip, keepaspectratio,height=0.28\textheight]{He3eq}
\includegraphics[trim={0cm 0cm 1cm 0cm},clip, keepaspectratio,height=0.28\textheight]{PP1DHet}
\includegraphics[trim={0cm 0cm 1cm 0cm},clip, keepaspectratio,height=0.28\textheight]{pplifetime}
	\end{figure}
	\end{column}
\end{columns}
\end{frame}

\begin{frame}{Network reazioni PP completo}

\begin{columns}[T]
\begin{column}{0.55\textwidth}
Rapid $^8B/^8Be$ decays: $^7Be(p,\gamma)^8B(\APelectron,\nu)^8Be(\alpha)\alpha$ as $^7Be+p\to2\alpha+\gamma$, $\TDof{t}(^7Li+^7Be)\approx0$
\begin{align*}
&Q_{PPI}=27.73-2\exv{E}_{\nu}^{pp}=\SI{26.19}{\mega\ev}\\
&Q_{PPII}=26.73-\exv{E}_{\nu}^{^7Be}-\exv{E}_{\nu}^{pp}=\SI{25.65}{\mega\ev}\\
&Q_{PPIII}=26.73-\exv{E}_{\nu}^{^8B}-\exv{E}_{\nu}^{pp}=\SI{19.75}{\mega\ev}\\
&\epsilon_{PP}=\frac{Q_{4H\to^4He}}{\rho}\dot{^4He}[0.98F_{PPI}+0.96F_{PPII}\\
&+0.74F_{PPIII}],\ f_{PPI}=\frac{r_{^3He^3He}}{r_{^3He^3He}+r_{\alpha^3He}}\\
&F_{PPII}=(1-F_ {PPI})\frac{r_{e^7Be}}{r_{e^7Be}+r_{p^7Be}}
\end{align*}
\end{column}
\begin{column}{0.45\textwidth}
\begin{figure}[!ht]
	\includegraphics[trim={0cm 0cm 1cm 0cm},clip, keepaspectratio,width=0.72\textwidth]{ppchainsequi}
	\includegraphics[trim={0cm 0cm 1cm 0cm},clip, keepaspectratio,width=0.72\textwidth]{ppfraction}
\end{figure}
\end{column}
\end{columns}
\begin{align*}
&\dot{(^3He)}=\frac{H^2}{2}\exv{\sigma v}_{pp}-2(^3He)^2\frac{\exv{\sigma v}_{^3He^3He}}{2}-(^3He)(^4He)\exv{\sigma v}_{\alpha^3He}\\
&(^3He)_e=\frac{-(^4He)\exv{\sigma v}_{\alpha^3He}+\sqrt{(^4He)^2\exv{\sigma v}_{\alpha^3He}^2+2H^2\exv{\sigma v}_{\alpha^3He}\exv_{\sigma v}_{^3He^3He}}}{2\exv{\sigma v}_{^3He^3He}}
\end{align*}
\end{frame}


\subsection{Ciclo CN-NO:}\linkdest{epsiloncno}

\begin{frame}{Biciclo CNO}
dipendenza da T
flusso neutrini
Modalit\'a combustione H in He: sequenza principale inferiore/superiore
\end{frame}

\begin{frame}{CNOF Network reactions: $4^1H\to^4He+2\APelectron+2\Pnue$}
\begin{columns}[T]
\begin{column}{0.5\textwidth}
\begin{itemize}
\item CNOF elements acts as catalyst: relative initial CNOF abundance are important - produced in previous gen stars at He-burning stages ($^{12}C$, $^{16}O$, less $^{14}N$: solar $^{12}C:^{14}N:^{16}O=10:3:24$)
\item 4 cycle: active cycle influences heavy element abundance - $(p,\gamma)$ compete with $(p,\alpha)$ on $^{15}N$, $^{17}O$, $^{18}$, $^{19}F$: $(p,\alpha)$ faster over entire T-range except $^{17}O$/$^{18}O$ at $T<\SI{20}{\mega\kelvin}$
\item At hydro-burning ($T<\SI{55}{\mega\kelvin}$) much faster than p-induced reactions, at $T>\SI{100}{\mega\kelvin}$ also other reactions are important (HCNO)
\end{itemize}
\end{column}
\begin{column}{0.5\textwidth}
	\begin{figure}[!ht]
	\includegraphics[trim={0cm 0cm 1cm 0cm},clip, keepaspectratio,height=0.28\textheight]{CNO}
	\includegraphics[trim={0cm 0cm 1cm 0cm},clip, keepaspectratio,height=0.28\textheight]{HCNO}
\end{figure}
\end{column}
\end{columns}
\end{frame}

\begin{frame}{CNO1: equilibrium properties}
\begin{columns}[T]
	\begin{column}{0.5\textwidth}
		\begin{itemize}
			\item Elements involving beta-decay reaches equilibrium in few minutes: $\dot{^{13}N}=0$ - $(^{13}N)_t=\frac{\tau_{\beta}(^{13}N)}{\tau_p{^{12}C}}^{12}C[1-\exp{-\frac{t}{\tau_{\beta}(^{13}N)}}]$ - $(\frac{^{13}N}{^{12}C})_e=\frac{\tau_{\beta}(^{13}N)}{\tau_p(^{12}C)}$
			\item At equilibrium ratio of abundance of $^{12}C$, $^{13}C$, $^{14}N$, $^{15}N$ are given by inverse ratio of reaction time (ie $(\frac{^{14}N}{^{12}C})_e=\frac{\exv{\sigma v}_{^{12}C(p,\gamma)}}{\exv{\sigma v}_{^{14}N(p,\gamma)}}=\frac{\tau_p(^{14}N)}{\tau_p(^{12}C)}$), fractional abundance $\frac{(^{12}C)_e}{\sum CNO1}=\frac{\tau_p(^{12}C)}{\tau_p(^{12}C)+\tau_p(^{13}C)+\tau_p(^{14}N)+\tau_p(^{15}N)}$
		\end{itemize}
	\end{column}
	\begin{column}{0.5\textwidth}
		\begin{figure}[!ht]
			\includegraphics[trim={0cm 0cm 1cm 0cm},clip, keepaspectratio,height=0.28\textheight]{CNO1ab}
			\includegraphics[trim={0cm 0cm 1cm 0cm},clip, keepaspectratio,height=0.28\textheight]{CNOXivsT}
		\end{figure}
	\end{column}
\end{columns}
\end{frame}

\begin{frame}{CNO1: energy production rate}

\begin{align*}
&\epsilon_{CNO1}=\sum_{i\to j}\epsilon_{i\to j}=\frac{1}{\rho}(Q_{i\to j}-\exv{E}_{\nu}^{i\to j})r_{i\to j}\\
&\rho\epsilon^e_{CNO1}=\SI{3.458}{\mega\ev}\frac{(^{12}C)_e}{\tau_p(^{12}C)}+\SI{7.551}{\mega\ev}\frac{(^{13}C)_e}{\tau_p(^{13}C)}+\SI{9.055}{\mega\ev}\frac{(^{14}N)_e}{\tau_p(^{14}N)}\\
&+\SI{4.966}{\mega\ev}\frac{(^{15}N)_e}{\tau_p(^{15}N)}\\
&\epsilon^e_{CNO1}\approx\SI{25.030}{\mega\ev}N_A\exv{\sigma v}_{^{14}N(p,\gamma)}(\sum_{CNO1}\frac{X_i}{M_i})\frac{X_H}{M_H}\rho N_A\si{\mega\ev\per\second\per\gram}\\
&\epsilon^e_{CNO1}=\epsilon^e_{CNO1}(T_0)(\frac{T}{T_0})\expy{16.7},\ T_0\approx\SI{25}{\mega\kelvin}\\
%%Q
&Q_{^{12}C(p,\gamma)^{13}N(\beta^+\nu)}-\exv{E}_{\nu}^{^{13}N(\beta^+\nu)}=(1.944+2.22-0.706)\si{\mega\ev}\\
&Q_{^{13}C(p,\gamma)}=\SI{7.551}{\mega\ev}\\
&Q_{^{14}N(p,\gamma)^{15}O(\beta^+\nu)}-\exv{E}_{\nu}^{^{15}O(\beta^+\nu)}=(7.297+2.754-0.996)\si{\mega\ev}\\
&Q_{^{15}N(p,\alpha)}=\SI{4.966}{\mega\ev}
\end{align*}

\end{frame}

\subsection{He-C-Ne-O-Si-Burning}\linkdest{epsilonheavy}

\begin{frame}{Combustion He}
\begin{columns}[T]
	\begin{column}{0.5\textwidth}
	\begin{itemize}
	\item Deps $M_*$ amd $Z$: hydrostatic He burning in massive star $\rho=\SIrange{e2}{e5}{\gram\per\cubic\cm}$, $T=\SIrange{0.1}{0.4}{\giga\kelvin}$
	\item Reactiona during He-buring
	\begin{align*}
&^4He(\alpha\alpha,\gamma)^{12}C\ Q=\SI{7274.7}{\kilo\ev}\\
&^{12}C(\alpha,\gamma)^{16}O\ Q=\SI{7161.9}{\kilo\ev}\\
&^{16}O(\alpha,\gamma)^{20}Ne\ Q=\SI{4729.8}{\kilo\ev}\\
&^{20}Ne(\alpha,\gamma)^{24}Mg\ Q=\SI{9316.6}{\kilo\ev}
	\end{align*}
\item Energy production by $3\alpha$:
\begin{align*}
&\epsilon_{3\alpha}=\frac{Q_{3\alpha}}{\rho}r_{3\alpha}=\frac{Q_{3\alpha}}{\rho}\frac{N_{\alpha}\lambda_{3\alpha}}{3}\\
&=\num{3.1771e14}\frac{\rho^2X_{\alpha}^3}{T_9^3}\exp{-\frac{4.4040}{T_9}}\si{\mega\ev\per\gram\per\second}\\
&\epsilon_{3\alpha}(T)=\epsilon_{3\alpha}(T_0)(\frac{T}{T_0})^{41}
\end{align*}
		\end{itemize}
	\end{column}
	\begin{column}{0.5\textwidth}
		\begin{figure}[!ht]
			\includegraphics[trim={0cm 0cm 1cm 0cm},clip, keepaspectratio,height=0.28\textheight]{Heburningelems}
			\includegraphics[trim={0cm 0cm 1cm 0cm},clip, keepaspectratio,height=0.28\textheight]{3alphalevels}
		\end{figure}
	\end{column}
\end{columns}
\end{frame}


\begin{frame}{Produzione nuclei fino al Fe56}
3$\alpha$: $He4+\alpha\to C12$, $C12+\alpha\to O16$
Produzione neutroni liberi
Fusione $C12$, fotodisintegrzione $Ne20$, fusione $O16$, fotodisintegrzione $Si28$, catture $\alpha$ su nuclei fino a produzione $Fe56$
\end{frame}

\begin{frame}{Cattura neutronica: processi r e s}
\todo{picchi r e s nella nella curva universale delle abbondanze}
\end{frame}

\begin{frame}{Nuclear burning efficiency}
\begin{equation*}
\epsilon_{ij}(\rho,T,X_i)=Q_{ij}\frac{n_in_j}{\rho(1+\delta_{ij})}\lambda_{ij}=\frac{1}{1+\delta_{ij}}Q_{ij}\frac{\rho N_A^2X_jX_k}{{A_iA_j}}\exv{\sigma v}_{ij}
\end{equation*}
dove $Q_{ij}$ \'e l'energia liberata per reazione tra nucleo di specie i e j e $\exv{\sigma v}_{ij}$ \'e il rate di reazione per coppia di particelle; $X_i$ indica la frazione in  massa della specie i
\begin{columns}[T]
	\begin{column}{0.5\textwidth}
		\begin{align*}
		&PP\ \exv{\sigma v}\propto T\expy{3.9}\ E_C=\SI{0.55}{\mega\ev}\\
		&P^{14}N\ \exv{\sigma v}\propto T\expy{20}\ E_C=\SI{2.27}{\mega\ev}\\
		&\alpha+^{12}C\ \exv{\sigma v}\propto T\expy{42}\ E_C=\SI{3.43}{\mega\ev}\\
		&^{16}O+^{16}O\ \exv{\sigma v}\propto T\expy{182}\ E_C=\SI{14.07}{\mega\ev}
		\end{align*}
	\end{column}
	\begin{column}{0.5\textwidth}
		\begin{align*}
		&E_C\approx Z_1Z_2\si{\mega\ev}
		\end{align*}
	\end{column}
\end{columns}
\end{frame}

\begin{frame}{Ripasso perdita neutrini}\linkdest{epsilonnu}
    
\end{frame}


\section{Ordini di grandezza}

\begin{frame}{Energia interna}
\begin{equation*}
E_i=\int_0^Mu\,dm=\frac{3}{2}\int_M\frac{P}{\rho}\,dm\label{eq:traslintenergy}
\end{equation*}
\end{frame}

\section{Metodi per integrazione equazioni di struttura e raccordo con modelli dell'atmosfera stellare}\linkdest{stellarsolve}

\subsection{4 Structure ODE with boundary conditions}\linkdest{fourODE}

\begin{frame}{Equazioni struttura di equilibrio}
\begin{align*}
&\TDy{m}{r}=\frac{1}{4\pi r^2\rho}\\
&\TDy{m}{P}=-\frac{Gm}{4\pi r^4}\overbrace{[-\frac{1}{4\pi r^2}\PtwoDy{t}{r}]}^{\tau_{hyd}}\\
&\TDy{m}{T}=-\nabla\frac{T}{p}\frac{Gm}{4\pi r^4}\\
&\TDy{m}{L}=\epsilon-\epsilon_{\nu} \underbrace{-c_P[\TDy{t}{T}-\nad\frac{T}{P}\TDy{t}{P}]}_{-c_P\PDy{t}{T}+\frac{\delta}{\rho}\PDy{t}{P}: \tkh}\\
&\TDy{t}{X_s}\frac{1}{A_s}=\sum_{production}\rho^{n_h+n_k-1}n_p\frac{X_h^{n_h}X_k^{n_k}}{A_h^{n_h}A_k^{n_k}}\frac{\exv{\sigma v}_{hk}}{m_H^{n_h+n_k-1}n_h!n_k!}\\
&-\sum_{distruction}\rho^{n_d+n_j-1}n_d\frac{X_s^{n_d}X_j^{n_j}}{A_s^{n_d}A_j^{n_j}}\frac{\exv{\sigma v}_{sj}}{m_H^{n_d+n_j-1}n_d!n_j!}
\end{align*}
\end{frame}

\begin{frame}{Condizioni al bordo: espansione condizioni al centro ($m\to0$)}
\begin{itemize}
\item Le 4+I equazioni determinano $r$, $P$, $T$, $L$, $X_s$ specificata la massa e composizione iniziale (omogenea)
\item $\tau_n\gg\tkh\gg\tau_{dyn}$: solve 4 structure equations at time t - do time step $\Delta t$ and determine new composition - solve structure at $t+\Delta t$ with new composition
\item Solution of 4 structure equations require 4 boundary condition: 2 at surface (atmospere model without diffusion approx, PP geometry), parametri $\rho_c,T_c$; 2 at center (via Taylor expansion for $m=m'$)
\end{itemize}
\begin{columns}[T]
\begin{column}{0.65\textwidth}
\begin{align*}
&r=(\frac{3}{4\pi\rho_c})\expy{1/3}{m'}\expy{1/3}\\
&P=P_c-\frac{3G}{8\pi}(\frac{4\pi\rho_c}{3})\expy{4/3}{m'}\expy{2/3}\\
&L=\epsilon_cm'\\
&T^4=T_c^4-\frac{1}{2ac}(\frac{3}{4\pi})\expy{2/3}\kappa_c\epsilon_c\rho_c\expy{4/3}{m'}\expy{2/3}\tag*{rad}\\
&\ln{T}=\ln{T_c}-(\frac{\pi}{6})\expy{1/3}G\frac{{\nad}_c\rho_c\expy{4/3}}{P_c}{m'}\expy{2/3}\tag*{con}
\end{align*}
\end{column}
\begin{column}{0.35\textwidth}
At surface $m=M$, $L=L_s$, atmospheric model for $P$, $T$ - atmosphere defined by $g=\frac{GM}{R^2}$, $T_e$, composition: provides $P_s$ at $\tau$ where diffusion approx. starts to be valid
\end{column}
\end{columns}
\end{frame}

\begin{frame}[fragile]{Al centro: 1-order expansion of structure equations as $m\to0$}
	\begin{align*}
	&d(r^3)=\frac{3}{4\pi\rho}:\ r=(\frac{3}{4\pi\rho_c})^{1/3}m^{1/3}\\
	&l=(\epsilon_n-\epsilon_{\nu}+\epsilon_g)_cm\\
	&\TDy{m}{P}=-\frac{GM}{4\pi r^4}=-\frac{G}{4\pi}(\frac{4\pi\rho_c}{3})^{4/3}m^{-1/3}\\
	&P-P_c=-\frac{3G}{8\pi}(\frac{4\pi\rho_c}{3})^{4/3}m^{2/3}\\
	&\TDy{r}{P}\propto\frac{m}{r^2}\to0\\
	&\TDy{m}{T}=-\frac{T}{P}\frac{Gm}{4\pi r^4}\nabla:\\
	&T^4-T_c^4=-\frac{1}{2ac}(\frac{3}{4\pi})^{2/3}\kappa_c(\epsilon_n-\epsilon_{\nu}+\epsilon_g)_c\rho_c^{4/3}m^{2/3}\tag{Radiative}\\
	&\log{T}-\log{T_c}=-(\frac{\pi}{6})^{1/3}G\frac{\nad{}_{,c}\rho_c^{4/3}}{P_c}m^{2/3}\tag{convective}
	\end{align*}
\end{frame}

\begin{frame}{Condizione alla superficie: fitting mass}
Fotosfera: regione di emissione della radiazione $\tau=\int_R^{\infty}\kappa\rho\,dr=\exv{\kappa}\int_R^{\infty}\rho\,dr=\frac{2}{3}$. HE: pressure is given by weight of matter above $P(R)=\frac{2}{3}\frac{GM}{R^2}\frac{1}{\exv{\kappa}}$. $L=4\pi R^2\sigma T_e^4$.
Fitting mass: interior equations still valid, $l\approx L$, $m\approx M$, $M-m_F$ can have thermal readjustment.
Model of stellar atmosphere for given M, $X_i(m)$: two parameter solution $(R,T_e)$ for $l=l_F^{ex}$, $P=P_F^{ex}$, $T=T_F^{ex}$, $r=r_F^{ex}$ integrated down to $m_F$ to be matched with interior solutions.
4 equations for 4 DOF: $T_c$, $P_c$, $T_e$, $R$, and we can invert $r_F^{ex}(R,L)$
\begin{columns}[T]
\begin{column}{0.5\textwidth}
\begin{align*}
&P_F^{ex}(R(r_F^{ex},L),L)=\pi(r_F^{ex},L)\\
&T_F^{ex}(R(r_F^{ex},L),L)=\theta(r_F^{ex},L)
\end{align*}
\end{column}
\begin{column}{0.445\textwidth}
	Outer boundary conditions for interior solution such that there is alwaysan outer solution that matches interior
\begin{align*}
&P_F^{in}=\pi(r_F^{in},L)\\
&T_F^{in}=\theta(r_F^{in},L)
\end{align*}
\end{column}
\end{columns}

\end{frame}

\begin{frame}{Deps of interior solution on surface condition}
m,l approx const: raidative envelope $\PDy{P}{T}=\frac{3}{64\pi\sigma G}\frac{\kappa l}{T^3m}$, $\kappa=\kappa_0P^aT^b$ con $a>0(=1)$, $b<0(=-4.5)$ si ha $\frac{T^{3-b}}{P^a}\PDy{P}{T}=\frac{3\kappa_0}{64\pi\sigma G}\frac{l}{m}$: solution $T^{4-b}=B(P^{1+a}+C)$ (''adiabat'')
\begin{itemize}
	\item $C=0\Rightarrow \nabla=\TDly{P}{T}\approx0.235<\nad=\frac{2}{5}$ ($P=0|_{T=0}$)
	\item $C>0: \nabla<0.235$ (envelope sol. for $T_e>\SI{e4}{\kelvin}$): $P^2\ll C: T\approx\const{}$ (toward surface), $P\gg C$ (converge to $C=0$ solution)
	
\end{itemize}
Condition at surface do not affect greatly interior solution.
\begin{columns}[T]
\begin{column}{0.45\textwidth}
\begin{itemize}
	\item $0.4>\nabla>0.235$ (solutions below those with $C=0$): $\nad=\frac{2}{5}$ (lower in ionization regions),
	$\nabla>\nad$ (interior solution have to be terminated at convection border).
\end{itemize}
Small variation of $T_e$ (numerical/physical uncert.or super-adiabat) lead to curves widely separated in interior.
\end{column}
\begin{column}{0.5\textwidth}
\begin{figure}[!ht]
	\includegraphics[trim={0cm 0cm 1cm 0cm},clip, keepaspectratio,width=0.99\textwidth]{envelopelnTlnP}\label{fig:envelopelnTlnP}
\end{figure}
\end{column}
\end{columns}

\end{frame}

\begin{frame}{Simplified atmosferic model: grey atmosphere}
Atmosphere model: usually PP geometry and solve HE equation using non grey radiative transport and EOS and convection if needed
\begin{align*}
&\TDy{\tau}{P}=\frac{g}{\kappa}\tag*{HE using $\tau$ as indip var}\\
&T^4=\frac{3}{4}T_e^4(\tau+\frac{2}{3})\tag*{or solar $T(\tau)$ empirical relation}
\end{align*}
integration from $\tau\approx0$ where $T\approx0$, $P\approx0$ down to $\tau=\frac{2}{3}$ where $T=T_e$ - using shooting method.
\end{frame}

\begin{frame}{Chemical mixing: diffusion and convection}

\end{frame}

\subsection{Approssimazione politropa}\linkdest{poly}

\begin{frame}{Polytropic relation}
    \framesubtitle{$P=K\rho^{\gamma}=K\rho^{1+\frac{1}{n}}$}
    \begin{itemize}
\item[*] Additional relation between T-P: ie $1^{st}$ law $\TDy{T}{Q}=\TDy{T}{U}+P\TDy{T}{V}=C=c_V+P(\TDy{T}{V})_P$ - ideal gas $c_P-c_V=\frac{R}{\mu}$: $\frac{dT}{T}+\frac{1}{n}\frac{dV}{V}=0$, $n=\frac{c_V-C}{c_P-c_V}$. \keyword{trasformazione politropa}: t. quasi-statica in  con $c=\TDy{Q}{T}$ (calore specifico) assegnato - K is free parameter.
%$TV\expy{\gamma-1}=\const$, $PV\expy{\gamma}=\const$, $P\expy{1-\gamma}T\expy{\gamma}=\const$ ($0=dE-\frac{P}{\rho^2}d\rho$). 
\item [*] EOS of the form $P=K\rho^{\gamma}$, K fixed.
    \end{itemize}
	\begin{block}{Poisson eq.}
		\begin{columns}[T]
			\begin{column}{0.6\textwidth}
				\begin{align*}
				&\TDy{r}{P}=-\TDy{r}{\phi}\rho\\
				&\frac{1}{r^2}\TDof{r}(r^2\TDy{r}{\phi})=4\pi G\rho\\
				&\TDy{r}{\phi}=-\gamma K\rho\expy{\gamma-2}\TDy{r}{\rho}:\ \rho=(\frac{-\phi}{(n+1)K})^n\\
				&\TtwoDy{r}{\phi}+\frac{2}{r}\TDy{r}{\phi}=4\pi G(\frac{-\phi}{(n+1)K})^n
				\end{align*}
			\end{column}
			\begin{column}{0.4\textwidth}
				\begin{align*}
				&\gamma=5/3, n=3/2\tag{NR deg}\\
				&\gamma=4/3, n=3\tag{R deg}\\
				&\gamma=1, n=\infty\tag{isot.}\\
				&\nad\approx\frac{2}{5}\tag{conv}
				\end{align*}
		\end{column}\end{columns}
	\end{block}
$w(0)=1$, $w'(0)\propto\TDy{r}{\phi}=|g|\to0$, $w(z\approx0)\approx1-\frac{1}{6}z^2+\ldots$
\end{frame}

\begin{frame}{Equazione Lane-Emden}
	\begin{columns}[T]
		\begin{column}{0.5\textwidth}
			\begin{align*}
			&\rho=(\frac{-\phi}{(n+1)K})^n\tag{HE}\\
			&\TtwoDy{r}{\phi}+\frac{2}{r}\TDy{r}{\phi}=4\pi G(\frac{-\phi}{(n+1)K})^n\tag{Poiss}\\
			&\TtwoDy{z}{w}+\frac{2}{z}\TDy{z}{w}+w^n=0\\
			&\frac{1}{z^2}\TDof{z}(z^2\TDy{z}{w})+w^n=0\tag{Lane-Emden}
			\end{align*}
		\end{column}
		\begin{column}{0.5\textwidth}
			\begin{align*}
			&z=Ar\tag{new vars}\\
			&A^2=\frac{4\pi G}{(n+1)^nK^n}(-\phi_c)\expy{n-1}\\
			&=\frac{4\pi G}{(n+1)K}\rho_c\expy{\frac{n-1}{n}}\\
			&w=\frac{\phi}{\phi_c}=(\frac{\rho}{\rho_c})\expy{1/n}\\
			&(\rho=\rho_cw^n)
			\end{align*}
	\end{column}\end{columns}
Surface where $w(z_n)=0$ and $\rho=0$ ($\frac{r}{z}=\frac{1}{A}=\frac{R}{z_n}$)
\end{frame}

\begin{frame}{Polytropic star model: given $n<5, M, R$}
\begin{block}{Possible if K not fixed by eos}
\begin{align*}
&m(r)=\int_0^r4\pi\rho r^2\,dr=4\pi\rho_c\int_0^rw^nr^2\,dr=4\pi\rho_c\frac{r^3}{z^3}\int_0^zw^nz^2\,dz\\
&=4\pi\rho_cr^3(-\frac{1}{z}\TDy{z}{w}),\ M=4\pi\rho_cR^3(-\frac{1}{z}\TDy{z}{w})_{z=z_n}\\
&\frac{\exv{\rho}}{\rho_c}=[-\frac{3}{z}\TDy{z}{w}]_{z_n}\tag{model}\\
&\rho=\rho_cw^n(z),\ P=K(A,\rho_c)\rho^{\frac{n+1}{n}}
\end{align*}
\end{block}
\begin{columns}[T]
\begin{column}{0.55\textwidth}
\begin{block}{ie radiative pressure}
	\begin{align*}
	&P=\frac{R\rho}{\mu}+\frac{a}{3}T^4=\frac{R}{\mu\beta}\rho T,\ \beta=\frac{P_{gas}}{P}\\
	&1-\beta=\frac{P_r}{P}=\frac{aT^4}{3P}\Rightarrow P=(\frac{3R^4}{a\mu^4})^{\frac{1}{3}}(\frac{1-\beta}{\beta^4})^{\frac{1}{3}}\rho^{\frac{4}{3}}
	\end{align*}
\end{block}
\end{column}
\begin{column}{0.45\textwidth}
\begin{block}{isothermal sphere: $\gamma=1, n\to\infty$}
$K=\frac{RT}{\mu}$ free param - $\frac{\phi}{-K}=\ln{\rho}-\ln{\rho_c}$
\begin{align*}
%&\TDy{r}{\phi}=-\gamma K\rho^{\gamma-2}\TDy{r}{\rho}:\  \frac{\phi}{-K}=\ln{\rho}-\ln{\rho_c}\\
&\TtwoDy{z}{w}+\frac{2}{z}\TDy{z}{w}=\exp{-w}\\
&w(0)=w'(0)=0
\end{align*}
\end{block}
\end{column}
\end{columns}
\end{frame}

\begin{frame}{Fixed K. $M_{ch}$}
\begin{block}{NR degenerate \Pelectron gas}
$n=\frac{3}{2}$, $K=\frac{1}{20}(\frac{3}{\pi})^{\frac{3}{2}}\frac{h^2}{m_e}\frac{1}{(\mu_em_p)^{\frac{5}{3}}}$: given $\rho_c$ the poly model gives $\rho=\rho_cw^n$, $R=\frac{z_n}{A}\propto\rho_c^{\frac{1-n}{2n}}$, $M\propto\rho_cR^3=C_1\rho_c^{\frac{3-n}{2n}}$ quindi $R\propto M^{\frac{1-n}{3-n}}$. Given K, n the only param is $R$ or $M$ (or $\rho_c$)
\end{block}
\begin{block}{WD model}
As M increases the inner R degenerate part increases ($\rho>\SI{e6}{\gram\per\cubic\cm}$): R degenerate core with $n=3$ fitted to outer NR degenerate envelope $n=\frac{3}{2}$. For $n=3$:
\begin{align*}
M=C_1=4\pi(\frac{w'}{z})_{z_3}z_3^3(\frac{K}{4\pi G})^{\frac{3}{2}}=M_{ch}=\frac{5.836}{\mu_e^2}\msun
\end{align*}
Model tend to finite mass as \xaumenta{\rho_c}
\end{block}
\end{frame}

\begin{frame}{Energy in poly}
Energia gravitazionale:
\begin{align*}
&E_g=-G\int_0^M\frac{m}{r}\,dm\, (=-3\int_0^M\frac{P}{\rho}\, dm)\\
&=-\frac{1}{2}\frac{GM^2}{R}-\frac{1}{2}\int_0^R\TDy{r}{\phi}m\,dr=-\frac{1}{2}\frac{GM^2}{R}+\frac{1}{2}\int_0^M\phi\,dm\\
&\phi=-\frac{K\gamma}{\gamma-1}\rho^{\gamma-1}=-\frac{\gamma}{\gamma-1}\frac{P}{\rho}\tag{poly}\\
&E_g=-\frac{1}{2}\frac{GM^2}{R}+\frac{1}{6}(n+1)E_g: E_g=-\frac{3}{5-n}\frac{GM^2}{R}
\end{align*}
Energia interna $E_i$ della stella - $u$ per unit mass:
\begin{align*}
&\frac{P}{\rho}=\frac{R}{\mu}T=(\gamma-1)c_VT\xrightarrow{\text{mono gas}}\frac{2}{3}u:\ \zeta=\frac{3P}{\rho u}\tag{virial for general EOS}\\
&E_i=-\frac{1}{\zeta}E_g=\frac{3}{\zeta(5-n)}\frac{GM^2}{R},\ W=E_i+E_g=\frac{3}{5-n}(\frac{1}{\zeta}-1)\frac{GM^2}{R}\\
&\zeta\,du=3\frac{dP}{\rho}-3\frac{P}{\rho^2}\,d\rho,\ du=\frac{P}{\rho^2}\,d\rho\Rightarrow\zeta=3\frac{\rho}{P}\TDy{\rho}{P}-3=3(\gamma_{ad}-1)
\end{align*}
\end{frame}

\begin{frame}{Homologous contraction (poly index n)}
$z\propto\frac{r}{R}$: LE gives $m(z)$ so mass elements at homologous points
\begin{align*}
&r'=r+\dot{r}\Delta t\Rightarrow\frac{r}{r'}=1+\frac{\dot{r}}{r}\Delta t\\
&\frac{r'}{r}=\frac{R'}{R}=\const{}\Rightarrow\frac{\dot{r}}{r}=\frac{\dot{R}}{R}=\const{}\ (\PDof{m}\PDy{t}{\ln{r}})=0\\
&\PDof{t}(\frac{1}{r}\PDy{m}{r})=0=\PDof{t}(\frac{1}{4\pi r^3\rho})\Rightarrow\frac{\dot{\rho}}{\rho}=-3\frac{\dot{r}}{r}\\
&P=\int_m^M\frac{Gm}{4\pi r^4}\,dm:\ \dot{P}=\int_m^M\PDof{t}(\frac{1}{r^4})\frac{Gm}{4\pi}\tag{HE}\\
&\rho\propto P^{\alpha}T^{-\delta}\Rightarrow\frac{\dot{\rho}}{\rho}=\alpha\frac{\dot{P}}{P}-\delta\frac{\dot{T}}{T}\tag{EOS}\\
&\frac{\dot{T}}{T}=-\frac{4\alpha-3}{\delta}\frac{\dot{r}}{r}\\
&\epsilon_g=c_PT(\nad\frac{\dot{P}}{P}-\frac{\dot{T}}{T})=c_PT[-4\nad+\frac{4\alpha-3}{\delta}]\frac{\dot{R}}{R}\\
&\to-\frac{3}{5}c_PT\frac{\dot{R}}{R}\tag{ideal mono $\nad=\frac{2}{5}$, $\alpha=\delta=1$}
\end{align*}

\end{frame}

\begin{frame}{Strutture isoterme e modello solare}
	
\end{frame}

\subsection{Metodo di Henyey: modelli evolutivi}\linkdest{henyey}

\begin{frame}{Metodo di Henyey $[96]$}
N mass-shell with boundaries at $m_j$ $m_1=0,m_2=m',\ldots,m_N=M$
\begin{columns}[T]
\begin{column}{0.5\textwidth}
\begin{align*}
&\frac{r_{j+1}-r_j}{m_{j+1}-m_j}=\frac{1}{4\pi r_{j+1/2}^2\rho_{j+1/2}}\\
&\frac{P_{j+1}-P_j}{m_{j+1}-m_j}=-\frac{Gm_{j+1/2}}{4\pi r_{j+1/2}}\\
&\frac{L_{j+1}-L_j}{m_{j+1}-m_j}=\epsilon_n(T_{j+1/2},\rho_{j+1/2},X_{s,j+1/2})+!!!\\
&\frac{T_{j+1}-T_j}{m_{j+1}-m_j}=-\frac{T_{j+1/2}}{P_{j+1/2}}\nabla_{j+1/2}\frac{Gm_{j+1/2}}{4\pi r^4_{j+1/2}}
\end{align*}
\end{column}
\begin{column}{0.5\textwidth}
\begin{align*}
&(Y^1=r, y^2=P, y^3=T, y^4=L)\\
&E_j^{i=1,\ldots,4}=\frac{y_{j+1}^i-y^i_j}{m_{j+1}^i-m^i_j}\\
&-f_i(y_{j+1/2}^1,\ldots,y_{j+1/2}^4)=0
\end{align*}
$j=2,\ldots,N-2$: 4N-8 vincoli e le condizioni al bordo
\begin{align*}
&J=N: S_1=y_N^2-f_S(y_N^1,y_N^4)=0\\
&S_2:y^3_N-g_S(y_N^1,y_N^4)=0\\
&j=1: C_i(y^1_2,y_2^2,y_2^3,y_2^4,y_1^2,y_1^3)=0\\
&y_1^1=y_1^4=0
\end{align*}
\end{column}
\end{columns}
$4N-2$ equazioni in $4N-2$ incognite
\end{frame}

\begin{frame}{Metodo di Henyey: solve system of algebraic equations}
\'A la Newton-Rapshon - Trial solution (solution at step $t-\Delta t$): $(S_i)_1\neq0$, $(C_i)_1\neq0$, $(E_i^j)_1\neq0$ so we have to find correction to trial solution $(y_i^j)_2=(y_i^j)_1+\delta y_i^j$ - by Taylor expansion we can express $\delta S_i$, $\delta C_i$, $\delta E^i_j$ as function of the unknown small correction:
\begin{align*}
&(S_i)_1+\delta S_i=0, (C_i)_1+\delta C_i=0, (E_i^j)_1+\delta E_i^j=0\\
&\left\{\begin{array}{l}\TDy{y_N^1}{S_i}\delta y_N^1+\TDy{y_N^2}{S_i}\delta y_N^2+\TDy{y_N^3}{S_i}\delta y_N^3+\TDy{y_N^4}{S_i}\delta y_N^4=-(S_i)_1\\
\TDy{y_2^1}{C_i}\delta y_2^1+\ldots+\TDy{y_2^4}{C_i}\delta y_2^4+\TDy{y_1^2}{C_i}\delta y_1^2+\TDy{y_1^3}{C_i}\delta y_1^3=-(C_i)_1\\
\TDy{y_j^1}{E_j^i}\delta y_j^1+\ldots+\TDy{y_j^4}{E_j^i}\delta y_j^4+\TDy{y_{j+1}^1}{E_j^i}\delta y_{j+1}^1\ldots+\TDy{y_{j+1}^4}{E_j^i}\delta y_{j+1}^4=-(E_j^i)_1\\
\end{array}\right.
\end{align*}
con $i=1,\ldots,4$, $j=2,\ldots,N$: sistema algebrico di $4N-2$ equazioni in $4N-2$ $\delta y_j^i$ incognite ha matrice dei coefficienti (\keyword{Henyey matrix}) non-zero only near the diagonal e determinante non nullo - soluzione con metodi algebrici standard - fisso $\epsilon>0$ accuratezza con cui voglio risolvere le equazioni della struttura stellare e ripeto la procedura finch\'e $S_i<\epsilon$, $C_i<\epsilon$ - local errors doesn't propagate to other mesh
\end{frame}

\subsection{Metodo di shooting: soluzione primo modello stellare con condizioni al bordo}\linkdest{shooting}

\begin{frame}{Step temporale e problema modello iniziale}
Step temporale $\Delta t$: uso metodo di Henyey per determinare le nuove abbondanze with I equations as I elements accounted for.
\keyword{Problem of first model}: i) Turn on $\epsilon_g$ within few models ii) Initial model of evolutionary sequence is evaluated with \keyword{shooting method}
starting from outer mesh with 4 boundary conditions for $T_e$, $L_s$, $P_s$, $R$ and using first order approx. $y_j^i=y_{j\pm1}^i+\TDy{m}{y_{j\pm1}}\,dm$: we determine $(y_f^i)_{surf}$ until some mesh f midway between center and surface and then integrate from centerup to $j=f$ using central trial conditions - we have $(y_f^i)_{center}\neq(y_f^i)_{surf}$ and have to correct trial central boundary conditions
\begin{columns}[T]
	\begin{column}{0.5\textwidth}
\begin{align*}
&\Delta(y_f^i)_{surf}=\TDy{y_N^1}{(y^i_f)_{surf}}\delta y_N^1+\TDy{y_N^4}{(y^i_f)_{surf}}\delta y_N^4\\
&\Delta(y_f^i)_{center}=\TDy{y_1^2}{(y^i_f)_{center}}\delta y_1^2+\TDy{y_1^3}{(y^i_f)_{center}}\delta y_1^3
\end{align*}
	\end{column}
	\begin{column}{0.45\textwidth}
correzione ai 4 valori al contorno $\delta y_N^1=\delta R$, $\delta y_N^4=\delta L_s$ e $\delta y_1^2=P_c$, $\delta y_1^3=T_c$ found solving \[-\Delta_f=\Delta(y_f^i)_{center}-\Delta(y_f^i)_{surf}\]
	\end{column}
\end{columns}
Fails in advanced stages when structure has strong gradient - In massive stars late phases we can't ignore acceleration term
\end{frame}

\section{Equazioni struttura stellare}\linkdest{stellarmodel}

\begin{wordonframe}{da fare: kippenhahn wiegert}
\begin{itemize}
\item strutture autogravitanti 1-62' (43). EQuilibrio idrostatico, vento stellare, stabilit\'a e pulsazioni
\item metodi numerici 77'-84(44-48)
\item esistenza e unicit\'a 85'-99'(48-56)
\item Properties of stellar matter: ideal gas with radiation, ionization, degenerate electron gas, equazione di stato, opacit\'a 102'-144'(57-78)
\item produzione energia reazioni nucleari:  146'-172'(79-92)
\item politrope 174'-190' (93-102)
\end{itemize}
\end{wordonframe}

\subsection{Struttura di equilibrio}\linkdest{stellarstructure}

\begin{frame}{Equazioni struttura di equilibrio}

\begin{align*}
&\TDy{m}{r}=\frac{1}{4\pi r^2\rho}\\
&\TDy{m}{P}=-\frac{Gm}{4\pi r^4}\overbrace{[-\frac{1}{4\pi r^2}\PtwoDy{t}{r}]}^{\tau_{hyd}}\\
&\TDy{m}{T}=-\nabla\frac{T}{p}\frac{Gm}{4\pi r^4}\\
&\TDy{m}{L}=\epsilon-\epsilon_{\nu} \underbrace{-c_P[\TDy{t}{T}-\nad\frac{T}{P}\TDy{t}{P}]}_{-c_P\PDy{t}{T}+\frac{\delta}{\rho}\PDy{t}{P}}\\
&\TDy{t}{X_s}\frac{1}{A_s}=\sum_{production}\rho^{n_h+n_k-1}n_p\frac{X_h^{n_h}X_k^{n_k}}{A_h^{n_h}A_k^{n_k}}\frac{\exv{\sigma v}_{hk}}{m_H^{n_h+n_k-1}n_h!n_k!}\\
&-\sum_{distruction}\rho^{n_d+n_j-1}n_d\frac{X_s^{n_d}X_j^{n_j}}{A_s^{n_d}A_j^{n_j}}\frac{\exv{\sigma v}_{sj}}{m_H^{n_d+n_j-1}n_d!n_j!}
\end{align*}
\end{frame}

\begin{wordonframe}{Chemical evolution: nuclear burning, diffusion and convective mixing}\linkdest{diffusion}\

\end{wordonframe}

\begin{frame}{Energia gravitazionale}
$(\TDy{r}{L}=4\pi r^2[\rho\epsilon-\rho\TDof{t}(\frac{u}{\rho}+\frac{P}{\rho}\TDy{t}{\rho})])$
\end{frame}

\begin{frame}{Trasporto radiativo e convettivo: \'e valida approx idrostatica}

\end{frame}

\subsection{Relazioni approssimate per grandezze stellari fondamentali}\linkdest{omrel}

\begin{frame}{Teorema del viriale}

\end{frame}

\begin{frame}{Relazione massa, densit\'a, temperatura/ massa, peso molecolare, opacit\'a}

\end{frame}

\section{Trasporto}\linkdest{transport}

\subsection{Trasporto radiativo}\linkdest{trarad}

\begin{frame}{Trasporto radiativo}
media di rosseland
\end{frame}

\begin{frame}{Opacit\'a radiativa (analitico)}
Scattering elettronico, processi ff, fotoionizzazione; opacit\'a atmosfera: ione H-
\end{frame}

\subsection{Conduzione}\linkdest{tracond}

\begin{frame}{Diffusione per conduzione}
stima opacit\'a per conduzione gas degenere NR
\end{frame}

\subsection{Convezione}\linkdest{traconv}

\begin{frame}{Convective regions and temperature gradient}
\begin{block}{Criterio di \sch e Ledoux: regioni convettive}
\begin{align*}
&\nrad{}>\nad+\frac{\phi}{\delta}\nmu{}\\
&\nrad{}>\nad
\end{align*}
\end{block}
\begin{block}{T gradient: Mixing length}
Gas with negligible radiation pressure
\begin{align*}
&F=\frac{L}{4\pi r^2}=F_{rad}+F_{con}=-\frac{4acT^3}{3\kappa\rho}\TDy{r}{T}|_{amb}+\frac{1}{2}\rho vc_p[\TDy{r}{T}|_{Ad}-\TDy{r}{T}|_{amb}]\Lambda\\
&v^2=\frac{1}{8}g\frac{\Lambda}{H_P}Q(\nabla-\nad),\ Q=1-\TDly{T}{\mu}\\
&F_{con}=\frac{1}{2}\rho vc_PT\frac{\lambda}{H_P}(\nabla-\nad{})
\end{align*}
$\nabla\to\nad{}$ in interni stellari convettivi
\end{block}
\end{frame}

\begin{frame}{Mixing length per esterni stellari}
%Calcolo altezza scala di pressione, flusso convettivo e gradiente ambientale in funzione della velocit\'a media degli elementi
Le stelle con massa $M\leq1.1\msun{}$ hanno una regione radiativa interna mentre la parte esterna \'e convettiva
\end{frame}

\begin{wordonframe}{Forza di archimede}
Una regione stellare \'e convettivamente stabile se una perturbazione di densit\'a infinitesima non cresce ad ampiezza finita.
\begin{equation*}\label{eq:buoyancyEOM}
\rho\PtwoDy{t}{(\Delta r)}=-g\Delta\rho=-g[\Dcvar{\TDy{r}{\rho}}{e}-\Dcvar{\TDy{r}{\rho}}{amb}]\Delta r
\end{equation*}

La forza di Archimede ha verso opposta alla perturbazione se
\begin{equation*}\label{eq:Acriterion}
[\Dcvar{\TDy{r}{\rho}}{e}-\Dcvar{\TDy{r}{\rho}}{amb}]>0
\end{equation*}
\end{wordonframe}

\begin{wordonframe}{EOS e $\nad{}$}

Considero un'equazione di stato generica $\rho(P,T,\mu)$ e definita tramite:
\begin{align*}
&\frac{d\rho}{\rho}=\alpha\frac{dP}{P}-\delta\frac{dT}{T}+\phi\frac{d\mu}{\mu}\\
&P=\frac{\rho\gasconstant{}T}{\mu}\quad\Rightarrow\quad\alpha=\delta=\phi=1
\end{align*}

Definisco le lunghezze caratteristiche per variazione di densit\'a e pressione:
\begin{equation*}
\densityscale{}=-\frac{dr}{d\ln{\rho}},\ H_P=-\frac{dr}{d\ln{P}}
\end{equation*}
e i gradienti termici per il blob, l'ambiente e il gradiente di composizione chimica ambientale
\begin{equation*}
\nabla=\Dcvar{\TDly{P}{T}}{amb},\ \nabla_e=\Dcvar{\TDly{P}{T}}{e},\ \nmu{}=\Dcvar{\TDly{P}{\mu}}{amb}
\end{equation*}
\end{wordonframe}

\begin{wordonframe}{EOS e EOM}
Riscrivo l'equazione del moto utilizzando l'equazione di stato per scrivere la differenza di densit\'a in termini dei gradienti termici e di composizione chimica; inoltre supponendo il moto dell'elemento in equilibrio di pressione con l'ambiente e assumendo $\nmu{}_{blob}\approx0$ risulta:
\begin{equation*}
\PtwoDy{t}{(\Delta r)}=-g\frac{\delta}{H_P}[\nabla_e-\nabla-\frac{\phi}{\delta}\nmu{}]\Delta r
\end{equation*}

\end{wordonframe}

\begin{wordonframe}{Criterio di \sch/Ledoux}

Infine per ricavare il criterio di stabilit\'a per convezione suppongo  il moto del blob adiabatico:
\begin{equation*}
dq=c_P\,dT-\frac{\delta}{\rho}\,dP
\end{equation*}
da cui risulta:
\begin{equation*}
\nabla_e=\nabla_{ad}=\frac{P\delta}{T\rho c_P}
\end{equation*}
cio\'e una regione solare \'e stabile per convezione se
\begin{equation*}
\nrad{}<\nad+\frac{\phi}{\delta}\nmu{}\label{eq:ledoux}
\end{equation*}
dove ho usato $\nabla_{amb}=\nrad{}$, cio\'e il gradiente che si ha nel caso l'energia sia trasportata dai fotoni.
\end{wordonframe}

\begin{wordonframe}{Stabilit\'a convettiva e frequenza di \bv{}}
Gradiente adiabatico: riscrivo prima legge della termodinamica come $dq=c_P\,dT-\frac{\delta}{\rho}\,dP$

Introduco la frequenza di \bv{}:
\begin{equation*}
N^2=g(\frac{1}{\Gamma_1P}\TDy{r}{P}-\frac{1}{\rho}\TDy{r}{\rho})=g(\frac{1}{\densityscale{}}-\frac{g}{c_s^2})\label{eq:bvfs}
\end{equation*}
$N^2$ rappresenta la massima frequenza sotto cui pu\'o oscillare una particella di fluido sottoposta a onde di gravit\'a mantenendo l'equilibrio di pressione con l'ambiente.

\begin{equation*}
\PtwoDy{t}{(\Delta r)}=-N^2\Delta r
\end{equation*}
che descrive un comportamento oscillatorio per $N^2>0$.

\end{wordonframe}

\subsection{Teoria della mixing-length.}

\begin{wordonframe}{Convezione in esterni stellari}

In presenza di convezione il flusso di energia verso l'esterno ha una componente radiativa, determinata dal gradiente di temperatura, e una componente dominante convettiva 
\begin{equation*}\label{eq:radconvflux}
F=F_{con}+F_{rad}=\frac{\lsun{}}{4\pi r^2}
\end{equation*}

Una maggiore efficienza del trasporto convettivo di energia si riflette in una minore differenza tra il gradiente di temperature adiabatico ed effettivo.

\begin{figure}[!h]
%   \includegraphics[ width=0.99\textwidth,keepaspectratio]{proportionflux}
%   \subcaption{Profilo radiale (profondit\'a in \si{\kilo\meter}) del flusso convettivo $F_c$ rispetto al flusso totale $F$, della super-adiabaticit\'a $\nabla-\nad{}$ e regioni di ionizzazione idrogeno e $\cel{He}{4}{}{}$. Da \cite{christensen1997effects}.}\label{fluxproportion}

%\includegraphics[keepaspectratio,width=0.9\textwidth]{specificheatnablaa}
%\subcaption{Profilo radiale di $c_P$ e $\nabla_a$: si ha cambiamento di comportamento nelle regioni di ionizzazione parziale di idrogeno ed elio. Da \cite{stix91sun}.}\label{specificheatnablaa}
\end{figure}

Per determinare il gradiente di temperatura effettivo $\nabla$ uso la teoria della mixing-length:
si considera l'eccesso di calore trasportato dai blob di gas nel moto convettivo $c_P\Delta T$ rispetto all'ambiente, il cui cammino libero medio \'e la mixing-length $l_m=\alpha H_P$, che da luogo al flusso di energia
\begin{equation*}
F_{con}=\exv{\rho vc_P\Delta T}\label{eq:convectiveflux}
\end{equation*}
dove $\exv{}$ indica una media opportuna sulla sfera di raggio r. Determino il valor medio della differenza di temperatura prendendo come valore caratteristico dello spostamento del blob $\Delta r\approx\frac{l_m}{2}$:
%, considerando moti in entrambi i versi,
\begin{equation*}
\frac{\Delta T}{T}\approx\frac{1}{T}\PDy{r}{(\Delta T)}\frac{l_m}{2}=(\nabla-\nabla_e)\frac{l_m}{2}\frac{1}{H_P}\label{eq:blobambdiff}
\end{equation*}

Assumo il lavoro medio fatto dalla forza di galleggiamento per unit\'a di massa $-g\frac{\Delta\rho}{\rho}$ uguale al valore medio della forza, cio\'e la met\'a di quello alla superficie sferica data, moltiplicato lo spostamento medio $\frac{l_m}{2}$ quindi, assumendo in oltre che in media met\'a del lavoro fatto dalla forza di galleggiamento sia trasformato in energia cinetica del blob si ottiene
\begin{equation*}
v^2=g\delta(\nabla-\nabla_e)\frac{l_m^2}{8H_P}\label{eq:blobvelocity}
\end{equation*}

Infine determino gli scambi radiative del blob: il modulo del flusso radiativo \'e proporzionale al gradiente termico in direzione normale alla superficie del blob
\begin{equation*}
f=\frac{4acT^3}{3\kappa\rho}|\PDy{n}{T}|
\end{equation*}
quindi l'energia scambiata dall'intera superficie S del blob \'e $\lambda=Sf$ che determina, per la prima legge della termodinamica, una variazione di temperatura per unit\'a di tempo:
\begin{equation*}
\PDy{t}{T_e}=-\frac{\lambda}{\rho Vc_P}
\end{equation*}
indicato con $V$ il volume del blob.

La variazione della temperatura del blob per unit\'a distanza percorsa \'e quindi
\begin{equation*}
\Dcvar{\TDy{r}{T}}{e}=\Dcvar{\TDy{r}{T}}{ad}-\frac{\lambda}{\rho Vc_Pv}\label{eq:Tchangelength}
\end{equation*}
e approssimando il gradiente normale alla superficie con $\exv{\Delta T}$ ed usando le definizioni \eqref{eq:nablavitense} si ottiene:
\begin{equation*}
\frac{\nabla_e-\nad{}}{\nabla-\nabla_e}=\frac{6acT^3}{\kappa\rho^2c_Pl_mv}
\end{equation*}
Il gradiente termico ambientale $\nabla$ e del blob $\nabla_e$ sono determinati da \eqref{eq:radconvflux} e \eqref{eq:Tchangelength} inserendo le espressioni per il flusso radiativo \eqref{eq:radiativeflux} e il flusso convettivo \eqref{eq:convectiveflux}.

In figura (\subref{fluxproportion}) si mostrano l'andamento di $\nabla-\nad{}$, il profilo termico e la frazione di flusso totale trasportato dalla convezione; in figure (\subref{specificheatnablaa}) si mostrano il profilo del calore specifico per unit\'a di massa e del gradiente adiabatico.

Le 5 equazioni del flusso convettivo

Le 5 equazioni determinano completamente le variabili $F_{rad}, F_{con}, v, \nabla_e, \nabla$ in funzione di $P,T,l(r),m(r),c_P,\nad{},\nrad{},g$.

Come determino il gradiente effettivo ??

Determino $\nabla-\nabla_e$
cubic equation for $(\nabla-\nabla_e)$

\end{wordonframe}

\subsection{Approssimazione politropa}\linkdest{poly}

\begin{frame}{Trasformazioni politropiche}
\begin{block}{Gener. T. adiabatica}
	Il rapporto $\gamma=\frac{c_P}{c_V}$ costante per gas perfetto di sole particelle totalmente ionizzato.
	T. adiabatica:
	\[TV\expy{\gamma-1}=\const,\ PV\expy{\gamma}=\const,\ P\expy{1-\gamma}T\expy{\gamma}=\const\]
$0=dE-\frac{P}{\rho^2}d\rho$
Caso pi\'u generale delle trasformazioni adiabatiche: \keyword{trasformazione politropa} trasformazione quasi-statica in maniera che $c=\TDy{Q}{T}$ (calore specifico) vari in maniera assegnata. (adiabatica: $c=0$, isoterma: $c=\infty$, isometrica: $c=c_V$, ...)
\end{block}

\end{frame}

\begin{frame}{Equazione Lane-Emden}

\end{frame}

\begin{frame}{Strutture isoterme e modello solare}

\end{frame}


\section{Descrizione materia stellare}\linkdest{descstellarmatter}

\subsection{Equazione di stato}\linkdest{eos}

\begin{frame}{Approssimazione gas perfetto monoatomico}

\end{frame}

\begin{frame}{Degenerazione e- ed effetti coulombiani. Equazione di saha}

\end{frame}

\subsection{Reazioni nucleari}\linkdest{nuclearreactions}

\begin{frame}{Sezione d'urto nucleare}
schermaggio elettronico nelle stelle
\end{frame}

\begin{frame}{Catena PP}
dipendenza da T
flusso neutrini
\end{frame}

\begin{frame}{Biciclo CNO}
dipendenza da T
flusso neutrini
Modalit\'a combustione H in He: sequenza principale inferiore/superiore
\end{frame}

\begin{frame}{Combustion He}

\end{frame}

\begin{frame}{Produzione nuclei fino al Fe56}
3$\alpha$: $He4+\alpha\to C12$, $C12+\alpha\to O16$
Produzione neutroni liberi
Fusione $C12$, fotodisintegrzione $Ne20$, fusione $O16$, fotodisintegrzione $Si28$, catture $\alpha$ su nuclei fino a produzione $Fe56$
\end{frame}

\begin{frame}{Cattura neutronica: processi r e s}
picchi r e s nella nella curva universale delle abbondanze
\end{frame}

\section{Metodi per integrazione equazioni di struttura e raccordo con modelli dell'atmosfera stellare}\linkdest{nummod}


\begin{frame}{Equazioni struttura di equilibrio}

\begin{align*}
&\TDy{m}{r}=\frac{1}{4\pi r^2\rho}\\
&\TDy{m}{P}=-\frac{Gm}{4\pi r^4}\overbrace{[-\frac{1}{4\pi r^2}\PtwoDy{t}{r}]}^{\tau_{hyd}}\\
&\TDy{m}{T}=-\nabla\frac{T}{p}\frac{Gm}{4\pi r^4}\\
&\TDy{m}{L}=\epsilon-\epsilon_{\nu} \underbrace{-c_P[\TDy{t}{T}-\nad\frac{T}{P}\TDy{t}{P}]}_{-c_P\PDy{t}{T}+\frac{\delta}{\rho}\PDy{t}{P}: \tkh}\\
&\TDy{t}{X_s}\frac{1}{A_s}=\sum_{production}\rho^{n_h+n_k-1}n_p\frac{X_h^{n_h}X_k^{n_k}}{A_h^{n_h}A_k^{n_k}}\frac{\exv{\sigma v}_{hk}}{m_H^{n_h+n_k-1}n_h!n_k!}\\
&-\sum_{distruction}\rho^{n_d+n_j-1}n_d\frac{X_s^{n_d}X_j^{n_j}}{A_s^{n_d}A_j^{n_j}}\frac{\exv{\sigma v}_{sj}}{m_H^{n_d+n_j-1}n_d!n_j!}
\end{align*}
\end{frame}

\begin{frame}{Equazioni struttura di equilibrio: condizioni al bordo}
\begin{itemize}
\item Le 4+I equazioni determinano $r$, $P$, $T$, $L$, $X_s$ specificata la massa e composizione iniziale (omogenea)
\item $\tau_n\gg\tkh\gg\tau_{dyn}$: solve 4 structure equations at time t - do time step $\Delta t$ and determine new composition - solve structure at $t+\Delta t$ with new composition
\item Solution of 4 structure equations require 4 boundary condition: 2 at surface (atmospere model without diffusion approx, PP geometry), parametri $\rho_c,T_c$; 2 at center (via Taylor expansion for $m=m'$)
\end{itemize}
\begin{columns}[T]
\begin{column}{0.65\textwidth}
\begin{align*}
&r=(\frac{3}{4\pi\rho_c})\expy{1/3}{m'}\expy{1/3}\\
&P=P_c-\frac{3G}{8\pi}(\frac{4\pi\rho_c}{3})\expy{4/3}{m'}\expy{2/3}\\
&L=\epsilon_cm'\\
&T^4=T_c^4-\frac{1}{2ac}(\frac{3}{4\pi})\expy{2/3}\kappa_c\epsilon_c\rho_c\expy{4/3}{m'}\expy{2/3}\tag*{rad}\\
&\ln{T}=\ln{T_c}-(\frac{\pi}{6})\expy{1/3}G\frac{{\nad}_c\rho_c\expy{4/3}}{P_c}{m'}\expy{2/3}\tag*{con}
\end{align*}
\end{column}
\begin{column}{0.35\textwidth}
At surface $m=M$, $L=L_s$, atmospheric model for $P$, $T$ - atmosphere defined by $g=\frac{GM}{R^2}$, $T_e$, composition: provides $P_s$ at $\tau$ where diffusion approx. starts to be valid
\end{column}
\end{columns}
\end{frame}

\begin{frame}{Simplified atmosferic model: grey atmosphere}
Atmosphere model: usually PP geometry and solve HE equation using non grey radiative transport and EOS and convection if needed
\begin{align*}
&\TDy{\tau}{P}=\frac{g}{\kappa}\tag*{HE using $\tau$ as indip var}\\
&T^4=\frac{3}{4}T_e^4(\tau+\frac{2}{3})\tag*{or solar $T(\tau)$ empirical relation}
\end{align*}
integration from $\tau\approx0$ where $T\approx0$, $P\approx0$ down to $\tau=\frac{2}{3}$ where $T=T_e$ - using shooting method.
\end{frame}

\begin{frame}{Chemical mixing: diffusion and convection}

\end{frame}

\begin{frame}{Metodo del fitting}

\end{frame}

\begin{frame}{Metodo di Henyey $[96]$}
N mass-shell with boundaries at $m_j$ $m_1=0,m_2=m',\ldots,m_N=M$
\begin{columns}[T]
\begin{column}{0.5\textwidth}
\begin{align*}
&\frac{r_{j+1}-r_j}{m_{j+1}-m_j}=\frac{1}{4\pi r_{j+1/2}^2\rho_{j+1/2}}\\
&\frac{P_{j+1}-P_j}{m_{j+1}-m_j}=-\frac{Gm_{j+1/2}}{4\pi r_{j+1/2}}\\
&\frac{L_{j+1}-L_j}{m_{j+1}-m_j}=\epsilon_n(T_{j+1/2},\rho_{j+1/2},X_{s,j+1/2})+!!!\\
&\frac{T_{j+1}-T_j}{m_{j+1}-m_j}=-\frac{T_{j+1/2}}{P_{j+1/2}}\nabla_{j+1/2}\frac{Gm_{j+1/2}}{4\pi r^4_{j+1/2}}
\end{align*}
\end{column}
\begin{column}{0.5\textwidth}
\begin{align*}
&(Y^1=r, y^2=P, y^3=T, y^4=L)\\
&E_j^{i=1,\ldots,4}=\frac{y_{j+1}^i-y^i_j}{m_{j+1}^i-m^i_j}\\
&-f_i(y_{j+1/2}^1,\ldots,y_{j+1/2}^4)=0
\end{align*}
$j=2,\ldots,N-2$: 4N-8 vincoli e le condizioni al bordo
\begin{align*}
&J=N: S_1=y_N^2-f_S(y_N^1,y_N^4)=0\\
&S_2:y^3_N-g_S(y_N^1,y_N^4)=0\\
&j=1: C_i(y^1_2,y_2^2,y_2^3,y_2^4,y_1^2,y_1^3)=0\\
&y_1^1=y_1^4=0
\end{align*}
\end{column}
\end{columns}
$4N-2$ equazioni in $4N-2$ incognite
\end{frame}

\begin{frame}{Metodo di Henyey: solve system of algebraic equations}
\'A la Newton-Rapshon - Trial solution (solution at step $t-\Delta t$): $(S_i)_1\neq0$, $(C_i)_1\neq0$, $(E_i^j)_1\neq0$ so we have to find correction to trial solution $(y_i^j)_2=(y_i^j)_1+\delta y_i^j$ - by Taylor expansion we can express $\delta S_i$, $\delta C_i$, $\delta E^i_j$ as function of the unknown small correction:
\begin{align*}
&(S_i)_1+\delta S_i=0, (C_i)_1+\delta C_i=0, (E_i^j)_1+\delta E_i^j=0\\
&\left\{\begin{array}{l}\TDy{y_N^1}{S_i}\delta y_N^1+\TDy{y_N^2}{S_i}\delta y_N^2+\TDy{y_N^3}{S_i}\delta y_N^3+\TDy{y_N^4}{S_i}\delta y_N^4=-(S_i)_1\\
\TDy{y_2^1}{C_i}\delta y_2^1+\ldots+\TDy{y_2^4}{C_i}\delta y_2^4+\TDy{y_1^2}{C_i}\delta y_1^2+\TDy{y_1^3}{C_i}\delta y_1^3=-(C_i)_1\\
\TDy{y_j^1}{E_j^i}\delta y_j^1+\ldots+\TDy{y_j^4}{E_j^i}\delta y_j^4+\TDy{y_{j+1}^1}{E_j^i}\delta y_{j+1}^1\ldots+\TDy{y_{j+1}^4}{E_j^i}\delta y_{j+1}^4=-(E_j^i)_1\\
\end{array}\right.
\end{align*}
con $i=1,\ldots,4$, $j=2,\ldots,N$: sistema algebrico di $4N-2$ equazioni in $4N-2$ $\delta y_j^i$ incognite ha matrice dei coefficienti (\keyword{Henyey matrix}) non-zero only near the diagonal e determinante non nullo - soluzione con metodi algebrici standard - fisso $\epsilon>0$ accuratezza con cui voglio risolvere le equazioni della struttura stellare e ripeto la procedura finch\'e $S_i<\epsilon$, $C_i<\epsilon$ - local errors doesn't propagate to other mesh
\end{frame}

\begin{frame}{Step temporale e problema modello iniziale}
Step temporale $\Delta t$: uso metodo di Henyey per determinare le nuove abbondanze with I equations as I elements accounted for.
\keyword{Problem of first model}: i) Turn on $\epsilon_g$ within few models ii) Initial model of evolutionary sequence is evaluated with \keyword{shooting method}
starting from outer mesh with 4 boundary conditions for $T_e$, $L_s$, $P_s$, $R$ and using first order approx. $y_j^i=y_{j\pm1}^i+\TDy{m}{y_{j\pm1}}\,dm$: we determine $(y_f^i)_{surf}$ until some mesh f midway between center and surface and then integrate from centerup to $j=f$ using central trial conditions - we have $(y_f^i)_{center}\neq(y_f^i)_{surf}$ and have to correct trial central boundary conditions
\begin{columns}[T]
	\begin{column}{0.5\textwidth}
\begin{align*}
&\Delta(y_f^i)_{surf}=\TDy{y_N^1}{(y^i_f)_{surf}}\delta y_N^1+\TDy{y_N^4}{(y^i_f)_{surf}}\delta y_N^4\\
&\Delta(y_f^i)_{center}=\TDy{y_1^2}{(y^i_f)_{center}}\delta y_1^2+\TDy{y_1^3}{(y^i_f)_{center}}\delta y_1^3
\end{align*}
	\end{column}
	\begin{column}{0.5\textwidth}
correzione ai 4 valori al contorno $\delta y_N^1=\delta R$, $\delta y_N^4=\delta L_s$ e $\delta y_1^2=P_c$, $\delta y_1^3=T_c$ found solving \[-\Delta_f=\Delta(y_f^i)_{center}-\Delta(y_f^i)_{surf}\]
	\end{column}
\end{columns}
Fails in advanced stages when structure has strong gradient - In massive stars late phases we can't ignore acceleration term
\end{frame}
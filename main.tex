%! TEX root = main.tex
\makeatletter%stellare(~/Fisica-Stellare)
\let\@starttocorig\@starttoc
\makeatother%%

\documentclass[10pt,xcolor={usenames},fleqn]{beamer}

\usefonttheme{serif} % default family is serif

%% colors
\definecolor{bittersweet}{rgb}{1.0, 0.44, 0.37}
\definecolor{brilliantlavender}{rgb}{0.96, 0.73, 1.0}
\definecolor{antiquefuchsia}{rgb}{0.57, 0.36, 0.51}
\definecolor{violetw}{rgb}{0.93, 0.51, 0.93}
\definecolor{Veronica}{rgb}{0.63, 0.36, 0.94}
\definecolor{atomictangerine}{rgb}{1.0, 0.6, 0.4}
\definecolor{darkgray}{rgb}{0.66, 0.66, 0.66}
\definecolor{brightcerulean}{rgb}{0.11, 0.67, 0.84}
\definecolor{cadmiumorange}{rgb}{0.93, 0.53, 0.18}
\definecolor{ochre}{rgb}{0.8, 0.47, 0.13}
\definecolor{midnightblue}{rgb}{0.1, 0.1, 0.44}
\definecolor{lemon}{rgb}{1.0, 0.97, 0.0}
\definecolor{grey}{rgb}{0.7, 0.75, 0.71}
\definecolor{amber}{rgb}{1.0, 0.75, 0.0}
\definecolor{almond}{rgb}{0.94, 0.87, 0.8}
\definecolor{bf}{RGB}{88, 86, 88}
\definecolor{bb}{RGB}{177, 177, 177}
\definecolor{keyword}{rgb}{0.25, 0.25, 0.28}
\definecolor{todo}{rgb}{0.75, 0.0, 0.2}
\definecolor{must}{rgb}{1.0, 0.31, 0.0}
% redefine frame
\setbeamertemplate{headline}{\vskip-3pt}
\makeatletter
\setbeamertemplate{frametitle}{
\vspace*{-1mm}
    \ifbeamercolorempty[bg]{frametitle}{}{\nointerlineskip}%
    \@tempdima=\textwidth%
    \advance\@tempdima by\beamer@leftmargin%
    \advance\@tempdima by\beamer@rightmargin%
    %\hspace*{1cm} %%%%%%%%%%%%% For example insert shift to right
    \begin{beamercolorbox}[sep=0.3cm,center,wd=\the\@tempdima]{frametitle}
        \usebeamerfont{frametitle}%
        \vbox{}\vskip-1ex%
        \if@tempswa\else\csname beamer@ftecenter\endcsname\fi%
        \strut\insertframetitle\strut\par%
        {%
            \ifx\insertframesubtitle\@empty%
            \else%
            {\usebeamerfont{framesubtitle}\usebeamercolor[fg]{framesubtitle}\insertframesubtitle\strut\par}%
            \fi
        }%
        \vskip-1ex%
        \if@tempswa\else\vskip-.0cm\fi% set inside beamercolorbox... evil here...
\end{beamercolorbox}%
\vspace*{-6mm}
}
\makeatother

%%%%%%%%%%%%%%%%%%%%%%%%%%%%%%%%%%% importa pacchetti
\usepackage{usepkg}
\usepackage{makerobust}
\MakeRobustCommand\usebeamertemplate
%%%%%%%%%%%%%%%%%%%%%%%%%%%%%%%%%%% Funzioni generali
\usepackage{functions}
%http://tex.stackexchange.com/questions/246/when-should-i-use-input-vs-include
%%
%beamer setup

%%% COUNTERS
\newcounter{cherrykey}%conta le keyword
\newcounter{sectionkey}% conta le section prima della attuale

 \makeatletter
\newcommand{\listofsecframes}{%listofsecframes
  \@starttocorig{secfr-\thesectionkey}
}
\newcommand{\linkdest}[1]{\Hy@raisedlink{\hypertarget{#1}{}}}
\makeatother
\usepackage{LocalF}
\makeatletter
\newif\ifframeinlbf
\frameinlbftrue
\newcommand\listofcherryframesbasic{\@starttocorig{cherryframes}}
%%frameintoc
\newcommand\listofcherryframesname{Important frames}
\newcommand\listofcherryframes{%
\listofcherryframesname\phantomsection: \listofcherryframesbasic}
\addtobeamertemplate{frametitle}{%
    \addtocontents{secfr-\thesectionkey}{%
        \protect\beamer@subsectionintoc{\protect\the\c@section}{0}{%
            \protect\insertframetitle%
        }{\protect\the\c@page}{\protect\the\c@part}{\protect\the\beamer@tocsectionnumber}%
    }%%
\ifframeinlbf
\mode<presentation>{%
        \only<1>{%
            \hypertarget{\protect\insertframetitle}{}%
                    \addcontentsline{cherryframes}{section}{%
                    %\protect\makebox[1em][l]{\normalsize\insertframenumber\hfill}
                        \normalsize\insertframetitle%
                    }%
    }%
 }%
             \else\fi%
}%
{}%
\makeatother
\usepackage{beamersetup}
\usepackage{sources}
\usepackage{mathOp}

\title{Fisica stellare}

% Let's get started
\begin{document}
\frameinlbffalse
%\input{tikzdir}%%contain tikz files as filecontents
\addtolength{\textheight}{-0pt}
\addtolength{\headheight}{0pt}
\addtolength{\footskip}{0pt}

\begin{frame}
  \titlepage
\end{frame}

% Section and subsections will appear in the presentation overview
% and table of contents.
%\frame{\tableofcontents[onlyparts]}

\begin{frame}[allowframebreaks,fragile]{Fisica stellare: argomenti del corso}%label={}
    \begin{columns}[T]
     \begin{column}{0.55\textwidth}
            \tableofcontents[soloparti]
        \end{column}
        \begin{column}{0.45\textwidth}
            \listofcherryframes
        \end{column}
    \end{columns}
\end{frame}

\begin{wordonframe}{Meta.}
    argomenti del corso: toc parts a sinistra e important frames a destra
    todo, must: before reglez

\end{wordonframe}

%\begin{wordonframe}{Perch\'e studio queste cose?? Sviluppi; futuro.}
%Concretezza, concentrazione, indipendenza
%\end{wordonframe}

\part{Intro}\linkdest{intro}
%\begin{frame}{TOC intro}
%\tableofcontents
%\end{frame}
%! TEX root = main.tex
\subsection{Fonti}

\begin{frame}{Appunti Quaderno nero}\linkdest{quadnerorosa}
            \begin{itemize}
                \item 28/04 - S49: protostella, pre-MS
                \item 01/05 - S51: Contrazione traccia Hayashi, ZAMS
                \item 04/05 - S53: Elementi leggeri
                \item 05/05 - S54: SPI/SPS, difficolt\'a nel predirre ZAMS teorica
                \item S56: Isocrona 3.5/12\si{\giga\year}
                \item S61: Isocrona di ammasso \SI{12}{\giga\year}, WD di elio
                \item S63: transizione RGB
                \item S65: TIP RGB
                \item S66: relazioni $\Pi-L$
                \item S67: Metodo orizzontale, metodo verticale, 
                \item S69: From ZAHB, ZAHB-candela standard, clump dell'elio
                \item 18/05-S73: Instability Strip
                \item S74: Autotrascinamento nucleo convettivo, semiconvezione, R2, Pulsi convettivi
                \item 19/05 - S76: Fine He core burning, Pulsi termici, produzione neutroni, tazza carbonio
                \item 22/05-S78: R1, SNII, $M_{up}$
                \item 25/05: Neutronizzazione esplosiva, WD, Legge Mestel
            \end{itemize}
\end{frame}

\begin{frame}{General}
\begin{itemize}
\item Astrofisica stellare (castellani)
\item Rolfs rodney Couldron in the cosmos
\item iliadis nuclear burning
\item kippenhahn wiekert
\item salaris cassisi: evoluzione; popolazioni stellari;
\end{itemize}
\end{frame}

\begin{frame}{Rotating stars}
\begin{itemize}
\item Tassoul - Theory of rotating stars
\item Maeder - Physics, formation and evolution of rotating stars
\end{itemize}
\end{frame}

\begin{frame}{Milky way}
Thin disk thick disk transition
\end{frame}

\begin{frame}{popolazioni stellari}
salaris cassisi ch 9, 10, 11
\end{frame}

\begin{frame}{Radiative transfer}
    \begin{itemize}
        \item Radiative transport equation and star interior: Chap 3 Maeder
    \end{itemize}
\end{frame}

\subsection{RegLez 24/25}

\begin{frame}{RegLex 24/25}
    \begin{itemize}
\item Mar 11/02/2025 - Descrizione del corso. Richiami alle definizioni di temperatura efficace e magnitudine. Richiami a modulo di distanza, indice di colore, estinzione ed arrossamento. Richiamo alla misura di distanza con il metodo della parallasse. Indice $[Fe/H]$. Relazione tra $[Fe/H]$ e Z per stelle di disco.
\item Gio 13/02/2025 - Cenni alla curva universale delle abbondanze (nel disco galattico). Cenni alla determinazione delle abbondanze degli elementi nel sistema solare. Discussione generale sulla stima dell'abbondanza di elio per le stelle di disco. Alpha enhancement e discussione della sua origine.
\item Lun 17/02/2025 - Relazione tra $[Fe/H]$ e Z per stelle di disco. Relazione tra $[Fe/H]$ e Z in caso di alpha enhancement (stella di alone). Descrizione generale della struttura della Via Lattea. Caratteristiche generali di ammassi aperti ed ammassi globulari. Discussione generale sulla nucleosintesi primordiale per produzione di elio ed elementi leggeri e differenze con la nucleosintesi stellare. Caratteristiche generali delle stelle di thick disk. Alone interno ed alone esterno. Scenario qualitativo generale della formazione della nostra Galassia. Caratteristiche di alone e relazione con la cattura di galassie nane durante la formazione della Galassia.
\item Mar 18/02/2025 - Discussione sulle definizioni di pop. I e pop. II. Spiegazione qualitativa della relazione massa-luminosit\'a-tempi di vita per le stelle. Generalit\'a su diagrammi colore-magnitudine di disco e di campo. Cenni su initial mass function e sua determinazione. Relazione massa-luminosit\'a. Cenni sui metodi di valutazione della Star Formation Rate e della relazione et\'a-metallicit\'a nel disco galattico. Discussione generale sulla condizione di validit\'a di simmetria sferica stellare. Stima dell'ordine di grandezza del tempo scala dinamico: richiami al tempo scala di free-fall. Le equazioni di equilibrio stellare: equilibrio idrostatico ed equazione di continuit\'a.
\item Gio 20/02/2025 - L'equazione di stato: es. equazione di stato dei gas perfetti + pressione di radiazione. Peso molecolare medio. Esempio: calcolo del peso molecolare medio per materia completamente ionizzata. Peso molecolare medio degli ioni e peso molecolare medio per elettrone e loro relazione con il peso molecolare medio del gas. Discussione del concetto di equilibrio termodinamico locale e calcolo per ordini di grandezza dell'anisotropia di densit\'a di radiazione negli interni stellari.
\item Lun 24/02/2025 - Pressione di radiazione. Definizione del gradiente radiativo. Equazione del trasporto radiativo. Gradiente radiativo e gradiente ambientale. Equilibrio termico per le stelle. Stima per ordini di grandezza del tempo scala termico. Equazione dell'equilibrio termico (di conservazione dell'energia). Energia persa per neutrini; cenni ai processi principali. Calcolo della produzione di energia gravitazionale.Equazioni di equilibrio stellare con variabile indipendente la massa. Integrazione delle equazioni di equilibrio stellare: interno, subatmosfera ed atmosfera. Profondit\'a ottica.
\item Mar 25/02/2025 - Integrazione delle equazioni di equilibrio stellare: problema della stima dell'energia gravitazionale del ''primo modello'' calcolato. Cenni ai metodi di integrazione dell'atmosfera (variabile indipendente profondit\'a ottica). Metodo iterativo per il calcolo delle variabili fisiche esterne in atmosfera. Cenni al problema della determinazione della pressione di turbolenza e del gradiente ambientale nelle zone convettive esterne. Metodi numerici di integrazione delle strutture stellari. Il metodo del fitting.Metodo di Henyey per l'integrazione delle equazioni di equilibrio stellare.
\item Gio 27/02/2025 - Relazione tra energia termica ed energia gravitazionale in una struttura stellare all'equilibrio idrostatico. Il teorema del viriale per le strutture stellari: gas perfetto monoatomico. Il teorema del viriale per le strutture stellari: gas perfetto non monoatomico ed EOS generica. Criterio di stabilita' per le strutture stellari in base al valore della costante adiabatica del gas
\item Lun 03/03/2025 - Relazioni per ordini di grandezza tra quantita' fisiche in strutture stellari ricavate utilizzando le equazioni di equilibrio stellare ed il teorema del viriale: relazione tra massa, densita' e temperatura, relazione tra luminosita', massa, peso molecolare medio ed opacità, relazione tra temperatura, densità e produzione di energia. Tempo scala di Kelvin-Helmholtz. Calcolo del gradiente radiativo di temperatura. Approssimazione di strutture stellari a politropiche. Equazione di Lane-Emden. Risoluzione dell'equazione di Lane-Emden. Cenni a strutture isoterme.
\item Mar 04/03/2025 - Il modello solare approssimato con una politropica. Definizione di strutture stellari omologhe. Le strutture politropiche come esempio di strutture omologhe.Introduzione all'integrazione di strutture stellari omologhe.
\item Gio 06/03/2025 - Equazioni per l'integrazione di strutture stellari omologhe. Cenni all'integrazione di modelli omologhi di sequenza principale di et\'a zero.
\item Lun 10/03/2025 - Generalit\'a sul criterio di innesco della convezione in strutture stellari. Criterio di Schwarzschild per l'innesco della convezione. Criterio di Ledoux per l'innesco della convezione nel caso di gas perfetto ed EOS generica. Cenni al problema dell'overshooting. Discussione sul valore del gradiente ambientale nelle stelle in presenza di convezione.
\item Mar 11/03/2025 - Metodo della mixing length per il trattamento della convezione negli esterni stellari: calcolo del flusso convettivo in funzione della velocit\'a media degli elementi di convezione, calcolo del gradiente ambientale, in funzione della velocit\'a media degli elementi di convezione, calcolo della velocit\'a media degli elementi di convezione. Generalit\'a su calcolo dell'opacit\'a radiativa.
\item Gio 13/03/2025 - Opacit\'a a basse temperature: interazione fotoni-ione H- , Collision Induced Opacity. La media di Rosseland dell'opacit\'a sulla distribuzione in frequenza dei fotoni. Calcolo dell'opacit\'a (in forma analitica) per: scattering elettronico
\item Mar 18/03/2025 - Calcolo dell'opacit\'a (in forma analitica) per: correzione all'opacit\'a di scattering elettronico Thompson dovuta alla presenza di scattering Compton, per processi free free e fotoionizzazione.Discussione di grafici relativi all'andamento dell'opacit\'a negli interni stellari. Generalit\'a sul trasporto di energia tramite conduzione elettronica.Coefficiente di conduttività termica e sua stima approssimata nel caso di degenerazione totale non relativistica.
\item Gio 20/03/2025 - Definizione dell'''opacit\'a di conduzione'' e sua stima approssimata nel gaso di gas degenere non relativistico. Generalità su meccanismi di fusione nucleare nelle stelle. Espressione per il rate di fusione nucleare nelle stelle in funzione della sezione d'urto.
\item Lun 24/03/2025 - Sezione d'urto di fusione tra particelle cariche ed espressione per il rate di fusione nucleare. Picco di Gamow. Esempi di misure sperimentali di sezioni d'urto nucleari e del fattore astrofisico. Calcolo approssimato dei rates di fusione nucleare tra particelle cariche. Dipendenza approssimata delle reazioni nucleari tra particelle cariche dalla temperatura.
\item Mar 25/03/2025 - Sezione d'urto risonante: risonanze strette. Lo schermaggio elettronico in laboratorio. Calcolo approssimato dell'effetto di schermaggio elettronico in laboratorio. Lo schermaggio elettronico del plasma stellare: schermaggio debole e forte. Calcolo dell'effetto di schermaggio debole nelle stelle. Correzione al rate di fusione dovuto alla presenza di schermaggio debole.
\item Gio 27/03/2025 - Reazioni di fusione di elementi leggeri: deuterio, litio, berillio e boro. Discussione generale sul problema del 7Li. Fusione di idrogeno in elio: generalit\'a e calcolo approssimato del tempo di vita del Sole in fase di combustione centrale di H.
\item Lun 31/03/2025 - Catena protone-protone: rami ppI , ppII, ppIII. Fusione di H in He: calcolo approssimato del flusso di neutrini solari. Reazione protone-protone: calcolo del tempo scala di fusione dei protoni nel Sole. Elementi primari ed elementi secondari in una catena di reazioni. Abbondanza di equilibrio degli elementi secondari. Esempio: calcolo dell' abbondanza di equilibrio del deuterio. Discussione sul raggiungimento dell'abbondanza di equilibrio dell'elio 3 per stelle in combustione di H centrale.Il bi-ciclo CN-NO: caratteristiche generali e reazioni del ciclo. Fusione di H in He ad alte temperature: cicli NeNa e MgAl.
\item Mar 01/04/2025 - Caratteristiche generali della reazione $3\alpaa$ per la fusione di elio in 12C. Cenni alle caratteristiche delle stelle in fase di combustione centrale di elio determinate dalla dipendenza dalla temperatura di tale combustione. Discussione generale sulla reazione $12C+\alpha$ e sul rapporto 16O/12C nei nuclei delle stelle. Influenza della sezione d'urto $12C+\alpha$ sul tempo di vita in fase di combustione di elio centrale. Catene di produzioni di neutroni liberi. Reazioni in fasi evolutive avanzate: fusione di 12C, fotodisintegrazione del 20Ne, fusione dell'16O , fotosintegrazione del 28Si, catene di catture alfa su nuclei fino alla produzione di 56Fe.
\item Gio 03/04/2025 - Catture neutroniche su nuclei: andamento della sezione d'urto con l'energia e con il peso atomico. Processi s e processi r. Tempo di vita di un nucleo per cattura neutronica. Stima del flusso di neutroni caratteristico di processi s ed r. Stima dell'abbondanza di equilibrio per processi s. Spiegazione qualitativa dell'andamento dei picchi ''s'' ed ''r'' nella curva universale delle abbondanze.
\item Lun 07/04/2025 - Descrizione qualitativa dell'evoluzione di protostella: primo e secondo core idrostatico (core di Larson). Cenni alle caratteristiche di protostella. Luminosità di accrescimento nella fase protostellare. Passaggio tra protostella e stella di Pre-Sequenza Principale. Evoluzione di PMS: stelle completamente convettive, la traccia di Hayashi. Ruolo dell'opacità dell'H- nella verticalit\'a della traccia di Hayashi. Modellizzazione delle stelle di presequenza come strutture politropiche.Combustione del deuterio in PMS.
\item Mar 08/04/2025 - Andamento della luminosit\'a con il tempo in PMS. La sequenza di Henyey e l'approccio alla sequenza principale per stelle di sequenza principale inferiore e superiore. La Zero Age Main Sequence, ZAMS. Caratteristiche generali delle stelle di sequenza principale inferiore e superiore e delle Very Low Mass Stars. Il profilo di abbondanza di equilibrio dell'3He e l'approccio all'equilibrio dell'elio 3 per stelle di sequenza principale inferiore. Andamento della ZAMS nel diagramma HR per stelle di sequenza principale inferiore e superiore e per very low mass stars. Dipendenza della posizione della ZAMS nel diagramma HR dall'abbondanza originale di elio e metalli. Evoluzione di sequenza principale per stelle di sequenza principale inferiore, superiore e VLM. Andamento della temperatura centrale in ZAMS in funzione della massa.Discussione di grafici relativi all'evoluzione di protostella (a partire dal secondo core di Larson) e di presequenza principale.
\item Gio 10/04/2025 - Discussione sulla dipendenza della posizione della ZAMS nel diagramma HR dalle incertezze negli input fisici utilizzati dai modelli, dai modelli di atmosfera e dall'efficienza della convezione esterna. Metodo del Main Sequenze fitting per la determinazione della distanza di ammassi. Metodo del confronto teoria-osservazione per ZAMS stars per la determinazione del rapporto di arricchimento elio-metalli, criticit\'a del metodo.
\item Lun 14/04/2025 - Massa minima e massima in sequenza principale.Modalit\'a di esaurimento dell'H centrale per stelle di sequenza principale inferiore e superiore.Fase di subgigante rossa (SGB).Limite di Schonberg-Chandrasekhar e gap di Hertzsprung. Dalla fase di subgigante a quella di gigante rossa.
\item Mar 15/04/2025 - Caratteristiche generali della fase di gigante rossa. Il primo dredge up. Il bump dell'RGB. Dipendenza della luminosita' del bump dalla massa e dalla composizione chimica. Innesco dell'elio a flash per stelle di piccola massa. Massa del nucleo di elio all'innesco dell'elio centrale in funzione della massa totale della stella. Luminosità del vertice del ramo delle giganti rosse (RGB tip) in funzione della massa totale della stella. Dipendenza della massa del nucleo di elio all'innesco e della luminosità di RGB tip dalla composizione chimica. Dipendenza della massa di RGB transition dalla composizione chimica. Cenni alla diffusione microscopica: effetti sulla struttura interna delle stelle di SPI e sull'abbondanza superficiale. Discussione sull'evoluzione delle abbondanze superficiali dalla PMS al tip dell'RGB. Tracce evolutive ed isocrone di ammasso. Il Turn Off/Overall Contraction come indicatore di età.
\item Gio 17/04/2025 - Morfologia del ramo delle giganti rosse per stelle di massa piccola ed intermedia. Cenni agli effetti della diffusione microscopica sull'evoluzione di stelle di sequenza principale inferiore. Il tip dell'RGB come indicatore di distanza in galassie esterne, calibrazione e criticita' del metodo.
\item Gio 24/04/2025 - Discussione di grafici relativi alla fase di RGB, al metodo del tip del ramo delle giganti rosse come indicatore di distanza ed ai diagrammi CM di ammasso.
\item Mar 29/04/2025 - Stelle in combustione centrale di elio: struttura e ZAHB. Dipendenza della luminosit\'a della ZAHB dalla composizione chimica. La Zero Age Horizontal Branch come indicatore di distanza. Il metodo verticale per la determinazione dell'et\'a di ammassi antichi. Cenni al metodo orizzontale per la determinazione dell'et\'a di ammassi antichi. Caratteristiche generali degli hot flashers. Il ramo orizzontale anomalo in ammassi stellari giovani a bassa metallicit\'a in galassie esterne: l'RGB clump Discussione dei parametri che determinano la morfologia del ramo orizzontale. Combustione di elio centrale per stelle di massa intermedia; il clump dell'elio in ammassi di et\'a intermedia.Il clump dell'elio come indicatore di distanza.
\item Lun 05/05/2025 - Combustione di elio centrale per stelle di massa intermedia-grande; il loop dell'elio. Discussione di grafici riguardati caratteristiche evolutive discusse nella lezione precedente. Morfologia del clump dell'elio per stelle di campo locale. Autotrascinamento e semiconvezione in combustione centrale di elio. Discussione sul fenomeno dei pulsi convettivi. Il parametro R2. Il parametro R come indicatore di abbondanza originale di elio negli ammassi globulari. Ingresso in fase di ramo asintotico per stelle di massa $<2.5\msun{}$: il clump dell'AGB.
\item Mar 06/05/2025 - AGB manqu\'e. Evoluzione di ramo asintotico: il secondo dredge-up. Pulsi termici e terzo dredge-up. Produzione di litio in AGB. Nucleosintesi di elementi s in AGB: la tasca di 13C. Hot bottom burning. Variabili pulsanti: cenni al meccanismo di instabilit\'a, alla striscia di instabilit\'a ed alla relazione periodo-luminosit\'a-temperatura effettiva-massa. Cenni all'utilizzo delle RR Lyrae e delle cefeidi classiche come indicatori di distanza.
\item Gio 08/05/2025 - Definizione di masse grande, intermedie, piccole e very low mass stars. Destino finale di stelle di varia massa: nane bianche di He, C/O, O/Ne, supernovae di tipo II da deflagrazione del carbonio e da cattura elettronica su nuclei, supernovae di tipo II core collapse.
\item Lun 12/05/2025 - Destino finale di stelle di massa superiore a circa 10 masse solari. Supernovae di tipo II: caratteristiche di pre-supernova. Emissione di neutrini e loro interazione con i nuclei del nocciolo denso della supernova. Intrappolamentp e tempo scala di emissione dei neutrini. Ordini di grandezza dell'energia emessa sotto varie forme (neutrini, fotoni, energia del fronte di shock). Neutronizzazione esplosiva e meccanismo di esplosione ritardata. Cenni alla nucleosintesi esplosiva. Discussioni di grafici relativi all'evoluzione di ramo asintotico. Classificazione di SN II. Discussione generale sulle curve di luce delle SNII e sull'osservazione multibanda della SN 1987 A. Nane bianche: caratteristiche generali e relazione massa-raggio.
\item Mar 13/05/2025 - Trattamento semplificato del raffreddamento di nana bianca: relazione luminosit\'a temperatura del core, tempo di raffreddamento - temperatura del core e tempo di raffreddamento-et\'a (Mestel law). Effetti non considerati nella legge di Mestel: energia da cristallizzazione e separazione chimica, perdite energetiche per neutrini, combustione alla base dell'inviluppo di H, presenza di convezione nell'inviluppo.Classificazione spettroscopica delle nane bianche. Influenza dell'intercertezza sulla composizione chimica (Y e Z) e sull'efficienza della diffuzione microscopica nella determinazione di et\'a di ammassi globulari.
\item Gio 15/05/2025 - Effetto dell'efficienza dell'overshooting sulle caratteristiche evolutive a massa fissata, sull'isocrona e sulla determinazione di et\'a di un ammasso stellare tramite il metodo verticale. Valutazione dell'incertezza totale sulla stima di et\'a di ammassi stellari di et\'a antica ed intermedia.
        \end{itemize}<++>
\end{frame}

\subsection{RegLez 23/24}

\begin{frame}[allowframebreaks]{RegLex 23/24}\linkdest{rl2324}

\begin{itemize}
\item Lun 12 Feb lezione: Descrizione del corso. Richiami alle definizioni di temperatura efficace e magnitudine. Richiami a modulo di distanza, indice di colore, estinzione ed arrossamento. Richiamo alla misura di distanza con il metodo della parallasse. 
\item Mar 13 Feb lezione: Descrizione generale della struttura della Via Lattea. Caratteristiche generali di ammassi aperti ed ammassi globulari. Cenni alla curva universale delle abbondanze (nel disco galattico). Alpha enhancement e discussione della sua origine. Discussione sulle definizioni di pop. I e pop. II.Spiegazione qualitativa della relazione massa-luminosita-tempi di vita per le stelle. 
\item Gio 15 Feb lezione: Indice [Fe/H].Cenni alla determinazione delle abbondanze degli elementi nel Sole e nelle stelle di disco. Alpha enhancement e discussione della sua origine. 
\item Lun 19 Feb lezione: Relazione tra [Fe/H] e Z per stelle di disco. Relazione tra [Fe/H] e Z in caso di alpha enhancement (stella di alone). Cenni alla determinazione delle abbondanze degli elementi nel sistema solare. Discussione generale sulla nucleosintesi primordiale per produzione di elio ed elementi leggeri e differenze con la nucleosintesi stellare. Caratteristiche generali delle stelle di thick disk. Alone interno ed alone esterno. Scenario qualitativo generale della formazione della nostra Galassia. Caratteristiche di alone e relazione con la cattura di galassie nane durante la formazione della Galassia. 
\item Mar 20 Feb lezione: Generalità su diagrammi colore- magnitudine di disco e di campo. Cenni su initial mass function e sua determinazione. Relazione massa-luminosit\'a. Cenni sui metodi di valutazione della Star Formation Rate e della relazione et\'a-metallicit\'a nel disco galattico. Discussione generale sulla condizione di validità di simmetria sferica stellare. Stima dell'ordine di grandezza del tempo scala dinamico: richiami al tempo scala di free-fall. Le equazioni di equilibrio stellare: equilibrio idrostatico ed equazione di continuit\'a. L'equazione di stato: es. equazione di stato dei gas perfetti + pressione di radiazione. Peso molecolare medio. Esempio: calcolo del peso molecolare medio per materia completamente ionizzata. 
\item Gio 22 Feb lezione: Peso molecolare medio degli ioni e peso molecolare medio per elettrone e loro relazione con il peso molecolare medio del gas. Discussione del concetto di equilibrio termodinamico locale e calcolo per ordini di grandezza dell'anisotropia di densit\'a di radiazione negli interni stellari. Generalit\'a sull'equazione del trasporto radiativo e sull'opacit\'a. Cammino libero medio dei fotoni. Definizione del gradiente radiativo. 
\item Lun 26 Feb lezione: Equilibrio termico per le stelle. Stima per ordini di grandezza del tempo scala termico. Equazione dell'equilibrio termico (di conservazione dell'energia). Energia persa per neutrini; cenni ai processi principali. Calcolo della produzione di energia gravitazionale.Problema della stima dell'energia gravitazionale del "primo modello" calcolato. Equazioni di equilibrio stellare con variabile indipendente la massa. Integrazione delle equazioni di equilibrio stellare: interno, subatmosfera ed atmosfera. Profondità ottica. 
\item Mar 27 Feb lezione: Cenni ai metodi di integrazione dell'atmosfera (variabile indipendente profondità ottica). Metodo iterativo per il calcolo delle variabili fisiche esterne in atmosfera. Cenni al problema della determinazione della pressione di turbolenza e del gradiente ambientale nelle zone convettive esterne. Metodi numerici di integrazione delle strutture stellari. Il metodo del fitting.Metodo di Henyey per l'integrazione delle equazioni di equilibrio stellare. 
\item Gio 29 Feb lezione: Relazione tra energia termica ed energia gravitazionale in una struttura stellare all'equilibrio idrostatico. Il teorema del viriale per le strutture stellari: gas perfetto monoatomico. Il teorema del viriale per le strutture stellari: gas perfetto non monoatomico ed EOS generica. Criterio di stabilit\'a per le strutture stellari in base al valore della costante adiabatica del gas. Relazioni per ordini di grandezza tra quantit\'a fisiche in strutture stellari ricavate utilizzando le equazioni di equilibrio stellare ed il teorema del viriale: relazione tra massa, densit\'a e temperatura. 
\item Lun 04 Mar lezione: Relazioni per ordini di grandezza tra quantit\'a fisiche in strutture stellari ricavate utilizzando le equazioni di equilibrio stellare ed il teorema del viriale: relazione tra luminosita', massa, peso molecolare medio ed opacità, relazione tra temperatura, densità e produzione di energia. Tempo scala di Kelvin-Helmholtz. Calcolo del gradiente radiativo di temperatura. 
\item Mar 05 Mar lezione: Approssimazione di strutture stellari a politropiche. Equazione di Lane-Emden. Risoluzione dell'equazione di Lane-Emden. Cenni a strutture isoterme. Il modello solare approssimato con una politropica. Definizione di strutture stellari omologhe. Le strutture politropiche come esempio di strutture omologhe. 
\item Gio 07 Mar lezione: Equazioni per l'integrazione di strutture stellari omologhe. Cenni all'integrazione di modelli omologhi di sequenza principale di età zero. 
\item Mar 12 Mar lezione: Discussione sul valore del gradiente ambientale negli interni stellari. Metodo della mixing length per il trattamento della convezione negli esterni stellari:calcolo del flusso convettivo in funzione della velocità media degli elementi di convezione, calcolo del gradiente ambientale, in funzione della velocità media degli elementi di convezione, calcolo della velocità media degli elementi di convezione. 
\item Gio 14 Mar lezione: Generalit\'a su calcolo dell'opacit\'a radiativa. Opacita a basse temperature: interazione fotoni-ione H- , Collision Induced Opacity. La media di Rosseland dell'opacità sulla distribuzione in frequenza dei fotoni. Calcolo dell'opacit\'a (in forma analitica) per: scattering elettronico 
\item Lun 18 Mar lezione: Calcolo dell'opacit\'a (in forma analitica) per processi free free e fotoionizzazione.Discussione di grafici relativi all'andamento dell'opacit\'a negli interni stellari. Generalit\'a sul trasporto di energia tramite conduzione elettronica. 
\item Mar 19 Mar lezione: Coefficiente di conduttivit\'a termica e sua stima approssimata nel caso di degenerazione totale non relativistica. Definizione dell'"opacità di conduzione" e sua stima approssimata nel gaso di gas degenere non relativistico. Meccanismi di fusione nucleare nelle stelle. Sezione d'urto di fusione tra particelle cariche ed espressione per il rate di fusione nucleare. Picco di Gamow.Esempi di misure sperimentali di sezioni d'urto nucleari e del fattore astrofisico. 
\item Gio 21 Mar lezione: Calcolo approssimato dei rates di fusione nucleare tra particelle cariche. Dipendenza approssimata delle reazioni nucleari tra particelle cariche dalla temperatura. 
\item Lun 25 Mar lezione: Sezione d'urto risonante: risonanze strette. Lo schermaggio elettronico in laboratorio. Calcolo approssimato dell'effetto di schermaggio elettronico in laboratorio. Lo schermaggio elettronico del plasma stellare: schermaggio debole e forte. Calcolo dell'effetto di schermaggio debole nelle stelle. Correzione al rate di fusione dovuto alla presenza di schermaggio debole. 
\item Mar 26 Mar lezione: eazioni di fusione di elementi leggeri: deuterio, litio, berillio e boro. Fusione di idrogeno in elio: generalità e calcolo approssimato del tempo di vita del Sole in fase di combustione centrale di H. Catena protone-protone: rami ppI , ppII, ppIII. Fusione di H in He: calcolo approssimato del flusso di neutrini solari. Reazione protone-protone: calcolo del tempo scala di fusione dei protoni nel Sole. Elementi primari ed elementi secondari in una catena di reazioni. Abbondanza di equilibrio degli elementi secondari. Esempio: calcolo del\'l abbondanza di equilibrio del deuterio. Discussione sul raggiungimento dell'abbondanza di equilibrio dell'elio 3 per stelle in combustione di H centrale. 
\item Gio 28 Mar lezione: Il bi-ciclo CN-NO: caratteristiche generali e reazioni del ciclo. Fusione di H in He ad alte temperature: cicli NeNa e MgAl. 
\item Lun 08 Apr lezione: Caratteristiche generali della reazione 3 alfa per la fusione di elio in 12C. Cenni alle caratteristiche delle stelle in fase di combustione centrale di elio determinate dalla dipendenza dalla temperatura di tale combustione. Discussione generale sulla reazione 12C+alfa e sul rapporto 16O/12C nei nuclei delle stelle. Influenza della sezione d'urto 12C + alfa sul tempo di vita in fase di combustione di elio centrale. Catene di produzioni di neutroni liberi. Reazioni in fasi evolutive avanzate: fusione di 12C, fotodisintegrazione del 20Ne, fusione dell'16O , fotosintegrazione del 28Si, catene di catture alfa su nuclei fino alla produzione di 56Fe. 
\item Mar 09 Apr lezione: Catture neutroniche su nuclei: andamento della sezione d'urto con l'energia e con il peso atomico. Processi s e processi r. Tempo di vita di un nucleo per cattura neutronica. Stima del flusso di neutroni caratteristico di processi s ed r. Stima dell'abbondanza di equilibrio per processi s. Spiegazione qualitativa dell'andamento dei picchi "s" ed "r" nella curva universale delle abbondanze.Descrizione qualitativa dell'evoluzione di protostella: primo e secondo core idrostatico (core di Larson). Cenni alle caratteristiche di protostella. ezione: Luminosità di accrescimento nella fase protostellare. Passaggio tra protostella e stella di Pre-Sequenza Principale. 
\item Gio 11 Apr lezione: Evoluzione di PMS: stelle completamente convettive, la traccia di Hayashi. Ruolo dell'opacità dell'H- nella verticalit\'a della traccia di Hayashi. Dipendenza della traccia di Hayashi da massa, composizione chimica ed efficienza della convezione esterna. Modellizzazione delle stelle di presequenza come strutture politropiche. Andamento della luminosità con il tempo in PMS. Combustione del deuterio in PMS. 
\item Lun 15 Apr lezione: La sequenza di Henyey e l'approccio alla sequenza principale per stelle di sequenza principale inferiore e superiore. La Zero Age Main Sequence, ZAMS. Caratteristiche generali delle stelle di sequenza principale inferiore e superiore e delle Very Low Mass Stars. Il profilo di abbondanza di equilibrio dell'3He e l'approccio all'equilibrio dell'elio 3 per stelle di sequenza principale inferiore. Andamento della ZAMS nel diagramma HR per stelle di sequenza principale inferiore e superiore e per very low mass stars. Dipendenza della posizione della ZAMS nel diagramma HR dall'abbondanza originale di elio e metalli. Andamento della temperatura centrale in ZAMS in funzione della massa.Accenno al metodo di determinazione di DY/DZ dal confronto teoria-osservazione per stelle di disco locale parallassate. 
\item Mar 16 Apr lezione: Discussione di grafici relativi all'evoluzione di protostella (a partire dal secondo core di Larson) e di presequenza principale. Discussione sulla dipendenza della posizione della ZAMS nel diagramma HR dalle incertezze negli input fisici utilizzati dai modelli, dai modelli di atmosfera e dall'efficienza della convezione esterna. Metodo del Main Sequenze fitting per la determinazione della distanza di ammassi, criticit\'a del metodo. 
\item Gio 18 Apr lezione: Discussione del problema dell'abbondanza superficiale degli elementi leggeri nelle stelle. Massa minima in sequenza principale. 
\item Lun 22 Apr lezione: Evoluzione di sequenza principale per stelle di sequenza principale inferiore e superiore. Modalit\'a di esaurimento dell'H centrale per stelle di sequenza principale inferiore e superiore.Fase di subgigante rossa (SGB).Limite di Schonberg-Chandrasekhar e gap di Hertzsprung. Dalla fase di subgigante a quella di gigante rossa. 
\item Mar 23 Apr lezione: Caratteristiche generali della fase di gigante rossa. Il primo dredge up. Il bump dell'RGB. Dipendenza della luminosit\'a del bump dalla massa e dalla composizione chimica. Innesco dell'elio a flash per stelle di piccola massa. Perdite per plasma-neutrini ed inversione di temperatura nel nucleo di elio delle stella di RGB di piccola massa. 
\item Lun 29 Apr lezione: Morfologia del ramo delle giganti rosse per stelle di massa piccola ed intermedia: massa di "RGB transition". Massa del nucleo di elio all'innesco dell'elio centrale in funzione della massa totale della stella. Luminosità del vertice del ramo delle giganti rosse (RGB tip) in funzione della massa totale della stella. Dipendenza della massa del nucleo di elio all'innesco e della luminosità di RGB tip dalla composizione chimica. Dipendenza della massa di RGB transition dalla composizione chimica. Cenni alla diffusione microscopica: effetti sulla struttura interna delle stelle di SPI e sull'abbondanza superficiale. Discussione sull'evoluzione delle abbondanze superficiali dalla PMS al tip dell'RGB. Tracce evolutive ed isocrone di ammasso. Il Turn Off/Overall Contraction come indicatore di età. 
\item Mar 30 Apr lezione: Cenni agli effetti della diffusione microscopica sull'evoluzione di stelle di sequenza principale inferiore. Il tip dell'RGB come indicatore di distanza in galassie esterne, calibrazione e criticit\'a del metodo. Discussione di grafici relativi al metodo del Main Sequence fitting, alla fase di RGB ed al metodo del tip del ramo delle giganti ross ed ai diagrammi CM di ammasso. 
\item Gio 02 Mag lezione: Stelle in combustione centrale di elio: struttura e ZAHB. Dipendenza della luminosit\'a della ZAHB dalla composizione chimica. La Zero Age Horizontal Branch come indicatore di distanza. 
\item Lun 06 Mag lezione: Il metodo verticale per la determinazione dell'et\'a di ammassi antichi. Cenni al metodo orizzontale per la determinazione dell'et\'a di ammassi antichi. Caratteristiche generali degli hot flashers. Il ramo orizzontale anomalo in ammassi stellari giovani a bassa metallicit\'a in galassie esterne: l'RGB clump Discussione dei parametri che determinano la morfologia del ramo orizzontale. Combustione di elio centrale per stelle di massa intermedia; il clump dell'elio in ammassi di et\'a intermedia.Il clump dell'elio come indicatore di distanza. Combustione di elio centrale per stelle di massa intermedia-grande; il loop dell'elio. 
\item Mar 07 Mag lezione: Discussione di grafici riguardati caratteristiche evolutive discusse nella lezione precedente. Morfologia del clump dell'elio per stelle di campo locale. Variabili pulsanti: cenni al meccanismo di instabilità, alla striscia di instabilit\'a ed alla relazione periodo-luminosit\'a-temperatura effettiva-massa. Cenni all'utilizzo delle RR Lyrae e delle cefeidi classiche come indicatori di distanza. 
\item Gio 09 Mag lezione: Lezione non tenuta per partecipazione a commissione di ammissione al 40-esimo ciclo di dottorato in fisica 
\item Lun 13 Mag lezione: Effetto della variazione di composizione chimica (abbondanza di elio e metalli) sulle caratteristiche evolutive a massa fissata, sull'icrona e sulla determinazione di età di un ammasso stellare tramite il metodo verticale. Effetto dell'efficienza della diffusione e dell'overshooting sulle caratteristiche evolutive a massa fissata, sull'icrona e sulla determinazione di età di un ammasso stellare tramite il metodo verticale. Valutazione dell'incertezza totale sulla stima di et\'a di ammassi stellari di et\'a antica ed intermedia. 
\item Mar 14 Mag lezione: Autotrascinamento e semiconvezione in combustione centrale di elio. Discussione sul fenomeno dei pulsi convettivi. Il parametro R2. Il parametro R come indicatore di abbondanza originale di elio negli ammassi globulari. ngresso in fase di ramo asintotico: il clump dell'AGB. AGB manqu\'e. Evoluzione di ramo asintotico: il secondo dredge-up. Pulsi termici e terzo dredge-up.Produzione di litio in AGB. 
\item Gio 16 Mag lezione: Lezione non tenuta per partecipazione a commissione di ammissione al 40-esimo ciclo di dottorato in fisica 
\item Lun 20 Mag lezione: Nucleosintesi di elementi s in AGB: la tasca di 13C. Hot bottom burning. Definizione di masse grande, internedie, piccole e very low mass stars. Destino finale di stelle di varia massa: nane bianche di He, C/O, O/Ne, supernovae di tipo II da deflagrazione del carbonio e da cattura elettronica su nuclei, supernovae di tipo II core collapse. 
\item Mar 21 Mag lezione: Destino finale di stelle di massa superiore a circa 10 masse solari. Supernovae di tipo II: caratteristiche di pre-supernova. Emissione di neutrini e loro interazione con i nuclei del nocciolo denso della supernova. Intrappolamentp e tempo scala di emissione dei neutrini. Ordini di grandezza dell'energia emessa sotto varie forme (neutrini, fotoni, energia del fronte di shock). Neutronizzazione esplosiva e meccanismo di esplosione ritardata. Cenni alla nucleosintesi esplosiva. Discussioni di grafici relativi all'evoluzione di ramo asintotico. Classificazione di SN II. Discussione generale sulle curve di luce delle SNII e sull'osservazione multibanda della SN 1987 A. 
\item Gio 23 Mag lezione: Nane bianche: relazione massa-raggio. Trattamento semplificato del raffreddamento di nana bianca: relazione luminosità temperatura del core, tempo di raffreddamento - temperatura del core e tempo di raffreddamento-et\'a (Mestel law). Effetti non considerati nella legge di Mestel: energia da cristallizzazione e separazione chimica, perdite energetiche per neutrini. 
\end{itemize}
\end{frame}
\subsection{RegLez 21/22}\linkdest{rl2122}

\begin{frame}[allowframebreaks]{List of todos}
%\listoftodos
\end{frame}

%\begin{frame}[allowframebreaks]{List of Keywords}
%\listofkeywords
%\end{frame}

\begin{frame}[allowframebreaks]{List of Musts}
% \listofmusts
\end{frame}

\begin{frame}[allowframebreaks]{RegLeZ 21/22}
\begin{itemize}
\item 14/02/2022 Descrizione del corso. Richiamo alla misura di distanza con il metodo della parallasse. Precisione di misure di parallasse del satellite Gaia. Richiami alle definizioni di temperatura efficace e magnitudine.
\item 15/02/2022 Richiami a modulo di distanza, indice di colore, estinzione ed arrossamento. Descrizione generale della struttura della Via Lattea. Caratteristiche generali di ammassi aperti ed ammassi globulari.
\item 18/02/2022 Cenni alla curva universale delle abbondanze (nel disco galattico). Alpha enhancement e discussione della sua origine. Cenni alla determinazione delle abbondanze degli elementi nel sistema solare. Discussione sulle definizioni di pop. I e pop. II.
\item 21/02/2022 Indice [Fe/H] ed [alpha/Fe]. Caratteristiche generali delle stelle di thick disk. Alone interno ed alone esterno. Richiamo alla massa di Jeans. Spiegazione qualitativa della relazione massa-luminosita-tempi di vita per le stelle. Scenario qualitativo generale della formazione della nostra Galassia. Alone e relazione con la cattura di galassie nane durante la formazione della Galassia.
\item 22/02/2022 Cenni alle ipotesi di formazione del thick disk ed alla formazione inside out del disco. Generalit\'a su diagrammi colore- magnitudine di disco e di campo. Present day mass function ed initial mass function. Relazione massa-luminosit\'a. Cenni sui metodi di valutazione della Star Formation Rate e della relazione eta'-metallicita' nel disco galattico. Cenni alle caratteristiche della popolazione stellare nella piccola e grande nube di Magellano ed in Andromeda.
\item 25/02/2022 Discussione generale sulla nucleosintesi primordiale per produzione di elio ed elementi leggeri: principali catene di reazione. Reazioni di fusione di elementi leggeri nelle stelle. Discussione del problema della multipopolazione negli ammassi globulari della nostra galassia. Evidenza spettroscopica di anomalie chimiche e possibili ipotesi sull'origine delle stesse. Cenni alle evidenze fotometriche di multipopolazione. Cenni alle caratteristiche particolari della multipopolazione negli ammassi piu' massici della nostra Galassia e discussione sulla possibile causa.
\item 28/02/2022 Discussione generale sulla condizione di validit\'a di simmetria sferica stellare. Stima dell'ordine di grandezza del tempo scala dinamico. Richiami al tempo scala di free-fall. Le equazioni di equilibrio stellare: equilibrio idrostatico ed equazione di continuit\'a. L'equazione di stato: es. equazione di stai dei gas perfetti + pressione di radiazione. Peso molecolare medio. Esempio: calcolo del peso molecolare medio per materia completamente ionizzata. Peso molecolare medio degli ioni. Peso molecolare medio per elettrone.
\item 01/03/2022 Generalita' sull'equazione del trasporto radiativo e sull'opacit\'a. Discussione del concetto di equilibrio termodinamico locale e calcolo per ordini di grandezza dell'anisotropia di densit\'a di radiazione negli interni stellari. Equilibrio termico per le stelle. Equazione dell'equilibrio termico (di conservazione dell'energia). Energia persa per neutrini; cenni ai processi principali. Calcolo della produzione di energia gravitazionale. Problema della stima dell'energia gravitazionale del "primo modello" calcolato.
\item 04/03/2022 Equazioni di equilibrio stellare con variabile indipendente la massa. Integrazione delle equazioni di equilibrio stellare: interno, subatmosfera ed atmosfera. Profondit\'a ottica. Cenni ai metodi di integrazione dell'atmosfera (variabile indipendente profondit\'a ottica). Metodo iterativo per il calcolo delle variabili fisiche esterne in atmosfera. Cenni al problema della determinazione della pressione di turbolenza e del gradiente ambientale nelle zone convettive esterne.
\item 07/03/2022 Metodi numerici di integrazione delle strutture stellari. Il metodo del fitting. Metodo di Henyey per l'integrazione delle equazioni di equilibrio stellare. Relazione tra energia termica ed energia gravitazionale in una struttura stellare all'equilibrio idrostatico. Il teorema del viriale per le strutture stellari: gas perfetto monoatomico.
\item 08/03/2022 Il teorema del viriale per le strutture stellari: gas perfetto non monoatomico ed EOS generica. Criterio di stabilit\'a per le strutture stellari in base al valore della costante adiabatica del gas. Relazioni per ordini di grandezza tra quantit\'a fisiche in strutture stellari ricavate utilizzando le equazioni di equilibrio stellare ed il teorema del viriale: relazione tra massa, densit\'a e temperatura, relazione tra luminosit\'a, massa, peso molecolare medio ed opacit\'a, relazione tra temperatura, densit\'a e produzione di energia. Approssimazione di strutture stellari a politropiche.
\item 11/03/2022 Equazione di Lane-Emden. Risoluzione dell'equazione di Lane-Emden. Cenni a strutture isoterme. Il modello solare approssimato con una politropica.
\item 14/03/2022 Esempio di risoluzione di struttura stellare politropica: il modello solare approssimato con una politropica. Definizione di strutture stellari omologhe. Le strutture politropiche come esempio di strutture omologhe. Equazioni per l'integrazione di strutture stellari omologhe.
\item 15/03/2022 Es. cenni all'integrazione di modelli omologhi di sequenza principale di et\'a zero. Criterio di Schwarzschild per l'innesco della convezione.
\item 21/03/2022 Criterio di Ledoux per l'innesco della convezione nel caso di gas perfetto e non. Cenni al problema dell'overshooting. Relazioni tra gradiente radiativo, ambientale ed adiabatico. Il gradiente ambientale negli interni e negli esterni stellari. etodo della mixing length per il trattamento della convezione negli esterni stellari: calcolo approssimato dell'altezza di scala della pressione.
\item 22/03/2022 Metodo della mixing length per il trattamento della convezione negli esterni stellari:calcolo del flusso convettivo in funzione della velocit\'a media degli elementi di convezione, calcolo del gradiente ambientale, in funzione della velocit\'a media degli elementi di convezione, calcolo della velocit\'a media degli elementi di convezione. Generalita' su calcolo dell'opacita' radiativa. Opacita a basse temperature: interazione fotoni-ione H- , Collision Induced Opacity.
\item 25/03/2022 La media di Rosseland dell'opacit\'a sulla distribuzione in frequenza dei fotoni. Calcolo dell'opacita' (in forma analitica) per: scattering elettronico, processi free free, fotoionizzazione.
\item 28/03/2022 Discussione di grafici relativi all'andamento dell'opacita' negli interni stellari. Stima del tempo scala termico del Sole come tempo che impiega un fotone prodotto al centro a raggiungere la superficie. Tempo scala di Kelvin-Helmholtz. Generalit\'a sul trasporto di energia tramite conduzione elettronica. Coefficiente di conduttivit\'a termica e sua stima approssimata nel caso di degenerazione totale non relativistica.
\item 29/03/2022 Definizione dell'''opacit\'a di conduzione'' e sua stima approssimata nel gaso di gas degenere non relativistico. Meccanismi di fusione nucleare nelle stelle. Sezione d'urto di fusione tra particelle cariche ed espressione per il rate di fusione nucleare. Picco di Gamow.
\item 01/04/2022 Esempi di misure di sezioni d'urto nucleari. Calcolo approssimato dei rates di fusione nucleare tra particelle cariche. Dipendenza approssimata delle reazioni nucleari tra particelle cariche dalla temperatura.
\item 04/04/2022 Sezione d'urto risonante: risonanze strette. Lo schermaggio elettronico in laboratorio. Calcolo approssimato dell'effetto di schermaggio elettronico in laboratorio.
\item 05/04/2022 Lo schermaggio elettronico del plasma stellare: schermaggio debole e forte. Calcolo dell'effetto di schermaggio debole nelle stelle. Correzione al rate di fusione dovuto alla presenza di schermaggio debole. Reazioni di fusione di elementi leggeri: deuterio, litio, berillio e boro.
\item 08/04/2022 Fusione di idrogeno in elio: generalit\'a e calcolo approssimato del tempo di vita del Sole in fase di combustione centrale di H. Catena protone-protone: rami ppI , ppII, ppIII. Fusione di H in He: calcolo approssimato del flusso di neutrini solari. Reazione protone-protone: calcolo del tempo scala di fusione dei protoni nel Sole. Elementi primari ed elementi secondari in una catena di reazioni. Abbondanza di equilibrio degli elementi secondari. Esempio: calcolo dell' abbondanza di equilibrio del deuterio. Discussione sul raggiungimento dell'abbondanza di equilibrio dell'elio 3 per stelle in combustione di H centrale.
\item 11/04/2022 Il bi-ciclo CN-NO: caratteristiche generali. Fusione di H in He ad alte temperature: cicli NeNa e MgAl. CNO veloce. Dipendenza dalla temperatura della catena pp e del bi-ciclo CN-NO. Caratteristiche generali della reazione 3 alfa per la fusione di elio in 12C. Cenni alle caratteristiche delle stelle in fase di combustione centrale di elio determinate dalla dipendenza dalla temperatura di tale combustione. Discussione generale sulla reazione 12C+alfa e sul rapporto 16O/12C nei nuclei delle stelle. Influenza della sezione d'urto 12C + alfa sul tempo di vita in fase di combustione di elio centrale. Catene di produzioni di neutroni liberi. 
\item 26/04/2022 Descrizione qualitativa dell'evoluzione di protostella: primo e secondo core idrostatico (core di Larson). Cenni alle caratteristiche di protostella ed al passaggio tra protostella e stella di Pre-Sequenza Principale. Evoluzione di PMS: stelle completamente convettive, la traccia di Hayashi. Ruolo dell'opacit\'a dell'H- nella verticalit\'a della traccia di Hayashi.
\item 29/04/2022 Dipendenza della traccia di Hayashi da massa, composizione chimica ed efficienza della convezione esterna. La sequenza di Henyey e l'approccio alla sequenza principale per stelle di sequenza principale inferiore e superiore. La Zero Age Main Sequence, ZAMS. Caratteristiche generali delle stelle di sequenza principale inferiore e superiore e delle Very Low Mass Stars. Il profilo di abbondanza di equilibrio dell'3He e l'approccio all'equilibrio dell'elio 3 per stelle di sequenza principale inferiore.
\item 02/05/2022 Andamento della ZAMS nel diagramma HR per stelle di sequenza principale inferiore e superiore e per very low mass stars. Dipendenza della posizione della ZAMS nel diagramma HR dall'abbondanza originale di elio e metalli. Massa minima in sequenza principale. Cenni al problema della determinazione della massa massima in sequenza principale. Andamento della temperatura centrale in ZAMS in funzione della massa.
\item 03/05/2022 Evoluzione di sequenza principale per stelle di sequenza principale inferiore e superiore. Modalit\'a di esaurimento dell'H centrale per stelle di sequenza principale inferiore e superiore. Discussione sull'influenza sulla posizione della ZAMS del diagramma HR delle incertezze negli input fisici utilizzati dai modelli e nell'efficienza della convezione esterna.Accenno al metodo di determinazione di DY/DZ dal confronto teoria -osservazione per stelle di disco locale parallassate. Metodo del Main Sequenze fitting per la determinazione della distanza di ammassi, criticit\'a del metodo.
\item 06/05/2022 Fase di subgigante rossa (SGB).Limite di Schonberg-Chandrasekhar e gap di Hertzsprung. Dalla fase di subgigante a quella di gigante rossa. Caratterstiche generali della fase di gigante rossa. Il primo dredge up. Il bump dell'RGB. Dipendenza della luminosit\'a del bump dalla massa e dalla composizione chimica.
\item 09/05/2022 Innesco dell'elio a flash per stelle di piccola massa. Perdite per plasma-neutrini ed inversione di temperatura nel nucleo di elio delle stella di RGB di piccola massa. Dipendenza della massa del nucleo di elio all'innesco e della luminosità di RGB tip dalla composizione chimica. Morfologia del ramo delle giganti rosse per stelle di massa piccola ed intermedia: massa di ''RGB transition''. Massa del nucleo di elio all'innesco dell'elio centrale in funzione della massa totale della stella. Luminosità del vertice del ramo delle giganti rosse (RGB tip) in funzione della massa totale della stella. Dipendenza della massa di RGB transition dalla composizione chimica. Tracce evolutive ed isocrone di ammasso. Il Turn Off/Overall Contraction come indicatore di et\'a. Cenni alla diffusione microscopica: effetti sulla struttura interna delle stelle di SPI e sull'abbondanza superficiale.
\item 10/05/2022 Il tip dell'RGB come indicatore di distanza. Stelle in combustione centrale di elio: la ZAHB. Dipendenza della luminosita' della ZAHB dalla composizione chimica. La Zero Age Horizontal Branch come indicatore di distanza.
\item 13/05/2022 Il metodo verticale per la determinazione dell'et\'a di ammassi antichi. Cenni al metodo orizzontale per la determinazione dell'et\'a di ammassi antichi. Caratteristiche generali degli hot flashers. Il ramo orizzontale anomalo in ammassi stellari giovani a bassa metallicit\'a in galassie esterne: l'RGB clump Discussione dei parametri che determinano la morfologia del ramo orizzontale. Combustione di elio centrale per stelle di massa intermedia; il clump dell'elio in ammassi di et\'a intermedia. Il clump dell'elio come indicatore di distanza. Morfologia del clumo dell'elio per stelle di campo locale. Combustione di elio centrale per stelle di massa intermedia-grande; il loop dell'elio.
\item 16/05/2022 Discussione di grafici riguardati varie fasi evolutive discusse nelle lezioni precedenti. Autotrascinamento e semiconvezione in combustione centrale di elio. Discussione sul fenomeno dei pulsi convettivi. Il parametro R come indicatore di abbondanza originale di elio negli ammassi globulari.
\item Mar 17/05/2022 08:30-10:05 (2:0 h) lezione: Ingresso in fase di ramo asintotico: il clump dell'AGB. AGB manqu\'e. Evoluzione di ramo asintotico: il secondo dredge-up. Pulsi termici e terzo dredge-up. Nucleosintesi di elementi s in AGB: la tasca di 13C. Hot bottom burning. Produzione di litio in AGB.
\item 20/05/2022 Destino finale di stelle di varia massa: nane bianche di He, C/O, O/Ne, supernovae di tipo II da deflagrazione del carbonio e da cattura elettronica su nuclei, supernovae di tipo II da fotodisintegrazione del ferro. Supernovae di tipo II: caratteristiche di pre-supernova. Emissione di neutrini e loro interazione con i nuclei del nocciolo denso della supernova. Intrappolamentp e tempo scala di emissione dei neutrini. Ordini di grandezza dell'energia emessa sotto varie forme (neutrini, fotoni, energia del fronte di shock). Neutronizzazione esplosiva e meccanismo di esplosione ritardata.
\item 23/05/2022 Relazione massa-raggio e massa-densit\'a. Trattamento semplificato del raffreddamento di nana bianca: relazione luminosit\'a temperatura del core, tempo di raffreddamento - temperatura del core e tempo di raffreddamento-et\'a (Mestel law). Effetti non considerati nella legge di Mestel: energia da cristallizzazione e separazione chimica, perdite energetiche per neutrini.
\end{itemize}
\end{frame}

\subsection{RegLez 19}\linkdest{rl1819}

\begin{frame}[allowframebreaks]{List of todos}
\listoftodos
\end{frame}

%\begin{frame}[allowframebreaks]{List of Keywords}
%\listofkeywords
%\end{frame}

\begin{frame}[allowframebreaks]{List of Musts}
 \listofmusts
\end{frame}

\begin{frame}[allowframebreaks]{Reg Lez 19}
%
\begin{itemize}
\phantomsection\linkdest{febbraio}
\item 18/02/2019 - Descrizione del corso. Descrizione generale della via Lattea. Definizione di \keyword{ammasso stellare}. Ammassi aperti ed ammassi globulari. Caratteristiche generali delle stelle di disco e di alone. Richiami alle definizioni di magnitudine, indice di colore, modulo di distanza. Cenni alla \todo{curva universale delle abbondanze} (nel disco galattico). Alpha enhancement. 
\item 19/02/2019 - Caratteristiche generali del bulge e del thick disk. Discussione dell'origine dell'alpha enhancement. Scenario qualitativo generale della formazione della nostra Galassia. Alone interno ed alone esterno e relazione con la cattura di galassie nane durante la formazione della Galassia. Cenni alla formazione inside out del disco. Generalit\'a su \todo{diagrammi colore- magnitudine di disco e di campo}. Present day mass function ed initial mass function. \must{Relazione massa-luminosit\'a}. 
\item 22/02/2019 - Cenni alla derivazione dello Star formation rate e della relazione massa-luminosit\'a per il disco della nostra Galassia. Cenni alla produzione di elementi nella nucleosintesi primordiale. Generalit\'a sui \todo{metodi di determinazione delle abbondanze chimiche nel sistema solare}. 
\item 25/02/2019 - Discussione del problema della determinazione dell'elio primordiale e dell'elio in stelle di disco. Cenni alla struttura del gruppo locale ed alle caratteristiche delle popolazioni stellari all'interno delle galassie del Gruppo Locale. Discussione del problema della \todo{multipopolazione negli ammassi globulari}. 
\item 26/02/2019 - Generalit\'a su strutture autogravitanti. Stima del \must{tempo scala dinamico}. Richiamo al tempo di free fall. Equazioni di equilibrio stellare: \must{equilibrio idrostatico ed equazione di continuit\'a}. \must{Peso molecolare medio}. Esempio: calcolo del peso molecolare medio per materia completamente ionizzata. Peso molecolare medio degli ioni. Peso molecolare medio per elettrone. Generalit\'a sull'\must{equazione del trasporto radiativo} e sull'\must{opacit\'a}. Dimostrazione per ordini di grandezza che la presenza di un flusso negli interni stellari non contraddice l'assunzione di equilibrio termodinamico. 
\phantomsection\linkdest{marzo}
\item 01/03/2019 - Equilibrio termico per le stelle. Stima del \must{tempo scala termico}. \must{Equazione dell'equilibrio termico} (di conservazione dell'energia). Energia persa per neutrini; cenni ai processi principali. Calcolo della produzione di energia gravitazionale; problema della stima dell'energia gravitazionale del "primo modello" calcolato.

\item 04/03/2019 - \must{Integrazione delle equazioni di equilibrio stellare}: interno, subatmosfera ed atmosfera. Cenni ai metodi di integrazione dell'atmosfera (variabile indipendente profondit\'a ottica). Metodo iterativo per il calcolo delle variabili fisiche esterne in atmosfera. Cenni al problema della determinazione della \keyword{pressione di turbolenza} e del \must{gradiente ambientale nelle zone convettive esterne}. Metodi numerici di integrazione delle equazioni di equilibrio stellare: il \todo{metodo del fitting}. 
\item 05/03/2019 - \must{Metodo di Henyey} per l'integrazione delle equazioni di equilibrio stellare. Relazione tra energia termica ed energia gravitazionale in una struttura stellare all'equilibrio idrostatico. Il \must{teorema del viriale per le strutture stellari}: gas perfetto monoatomico e non monoatomico. Criterio di \must{stabilit\'a per le strutture stellari in base al valore della costante adiabatica del gas}. Relazioni per ordini di grandezza tra quantit\'a fisiche in strutture stellari ricavate utilizzando le equazioni di equilibrio stellare ed il teorema del viriale: relazione tra massa, densit\'a e temperatura, relazione tra luminosit\'a, massa, peso molecolare medio ed opacit\'a. 
\item 08/03/2019 - Approssimazione di strutture stellari a politropiche. \todo{Equazione di Lane-Emden e modalit\'a di risoluzione}. Cenni a strutture isoterme. Esempio. il modello solare approssimato con una politropica. 
\item 11/03/2019 Equazione del trasporto radiativo. Generalit\'a sul trasporto di energia tramite \todo{conduzione elettronica}. \todo{Coefficiente di diffusione per conduzione}. Definizione dell'"opacit\'a di conduzione" e sua stima approssimata nel caso di gas degenere non relativistico. Generalit\'a su calcolo dell'opacit\'a radiativa. Calcolo dell'opacit\'a (in forma analitica) per: \must{scattering elettronico}, \must{processi free free}, \must{fotoionizzazione}. Opacit\'a nelle atmosfere stellari: \must{opacit\'a di interazione fotoni-ione $H^-$}. 
\item 12/03/2019 - L'\must{equazione di stato delle strutture stellari}. \must{Gas degenere elettronicamente}. \must{Effetti coulombiani nelle strutture stellari}. Equazione di Saha.
\item 15/03/2019 - La \must{media di Rosseland} dell'opacit\'a sulla distribuzione in frequenza dei fotoni. Discussione di grafici relativi all'andamento dell'opacit\'a negli interni stellari. \must{Criterio di Schwarzschild} per l'innesco della convezione. Cenni al problema dell'\keyword{overshooting}.
\item 18/03/2019 - \must{Criterio di Ledoux} per l'innesco della convezione nel caso di gas perfetto e non. Relazioni tra gradiente radiativo, ambientale ed adiabatico. Il \must{gradiente ambientale negli interni e negli esterni stellari}. \must{Metodo della mixing length} per il trattamento della convezione negli esterni stellari: Calcolo approssimato dell'altezza di scala della pressione, calcolo del flusso convettivo e del gradiente ambientale, in funzione della velocit\'a media degli elementi di convezione.
\item 19/03/2019 - Teoria della mixing length per il calcolo del flusso convettivo: calcolo della velocit\'a media degli elementi di convezione. Meccanismi di fusione nucleare nelle stelle. \must{Sezione d'urto di fusione tra particelle cariche} ed espressione per il \must{rate di fusione nucleare}. \must{Picco di Gamow}. Calcolo approssimato dei rates di fusione nucleare tra particelle cariche. Dipendenza approssimata delle reazioni nucleari tra particelle cariche dalla temperatura.
\item 22/03/2019 - Esempi di misure di sezioni d'urto nucleari. \must{Sezione d'urto risonante}: risonanze strette. Lo schermaggio elettronico in laboratorio. Calcolo approssimato dell'effetto di schermaggio elettronico in laboratorio. Lo schermaggio elettronico del plasma stellare: \must{schermaggio debole e forte}.
\item 25/03/2019 - Schermaggio elettronico nelle stelle. Calcolo dell'effetto di schermaggio debole. Correzione al rate di fusione dovuto alla presenza di schermaggio debole. \must{Reazioni di fusione di elementi leggeri}: deuterio, litio, berillio e boro. \must{Fusione di idrogeno in elio}: generalit\'a e calcolo approssimato del flusso di neutrini solari. Reazione protone-protone: calcolo del tempo scala di fusione dei protoni nel Sole. Elementi primari ed elementi secondari in una catena di reazioni. Abbondanza di equilibrio degli elementi secondari. Esempio: calcolo dell'\keyword{abbondanza di equilibrio del deuterio}. Raggiungimento dell'\keyword{abbondanza di equilibrio dell'elio 3} nella catena protone protone.
\item 26/03/2019 - Il \must{bi-ciclo CN-NO}: caratteristiche generali. Cenni allo \must{spettro di neutrini solari}. Cicli di CNO veloce. Dipendenza dalla temperatura della catena pp e del bi-ciclo CN-NO. Cenni alla caratteristiche delle stelle di \must{sequenza principale inferiore e superiore} dovute alle modalit\'a di combustione di H in He. \must{Reazione 3 alfa} per la fusione di elio in 12C e 12C+alfa per la produzione di 16 O. Confronto 12C + alfa e 16O + alfa alle condizioni fisiche tipiche della fusione di elio al centro.
\item 29/03/2019 - Influenza della sezione d'urto 12C + alfa sul tempo di vita in fase di combustione di elio centrale. Catene di {produzioni di neutroni liberi}. Reazioni in fasi evolutive avanzate: \must{fusione di 12C}, \must{fotodisintegrazione del 20Ne}, \must{fusione dell'16O} , \must{fotosintegrazione del 28Si}, catene di catture alfa su nuclei fino alla {produzione di 56Fe}.
\phantomsection\linkdest{aprile}
\item 01/04/2019 - Catture neutroniche su nuclei: andamento della sezione d'urto con l'energia e con il peso atomico. \must{Processi s e processi r}. Tempo di vita di un nucleo per cattura neutronica. Stima del flusso di neutroni caratteristico di processi s ed r. Stima dell'abbondanza di equilibrio per processi s. Spiegazione qualitativa dell'\must{andamento dei picchi "s" ed "r" nella curva universale delle abbondanze}. Cenni alle caratteristiche di protostella ed al passaggio tra protostella e \must{stella di Pre-Sequenza Principale}.
\item 02/04/2019 - Lezione seminariale del dott. Emanuele Tognelli su formazione stellare: tempo di Kelvin-Helmholtz, accrescimento della protostella fino al primo ed al secondo core idrostatico. Evoluzione di PMS: \must{traccia di Hayashi}.
\item 05/04/2019 - Lezione seminariale del dott. Tognelli su: evoluzione di Pre-Sequenza Principale. Ruolo dell'opacit\'a dell'H- nella verticalit\'a della traccia di Hayashi, innesco della fusione del deuterio. \must{Stelle-PMS completamente convettive}. \must{Stelle che in PMS sviluppano un nucleo radiativo}.
\item 08/04/2019 - \must{Zero Age Main Sequence}. Approccio alla ZAMS per stelle di sequenza principale inferiore e superiore. Il \keyword{profilo di abbondanza di equilibrio dell'3He}. Caratteristiche generali delle stelle di sequenza principale inferiore e superiore. Dipendenza della posizione della ZAMS nel diagramma HR dall'abbondanza originale di elio e metalli. Accenno al metodo di \todo{determinazione di DY/DZ dal confronto teoria-osservazione per stelle di disco locale parallassate}, Dipendenza della massa di transizione dall'abbondanza originale di elio e metalli. Massa minima e massima in ZAMS.
\item 09/04/2019 - Influenza sulla \must{posizione della ZAMS del diagramma HR} delle incertezze negli input fisici utilizzati dai modelli e nell'efficienza della convezione esterna. \must{Metodo del Main Sequenze fitting} per la determinazione della distanza di ammassi, criticit\'a del metodo. Evoluzione di sequenza principale per stelle di sequenza principale inferiore e superiore. Modalit\'a di esaurimento dell'H centrale per stelle di sequenza principale inferiore e superiore: \must{Turn Off and Overall Contraction}. Fase di \must{subgigante rossa (SGB)}. \must{Gap di Hertzsprung}.
\item 12/04/2019 - Dalla fase di subgigante a quella di \must{gigante rossa}. \must{Tracce evolutive ed isocrone di ammasso}. Il \must{Turn Off/Overall Contraction come indicatore di et\'a}. Isocrone di ammassi giovani e vecchi. Evoluzione di gigante rossa: il \must{primo dredge up}.
\item 15/04/2019 - Morfologia del \must{ramo delle giganti rosse per stelle di massa piccola ed intermedia}: massa di \must{''RGB transition''}. Dipendenza della massa di RGB transition dalla composizione chimica. \must{Massa del nucleo di elio all'innesco} dell'elio centrale in funzione della massa totale della stella. Luminosit\'a del vertice del ramo delle giganti rosse (\must{RGB tip}) in funzione della massa totale della stella. Dipendenza della massa del nucleo di elio all'innesco e della \must{luminosit\'a di RGB tip} dalla composizione chimica. Dipendenza dell'isocrona e della \must{luminosit\'a del TO/OC} dall'abbondanza originale di elio e metalli. Influenza sull'isocrona e sulla luminosit\'a del TO della diffusione microscopica. Il \must{bump dell'RGB}. Dipendenza della luminosit\'a del bump dall'efficienza della convezione esterna, dall'et\'a dell'ammasso e dalla composizione chimica. Il \must{tip dell'RGB come indicatore di distanza}. Dipendenza della luminosit\'a del tip dalla composizione chimica. \must{Innesco dell'elio a flash per stelle di piccola massa}.
\item 16/04/2019 - Stelle in \must{combustione centrale di elio}: la \must{ZAHB}. Caratteristiche generali degli \todo{hot flashers}. Il \todo{ramo orizzontale anomalo in ammassi stellari giovani a bassa metallicit\'a} in galassie esterne: l'\must{RGB clump}. La \must{Zero Age Horizontal Branch come indicatore di distanza}. Dipendenza della luminosit\'a della ZAHB dalla composizione chimica. Cenni al metodo orizzontale per la determinazione dell'et\'a di ammassi antichi. Il \must{metodo verticale per la determinazione dell'et\'a} di ammassi antichi. Cenni alla \todo{discrepanza teoria-osservazione per la luminosit\'a del bump dell'RGB}.
\item 29/04/2019 - Evoluzione in combustione di elio per stelle di ramo orizzontale. \must{Combustione di elio centrale per stelle di massa intermedia}; il \todo{clump dell'elio in ammassi di et\'a intermedia}. Il \todo{clump dell'elio come indicatore di distanza}. \must{Combustione di elio centrale per stelle di massa intermedia-grande}; il \todo{loop dell'elio}. Incertezza nella determinazione di et\'a di ammassi antichi tramite il metodo verticale.
\item 30/04/2019 - \must{Effetto dell'overshooting sui modelli di sequenza principale superiore} e sulla determinazione di et\'a di ammassi di et\'a giovane-intermedia. Incertezze sulla determinazione di et\'a di ammassi di et\'a giovane-intermedia. Discussione sui parametri che influenzano la \must{morfologia di ramo orizzontale}.
\phantomsection\linkdest{maggio}
\item 03/05/2019 - Effetto della presenza di diffusione microscopica sulla determinazione dell'et\'a tramite il metodo verticale. Il \must{parametro R per la determinazione dell'elio}. Esaurimento dell'elio centrale: \todo{autotrascinamento del nucleo}, \must{semiconvezione} e \must{pulsi convettivi}.
\item 06/05/2019 - Ingresso in fase di ramo asintotico: il \must{clump dell'AGB}. \todo{AGB manqu\'e}. Evoluzione di ramo asintotico: il \must{secondo dredge-u}p. \must{Pulsi termici} e \must{terzo dredge-up}. \must{Nucleosintesi di elementi s in AGB}: la tasca di 13C. \keyword{Produzione di litio in AGB}.
\item 07/05/2019 - Lezione seminariale del dott. Emanuele Tognelli su: abbondanza superficiale di elementi leggeri in fase di Pre-Sequenza Principale. Il metodo del \must{''lithium depletion boundary''} per la datazione di ammassi giovani.
\item 10/05/2019 - Caratteristiche generali dell'\must{evoluzione avanzata di stelle massicce}. Destino finale di stelle di varia massa: \must{nane bianche di He, C/O, O/Ne}, \must{supernovae di tipo II} da deflagrazione del carbonio e da cattura elettronica su nuclei, supernovae di tipo II da fotodisintegrazione del ferro.
\item 13/05/2019 - lezione: Discussione sul \todo{60Fe come elemento indicatore di esplosioni di supernova} nelle vicinanze della Terra negli ultimi milioni di anni. \must{Supernovae di tipo II: caratteristiche di pre-supernova}. Emissione di neutrini e loro interazione con i nuclei del nocciolo denso della supernova. \todo{Intrappolamentp e temposcala di emissione dei neutrini}. Stima per ordini di grandezza dell'energia emessa sotto varie forme (neutrini, fotoni, energia del fronte di shock). \must{Neutronizzazione esplosiva} e \must{meccanismo di esplosione ritardata}. Discussione sulle caratteristiche della \todo{supernova 1987A}, sul flusso di neutrino osservato e sulla sua osservazione in varie bande elettromagnetiche nelle varie fasi di supernova. \must{Classificazione dei vari tipi di supernovae} in base allo spettro ed alla morfologia della curva di luce.
\item 14/05/2019 - Lezione seminariale del prof. Prada Moroni su \must{evoluzione delle nane bianche}.
\item 17/05/2019 - Caratteristiche e metodo di calcolo del \todo{modello solare standard}. Generalit\'a sull'eliosismologia e sui modelli solari eliosismologici. Generalit\'a sulla rivelazione dei \todo{neutrini solari} come test aggiuntivo del modello solare standard.
\item 20/05/2019 - \must{Stelle variabili pulsanti}: cenni al meccanismo di pulsazione. La \must{striscia di instabilit\'a} nel diagramma HR ed i vati tipi di stelle variabili pulsanti. \must{RR Lyrae}: diagramma di Bailey, curve di luce, relazione periodo-luminosit\'a-massa-temperatura effettiva. Le \must{RR Lyrae come indicatori di distanza}. Il \must{parametro A come indicatore di elio}. Stelle \must{cefeidi classiche}. Uso delle \must{stelle cefeidi come indicatori di distanza} di ammasso e di campo. La \todo{dicotomia di Oosterhoff}.
\item 21/05/2019 - Lezione seminariale del dott. Cignoni su studi di popolazione nella galassie esterne: \todo{recupero della star formation history}. Analisi di \todo{popolazioni non risolte semplici e complesse}.
\item 24/05/2019 - lezione: Caratteristiche generali delle \must{SNIa}. Possibili progenitori della SNIa: generalit\'a su meccanismi di scambio di massa in sistemi binari, sistemi finali dopo due common envelope, possibili scenari di accrescimento di H/He/C su una compagna degenere. La \must{SNIa come indicatori di distanza}.
\end{itemize}
\end{frame}

\subsection{RegLez 17}\linkdest{rl1617}

\begin{frame}[allowframebreaks]{Reg Lez 17}
\begin{itemize}
\item testo: Popolazioni stellari: formazione stellare (fenomenologia)
\item lezione: Descrizione generale della struttura della nostra Galassia. Concetti base: parallasse, magnitudine assoluta ed apparente, modulo di distanza, estinzione ed arrossamento. Caratteristiche generali di ammassi aperti e globulari. Alpha enhancement.
\item lezione: Caratteristiche fotometriche, dinamiche e chimiche delle popolazioni stellari della nostra Galassia. Caratteristiche generali delle stelle di thick disk. Relazione massa-luminosit\'a per le stelle di sequenza principale. Relazione generale tra massa e tempo di vita di una stella. Initial Mass Function and Present Day Mass Function.
\item lezione: Differenze generali tra stelle di ammasso e stelle di campo. Esempi di diagrammi Colore-Magnitudine di stelle di ammasso e stelle di campo. Discussione generale sullo studio delle caratetristiche delle stelle di campo. Cenni alla determinazione dello ''Star Formation Rate'' per il disco della nostra Galassia ed alla determinazione della relazione et\'a-metallicit\'a. 
\item lezione: Scenario generale di formazione della Via Lattea. Cenni alle galassie del Gruppo Locale. Cenni alle popolazioni stellari nella Via Lattea e nella galassie del Gruppo Locale. Metodi di determinazione delle abbondanze stellari. La curva ''universale'' delle abbondanze. Cenni alla nucleosintesi primordiale.
\item lezione: Discussione generale sulla multipopolazione negli ammassi globulari della nostra Galassia. 
\item Testo: Equazioni struttura stellare: fenomenologia e metodi numerici
\item lezione: Equilibrio idrostatico nelle stelle. Equazioni di equilibrio stellare: equilibrio idrostatico ed equazione di continuit\'a.
\item lezione: Peso molecolare medio. Esempio: calcolo del peso molecolare medio per materia completamente ionizzata. Peso molecolare medio degli ioni. Peso molecolare medio per elettrone. Generalit\'a sull'equazione del trasporto radiativo. Equilibrio termico. Le equazioni di equilibrio stellare. Calcolo dell'energia "gravitazionale".
\item lezione: Calcolo approssimato del tempo scala termico nell'interno del Sole. L'equazione del trasporto nelle atmosfere stellari. Profondit\'a ottica. Equazioni di equilibrio stellare in atmosfera. Generalit\'a sui metodi numerici di integrazione delle equazioni di equilibrio stellare. Il metodo del fitting.
\item lezione: Seminario del dott. Tognelli su: l'equazione di stato delle strutture stellari. Gas degenere elettronicamente. Effetti coulombiani nelle strutture stellari. Equazione di Saha.
\item lezione: Il metodo di Henyey per la risoluzione delle equazioni di equilibrio stellare. Integrazione dell'atmosfera. L'equazione del trasposto radiativo. Il teorema del viriale per le strutture stellari: caso di gas perfetto monoatomico. Tempo scala di Kelvin-Helmholtz.
\item lezione: Il teorema del viriale: gas perfetto non monoatomico. Criterio di stabilit\'a delle strutture stellari. Utilizzo delle equazioni di equilibrio stellare e del teorema del viriale per ottenere relazioni per ordini di grandezza tra: 1) massa, densit\'a e temperatura delle stelle 2) massa-luminosit\'a.
\item lezione: Calcolo approssimato dell'opacit\'a da scattering Thompson nel caso di ionizzazione totale. Formula di Kramer per opacit\'a free-free e bound free. Opacit\'a legata agli ioni H-.
\item lezione: La conduzione elettronica. Opacit\'a conduttiva. Equazione del trasporto in presenza di opacit\'a conduttiva. Criterio di Schwarzschild e di Ledoux per l'innesco della convezione in ambiente stellare. Cenni al fenomeno dell'overshooting.
\item lezione: Il metodo della mixing lenght per il trattamento della convezione negli inviluppi esterni stellari. Calcolo approssimato dell'altezza di scala della pressione. 
\item lezione: La teoria della mixing lenght per il trattamento della convezione negli esterni stellari: calcolo del flusso convettivo, della velocit\'a media delle bolle di convezione e del gradiente ambientale negli esterni stellari. Modelli politropici di strutture stellari: equazione di Lane-Emden e calcolo dell'andamento di pressione e densit\'a per modelli stellari politropici. Esempi: il modello solare.
\item Testo: Produzione enrgia: reazioni nucleari (evoluzione)
\item lezione: Calcolo dei rates di reazioni di fusione nucleare tra particelle cariche. La probabilit\'a di penetrazione della barriera coulombiana ed il fattore astrofisico. Il picco di Gamow. Espressione approssimata dei rates di fusione nucleare. Dipendenza approssimata delle reazioni dalla temperatura.
\item lezione: Lo schermaggio elettronico in laboratorio. Lo schermaggio elettronico nel plasma stellare: schermaggio debole, intermedio e forte. Trattamento dello schermaggio debole negli interni stellari secondo il metodo di Salpeter. 
\item lezione: Reazioni nucleari di combustione di elementi leggeri. Elementi primari ed elementi secondari. Concentrazione di equilibrio per gli elementi secondari. Reazioni di fusione di H in He: la catena protone-protone ed il bi-clo CN-NO. Neutrini solari.
\item lezione: Reazioni del ciclo CNO veloce. Reazioni di fusione di elio in C ed O. Catene di produzione di neutrini liberi in fasi evolutive avanzate.
\item lezione: Reazioni di fusione del C e dell'O. Reazioni nucleari successive fino alla produzione degli elementi del picco del ferro. Struttura a cipolla di pre-supernova e nucleosintesi esplosiva. Catture neutroniche su nuclei: processi s e processi r. Sezione d'urto per cattura neutronica ed andamento con l'energia del rate delle reazioni di cattura neutronica. Stima del flusso di neutroni caratteristico di processi s ed r. Stima dell'abbondanza di equilibrio per processi s. Spiegazione qualitativa dell'andamento dei picchi "s" ed "r" nella curva universale delle abbondanze.
\item Testo: MS-PMS
\item lezione: Seminario del dott. Emanuele Tognelli su caratteristiche delle fasi di protostella e di Pre-Sequenza Principale (PMS). Traccia di Hayashi. Effetto di variazione di massa e composizione chimica in PMS. Combustione del deuterio ed effetti sulle strutture di PMS.
\item lezione: Caratteristiche e metodo di calcolo del solare standard. Generalit\'a sull'eliosismologia e sui modelli solari eliosismologici.
\item lezione: Seminario del dott. Tognelli su evoluzione di Pre-sequenza principale, evoluzione temporale dell'abbondanza superficiale di elementi leggeri. Ingresso in ZAMS per stelle di sequenza principale inferiore e superiore.
\item lezione: Dipendenza della posizione di PMS nel diagramma HR dalla composizione chimica e dall'efficienza della convezione esterna. Zero Age Main Sequence. Dipendenza della posizione di ZAMS nel diagramma HR dalla composizione chimica e dall'efficienza della convezione esterna. Stelle di sequenza principale inferiore e superiore. Approccio alla ZAMS: sviluppo di un piccolo nucleo convettivo per stelle di SPI durante il raggiungimento dell'abbondanza di equilibrio dell'3He. Profilo dell'abbondanza di equilibrio dell'3He per stelle di SPI.Very low mass. Calcolo approssimativo della massa minima per l'innesco della fusione di H in He. 
\item lezione: Evoluzione di sequenza principale. Modalit\'a di esaurimento dell'idrogeno centrale per stelle di sequenza principale inferiore e superiore. Fase di subgigante rossa. Isocrona di ammasso. La luminosit\'a all'esaurimento dell'idrogeno centrale come indicatore di et\'a di una popolazioen stellare semplice. 
\item lezione: Incertezze nelle previsioni teoriche di MS. Generalit\'a sulle isocrone di ammasso. Il metodo del fitting della MS per la determinazione della disatnza degli ammassi globulari. La luminosit\'a del TO/OC come indicatore di et\'a. Evoluzione di sub-gigante rossa. Dipendenza delle tracce di MS/SGB dalla composizione chimica. Influenza della diffusione sulla traccia di stelle di data massa nel diagramma HR.
\item Testo: Post Hydrogen
\item lezione: Evoluzione di gigante rossa: il primo dredge up ed il bump dell'RGB. Dipendenza della luminosit\'a del bump dalla composizione chimica. Perdite di massa in RGB. Il flash dell'elio. Morfologia dell'RGB per ammassi stellari giovani ed antichi.
\item lezione: Massa dell'elio all'innesco dell'elio centrale in funzione della massa totale della stella. Massa di RGB transition e sua dipendenza dalla composizione chimica. Dipendenza della luminosit\'a del vertice del ramo della giganti rosse dalla composizione chimica. Il tip dell'RGB come indicatore di distanza.
\item lezione: Dipendenza dell'isocrona dalla composizione chimica e dalla diffusione. Fase di combustione di elio per stelle di ammasso antico: il ramo orizzontale, HB. Zero Age Horizontal Branch. Il ramo orizzontale come candela campione, metodo verticale per la determinazione dell'et\'a di ammassi antichi. Dipendenza della ZAHB dalla composizione chimica.
\item lezione: Influenza della diffusione sulla determinazione dell'et\'a degli ammassi tramite il metodo verticale. Evoluzione di ramo orizzontale. Morfologia del ramo orizzontale: dipendenza da metallicit\'a ed et\'a dell'ammasso. Hot flashers. Il clump dell'elio ed il loop dell'elio. Il clump dell'elio come indicatore di distanza.
\item lezione: Effetto della presenza di overshooting sulla determinazione di et\'a di ammassi giovani. Discussione dell'incertezza sulla detrminazione di et\'a in ammassi stellari. Evoluzione in combustione di elio centrale: autotrascinamento e semiconvezione. Il parametro R per la stima dell'abbondanza di elio in ammassi antichi.
\item lezione: Cenni su stelle variabili come indicatori di distanza: RR Lyrae e cefeidi classiche. Ingresso in ramo asintotico. Il clump dell'AGB in stelle di piccola massa. Caratteristiche generali per stelle di AGB. Il secondo dredge up. Nucleosintesi in AGB. Pulsi termici.
\item lezione: Evoluzione della luminosit\'a durante i pulsi termici. Terzo dredge up, hot bottom burning. Catene di produzione di neutroni liberi e processi s in fase di ramo asintotico.
\item lezione: Evoluzione finale di stelle di varia massa. Nane bianche di C/O e O/Ne, supernovae di tipo II da deflagrazione del carbonio e da cattura elettronica su nuclei. Supernovae di tipo II da fotodisintegrazioen del ferro.
\item Lezione seminariale del prof. Prada Moroni su caratteristiche strutturali delle nane bianche, curva di raffreddamento di nana bianca. Misura di distanza ed et\'a in ammasso tramite la curva di raffreddamento delle nane bianche.
\item lezione: Lezione seminariale del prof. Cignoni su popolazioni stellari complesse nelle vicinanze del Sole e nelle galassie nane. recupero della star formation rate in popolazioni complesse.
\end{itemize}
\end{frame}

%! TEX root = main.tex
\subsection{Radiation}

\begin{frame}{Radiation}
    \begin{columns}[T]
        \begin{column}{0.6\textwidth}
    \begin{align*}
        &F=\int I\cos{\theta}d\Omega=2\pi\int_{-1}^1I\mu d\mu\si{\erg\per\second\per\square\cm}\tag{Tot z-Flux}\\
        &\TDy{s}{I}=0, I=\frac{j}{\kappa}=\frac{c}{4\pi}aT^4=\frac{\sigma}{\pi}T^4\tag{LTE}\\
        &U=\frac{2\pi}{c}\int_{-1}^1I\,d\mu=\frac{4\pi}{c}I\tag{LTE}\\
        &\mu\PDy{r}{I_{\nu}}+[\frac{(1-\mu^2)}{r}\PDy{\mu}{I_{\nu}}]=j_{\nu}-\kappa_{\nu}I_{\nu}\tag{Rad.Transf}\\
        &\TDof{s}=\mu\PDof{r}+\frac{1-\mu^2}{r}\PDof{\mu}\\
        &\mu\TDy{x}{I_{\nu}(\mu,x)}=\j_{\nu}(x)-\kappa_{\nu}(x)I_{\nu}(\mu,x)\tag{plane p.}\\
        &\mu\TDy{\tau_{\nu}}{I_{\nu}(\mu,\tau_{\nu})}=I_{\nu}(\mu,\tau_{\nu})-S_{\nu}(\tau_{\nu})\\
        &\TDof{\tau}[\exp{-\frac{\tau}{\mu}}I]=-\exp{-\frac{\tau}{\mu}}\frac{S}{\mu}\tag{Int.factor}\\
        &\Rightarrow I(\tau,\mu)=\exp{-\frac{(\tau_0-\tau)}{\mu}}I(\tau_o,\mu)+\int_{\tau}^{\tau_0}\exp{-\frac{(t-\tau)}{\mu}}\frac{S(t)}{\mu}\,dt\\
        &\vec{F}_{\nu}=\int d\Omega I(\vec{x},t;\hat{n},\nu)\hat{n}=(\int I_{\nu}n_xd\Omega,\int I_{\nu}n_yd\Omega,\int I_{\nu}n_zd\Omega)\\
        &n_x=\sqrt{1-\mu^2}\cos{\phi},n_y=\sqrt{1-\mu^2}\sin{\phi},n_z=\mu
    \end{align*}
            
        \end{column}
        \begin{column}{0.4\textwidth}
            \begin{itemize}
                \item Photon momentum: $\frac{E}{c}$
                \item Mass emission coeff. $j(\theta)$ (\si{\erg\per\second\per\gram}): Put into $\theta$ - $dI=j(\theta)\rho\,ds$
                \item Absorption coeff $\kappa$: Taken out of $\theta$ - $dI=-\kappa\rho I(\theta)\,ds$
                \item Optical depth: $d\tau_{\nu}=-\kappa_{\nu}\rho\,dz$, $\tau_{\nu}(z)=\tau_{\nu,0}-\int_{z_0}^z\kappa_{\nu}\rho\,dz$ if $z_0$ is true surf. $\tau_{\nu,0}=0$
                \item BlackBody: $B_{\nu}$: BlackBody Intensity, $\pi B_{\nu}$: BlackBody Flux
                \item First Moment - Eddington Flux: $H_{\nu}=\frac{1}{4\pi}\int I_{\nu}\cos{\theta}\,d\Omega=\frac{1}{2}\int_{-1}^1I_{\nu}\mu\,d\mu=\frac{F_{\nu}}{4}$
                \item Mean intensity(0-th moment of rad-field over angles): $J_{\nu}=\frac{1}{4\pi}\int\,d\Omega I(\vec{x},t;\hat{n},\nu)$. PP: $J_{\nu}=\frac{1}{2}\int_{-1}^1I(z,t;\mu,\nu)d\mu$.
                \end{itemize}
        \end{column}
    \end{columns}
    
\end{frame}

\begin{frame}{BlackBody}
    
\end{frame}

\begin{frame}{Radiation Transfer in far interior and Diffusion analogy}
    \begin{columns}[T]
        \begin{column}{0.5\textwidth}
            \begin{align*}
                &I_{\nu}^{out}(\tau_{\nu},\mu)=\int_{\tau_{\nu}}^{\infty}[B_{\nu}(\tau_{\nu})+\TDy{\tau_{\nu}}{B_{\nu}}(t-\tau_{\nu})]\exp{-\frac{(t-\tau_{\nu})}{\mu}}\frac{dt}{\mu}\\
                    &=\int_0^{\infty}[B_{\nu}(\tau_{\nu})+\TDy{\tau_{\nu}}{B_{\nu}}\mu u]\exp{-u}\,du\\
                    &\RightarrowI(\tau,\mu\geq0)=B(\tau)+\mu(\PDy{\tau}{B})_{\tau}\\
                    &H_{\nu}=\frac{F_{\nu}}{4}=\frac{1}{2}\int_{-1}^1\mu I_{\nu}\,d\mu=\frac{1}{3}\TDy{\tau_{\nu}}{B_{\nu}}\\
                    &=-\frac{1}{3}\frac{1}{\kappa_{\nu}}\TDy{\tau_{\nu}}{B_{\nu}}=-\frac{1}{3\kappa_{\nu}}\TDy{T}{B_{\nu}}\TDy{x}{T}
            \end{align*}
        \end{column}
        \begin{column}{0.5\textwidth}
            \begin{itemize}
                \item Diffusion approx: $\tau_{\nu}\gg1$
                    \item Solution in far interior:
                        \begin{align*}
                            &J_{\nu}(\tau_{\nu})=B_{\nu}(\tau_{\nu})+\frac{1}{3}(\PtwoDy{\tau_{\nu}}{B_{\nu}})+\ldots\\
                            &H_{\nu}(\tau_{\nu})=\frac{1}{3}\PDy{\tau_{\nu}}{B_{\nu}}+\frac{1}{5}\partial^3\ldots\\
                            &K_{\nu}(\tau_{\nu})=\frac{1}{3}\PDy{\tau_{\nu}}{B_{\nu}}+\frac{1}{5}\PDy{\tau_{\nu}}{B_{\nu}}+\ldots\\
                            &\Rightarrow J_{\nu}=3K_{\nu}=B_{\nu}, H_{\nu}=\frac{1}{3}\PDy{\tau_{\nu}}{B_{\nu}}
                        \end{align*}
            \end{itemize}
        \end{column}
    \end{columns}
    Integrating over freq:
    \begin{align*}
        &F=-\frac{4\pi}{3}[\int_0^{\infty}\chi_{\nu}^{-1}\PDy{\tau_{\nu}{B_{\nu}}}\,d\nu]\TDy{r}{T}\tag{int $H_{\nu}$ over freq.}\\
        &\vec{F}=-K_R\nabla T,K_R=\frac{4\pi}{3\chi_R}\TDy{T}{B}=\frac{4}{3}c\lambda_RaT^3\\
        &\chi_R^{-1}=\frac{\int_0^{\infty}\chi_{\nu}^{-1}\TDy{T}{B_{\nu}}\,d\nu}{\int_0^{\infty}\TDy{T}{B_{\nu}}\,d\nu}
        &\chi=\kappa+\sigma
    \end{align*}
\end{frame}

\subsection{Energies}

\begin{frame}{Q-valore, Mass excess - PP chain}
    \begin{columns}[T]
        \begin{column}{0.5\textwidth}
\begin{itemize}
    \item $m_{atom}(A,Z)=m_{nuc}(A,Z)+Zm_e-B_e(Z)$, $\massexcess{}=(m_{atom}-Am_u)c^2$, $Q_{I\to F}=\sum_Im_Ic^2-\sum_Fm_Fc^2=\sum_I\massexcess{I}-\sum_F\massexcess{F}$, $B(Z,A)=(Zm_p+Nm_n-m_{nuc})c^2$, $m_{nuc}=Zm_p+Nm_n-\Delta m$, $m_uc^2=\SI{931.494}{\mega\ev}$.
    \item $^{17}O+p\to\alpha+^{14}N$:
        \begin{align*}   
        &Q=\massexcess{^{17}O}+\massexcess{^1H}-\massexcess{^{14}N}-\massexcess{\alpha}\\
        &=[\num{-808.81}+\num{7288.97}-\num{2863.42}-\num{2424.92}]\si{\kilo\ev}\\
        &=m_{^{17}O}+m_{1^H}-m_{^{14}N}-m_{\alpha}
    \end{align*}
    \item $p+p\to^2H\APelectron+\Pnue$:
        \begin{align*}
            &Q=(2*m_{^1H}-m_{^2H})c^2\\
            &=2*\massexcess{^1H}-\massexcess{^2H}\tag{contains $2m_ec^2$}\\
            &=2*\SI{7288.97}{\kilo\ev}-\SI{13135.72}{\kilo\ev}=\SI{1442.22}{\kilo\ev}\\
            &Q=[2\overbrace{(m_p+m_e)}^{m_{^1H}}-(m_d+m_e)]c^2\\
            &=[(m_p+m_p-m_d-m_e)+2m_e]c^2
        \end{align*}
    \end{itemize}
        \end{column}
        \begin{column}{0.5\textwidth}
            \begin{figure}[!ht]\includegraphics[trim={0.0cm 0cm 0.0cm 0},clip, keepaspectratio,height=0.9\textheight]{mass-excess}\label{fig:mass-excess}
			\end{figure}
        \end{column}
    \end{columns} 
\end{frame}

\begin{frame}{Binding Energies}
    \begin{columns}[T]
        \begin{column}{0.5\textwidth}
            \begin{align*}
                &m_uc^2=\SI{931.494}{\mega\ev}\\
                &m_{nuc}=Zm_p+Nm_n-\Delta m\\
                &B(Z,A)=(Zm_p+Nm_n-m_{nuc})c^2\\
                &2d\to\alpha:\\
                &\frac{B(d)}{A}=\SI{1.112}{\mega\ev},\frac{B(\alpha)}{A}=\SI{7.074}{\mega\ev}\\
                &Q=\SI{28.296}{\mega\ev}-\SI{2.224}{\mega\ev}-\SI{2.224}{\mega\ev}\\
                &=\SI{23.85}{\mega\ev}\tag{En. release}
            \end{align*}
        \end{column}
        \begin{column}{0.5\textwidth}
            \begin{figure}[!ht]\includegraphics[trim={0.0cm 0cm 0.0cm 0},clip, keepaspectratio,height=0.9\textheight]{binding-energy}\label{fig:binding-energy}
			\end{figure}
        \end{column}
    \end{columns}
\end{frame}

\subsection{EOS}

\begin{frame}{Gas perfetto, elettroni degeneri}
    \begin{columns}[T]
        \begin{column}{0.5\textwidth}
            \begin{align*}
                &P_g=P_I+P_e=\frac{\rho}{\mu}RT=nKT\\
                &\frac{d\rho}{\rho}=\frac{dP}{P}-\frac{dT}{T}+\frac{d\mu}{\mu}\\
                &\mu=\frac{1}{\exv{n_H}X+\exv{n_{He}}+\exv{n_Z}Z}\tag{massa media free part}\\
                &\invers{\mu}=\sum_iX_i\frac{1+f_i}{A_i}\\
                &\mu_0=\frac{1}{X+\frac{Y}{4}+\frac{Z}{\exv{A}}}\\
                &\mu_e\approx(\frac{1}{1}X+\frac{2}{4}Y)^{-1}=\frac{2}{1+X}\tag{full ion.}\\
                &u=\frac{1}{\rho}\sum\int f^{(0)}(p_i)\frac{p_i^2}{2m_i}d^3p_i\tag{internal ener. per unit mass}\\
                &=\frac{3}{2}\frac{P}{\rho}=\frac{3}{2}\frac{RT}{\mu}
            \end{align*}
        \end{column}
        \begin{column}{0.5\textwidth}
            Degenerazione elettronica
            \begin{align*}
                &f(\vec{p})=\left\{\begin{array}{l}
                        \frac{8\pi p^2}{h^3}: p\leq p_F\\
                        0: p>p_F\\
                \end{array}\\\right.
                    &n_edV=dV\int_0^{p_F}\frac{8\pi}{h^3}p^2dp=\frac{8\pi}{3h^3}p_F^3\,dV\\
                    &\Rightarrow p_F\propro n_e^{\frac{1}{3}}, E_F=\frac{p_F^2}{2m_e}\propto n_e^{\frac{2}{3}}\\
                    &P_e=\int \frac{d\Omega_s}{4\pi}\int_0^{\infty}f(p)v(p)p\cos^2{\theta}dp\\
                    &=\frac{8\pi}{3h^3}\int_0^{p_F}p^3v(p)dp=\frac{8\pi c}{3h^3}\int_0^{p_F}\frac{\frac{p}{m_ec}dp}{\sqrt{1+\frac{p^2}{m_e^2c}}}\\
                    &=\frac{\pi m_e^4c^5}{3h^3}f(x)=\left\{\begin{array}{l}
                            x\ll1: \frac{8\pi m_e^4c^5}{15h^3}x^5=\frac{2}{3}U_e\\
                            x\gg1: \frac{2\pi m_e^4c^5}{3h^3}x^4=\frac{1}{3}U_e\\
                    \end{array}\right.
            \end{align*}
        \end{column}
    \end{columns}
    \begin{columns}[T]
        \begin{column}{0.55\textwidth}
    \begin{align*}
                    &U_e=\int_0^{p_F}f(p)E(p)d^3p=\frac{8\pi}{h^3}\int_0^{p_F}E(p)p^2dp=\frac{\pi m_e^4c^5}{3h^3}g(x)\\
                    &x=\frac{p_F}{m_ec}, x\ll1:\left\{\begin{array}{l}
                            f(x)\approx \frac{8}{5}x^5\\
                            g(x)\approx \frac{12}{5}x^5\\
                    \end{array}\right.,\  x\gg1:\left\{\begin{array}{l}
                            f(x)\approx 2x^4\\
                            g(x)\approx 6x^4\\
                    \end{array}\right.
    \end{align*}
        \end{column}
        \begin{column}{0.45\textwidth}
            \begin{align*}
            &P_e= \left\{\begin{array}{l}
                    x\ll1: \frac{1}{20}(\frac{3}{\pi})^{\frac{2}{3}}\frac{h^2}{m_e}n_e^{\frac{5}{3}}\\
                            x\gg1: (\frac{3}{\pi})^{\frac{1}{3}}\frac{hc}{8}n_e^{\frac{4}{3}}\\
                    \end{array}\right.\\
            &\left\{\begin{array}{l}
                            \frac{1}{20}(\frac{3}{\pi})^{\frac{2}{3}}\frac{h^2}{m_em_u^{\frac{5}{3}}}(\frac{\rho}{\mu_e})^{\frac{5}{3}}=\num{1.0036e13}(\frac{\rho}{\mu_e})^{\frac{5}{3}}\si{\cgs}\\
                            (\frac{3}{\pi})^{\frac{1}{3}}\frac{hc}{8m_u^{\frac{4}{3}}}(\frac{\rho}{\mu_e})^{\frac{4}{3}}=\num{1.2435e15}(\frac{\rho}{\mu_e})^{\frac{4}{3}}\si{\cgs{}}\\
                    \end{array}\right.
                \end{align*}
        \end{column}
    \end{columns}
    
\end{frame}

\subsection{CrossSection}

\begin{frame}{Nuclear elastic scattering: \schr{} eq., partial waves}
    \begin{columns}[T]
        \begin{column}{0.55\textwidth}
            \begin{align*}
                &\psi=f(r)Y_l^m(\theta,\phi), L=\vecp{r}{p}=\frac{\hbar}{i}\vecp{r}{\nabla}\\
                &-\frac{\hbar^2}{2\mu}\nabla^2\psi+V(r)\psi=E\psi\\
                &-\frac{\hbar^2}{2\mu}=-\frac{\hbar^2}{2\mu}(\PtwoDof{r}+\frac{2}{r}\PDof{r})+\frac{L^2}{2\mu r^2}=\frac{p_r^2}{2\mu}+\frac{L^2}{2\mu r^2}\\
                &-\frac{\hbar^2}{2\mu}\TtwoDy{r}{\chi_l}+[\frac{l(l+1)\hbar^2}{2\mu r^2}+V(r)-E]\chi_l=0\tag{$f_l=\frac{\chi_l(r)}{r}$}\\
                &\psi(\vec{r})=N[\exp{i\scap{k}{r}}+f(\theta)\frac{\exp{ikr}}{r}]\tag{beam+target, $r\to\infty$}\\
                &\exp{i\scap{k}{r}}\to\exp{ikz}:u_l^{f.p.}=(kr)j_l(kr)\tag{free p.: Sol $j_l(kr)$ sph. Bessel}\\
                &\xrightarrow{\to\infty}\sin{(kr-\frac{l\pi}{2})},u_{l=0}^{f.p.}=\sin{(kr)}\\
                &\psi_T^{f.p.}=\exp{ikz}=\sum_{l=0}^{\infty}\underbrace{(2l+1)i^l}_{c_l}j_l(kr)P_l(\cos{\theta})\\
                &\xrightarrow{\to\infty}\frac{1}{2kr}\sum_{l=0}^{\infty}(2l+1)i^{l+1}[\exp{-i(kr-\frac{l\pi}{2})}-\exp{i(kr-\frac{l\pi}{2})}]P_l(\cos{\theta})\\
                &u_l=\sin{(kr-\frac{l\pi}{2}+\delta_l)}\tag{$r\to\infty$: $V=0$ same eq.}\\
                &\psi_T=\exp{ikz}+f(\theta)\frac{\exp{ikr}}{kr}\\
                &\xrightarrow{r\to\infty}\sum_{l=0}^{\infty}(2l+1)i^l\exp{i\delta_l}\frac{\sin{(kr-\frac{l\pi}{2}+\delta_l)}}{kr}P_l(\cos{\theta})\\
                &=\frac{1}{2kr}\sum_l^{\infty}(2l+1)i^{l+1}[\exp{-i(kr-\frac{l\pi}{2})}-\exp{2i\delta_l}\exp{i(kr-\frac{l\pi}{2})}]P_l(\cos{\theta})
            \end{align*}
        \end{column}
        \begin{column}{0.45\textwidth} 
            \begin{figure}[!ht]\includegraphics[trim={0.0cm 0cm 0.0cm 0},clip, keepaspectratio,width=0.9\textwidth]{nuclear-potential}\label{fig:nuclear-potential}
			\end{figure}
            \begin{align*}
                &R\approx1.4(A_1^{\frac{1}{3}}+A_2^{\frac{1}{3}})\SI{e-13}{\cm}\\
                &P=\psi^*\psi,j=vP\tag{particle-den, current-d.}\\
                &j_b=N^2v_b, j_s=v_sN^2|f(\theta)|^2 \frac{1}{r^2},dF=r^2d\Omega\\
                &\frac{N_e^{d\Omega}}{t}=\TDy{\sigma}{\Omega}\frac{N_b/t}{A}N_t\,d\Omega\tag{emitted}\\
                &\TDy{\sigma}{\Omega}=\frac{j_{et}\,dF}{j_b\,d\Omega}=\frac{j_sr^2}{j_b}=|f(\theta)|^2,j_{et}=\frac{N_{et}^{d\Omega}}{dF}
            \end{align*}
        \end{column}
    \end{columns}
\end{frame}

\begin{frame}{Elastic Scattering CrossSection}
    \begin{columns}[T]
        \begin{column}{0.7\textwidth}
            \begin{align*}
                &f(\theta)\frac{\exp{ikr}}{kr}=\psi_T-\psi_T^{f.p.}=\frac{1}{2kr}\sum_l(2l+1)i^{l+1}[\exp{i(kr-\frac{l\pi}{2})}(1-\exp{2i\delta_l})]P_l(\cos{\theta})\\
                &\exp{i \frac{\pi l}{2}}=\cos{\frac{\pi l}{2}}+i\sin{\frac{\pi l}{2}}=i^l, \exp{i\delta}\sin{\delta}=\frac{i}{2}(1-\exp{2i\delta}):\\
                &f(\theta)=\frac{1}{k}\sum_l(2l+1)\exp{i\delta_l}\sin{\delta_l}P_l(\cos{\theta})\\
                &(\TDy{\Omega}{\sigma})_{el}=f^*(\theta)f^*(\theta)=\frac{1}{k^2}|\sum_l(2l+1)\sin{\delta_l}P_l(\cos{\theta})|^2\\
                &\int_{d\Omega}P_l(\cos{\theta}P_{l'}(\cos{\theta}))\,d\Omega=\frac{4\pi}{2l+1}\delta_{ll'}:\\
                &\sigma_{el}=\sum_l\sigma_{el,l}=\sum_l \frac{\pi}{k^2}(2l+1)|1-\exp{2i\delta_l}|^2=\frac{4\pi}{k^2}(2l+1)\sin^2{\delta_l}
            \end{align*}
        \end{column}
        \begin{column}{0.3\textwidth}
        \end{column}
    \end{columns}
    \begin{columns}[T]
        \begin{column}{0.6\textwidth}
            \begin{align*}
                &(\TDy{\Omega}{\sigma})_{el,0}=\frac{1}{k^2}\sin^2{\delta_0},\sigma_{el,0}=\frac{4\pi}{k^2}\sin^2{\delta_0}\\
                &\delta_l\to\delta_l+\sigma_l:\tag{charged particles}\\
                &f(\theta)=\frac{i}{2k}\sum_l(2l+1)[1-\exp{2i(\delta_l+\sigma_l)}]P_l(\cos{\theta})\\
                &=\frac{i}{2k}\sum_l(2l+1)(1-\exp{2i\sigma_l})P_l(\cos{\theta})\tag{Rutherford s.}\\
                &+\frac{i}{2k}\sum_l(2l+1)\exp{2i\sigma_l}(1-\exp{2i\delta_l})P_l(\cos{\theta})\tag{N+C}
            \end{align*}
        \end{column}
        \begin{column}{0.4\textwidth}
            \begin{align*}
                &1-\exp{2i(\delta_l+\sigma_l)}\\
                &=(1-\exp{2i\sigma_l})+\exp{2i\sigma_l}(1-\exp{2i\delta_l})
            \end{align*}
        \end{column}
    \end{columns}
    
\end{frame}

\begin{frame}{Reaction CrossSection}
    \begin{columns}[T]
        \begin{column}{0.5\textwidth}
            \begin{align*}
                &\int_{\Omega}j_T\,d\Omega=0\tag{elastic}\\
                &\sigma_{re}=\frac{r^2}{j_b}\int j_T\,d\Omega\tag{non elastic: net current of particles}\\
                &j_b=\frac{\hbar}{2mi}[\exp{-ikz}(\exp{ikz}ik)-\exp{-ikz}(-ik)\exp{ikz}]=\frac{\hbar k}{m}\\
                &j_T=\frac{\hbar}{4mkr^2}[|\sum_l(2l+1)i^{l+1}\exp{\frac{il\pi}{2}}P_l(\cos{\theta})|^2\\
                &-|\sum_l(2l+1)i^{l+1}\exp{2i\delta_l}\exp{\frac{-il\pi}{2}}P_l{\cos{\theta}}|^2]\\
                &\sigma_{re,l}=\frac{\pi}{k^2}(2l+1)(1-|\exp{2i\delta_l}|^2)\geq0\\
                &|\exp{2i\delta_l}|^2=1\tag{If $\delta_l$ real: only elastic-sc}\\
                &\sigma_{el,l}^{max}=\frac{4\pi}{k^2}(2l+1), \sigma_{re,l}=0\tag{max elastic-c.s.}\\
                &\sigma_{re,l}^{max}=\sigma_{el,l}=\frac{\pi}{k^2}(2l+1)\tag{max reaction-c.s.}
            \end{align*}
        \end{column}
        \begin{column}{0.5\textwidth}
            \begin{itemize}
                \item  $\psi_T$ represents elastic scattering w.f.
                    \item $j=\frac{\hbar}{2mi}(\psi^*\TDy{r}{\psi}-\TDy{r}{\psi^*}\psi)$
                \end{itemize}
            \begin{figure}[!ht]
                \includegraphics[trim={0.0cm 0cm 0.0cm 0},clip, keepaspectratio,width=0.9\textwidth]{scattering-reaction-cs}\label{fig:scattering-reaction-cs}
			\end{figure}
        \end{column}
    \end{columns}
\end{frame}

\begin{frame}{Transmission prob., phase shift and resonance phenomenon: T defined only in 1D}
    \begin{columns}[T]
        \begin{column}{0.75\textwidth}
        \begin{align*}
            &\TtwoDy{r}{u}+\frac{2m}{\hbar^2}[E-V(r)]u=0: \TtwoDy{r}{u}+\hat{k}u=0\tag{$l=0$, V const}\\
            &u=\alpha\exp{i\hat{k}r}+\beta\exp{-i\hat{k}r}, r<R_0: K^2=\frac{2m}{\hbar^2}(E+V_0), r>R_0: k^2=\frac{2m}{\hbar^2}E\\
            &u_{in}=A'\exp{iKr}B'\exp{-iKr}, u_{in}(0)=0=A'+B'\Rightarrow u_{in}=A\sin{(Kr)}\\
            &u_{out}=C'\exp{ikr}+D'\exp{-ikr}=\tag{$\sin{(x\pm y)}=\sin{x}\cos{y}\pm\cos{x}\sin{y}$}\\
            &C''\sin{(kr)}+D''\cos{(kr)}=C[\sin{(kr)}\cos{\delta_0}+\cos{(kr)}\sin{\delta_0}]=C\sin{(kr+\delta_0)}\\
            &\exp{-i\omega t}\tag{temporal evol: B' propagate negative x, C',D' reflected/moving toward $R_0$}\\
            &j_{tr}=v_{in}|B'|^2, j_{refl}=v_{out}|C'|^2,j_{inc}=v_{out}|D'|^2: T=\frac{j_{tr}}{j_{inc}}=\frac{v_{in}|B'|^2}{v_{out}|D'|^2}=\frac{K|B'|^2}{k|D'|^2}\\
            &\left.\begin{array}{l}&A'\exp{iKR_0}+B'\exp{-iKR_0}=C'\exp{ikR_0}+D'\exp{-ikR_0}\\&\frac{K}{k}(A'\exp{iKR_0}-B'\exp{-iKR_0})=(C'\exp{ikR_0}-D'\exp{-ikR_0})\\
            \end{array}\right\}
            \tag{cont cond}\\
            &\frac{K}{k}(-B'\exp{-iKR_0})=B'\exp{-iKR_0}-2D'\exp{-ikR_0}: \frac{B'}{D'}=2 \frac{\exp{-ikR_0}}{\exp{-iKR_0}}\frac{k}{k+K},A'=0\\
            &T=\frac{K|B'|^2}{k|D'|^2}=4 \frac{kK}{(k+K)^2}=4 \frac{\frac{2m}{\hbar^2}\sqrt{(E+V_0)E}}{[\sqrt{\frac{2m}{\hbar^2}(E+V_0)}+\sqrt{\frac{2m}{\hbar^2}E}]^2}\\
            &\left.\begin{array}{l}&A\sin{(KR_0)}=C\sin{(kR_0+\delta_0)}\\&AK\cos{(KR_0)}=Ck\cos{(kR_0+\delta_0)}\\
            \end{array}\right\}\Rightarrow \frac{1}{K}\tan{(KR_0)}=\frac{1}{k}\tan{(kR_0+\delta_0)}
        \end{align*}
        \end{column}
        \begin{column}{0.25\textwidth}
            \begin{figure}[!ht]
            \includegraphics[trim={0.0cm 0cm 0.0cm 0},clip, keepaspectratio,width=0.8\textwidth]{pot-shift}\label{fig:pot-shift}
			\end{figure}
        \end{column}
    \end{columns}
    \begin{align*}
        &\delta_0=-kR_0+\arctan{[\frac{k}{K}\tan{(KR_0)}]}=-\frac{\sqrt{2mE}}{\hbar}R_0+\arctan{[\sqrt{\frac{E}{E*V_0}}\tan{(\frac{\sqrt{2m(E+V_0)}}{\hbar}R_0)}]}\\
        &
    \end{align*}
\end{frame}

\begin{frame}{Resonances}
    Squarin and addin continuity conditions:
    \begin{columns}[T]
        \begin{column}{0.6\textwidth}
        \begin{align*}
            &\frac{|A|^2}{|C|^2}=\frac{k^2}{k^2+[K^2-k^2]\cos^2{(KR_0)}}=\frac{E}{E+V_0\cos^2{(\frac{\sqrt{2m(E+V_0)}}{\hbar}R_0)}}\\
            &\cos^2{KR_0}=0: KR_0=(n+\frac{1}{2})\pi: K=\frac{(n+\frac{1}{2})\pi}{R_0}=\frac{2\pi}{\lambda_{in}}\\
            &\lambda_{in}=\frac{2R_0}{(n+\frac{1}{2})}=\frac{R_0}{(\frac{n}{2}+\frac{1}{4})}\tag{$\lambda_{in}$ wavelength in interior}\\
            &E_n=\frac{\hbar^2}{2m}\frac{\pi^2}{R_0^2}(n+\frac{1}{2})^2-V_0\tag{resonance energy: $\frac{n}{2}+\frac{1}{4}$ wavelength fit into interior}
        \end{align*}
        \begin{columns}[T]
            \begin{column}{0.45\textwidth}
            \begin{figure}[!ht]
            \includegraphics[trim={0.0cm 0cm 0.0cm 0},clip, keepaspectratio,width=0.9\textwidth]{wavefunction-phaseshift}\label{fig:wavefunction-phaseshift}
			\end{figure}
            \end{column}
            \begin{column}{0.55\textwidth}
            \end{column}
        \end{columns}
    \end{column}
        \begin{column}{0.4\textwidth}
            \begin{figure}[!ht]
            \includegraphics[trim={0.0cm 0cm 0.0cm 0},clip, keepaspectratio,width=0.9\textwidth]{phaseshift-res}\label{fig:phaseshift-res}
			\end{figure}
            At resonance energies $E_i$ prob finding particle inside $r<R_0$ is at maximum; each res shift phase $\delta_0$ by some amount
        \end{column}
    \end{columns}
\end{frame}

\begin{frame}{Resonance in square barrier potential}
                \begin{align*}
                    &u_I=A'\sin{(Kr)},u_{II}=C\exp{-\kappa r}+D\exp{\kappa r}, u_{III}=F'\sin{(kr+\delta_0')},\Delta=R_1-R_0\\
                    &\delta_0=-kR_1+\arctan{[\frac{k}{\kappa}\frac{\sin{(KR_0)}(\exp{-\kappa\Delta}+\exp{\kappa\Delta})+\frac{K}{\kappa}\cos{(KR_0)}(\exp{\kappa\Delta}-\exp{-\kappa\Delta})}{\sin{(KR_0)}(\exp{\kappa\Delta}-\exp{-\kappa\Delta})+\frac{K}{\kappa}\cos{(KR_0)}(\exp{-\kappa\Delta}-\exp{\kappa\Delta})}]}\\
                    &\frac{|F'|^2}{|A'|^2}=\sin^2{KR_0}+(\frac{K}{k})^2\cos^2{(KR_0)}+\sin^2{(KR_0)}\sinh^2{(\kappa\Delta)}[1+(\frac{\kappa}{k})^2]+\cos^2{(KR_0)}\sinh^2{(\kappa\Delta)}[(\frac{K}{\kappa})^2\\
                    &+(\frac{K}{k})^2]+\sin{(KR_0)}\cos{(KR_0)}\sinh{(2\kappa\Delta)}[(\frac{K}{\kappa})+(\frac{K}{\kappa})(\frac{\kappa}{k}^2)]
                \end{align*}
    \begin{columns}[T]
        \begin{column}{0.33\textwidth}
            \begin{figure}[!ht]
            \includegraphics[trim={0.0cm 0cm 0.0cm 0},clip, keepaspectratio,width=0.9\textwidth]{resonance-squarebarrier}\label{fig:resonance-squarebarrier}
			\end{figure}
        \end{column}
        \begin{column}{0.33\textwidth}
            \begin{figure}[!ht]
            \includegraphics[trim={0.0cm 0cm 0.0cm 0},clip, keepaspectratio,width=0.9\textwidth]{wavefunction-resonance-barrier}\label{fig:wavefunction-resonance-barrier}
			\end{figure}
        \end{column}
        \begin{column}{0.33\textwidth}
            \begin{figure}[!ht]
            \includegraphics[trim={0.0cm 0cm 0.0cm 0},clip, keepaspectratio,width=0.9\textwidth]{A2_F2-barrier}\label{fig:A2_F2-barrier}
			\end{figure}
        \end{column}
    \end{columns}
\end{frame}

\begin{frame}{Scattering: square barrier potential transmission (1D)}
    \begin{columns}[T]
        \begin{column}{0.7\textwidth}
            \begin{align*}
                &(u_I)_{R_0}=(u_{II})_{R_0}, (\TDy{x}{u_I})_{R_0}=(\TDy{x}{u_{II}})_{R_0}\\
                &(u_{II})_{R_1}=(u_{III})_{R_1}, (\TDy{x}{u_{II}})_{R_1}=(\TDy{x}{u_{III}})_{R_1}\\
                &A\alpha\exp{iKR_0}+B\alpha^*\exp{-iKR_0}=\exp{-\kappa(R_1-R_0)}(F\beta\exp{ikR_1}+G\beta^*\exp{-ikR_1})\\
                &A\alpha^*\exp{iKR_0}+B\alpha\exp{-iKR_0}=\exp{\kappa(R_1-R_0)}(F\beta^*\exp{ikR_1}+G\beta\exp{-ikR_1})\\
                &\alpha=1+i \frac{K}{\kappa}, \beta=1+i\frac{k}{\kappa}, \Delta=R_1-R_0\\
                &B[\alpha^*\beta^*\exp{\kappa\Delta}-\alpha\beta\exp{-\kappa\Delta}]=G[(\beta^*)^2-\beta^2]\exp{-i(kR_1-KR_0)}=-2i \frac{k}{\kappa}G\exp{-i(kR_1-KR_0)}\\
                &T=\frac{K|B|^2}{k|G|^2}=\frac{4Kk/\kappa^2}{|\alpha^*\beta^*\exp{\kappa\Delta}-\alpha\beta\exp{-\kappa\Delta}|^2}\\
                &=\frac{Kk}{(k+K)^2+(\kappa^2+K^2+k^2+K^2k^2/\kappa^2)\sinh^2{\kappa\Delta}}\\
                &\frac{1}{T}=\frac{1}{\sqrt{E(E+V_0)}}[(2E+V_0+2 \sqrt{E(E+V_0)})+\\
                &+(E+V_0+V_1+\frac{E(E+V_0)}{V_1-E})\sinh^2{(\sqrt{\frac{2m}{\hbar^2}(V_1-E)}\Delta)}]\\
                &\kappa\Delta=\frac{\sqrt{2m(V_1-E)}}{\hbar}(R_1-R_0)\gg1\tag{low bombarding E/Thick barrier}\\
                &|\alpha^*\beta^*\exp{\kappa\Delta}-\alpha\beta\exp{-\kappa\Delta}|^2\approx|\alpha^*\beta^*\exp{\kappa\Delta}|^2\Rightarrow T\approx4 \frac{\sqrt{E(E+V_0)}(V_1-E)}{V_1(V_0+V_1)}\exp{-2\kappa(R_1-R_0)}\\
                &T\approx\exp{-\frac{2}{\hbar}\sqrt{2m(V_1-E)}(R_1-R_0)}\tag{Strict: neutral, $l=0$ scatt.}
            \end{align*}
        \end{column}
        \begin{column}{0.25\textwidth}
            \begin{figure}[!ht]
                \includegraphics[trim={0.0cm 0cm 0.0cm 0},clip, keepaspectratio,width=0.8\textwidth]{nuclear-square-barrier}\label{fig:nuclear-square-barrier}
                \includegraphics[trim={0.0cm 0cm 0.0cm 0},clip, keepaspectratio,width=0.8\textwidth]{transcoeff-squarebarrier}\label{fig:transcoeff-squarebarrier}
			\end{figure}
            1D radial wave-function:
            \begin{align*}
                &u_I=A\exp{iKx}+B\exp{-iKx}\\
                &u_{II}=C\exp{-\kappa x}+D\exp{\kappa x}\\
                &u_{III}=F\exp{ikx}+G\exp{-ikx}\\
                &T=\frac{j_{trans}}{j_{inc}}=\frac{K|B|^2}{k|G|^2}\\
                &A=0\tag{no wave from left}\\
                &F=0
            \end{align*}
        \end{column}
    \end{columns}
    
\end{frame}

\subsection{Opacity}

\subsection{Stellar model}

\subsection{Stars Formation}

\begin{frame}{Phenomenology}
    \begin{itemize}
        \item GMC: $A_V(\si{\mag}=2$)$, $n_T=\SI{100}{\per\cubic\cm}$, $L=\SI{50}{\parsec}$, $T=\SI{15}{\kelvin}$, $M=\SI{e5}{\solarmass}$
            \item OB associtiation correlated with GMC: obs \SI{2.6}{\milli\meter} line of $^{12}C^{16}O$.
            \item Dense core/Bok Globules: $A_V(\si{\mag}=10$)$, $n_T=\SI{e4}{\per\cubic\cm}$, $L=\SI{0.1}{\parsec}$, $T=\SI{10}{\kelvin}$, $M=10\msun{}$ - Collapse of dense core into protostars
            \item Virial: $\frac{1}{2}\PtwoDy{t}{I}=-\frac{GM^2}{R}$ - $\Omega=-\frac{GM^2}{R}$, $I\approx MR^2$ quindi $t_{ff}\propto\sqrt{\frac{R^3}{GM}}\approx\sqrt{\frac{3\pi}{32G\rho}}$: $t_{ff}\approx\SI{7e6}{\year}(\frac{M}{\num{e5}\msun{}})^{-\frac{1}{2}}(\frac{R}{\SI{25}{\parsec}})^{\frac{3}{2}}$
            \item Birthline - Protostar becomes detectable in visible - Locus in HDR of pre-MS stars with protostellar radius: Surface $L,T_e$ are set by infalling materials dynamics, R is determined by internal structure: $\tkh{}\approx\frac{GM^2}{RL}$, $\dot{R}\approx-\frac{R_*}{\tkh{}}$, $\dot{L}\propto-\frac{L_*}{\tkh{}}$
            \item Dust and $H_2$ formation: $20\msun{}\si{\per\square\cm}$ column density needed to shield $H_2$ from UV (obs: $100\msun{}\si{\per\square\cm}$) - $T\approx \SIrange{10}{20}{\kelvin}$ - thermal radiation by dust increases cooling efficiency - dust increases $H_2$ formation: density of $H_2$ measured indirect. via coll. exc. of other species or decreses of \SI{21}{\cm} line abs/em from depletion of atomic H - \item Inside-out collapse: denser inner regions - $t_{ff}\propto\invers{\rho}$
                \item Rotation: angular momentum problem
        \end{itemize}
\end{frame}

\begin{frame}{Isothermal cloud structure}
    \begin{align*}
        &-\frac{1}{\rho}\nabla P-\nabla\phi_g=0, P=\rho a_T^2\Rightarrow \ln{\rho}+\frac{\phi_g}{a_T^2}=\const{}:\rho=\rho_c\exp{-\frac{\phi_g}{a_T^2}}\\
        &\xi=\sqrt{\frac{4\pi G\rho_c}{a_T^2}}r, \psi=\frac{\phi_g}{a_T^2}\Rightarrow \frac{1}{\xi^2}\TDof{\xi}(\xi^2\TDy{\xi}{\psi})=\exp{-\psi}\\
        &\xi\gg1: \frac{\rho}{\rho_c}\to \frac{2}{\xi}\Rightarrow \psi=\ln{\frac{\xi^2}{2}}, \rho(r)=\frac{a_T^2}{2\pi Gr^2},\rho_0=\frac{P_0}{a_T^2}, r_0=\frac{\xi_0}{\sqrt{\frac{4\pi G\rho_c}{a_T^2}}}\tag{SIS: large distance}\\
        &M=4\pi\int_0^{r_0}\rho r^2\,dr= 4\pi\rho_c(\frac{a_T^2}{4\pi G\rho_c})^{\frac{3}{2}}\int_0^{\xi_0}\exp{-\psi}\xi^2\,d\xi\\
        &\omega^2=k^2a_T^2-4\pi G\rho_0\Rightarrow \lambda_J=\sqrt{\frac{\pi a_T^2}{G\rho_0}}=\SI{0.19}{\parsec}\sqrt{\frac{T}{\SI{10}{\kelvin}}}(\frac{n_{H_2}}{\SI{e4}{\per\cubic\cm}})^{-\frac{1}{2}}: M_J=M_{BE}=\frac{\overbrace{m_1}^{1.18} a_T^4}{P_0^{\frac{1}{2}}G^{\frac{3}{2}}}\tag{G. stability}
    \end{align*}
\end{frame}

\begin{frame}{Protostars: structure}
    
\end{frame}

\begin{frame}{Protostars: deuterium thermostat}
    
\end{frame}

\subsection{Model computation}

\subsection{Sun model and helioseism}

\subsection{Composition: initial params and evolution}

\begin{frame}{programma per questa sezione}
    \begin{itemize}
        \item Def Z etc composizione solare, BBN: 210216
        \item Structural deps on composition: 20210315
        \item Composizione e opacit\'a/T: 20210326, 20210329, 20210330, 20210426
        \item CNO burning
        \item processi s
        \item produzione elementi pesanti:20210420
        \item Zams: variazione con comp: 20210503
        \item Helium to metal enrichment $\frac{\Delta Y}{\Delta Z}$, broadening observed MS
        \item Dredge-up: 20210507
        \item diffusione: 20210507
        \item Deps tip rgb on comp: 20210512
        \item Deps on chem of vertical/horizontal method: 20210517
        \item R param deps on Y: 20210521
        \item DredgeUpII: 20210521
        \item DredgeUpIII: 20210524
    \end{itemize}
\end{frame}

\subsection{Polytropic}

\subsection{Homologous stars}

\begin{frame}{H.models}
\begin{itemize}
    \item  Homologus star:they have the same $\frac{T}{T_c}$, $\frac{P}{P_c}$, $\frac{\rho}{\rho_c}$ when expressed in terms of $\frac{r}{R}$.
\end{itemize}
    
\end{frame}

\subsection{Grandezze Fondamentali}\linkdest{succo}

\begin{frame}{Stime Euristiche: Tempo scala, Pressione e Temperatura centrali}
    \begin{columns}[T]
        \begin{column}{0.5\textwidth}
            \begin{align*}
                &\PDy{m}{P}=-\frac{Gm}{4\pi r^4}\Rightarrow \frac{P_0-P_c}{M}\approx \frac{2G(M/2)^2}{4\pi(R/2)^4}\\
                &\Rightarrow P_c\approx \frac{2GM^2}{\pi R^4}\\
                &\rho\xrightarrow{\text{ideal g.}}\frac{\mu}{R}\frac{P}{T}\Rightarrow T_c\approx \frac{P_c}{\rho_c}\frac{\mu}{R}\\
                &=P_c\frac{\mu}{R}\underbrace{\frac{\bar{\rho}}{\rho_c}}_{\approx0.01-0.03}\Rightarrow T_c<\frac{8}{3}\frac{G\mu}{R}\frac{M}{R}
            \end{align*}
        \end{column}
        \begin{column}{0.5\textwidth}
            \begin{align*}
                &f_P=-\PDy{m}{P}\,dm\\
                &f_g=-\frac{g\,dm}{4\pi r^2}=-\frac{Gm}{r^2}\frac{dm}{4\pi r^2}\\
                &\frac{dm}{4\pi r^2}\PtwoDy{t}{r}=f_P+f_g\Rightarrow \frac{1}{4\pi r^2}\PtwoDy{t}{r}=-\PDy{m}{P}-\frac{Gm}{4\pi r^4}\\
                &|\PtwoDy{t}{r}|\to \frac{R}{\tau_{ff}}\Rightarrow \frac{R}{\tau_{ff}}\approx g\Rightarrow\tau_{ff}\approx\sqrt{\frac{R}{g}}\\
                &\to \frac{R}{\tau_{expl}^2}\Rightarrow \frac{R}{\tau_{expl}^2}=\frac{P}{\rho R}=4\pi r^2\PDy{m}{P}=\PDy{m}{P}/\rho\\
                &\Rightarrow\tau_{expl}\approx R\sqrt{\frac{\rho}{P}}
            \end{align*}
        \end{column}
    \end{columns}
R\end{frame}

\begin{frame}{Masse limite}
    \begin{itemize}
        \item $M_{up}$: Highest *-mass at which \Pelectron-degeneracy prevent C-ignition in CO core. Depends strongly on chem. composition - $M_{up}\approx8\msun{}$ at solar metallicity.
    \end{itemize}
\end{frame}

\subsection{Equazioni struttura: Energy transport}

\begin{frame}{Radiative transport}
    \begin{columns}[T]
        \begin{column}{0.5\textwidth}
            \begin{align*}
                &I(\theta)\,d\Omega=cu(\theta)\,d\Omega\\
                &u=\int^{4\pi}u(\theta)\,d\Omega=\frac{1}{c}\int I(\theta)\,d\Omega\tag{Energy density}\\
                &J(\vec{r},\nu,t)=\invers{(4\pi)}\int I(\vec{r},\hat{n},\nu,t)\,d\Omega\tag{Mean Intensity}\\
                &P_r=\frac{1}{3}\int_0^{\infty}\frac{h\nu}{c}cn(\nu)\,d\nu=\frac{1}{3}u\tag{rad Press}\\
                &P_r=\frac{1}{c}\int I(\theta)\cos^2{\theta}\,d\Omega\tag{Rad Press}\\
                &=\int^{4\pi}\frac{I(\theta)\cos{\theta}}{c}\cos{\theta}\,d\Omega\\
                &=\frac{2\pi}{c}\int_0^{2\pi}I(\theta)\cos{\theta}\sin{\theta}\,d\theta\\
                &H=\int I(\theta)\cos{\theta}\,d\Omega\\
                &=2\pi\int_0^{\pi}I(\theta)\cos{\theta}\sin{\theta}\,d\theta\tag{Net En. Flux polar dir}
            \end{align*}
        \end{column}
        \begin{column}{0.4\textwidth}
            Pressione di radiazione: radiazione contenuta in parallelepipedo di superficie unitaria lunghezza c in direzione $\theta$ (l'asse polare forma con la radiazione un angolo $\theta$) quindi la sezione d'urto geometrica in direzione polare contiene fattore $\cos{\theta}$, la proiezione del momento rispetto alla direzione polare ha un altro $\cos{\theta}$.
        \end{column}
    \end{columns}
\end{frame}

\begin{frame}{Flusso di energia proporzionale al gradiente termico}
\begin{block}{Diffusion Approx: Stellar Interior Near TE}
    \begin{columns}[T]
        \begin{column}{0.5\textwidth}
    \begin{align*}
                &U=aT^4\\
                &''\vec{j}=-D\nabla n''\tag{diffusion}\\
                &D=\frac{1}{3}vl_p\\
                &\vec{F}_{\nu}=-D_{\nu}\nabla U_{\nu}\\
                &D_{\nu}=\frac{1}{3}cl_{\nu}=\frac{c}{3\kappa_{\nu}}\rho
            \end{align*}
        \end{column}
        \begin{column}{0.5\textwidth}
            \begin{align*}
                &I(\theta)=I_0+I_1\cos{\theta}+\ldots\\
                &u=\frac{4\pi}{c}I_0\\
                &H=\frac{4\pi}{3}I_1\\
                &P_r=\frac{4\pi}{3c}I_0
            \end{align*}
        \end{column}
    \end{columns}
    
\end{block}

\begin{block}{Momentum transfer Rad-Mat}
    \begin{columns}[T]
        \begin{column}{0.6\textwidth}
    \begin{align*}
        &dp=\frac{dF_{Rad}}{c}=\frac{F_{Rad}}{c}\frac{dr}{l}\\
        &\TDy{r}{P_{Rad}}=-\frac{\kappa\rho}{c}F_{Rad}\\
        &-\frac{F_{rad}(\nu)}{c}\kappa_{\nu}\rho\,dr=\frac{4\pi}{3c}\TDy{r}{B_{\nu}(T)}\,dr\tag{L-grad T}
    \end{align*}
        \end{column}
        \begin{column}{0.4\textwidth}
            Flux of photons through volume matter at r, flux of energy $F_{rad}$. $dp$ momentum transfered from photons to volume element, $l$ photon mean free path: $\invers{l}=\kappa\rho$, $dp$ opposite of change of $dP_{Rad}$ of pressure exerted by photons over dr    
        \end{column}
    \end{columns}
    
\end{block}
\end{frame}

\subsection{Pre-MS}

\begin{frame}{Pre-MS}
    \begin{itemize}
        \item Lithium but not berylium burn s at bottom of convective zone for less massive then $1.3\msun{}$.
    \end{itemize}
\end{frame}

\subsection{Sub-Giant, RGB}

\begin{frame}{Sub-Giant, RGB}
    \begin{itemize}
        \item Tip-RGB: He-ignition ($3\alpha$)
        \item Hertzsprung-gap: int/massive stars move blue to red at const L - $\tkh{}$ few milion years.
        \item Low mass: ignite He in \Pelectron-degenerate core: $M_*\leq2.3\msun{}$
        \item $(\frac{M_c}{M_*})_{SC}=0.37(\frac{\mu_{env}}{\mu_{core}})^2$ (typical $0.37(\frac{0.6}{1.3})^2$) core mass approx $10\%$ total mass.
        \item Low mass: $L-M_{cHe}$ relation since L is provided by H-burning shell whose thermal properties are determined by He-core radius and mass.
        \item Low-mass: first dredge-up: As envelope expands convection goes deeper - Surf abundance of He increases monot.
        \item Low mass: RGB Bump. Convection receeds as H-burning shell moves outward and when encounters chemical disc. at max depth convection star have peculiar behaviour in HDR ($L_H\propto\mu^7$)
        \item Low mass-rgb: due to low density of convective envelope thermal gradient is largely superadiabatic.
        \item Low-mass: Thermal runaway - helium flash: He ignited when $M_{cHe}\approx0.48-0.5\msun{}$ at $T\approx\SI{e8}{\kelvin}$ in shell around deg core (ie when $T_{ion}$ \SI{e8}{\kelvin}??).
        \item Low-mass: as X exhaust in central regions max of $\epsilon_H$ move off-center.
        \item Low-mass - He-flash: $10^{10\lsun}$ absorbed by ND layers above: expansion and convection but mixing with H-burning shell is prevented - many flash to remove core degeneracy: $\tau_{flash}\approx\SI{e6}{\year}$, $5\%$ He converted into C.
        \item Location of RGB on HDR - Deps on params affecting size of convective envelope (as Hayashi track). Highly Z dependent: Z indicator for galaxies/star clusters. Computation shows (\xdiminuisce{M_*},\xdiminuisce{T_e}) - Changes in chemical composition causes changes in low T opacity: \xaumenta{Y} opacity decreases so extension of convective envelope decreases and star star is hotter. Metallicity is the param that most affect morphology of rgb: \xaumenta{Z}, \xdiminuisce{T_e}.
        \item RGB-phase transition: Luminosity deps on $M_{cHe}$ at Tip of rgb - for $M_*\leq1.8\msun$ L approx const (as $M_{cHe}$ at tip const: same level of electron degeneracy require same $M_{cHe}$ to ignite He) then decreases with increasing stellar mass as degeneracy of the core is at lower level (higher T, lower density), then increases again.
        \item Luminosity RGB bump: (\xaumenta{M_*},\xaumenta{L_{bump}}); (\xaumenta{Z},\xdiminuisce{L_{bump}}) opacity increases, convection goes deeper, H-burning shell find earlier chem disc; (\xaumenta{Y},\xaumenta{L_{bump}}): $L_H\propto\mu^7$ prevails; (\xaumenta{\alpha_{ml}},\xaumenta{L_{bump}}) decreases thermal gradient.
        \item Deps on params of tip-rgb:
            \item Low-mass star in rgb phase: mass loss causes structure readjust to new mass
        \end{itemize}
\end{frame}

\subsection{ZAHB e HB}

\begin{frame}{ZAHB e HB}
    \begin{itemize}
        \item He-burning - $3\alpha$ (\SI{1.2e8}{\kelvin}), ($^{16}O$ $^{20}Ne$ $^{24}Mg$ $^{28}Si$). $\frac{Q}{m(^{12}C)}=\SI{5.9e17}{\erg\per\gram}$ ($10\%$ di quella per H-burning) - $Q=\SI{7.27}{\mega\ev}$, $\tau_H\approx 100\tau_{He}$.
        \item low mass-ZAHB: L drops 1co.m. from tip rgb as core heating/expanding causes H-burning shell to cools down.
        \item low-mass star ZAHB: model where He is burned in chemical homogeneous core and H-burning in shell, He enriched by first dredge-up, C production during He-flashes.
        \item low mass - ZAHB morphology in HDR: spread in $T_e$ due to spread in $M_{env}$ - the higher the envelope's mass the cooler the star, slightly oblique as \xaumenta{\epsilon_H} (L is fixed by $M_{cHe}$ then by $M_e$), \xaumenta{M_e} - \SIrange{3500}{4000}{\kelvin}-$M_e\approx$\SIrange{e-4}{0.4}{\solarmass}.
        \item HB Brightness standard candles for popII stars: L fixed by $M_{cHe}$.
        \item ZAHB deps on composition: \xaumenta{Y_{in}}, (\xaumenta{\mu},\xaumenta{T}) \xdiminuisce{M_{cHe}}, \xaumenta{L_H}, \xdiminuisce{L_{cHe}}: $L^{ZAHB}$ approx const - blue (low mass envelope) part of ZAHB becomes fainter, R part brighter (massive envelope has more efficient shell). Increase in Z at fixed Y yield fainter ZAHB: \xaumenta{Z_{in}} -\xdiminuisce{M_{cHe}^{Flash}}/\xaumenta{\kappa}: \xdiminuisce{L^{ZAHB}}($L_H$ more efficient: less degenerate core)/\xdiminuisce{T_e^{ZAHB}}
        \end{itemize}
\end{frame}

\subsection{Equazioni da ricordare}

\begin{frame}{Pressione di radiazione}
Tutte le volte che un atomo emette/assorbe un fotone perde/guadagna quantit\'a di moto e dato che un atomo emette in maniera isotropa il momento netto \'e nullo una volta mediato su molte emissioni.
I processi di assorbimento non sono isotropicamente distribuiti dato il flusso uscente di energia per $cm^2$ per sec F: solo una frazione $\kappa$ del flusso di momento $\frac{F}{c}$ \'e assorbita dalla materia. Il trasferimento da parte della radiazione di momento alla materia per $cm^3$ per sec, cio\'e la forza esercitata dalla radiazione \'e $\kappa H \frac{1}{c}$.
Un elemento di volume $dS\,dr$ subisce per effetto dell'assorbimento della radiazione una variazione d'impulso $dq$, nel caso un fotone venga assorbito la variazione del flusso uscente \'e $dF<0$.
The distribution of photons over over different quantum states with energies $\epsilon_k=\hbar\omega_k$ (large volume $\omega_k\to\omega$)
\begin{align*}
\overline{n_k}=\frac{1}{\exp{\frac{\hbar\omega}{KT}}-1}
\end{align*}
Moltiplicando il numero di stati nel dato range di frequenze per la distribuzione di Plank (numero di occupazione) ottengo il numero di fotoni e l'energia radiativa nel range di frequenza
\begin{align*}
&dN_{\omega}=\frac{V}{\pi^2c^3}\frac{\omega^2\,d\omega}{\exp{\frac{\hbar\omega}{KT}}-1}\\
&dE_{\omega}=\frac{V\hbar}{\pi^2c^3}\frac{\omega^3\,d\omega}{\exp{\frac{\hbar\omega}{KT}}-1}
\end{align*}
\end{frame}

\begin{frame}{Gradiente per trasporto radiativo nell'interno stellare}
\begin{align*}
&dq=-(n_{\nu}\,dSc\,dt)*(\kappa\rho\,dr)*\frac{h\nu}{c}&\intertext{Il primo termine \'e il numero di fotoni pasanti per superficie $dS$ in tempo $dt$, il secondo \'e la probabilit\'a d'assorbimento attraverso spessore $dr$, il terzo \'e la quantit\'a di moto di ogni fotone.}\\
&dP_r=\TDy{S}{F}=\TDof{S}\TDy{t}{q}\\
&=-\int \,d\nu n_{\nu}c\kappa_{\nu}\rho\,dr\frac{h\nu}{c}\\
&F_{\nu}=n_{\nu}ch\nu,\\
&\TDy{r}{P(Rad)}=-\int\,d\nu\frac{F(Rad)}{c}\kappa_{\nu}\rho&\intertext{In condizioni di LTE posso confrontare $\uparrow$ con}\\
&P_{\nu}=\frac{1}{3}u_{\nu},\ P(Rad)=\frac{1}{3}aT^4&\intertext{e ricavare il gradiente di temperatura necessario per il flusso di energia $F(Rad)$:}\\
&\TDy{r}{T}=-\frac{3\kappa\rho l(r)}{16\pi acT^3r^2}
\end{align*}
\end{frame}

\begin{frame}{* formation}
	contenu...
\end{frame}

\frameinlbftrue
\begin{frame}[fragile]{Struttura di equilibrio}

\begin{itemize}
\item Equilibrio idrostatico: pressione in un mesh \'e il peso della materia sopra per unit\'a di superficie. Stabilit\'a e tempi reazione a perturbazione
\item Pressione radiativa. Una frazione $\kappa$ del flusso di momento $\frac{F}{c}$ \'e assorbita dalla materia (momentum transfer per $cm^3$ per sec): $dq=-(n_{\nu}\,dSc\,dt)*(\kappa\rho\,dr)*\frac{h\nu}{c}$, il primo termine \'e il numero di fotoni pasanti per superficie $dS$ in tempo $dt$, il secondo \'e la probabilit\'a d'assorbimento attraverso spessore $dr$, il terzo \'e la quantit\'a di moto di ogni fotone.
\begin{align*}
&dP_r=\TDy{S}{F}=\TDof{S}\TDy{t}{q}=-\int \,d\nu n_{\nu}c\kappa_{\nu}\rho\,dr\frac{h\nu}{c}\\
&F_{\nu}=n_{\nu}ch\nu,\ \TDy{r}{P(Rad)}=-\int\,d\nu\frac{F(Rad)}{c}\kappa_{\nu}\rho
\end{align*}
\begin{comment}
Un elemento di volume $dS\,dr$ subisce per effetto dell'assorbimento della radiazione una variazione d'impulso $dq$, nel caso un fotone venga assorbito la variazione del flusso uscente \'e $dF<0$
The distribution of photons over over different quantum states with energies $\epsilon_k=\hbar\omega_k$ (large volume $\omega_k\to\omega$) 
\begin{align*}
\overline{n_k}=\frac{1}{\exp{\frac{\hbar\omega}{KT}}-1}
\end{align*}
Moltiplicando il numero di stati nel dato range di frequenze per la distribuzione di Plank (numero di occupazione) ottengo il numero di fotoni e l'energia radiativa nel range di frequenza
\begin{align*}
&dN_{\omega}=\frac{V}{\pi^2c^3}\frac{\omega^2\,d\omega}{\exp{\frac{\hbar\omega}{KT}}-1}\\
&dE_{\omega}=\frac{V\hbar}{\pi^2c^3}\frac{\omega^3\,d\omega}{\exp{\frac{\hbar\omega}{KT}}-1}
\end{align*}
\end{comment}
\end{itemize}

\end{frame}
\frameinlbffalse

\begin{frame}{Rotazione, mass loss}
\begin{itemize}
\item Rotazione. \[\frac{\nabla P}{\rho}=-\nabla\phi+\vec{a}=-\nabla\phi+\Omega^2r_{\perp}=\vec{g}_{eff}\]
If $\nabla\wedge(\Omega^2r_{\perp})=0$: $\phi\to\phi-V$, $V=\int_0^{r_{\perp}}\Omega^2r_{\perp}\,dr_{\perp}$ (OK if $P(\rho)$, politrope, regioni convettive)
\end{itemize}
\end{frame}

\begin{frame}{Stability}

\end{frame}

\begin{frame}{Hayashi Line (HL)}

\end{frame}


\part{Modello Plasma stellare}\linkdest{stellarplasma}
\begin{frame}{this part toc}
    \begin{itemize}
        \item Radiative transfer equation and solution in far interior
        \item Solution of RE in stellar interior
        \item Nuclear Fusion
    \end{itemize}
\end{frame}
\section{From micro-physics to thermodynamics: Equazione di stato per ioni ed elettroni}

%\begin{multicols}{2}%https://newbedev.com/how-to-explicitly-split-long-toc-in-beamer
%   \tableofcontents[currentsection]{cherryframes}
%\end{multicols}

\begin{wordonframe}{Schema fisico/chimico: relazione $\rho(P,T)$ realistiche}
\begin{itemize}
\item schema  chimico: il primo considera atomi e molecole, la cui popolazione per stati eccitati e diversi gradi di ionizzazione \'e ottenuto minimizzando l'energia libera da cui sono ricavate le altre grandezze termodinamiche; utilizzando questo approccio \'e stata ricavata l'equazione di stato MHD (\cite{hummer1988equation})
\item schema fisico: nuclei ed elettroni come costituenti fondamentali interagenti tramite potenziale Coulombiano e trova le soluzione dell'equazione di Schr\"oedinger per un problema a molti corpi, questo approccio, usato per ricavare l'equazione di stato OPAL (\cite{rogers1986occupation})
\end{itemize}
Come illustrato in figura, per entrambe $\Gamma_1\approx\midfrac{5}{3}$ nell'interno solare e maggiori deviazioni si hanno nelle regioni di ionizzazione parziale degli elementi in particolare di idrogeno ed elio.
\end{wordonframe}

\begin{wordonframe}{Relazione tra $P(\rho, T)$: approx zero gas perfetto di atomi completamente ionizzati}
\begin{columns}[T]%
% \hspace{-19pt}\relax
\begin{column}{0.55\textwidth}%
EOS gas perfetto di ioni ed elettroni 
\begin{align*}
    &P_G=P_I+P_e=\frac{\rho}{\mu}\gasconstant{}T=nKT=\frac{\rho}{\mu m_u}KT\tag{$R=N_AK$}\\
&\frac{d\rho}{\rho}=\frac{dP}{P}-\frac{dT}{T}+\frac{d\mu}{\mu}\\
&(\frac{d\rho}{\rho}=\alpha\frac{dP}{P}-\delta\frac{dT}{T}+\phi\frac{d\mu}{\mu})
\end{align*}
Energia interna per unit\'a di massa: somma delle energie traslazionali delle particelle pesate secondo la distribuzione di equilibrio di Maxwell-Boltzmann per grammo di materia
 \begin{align*}
&u=\frac{1}{\rho}\sum_i\int f^{(0)}(\vec{p}_i)\frac{p^2_i}{2m_i}\,d^3p_i\\
&=\frac{3}{2}\frac{P}{\rho}=\frac{3}{2}\frac{\gasconstant T}{\mu}
\end{align*}
\end{column}
 %   \hspace{-22pt}\relax
\begin{column}{0.45\textwidth}
Peso molecolare medio: massa media in amu per particella libera
\begin{align*}
&\mu=\frac{1}{\bar{n}_HX+\bar{n}_{He}Y+\bar{n}_{Z}Z}\\
&\invers{\mu}=\sum_i X_i \frac{1+f_i}{A_i}\\
&\mu_0=\frac{1}{X+\midfrac{Y}{4}+\midfrac{Z}{\bar{A}}},\ \mu_e\approx\frac{2}{1+X}
\end{align*}
con $\bar{n}_i=\frac{1+f_i}{A_i}$ numero medio di particelle libere per unit\'a di massa atomica dovute alla specie i di peso atomico $A_i$ e $f_i$ numero medio di elettroni liberati da ione della specie i; peso atomico medio per ione $\mu_0$ ed elettrone libero (ionizzato) $\mu_e$.
\end{column}
\end{columns}
$f^{(0)}(\vec{p}_i)$: numero di particelle della specie i per unit\'a di volume con impulso in $[\vec{p}_i,\vec{p}_i+d\vec{p}_i]$
\end{wordonframe}

\begin{wordonframe}{Deviazioni dalla legge dei gas perfetti: radiazione e degenerazione elettronica}
\begin{itemize}
	\item Radiazione. Il contributo alla pressione ed energia interna per unit\'a di volume dei fotoni $P_R=\frac{a}{3}T^4$, $u_R=aT^4$, e $P-P_R=\beta P$.
	\item Degenerazione elettronica - Principio di Pauli: non pi\'u di 2 elettroni in volume di spazio delle fasi $h^3$. $n_e$ la densit\'a numerica di \Pelectron, $\psi(P,T)$ il parametro di degenerazione, tale che per $\psi\to-\infty$ si abbia la distribuzione di Boltzmann e per $\psi\to+\infty$ completa degenerazione
	\begin{align*}
	&n_e=\rho N_A\frac{1+X}{2}=\intzi{}\frac{8\pi p^2\,dp}{h^3(\exp{\frac{u_k}{KT}-\psi}+1)}=\frac{8\pi}{h^3}(2m_ekT)\expy{3/2}a(\psi)\\
	&P_e=\beta P-\rho\gasconstant{}(X+\frac{Y}{4}+\frac{Z}{\exv{A_Z}})=\frac{8\pi}{3h^3}\intzi{}p^3v(p)\frac{dp}{1+\exp{\epsilon/kT-\psi}}\\
	&U_e=\frac{8\pi}{h^3}\int_0^{\infty}\frac{p^2\epsilon(p)\,dp}{\exp{(-\psi+\midfrac{\epsilon}{KT})}+1}
	\end{align*}
\end{itemize}
\end{wordonframe}

\begin{frame}{degenerazione completa - NR limit}
    Quantum cell of volume in P.S. $d^3pd^3x=h^3$ each with $g_e=2$: in shell $[p,p+dp]$ there are $\frac{4\pi p^2dpdV}{h^3}$ so $f(p)dpdV\leq \frac{2*4\pi p^2\,dpdV}{h^3}$. Complete degeneracy: $f(p)=\left\{\begin{array}{l}
            \frac{8\pi p^2}{h^3}:\ p\leq p_F\\
            0\\
    \end{array}\right.$, so $n_edV=dV\int_0^{p_F}\frac{8\pi p^2}{h^3},dp=\frac{8\pi}{3h^3}p_F^3dV$. Given $n_e$ we get $p_F\propto n_e^{\frac{1}{3}}$:
    \begin{itemize}
            \item NR - $E_F=\frac{p_F^2}{2m_e}\propto n_e^{\frac{2}{3}}$
            \item R - If $n_e$ such that $p_F$ relativistic: $p=\frac{m_ev}{\sqrt{1-(\frac{v}{c})^2}}$,$E_T=\frac{m_ec^2}{\sqrt{1-(\frac{v}{c})^2}}=m_ec^2\sqrt{1+\frac{p^2}{m_e^2c^2}}$, quindi $\frac{1}{c}\TDy{p}{E_T}=\frac{\frac{p}{m_ec}}{\sqrt{1+\frac{p^2}{m_e^2c^2}}}$ e $E=E_T-m_ec^2$.
        \end{itemize}
    For EOS we need pressure (flux of momentum per unit surf per second: surf $d\sigma$, normal $\vec{n}$).Let's determine number of electrons goin through $d\sigma$ into small solid angle $d\Omega_s$ around dir $\vec{s}$ with $[p,p+dp]$: $f(p)\,dp \frac{d\Omega_s}{4\pi}$ electrons per unit volume with right p and dir, having $f(p)\,dpv(p)\cos{\theta}\,d\sigma \frac{d\Omega_s}{4\pi}$ electrons going through the surface per sec into $d\Omega_s$; total flux of momentum in dir $\vec{n}$: $P_e=\int_{2\pi}\frac{d\Omega_s}{4\pi}\int_0^{\infty}f(p)v(p)p\cos^2{\theta}\,dp=\frac{8\pi}{3h^3}\int_0^{p_F}p^3v(p)\,dp$ ($\frac{4\pi}{3}$ int of $\cos^2{\theta}$ over hemisphere).

x measures importance of rel. effect for electrons with high mom., $x=\frac{p_F}{m_ec}=\frac{\frac{v_F}{c}}{\sqrt{1-(\frac{v_f}{c})^2}}$ or $\frac{v_F^2}{c^2}=\frac{x^2}{1+x^2}$: if $x\ll1$ then $\frac{v_F}{c}\ll1$ (NR), if $x\gg1$ then $\frac{v_F}{c}\approx1$
\begin{align*}
    &x\to0: f(x)\to \frac{8}{5}x^5,\ g(x)\to \frac{12}{5}x^5\Rightarrow P_e=\frac{8\pi m_e^4c^5}{15h^3}x^5=\frac{2}{3}U_e\tag{$x\ll1$}\\
    &P_e=\frac{1}{20}(\frac{3}{\pi})^{\frac{2}{3}}\frac{h^2}{m_e}n_e^{\frac{5}{3}}=\frac{1}{20}(\frac{3}{\pi})^{\frac{2}{3}}\frac{h^2}{m_em_u^{\frac{5}{3}}}(\frac{\rho}{\mu_e})^{\frac{5}{3}}=\num{1.0036e13}(\frac{\rho}{\mu_e})\expy{5/3}\si{\cgs}\tag{EOS comp-deg-NR-\Pelectron}
\end{align*}
\end{frame}

\begin{frame}{EOS complete degenerate electron gas - UR limit}
\begin{align*}
    &P_e=\frac{8\pi c}{3h^3}\int_0^{P_F}\frac{p/(m_ec)}{\sqrt{1+p^2/(m_ec)^2}}=\frac{8\pi c^5 m_e^4}{3h^3}\int_0^x \frac{\xi^4\,d\xi}{\sqrt{1+\xi^2}}\tag{$\xi=\frac{p}{m_ec}$, $x=\frac{p_F}{m_ec}$: $n_e=\frac{\rho}{\mu_em_u}=\frac{8\pi m_e^3c^3}{3h^3}x^3$}\\
    &P_e=\frac{\pi m_e^4c^5}{3h^2}f(x),\ \int_0^x \frac{\xi^4\,d\xi}{\sqrt{1+\xi^2}}=\frac{1}{8}[x(2x^2-3)\sqrt{1+x^2}+3\sinh^{-1}{x}]\\
    &U_e=\int_0^{P_F}f(p)E(p)\,dp=\frac{8\pi}{h^3}\int_0^{P_F}E(p)p^2\,dp=\frac{\pi m_e^4c^5}{3h^3}g(x),\ g(x)=8x^3[\sqrt{x^2+1}-1]-f(x)\\
\end{align*}
\begin{align*}
    &x\to\infty:\ f(x)\to 2x^4,\ g(x)\to6x^4\\
    &\Rightarrow P_e=\frac{2\pi m_e^4c^5}{3h^3}x^4=\frac{1}{3}U_e\tag{$x\gg1$}\\
    &=(\frac{3}{\pi})^{\frac{1}{3}}\frac{hc}{8}n_e^{\frac{4}{3}}=(\frac{3}{\pi})^{\frac{1}{3}}\frac{hc}{8m_u^{\frac{4}{3}}}(\frac{\rho}{\mu_e})^{\frac{4}{3}}=\num{1.2435e15}(\frac{\rho}{\mu_e})\expy{4/3}\si{\cgs}\tag{EOS comp-deg-UR-\Pelectron}
\end{align*}
\end{frame}

\begin{frame}{Partial degeneracy of electron gas}
    \begin{columns}[T]
        \begin{column}{0.65\textwidth}
    For finite T not all \Pelectron will be in cell of lowest poss. moment: most prob. occupation of phase cells of shell $[p,p+dp]$ is determined by $f(p)\,dp\,dV=\frac{8\pi p^2\,dpdV}{h^3}\frac{1}{1+\exp{\frac{E}{KT}-\psi}}$, $\psi$ deg. param.
    \begin{align*}
        &E=\frac{p^2}{2m}\tag{NR}\\
        &n_e=\frac{8\pi}{h^3}\int_0^{\infty}\frac{p^2\,dp}{1+\exp{\frac{E}{KT}-\psi}}=\frac{8\pi}{h^3}(2m_eKT)^{\frac{3}{2}}a(\psi)
    \end{align*}
\end{column}
        \begin{column}{0.35\textwidth}
            \begin{align*}
                &n_e=\frac{8\pi}{h^3}\int_0^{\infty}\frac{p^2\,dp}{1+\exp{\frac{E}{KT}-\psi}}\\
                &P_e=\frac{8\pi}{3h^3}\int_0^{\infty}p^3v(p)\frac{p^2\,dp}{1+\exp{\frac{E}{KT}-\psi}}\\
                &U_e=\frac{8\pi}{h^3}\int_0^{\infty}\frac{Ep^2\,dp}{1+\exp{\frac{E}{KT}-\psi}}
            \end{align*}
        \end{column}
    \end{columns}
    \begin{columns}[T]
        \begin{column}{0.6\textwidth}
            \begin{align*}
                &a(\psi)=\int_0^{\infty}\frac{\eta^2\,d\eta}{1+\exp{\eta^2-\psi}},\ \eta=\frac{p}{\sqrt{2m_eKT}}\Rightarrow \psi=\psi(\frac{n_e}{T^{\frac{3}{2}}})\\
                &\exp{\psi}=\frac{h^3n_e}{2(2\pi m_eKT)^{\frac{3}{2}}}\tag{$\psi\to-\infty$, large T: $a(\psi)$ arb. small, $f(p)\to f_{MB}$}\\
                &\psi=\frac{E_0}{KT}\to\infty:\frac{1}{1+\exp{\frac{E}{KT}-\psi}}=\frac{1}{1+\exp{\psi(\frac{E}{E_0}-1)}}\approx\left\{\begin{array}{l}
                        1:\ E<E_0\\
                        0:\ E>E_0\\
                    \end{array}
            \end{align*}
        \end{column}
        \begin{column}{0.4\textwidth}
            \begin{align*}
                &m_e\,dE=p\,dp, n_e=\frac{\rho}{\mu_em_u}=\tag{NR}\\
                    &=\frac{4\pi}{h^3}(2m_eKT)^{\frac{3}{2}}F_{\frac{1}{2}}(\psi)\tag{moderate $\psi$}\\
                    &P_e=\frac{8\pi}{3h^3}(2m_eKT)^{\frac{3}{2}}KTF_{\frac{3}{2}}(\psi)\\
                    &U_e=\frac{4\pi}{h^3}(2m_eKT)^{\frac{3}{3}}KTF_{\frac{3}{2}}(\psi)=\frac{3}{2}P_e
            \end{align*}
        \end{column}
    \end{columns}
            Moderate $\psi$, UR: $p\to \frac{E}{c}$, $v\to c$ quindi $n_e=8\pi(\frac{KT}{hc})^3F_2(\psi)$, $P_e=\frac{8\pi}{3h^3c^3}(KT)^4F_3(\psi)$
            \begin{equation*}
        &F_{\nu}(\psi)=\int_0^{\infty}\frac{u^{\nu}}{\exp{(u-\psi)}+1}\,du\xrightarrow{\psi\gg1}\frac{\psi^{\nu+1}}{\nu+1}\{1+2[c_2(\nu+1)\nu\psi^{-2}+c_4(\nu+1)\nu(\nu-1)(\nu-2)\psi^{-4}+\ldots]\}\tag{FD Int} 
    \end{equation*}
                    \begin{columns}[T]
                        \begin{column}{0.5\textwidth}
                            \begin{align*}
                                &n_e=\frac{\rho}{\mu_em_u}=\frac{4\pi}{h^3}(2m_eKT)^{\frac{3}{2}}F_{\frac{1}{2}}(\psi)\tag{NR}\\
                                &P_e=\frac{8\pi}{3h^3}(2m_eKT)^{\frac{3}{2}}KTF_{\frac{3}{2}}(\psi),\ U_e=\frac{3}{2}P_e
                            \end{align*}
                        \end{column}
                        \begin{column}{0.5\textwidth}
                            \begin{align*}
                                &n_e=8\pi(\frac{KT}{hc})^2F_2(\psi)\tag{UR}\\
                                &P_e=\frac{8\pi}{3h^3c^3}(KT)^4F_3(\psi)
                            \end{align*}
                        \end{column}
                    \end{columns}
\end{frame}
\begin{wordonframe}{Elettrostatic screening of ions: weak screening}
La principale correzione che tiene conto dell'interazioni tra particelle \'e dovuta alle interazioni coulombiane: influence EOS and nuclear rection rates.

Screening of ion i with charge $Z_ie$ ar $\vec{r_i}$ in NR dilute plasma, motion of screened ions is slow compared to screening particles: continuum static equilibrium charge distribution (\Pelectron, light ions of mean charge $Z_p$)
\begin{equation*}
\nabla^2\phi=4\pi n_ee[\exp{(\frac{e\phi}{kT})}-\exp{-\frac{Z_pe\phi}{kT}}]-4\pi\sum_iZ_ie\delta(\vec{r}-\vec{r_i})
\end{equation*}
Regime di schermaggio debole, $e\phi\ll KT$: $\phi=\sum_i\phi_i$ potenziale attorno a ione pesante isolato:
\begin{align*}
%&\nabla^2\phi=-4\pi e\sum_Z Zn_Z-4\pi e\sum_i Z_i\delta(\vec{r}-\vec{r}_i)\\
&\phi_i=\frac{Z_ie}{r_i}\exp{-\frac{r_i}{r_D}}
&\frac{1}{r_D^2}=\frac{4\pi e^2}{kT}\sum Z^2\overline{n}_Z=\frac{4\pi e^2}{kT}N_A\zeta,\ \zeta=\sum_{i}(Z_i^2+Z_i)\frac{\rho X_i}{A_i}
\end{align*}
\end{wordonframe}

\begin{wordonframe}{Elettrostatic screening of ions: energy pressure correction}
Energy required to assemble uniform shere with charge $Ze$: $U_{ee}=\int_0^{R_Z}\frac{q_r}{r}\,dq=\frac{3}{5}\frac{(Ze)^2}{R_Z}$.
Energy required to assemble uniform cloud of charge $Ze$ around Z-nucleus: $U_{eZ}=-Ze\int_0^{R_Z}\frac{dq}{r}=-\frac{3}{2}\frac{(Ze)^2}{R_Z}$
Le correzioni dovute alle interazioni coulombiane sono dovute a numero sfere ioniche per unit\'a di volume $n_Z=\frac{\rho X_Z}{A_Z}N_0$ with average potential energy per electron $\exv{-e\phi}_Z=-\frac{9}{10}\frac{(Ze)^2}{R_Z}$ that contain Z \Pelectron.
\begin{align*}
&\rho u_c=(\frac{U}{V})_e=\frac{1}{2}\phi(\vec{r})\rho_c(\vec{r})\to \frac{1}{2}\sum_ZeZ\overline{n}_Z\phi_Z=-e^3\sqrt{\frac{\pi\rho}{kT}}(N_A\zeta)\expy{\frac{3}{2}},\ P_c=\frac{1}{3}\rho u_c\\
&E_0=\frac{(U/V)_e}{n_e}=\mu_e\sum_Z\exv{-e\phi}_Z\frac{ZX_Z}{A_Z}\approx-1.3(\mu_e^2\rho)\expy{1/3}[X+0.79Y+\sum_{Z>2}\frac{Z\expy{5/3}X_Z}{A_Z}]
\end{align*}
\end{wordonframe}

\begin{frame}{Equazione di Saha e continuum depression. Ioniozzazione da pressione}
L'equazione di Saha descrive la frazione relativa di ionizzazione
\begin{align*}
&\frac{n_{r+1}}{n_r}n_e=\frac{g_{r+1}}{g_r}f_r(T)\\
&f_r(T)=2\frac{(2\pi m_ekT)\expy{3/2}}{h^3}\exp{-\chi_r/(kT)}
\end{align*}
Saha limitation:
\begin{itemize}
\item LTE: is the case when collision dominate over radiative processes
\item Decreases ionization energy with increasing density: what is called pressure ionizzation is produced by coulomb interaction of bound electron with other electron in the plasma $\chi'_Z=\chi_Z-\frac{Ze^2}{R_D}$
\end{itemize}
\end{frame}

\begin{frame}{Crystallization and Neutronization}
\begin{block}{Crystallization}
Per $\rho$ alta e $T$ bassa (WD interior: $\Gamma_c=\frac{(Ze)^2}{r_{ion}kT}\approx180$) gli ioni formano reticolo quando energia termica uguale interazione coulombiana $\frac{3}{2}kT\approx E_c$ - melting $T_m\approx\frac{Z^2e^2}{\Gamma_ck}(\frac{4\pi\rho}{3\mu_0m_H})\expy{1/3}=\num{2.3e3}Z^2\mu_0\expy{-1/3}\rho\expy{1/3}$
\end{block}
\begin{block}{Neutronization}
If \Pelectron have $E_e>E^*=c^2(m_n-m_p)\approx\SI{1.3}{\mega\ev}$ they combine with protons and form neutrons: if we put $E=E_{kin}+m_ec^2=E_F+m_ec^2=c^2(m_n-m_p)\approx\SI{1.3}{\mega\ev}$ if $E_{kin}<E_F$ l'elettrone non trova stati liberi e il neutrone non decade ($x=\frac{p_F}{m_ec}\approx2.2$).
For $\rho>\SI{4e11}{\gram\per\cubic\cm}$ we have inverse $beta$-decay:nuclei capture electron and become neutron rich - neutron drip
\end{block}
\end{frame}

\section{Stat Phys of photon gas}

    \begin{frame}{Spazio delle fasi del sistema, stati microscopici, evoluzione}%
\begin{itemize}
    \item Macroscopic properties are ensemble average
    \item Macroscopic evolution: flow of density of states in PS (fluid is the large collection of identical systems in same macroscopic state)
    \item Microstate probability: $\#\text{ of states}=\rho(p,q)\,dp\,dq$
    \item Isolated System is in equilibrium iff accessible microstates are equiprobable
\item Ergodic system: trajectory in phase space are deterministic ($\dot{p}_i=-\PDy{q_i}{H}$, $\dot{q}_i=\TDy{p_i}{H}$) and starting from a point will get arb. closer to any other points.
\item Evolution of systems starting inside a small cube in phase space is described with $\TDy{t}{\rho}=\PDy{t}{\rho}+\{\rho,H\}=0$ preserve volume ($\TDof{t}(x+\delta x)=\dot{x}+\delta x\PDof{x}\dot{x}$,$\TDof{t}(p+\delta p)=\dot{p}+\delta p\PDof{p}\dot{p}$ quindi $\dot{V}=\delta x\delta p[\PDof{p}\dot{p}+\PDof{x}\dot{x}]=0$). Per sistemi in equilibrio termodinamico le medie non dipendono esplicitamente dal tempo quindi neanche $\rho(p,q)$: $\{\rho,H\}=0$.
\item Microcanonical ensemble: a particular solution is $\rho(p,q)=\const{}$ (all microstates accessible between $E$,$E+\delta E$, V and N are equiprobable)
\item Canonical ensemble: system in thermal bath at T $\rho(p,q)=\exp{-\frac{H(p,q)}{kT}}$, fixed V, N and T.
\item Gran-canonical ensemble: system in thermal/chemical equilibrium with reservoir, ie fixed T,$\mu$; $\rho(p,q)\propto\exp{-\frac{H(p,q)}{kT}}\exp{+\frac{\mu N}{kT}}$.
\item $d\omega=\frac{d^{3N}p\,d^{3N}q}{(2\pi\hbar)^N}$: $S(U)=k\ln{\Omega(U)}$, $\Omega(U)$ is volume PS $H(p,q)\leq U$
\end{itemize}
\end{frame}

\begin{frame}{Gas di fotoni: Ensemble canonico.}
    \begin{itemize}
        \item Microcanonical (NVE):
        $W=\exp{\frac{S}{k}}$ numero di stati microscopici tra $E$,$E+\Delta E$, $U=\exv{H}=E(S,V,N)$.
    \item Canonical: fixed T, partition function $Z(\beta)=\sum_{\substack{k\text{ microstates}:\\\text{V,N fixed}}}\exp{-\beta E_k}=\sum_{\text{Energies }i}g_i\exp{-\beta E_i}$.
        \begin{align*}
            &\exv{U}=-\PDof{\beta}\ln{Z_{CAN}}=\sum_s\frac{\sum_{n_s}n_sh\nu_s\exp{-\beta\sum_sn_sh\nu_s}}{\sum_{n_s}\exp{-\beta\sum_sn_sh\nu_s}}=-\sum_s\PDof{\beta}\ln{\sum_{n_s}\exp{-\beta n_sh\nu_s}}\\
&=\sum_s\frac{h\nu_s}{\exp{\beta h\nu_s}-1}\xrightarrow{\vec{k}=\frac{2\pi}{L}\vec{n}}\frac{8\pi V}{c^3}\int_0^{\infty}\frac{h\nu^3}{\exp{\frac{h\nu}{kT}}-1}\,d\nu\\
&\frac{U}{V}=\frac{4\sigma}{c}T^4=aT^4
\end{align*}
$F=U-TS$: processo isotermo rev., $\Delta Q$ system, $\Delta W=P\Delta V$ work to world; $\Delta U=\Delta Q-P\Delta V$, $\Delta S=\frac{\Delta Q}{T}$ quindi $P=-\frac{\Delta U-T\Delta S}{\Delta V}\to-\frac{\Delta F}{\Delta V}|_{T,N}$, $F=-\frac{4\sigma}{3c}VT^4=-kT\ln{Z}=KT\sum_{\vec{k},\gamma}\ln{[1-\exp{-\beta\omega\hbar}]}$
    \end{itemize}
\end{frame}


\frameinlbftrue
\begin{frame}{Plank Distro: energy density StefanBoltzmann law}
    \begin{itemize}
        \item Density of states, free wave confined into cube of edge L: $\psi=\frac{1}{\sqrt{V}}\exp{i\scap{k}{x}}$, $k_i=\frac{2\pi}{L}n_i$, $E=\hbar\omega=c\hbar\omega$
            \begin{align*}
                &\sum_{\vec{n}}\to\int\,d^3\vec{n}=\frac{V}{(2\pi)^3}\int\,d^3k=\frac{4\pi V}{(2\pi)^3}\int_0^{\infty}k^2\,dk\\
                &g(E)=\frac{VE^2}{\pi^2\hbar^3c^3}: g(E)\,dE=g(\omega)\,d\omega=\frac{V\omega^2}{\pi^2c^3}\,d\omega\tag{2 for polarizations}
            \end{align*}
            \item Partition function for fixed $\omega$: indep partition function multiplies:
            \begin{align*}
                &Z_{\omega}=\sum_N\exp{-\beta N \hbar\omega}=1+\exp{-\beta\hbar\omega}+\exp{-2\beta\hbar\omega}+\ldots=\frac{1}{1-\exp{-\beta\hbar\omega}}\tag{fixed $\omega$, summing over energies $E=N\hbar\omega$ - N photons}\\
                &\ln{Z}=\int_0^{\infty}g(\omega)\ln{Z_{\omega}}\,d\omega=-\frac{V}{\pi^2c^3}\int_0^{\infty}d\omega\omega^2\ln{(1-\exp{-\beta\hbar\omega})}\tag{Indep partition function mult(log adds)}
            \end{align*}
\item Energy density and flux
            \begin{align*}
                &E=-\PDof{\beta}\ln{Z}=\frac{V\hbar}{\pi^2c^3}\int\,d\omega\frac{\omega^3}{\exp{\beta\hbar\omega}-1}\\
                &\epsilon=\frac{E}{V}=\frac{\pi^2K^4}{15\hbar^3c^3}\tag{Energy Density}\\
                &\frac{\epsilon c}{4}=\sigma T^4\tag{Energy Flux}\\
                &(\frac{1}{4\pi}\int_0^{2\pi}d\phi\int_0^{\frac{\pi}{2}}d\theta\sin{\theta}(c\cos{\theta})=\frac{c}{4})
            \end{align*}
        \item Plank function: $B(\omega,T)=\epsilon(\omega)*\frac{c}{4\pi}=\frac{\hbar\omega^3}{4\pi^3c^2}\frac{1}{\exp{\beta\hbar\omega}-1}$.
    \end{itemize}
\end{frame}
\frameinlbffalse

\section{Opacity}\linkdest{opacitysources}

\subsection{Time-dependent perturbation: EM transition}

\begin{frame}{Time-dep perturbation: EM wave first order perturbation}
\begin{align*}
    &i\hbar\TDy{t}{\psi(t)}=H\psi(t)\\
    &H\psi_n=E_n\psi_n\tag{energy eigenstates}\\
    &\psi(t)=\sum_nc_n\exp{(-\frac{i}{\hbar}E_nt)}\psi_n: c_n=\braket{\psi_n|\psi(t_0)}\exp{(\frac{i}{\hbar}E_nt_0)}\\
    &H=H_0+V(t):\tag{perturb. EM pulse}\\
    &\psi(t)=\sum_nc_n(t)\Exp{(-\frac{i}{\hbar}E_n^{(0)}t)}\psi_n^{(0)}\\
    &c_s(t)\approx1\gg c_k(t): i\hbar\TDy{t}{c_k}=V_{ks}\exp{i\omega_{ks}t}\tag{Pert. short,weak: no mixin in initial state s}\\
    &c_k(+\infty)=-\frac{i}{\hbar}\int_{-\infty}^{+\infty}V_{ks}\exp{i\omega_{ks}t}\,dt\tag{V transient-$|c_k|^2$ prob transition to k}\\
    &H=\frac{[\vec{p}+\frac{e}{c}\vec{A}]^2}{2m}+V_c-e\phi=H_0 +\frac{e\scap{A}{p}}{mc}+\frac{e^2A^2}{2mc^2},\ \phi=0, \nabla\cdot\vec{A}=0, \nabla^2\vec{A}-\frac{1}{c^2}\TtwoDy{t}{\vec{A}}=0\tag{atom-e pert. by emwav-NoSorg}
\end{align*}
Final term gives much smaller effect than second: pertutbing potential for atoms is $V=\frac{e}{mc}\scap{A}{p}$, term in $A^2$ is important for scattering of EM-waves from free \Pelectron (to be considered if $\scap{A}{p}$ gives no first-order transition).
\end{frame}

\begin{frame}{Transition probability: emission/absorption}
    \begin{align*}
        &\vec{A}(\vec{r},t)=\int_{-\infty}^{+\infty}\vec{A}(\omega)\Exp{[-i\omega(t-\frac{\hat{n}\cdot\vec{r}}{c})]}\,d\omega\tag{Decomp. arb pulse into harmonic pw}\\
        &\vec{A}^*(\omega)=\vec{A}(-\omega), \nabla\cdot\vec{A}=0\Rightarrow\hat{n}\cdot\vec{A}(\omega)=0\tag{$\vec{A}$ real, transverse wave}\\
        &V=\frac{e}{mc}\int_{-\infty}^{+\infty}\Exp{-i\omega(t-\frac{\hat{n}\cdot\vec{r}}{c})}\vec{A}(\omega)\cdot\vec{p}\,d\omega: V_{ks}=\frac{e}{mc}\int_{-\infty}^{+\infty}\braket{k|\Exp{(i \frac{\omega}{c}\hat{n}\cdot\vec{r})}\vec{p}|s}\exp{-i\omega t}\vec{A}(\omega)\,d\omega\\
        &c_k(+\infty)=-\frac{ie}{\hbar mc}\int_{-\infty}^{+\infty}\int_{-\infty}^{+\infty}\braket{k|\Exp{i\frac{\omega}{c}\hat{n}\cdot\vec{r}}\vec{p}|s}\cdot\vec{A}(\omega)\exp{i(\omega_{ks}-\omega)t}\,dtd\omega\\
        &=-\frac{2\pi ie}{\hbar mc}\braket{k|\Exp{i \frac{\omega_{ks}}{c}\hat{n}\cdot\vec{r}}\vec{p}|s}\cdot\vec{A}(\omega_{ks}): \hbar\omega_{ks}=E_k-E_s\tag{Absorption at $\omega_{ks}$}\\
        &\tag{Stimulated emission at $\omega_{ks}$: excit $s\to k$}\\
        &\omega_{sk}=-\omega_{ks}, \int\psi_k^*V\psi_s=\int(V\psi_k)^*\psi_s: c_{k\to s}=c_{s\to k}^*\tag{Stimul. emiss. trans $k\to s$,V is Herm.}\\
        &\frac{B_{ks}}{B_{sk}}=\frac{2J_s+1}{2J_k+1}=\frac{g_s}{g_k}\tag{detailed bal(spin, state degeneracy) ratio transitio prob}\\
        &|c_k(+\infty)|^2=\frac{4\pi^2e^2}{\hbar^2m^2c^2}|A(\omega_{ks})|^2|\braket{k|\Exp{i \frac{\omega_{ks}}{c}\hat{n}\cdot\vec{r}}\vec{p}\cdot\hat{e}|s}|^2\tag{transition probability - Polariz $\hat{e}$: $\vec{A}(\omega)=A(\omega)\hat{p}$}
    \end{align*}
\end{frame}

\begin{frame}{Absorption crosssection}
    Mass absorption coeff deps on absorption crosssection which is ratio of number of photons in freq $d\omega$ absorbed per atom to total number in freq $d\omega$ per unit area incident onto atoms; but since radiation is treated classic. we consider average energy absorbed per atom
    \begin{align*}
        &\vec{S}=\frac{c}{4\pi}\vecp{E}{H}, \vec{E}=-\frac{1}{c}\TDy{t}{\vec{A}}, \vec{H}=\nabla\wedge\vec{A}\tag{Poynting's vec: energy flux in pulse}\\
        &\vec{S}=-\frac{1}{4\pi}\int_{-\infty}^{+\infty}-i\omega A(\omega)\hat{e}\Exp{[-i\omega(t-\frac{\hat{n}\cdot\vec{r}}{c})]}\,d\omega\wedge\frac{1}{c}\int_{-\infty}^{+\infty}i\omega'A(\omega')(\hat{n}\wedge\hat{e})\Exp{[-i\omega'(t-\frac{\hat{n}\cdot\vec{r}}{c})]d\omega'}\\
        &=-\frac{\hat{n}}{4\pi c}\iint\,d\omega d\omega' \omega\omega'A(\omega)A(\omega')\Exp{[-i(\omega+\omega')(t-\frac{\hat{n}\cdot\vec{r}}{c})]}\tag{transverse wave $\hat{e}\cdot\vec{n}=0$: $\hat{e}\wedge(\hat{n}\wedge\hat{e})=\hat{n}$}\\
        &E=\int_{-infty}^{+\infty}\vec{S}(t)\cdot\hat{n}\,dt=\frac{1}{c}\int_0^{+\infty}\omega^2|A(\omega)|^2\,d\omega=\int E(\omega)\,d\omega\tag{ener. carried by pulse per unit area}\\
        &\hbar\omega_{ks}|c_k(+\infty)|^2=\frac{4\pi^2e^2E(\omega_{ks})}{\hbar m^2c\omega_{ks}}|\braket{k|\Exp{(i \frac{\omega_{ks}}{c}\hat{n}\cdot\vec{r})\vec{p}\cdot\hat{e}}|s}|^2
    \end{align*}
    Absorption crosssection $\sigma(\omega)=\frac{\text{energy abs }/\text{initial state s at freq $\omega$}}{\text{Inc ener}/\text{unit area at freq $\omega$}}$
\end{frame}

\subsection{BB opacity}

\begin{frame}{Bound-Bound Abs.: finite lifetime \Pelectron state/oscillator strenght perspect.}
    \begin{itemize}
            \item Finite electron state lifetime:
    \begin{columns}[T]
        \begin{column}{0.5\textwidth}
            \begin{align*}
                &\int E(\omega)\sigma(\omega)\,d\omega\tag{pulse energy absorbed}\\
                &\sigma(\omega)=\frac{4\pi^2\alpha}{m^2\omega_{ks}}|\braket{k|\Exp{(i \frac{\omega_{ks}}{c}\hat{n}\cdot\vec{r})}\vec{p}\cdot\hat{e}|s}|^2\delta(\omega-\omega_{ks})\\
                &\tau\Delta E=\tau\Gamma=\hbar\tag{Initial/final \Pelectron states are not inf sharp}\\
                &E_{\frac{1}{2}}=E_k^{(0)}\pm \frac{\hbar}{2\tau}: \Gamma=\frac{\hbar}{\tau}\tag{Full-Width Half Max}\\
                &\sigma(E)=\frac{4\pi^2\alpha}{m^2\omega_{ks}}|\braket{k|\Exp{(i \frac{\omega_{ks}}{c}\hat{n}\cdot\vec{r})}\vec{p}\cdot\hat{e}|s}|^2*\\
                &*\frac{\frac{\Gamma}{2\pi\hbar}}{(\omega-\omega_{ks})^2+(\frac{\Gamma}{2\hbar})^2}\tag{line-abs cross-sec}
            \end{align*}
        \end{column}
        \begin{column}{0.5\textwidth}
            \begin{align*}
                &\int_{\Delta\omega}\sigma(\omega)\,d\omega=\frac{4\pi^2\alpha}{m^2\omega_{ks}}|\braket{k|\Exp{(i \frac{\omega_{ks}}{c}\hat{n}\cdot\vec{r})}\vec{p}\cdot\hat{e}|s}|^2\\
                &\hbar\Delta\omega\approx\hbar, E(\omega)\approx E(\omega_{ks})\\
                &\psi_k(t)\approx\Exp{(-\frac{t}{2\tau})}\psi_k^{(0)}\Exp{(-\frac{i}{\hbar}E_k^{(0)}t)}:\\
                &\int|\psi_k|^2\,dV=\Exp{(-\frac{t}{\tau})}\\
                &E_{op}=-\frac{\hbar}{i}\PDof{t}: \psi(t)=\int_{-\infty}^{+\infty}\phi(E)\Exp{(-\frac{i}{\hbar}Et)}\,dE\\
                &|\phi(E)|^2\tag{probability state has energy E}\\
                &\Rightarrow\phi(E)\propto\int_0^{+\infty}\Exp{[\frac{i}{\hbar}(E-E_k^{(0)})t-\frac{t}{2\tau}]}\\
                &P(E)\,dE=\frac{\hbar}{2\pi\tau}\frac{dE}{(E-E_k^{(0)})^2+(\frac{\hbar}{2\tau})^2}\tag{norm prob}
            \end{align*}
        \end{column}
    \end{columns}
\item Oscillator strengths of transition: for linear harmonic osc parallel to pol $\hat{e}$ $\int_{\Delta\omega}\sigma(\omega)\,d\omega=\frac{2\pi^2e^2}{mc}=\frac{2\pi^2\hbar\alpha}{m}$, lines are ref to this standard mult osc crosssec by o.s. f: $\sigma(\omega)=\frac{2\pi^2e^2}{mc}f_{ks}\frac{\frac{\Gamma}{2\pi\hbar}}{(\omega-\omega_{ks})^2+(\frac{\Gamma}{2\hbar})^2}$ so $f_{ks}=\frac{2}{\hbar m\omega_{ks}}|\braket{k|\Exp{(i \frac{\omega_{ks}}{c}\hat{n}\cdot\vec{r})}\vec{p}\cdot\hat{e}|s}|^2$; interior opacity is rel insens. to uncert. in line widths and shapes.
    implification for eval of matrix-element: wavelenght of incident light is usually long comp. to absorber size, $\lambda=\frac{2\pi c}{\omega_{ks}}\gg\bar{r}$, $\Exp{(i \frac{\omega_{ks}}{c}\hat{n}\cdot\vec{r})}\approx1+i \frac{\omega_{ks}}{c}\hat{n}\cdot\vec{r}+\ldots$ and matrix-el $\braket{k|\vec{p}\cdot\hat{e}|s}+\braket{k|i \frac{\omega_{ks}}{c}(\hat{n}\cdot\vec{r})\vec{p}\cdot\hat{e}|s}$, rapidity conv. $\hbar\omega_{ks}\approx \frac{Z^2e^2}{a_0}$, BE, whereas $\bar{r}\approx \frac{a_0}{Z}$: $\frac{\omega_{ks}\bar{r}}{c}=\frac{Z^2e^2}{\hbar ca_0}\frac{a_0}{Z}=Z\frac{e^2}{\hbar c}=\alpha Z=\frac{Z}{137}$; first term el-dipole (allow), second term if first zero are el-quadrupole/magn-dipole (''forbidd''): only allow. trans are important for stellar opacity (forbidd. important for low density)
        \end{itemize}
\end{frame}

\begin{frame}{Electric dipole transition: $\braket{k|\vec{p}|s}$}
    \begin{columns}[T]
        \begin{column}{0.5\textwidth}
            \begin{align*}
                &[H_0,\vec{r}]=-\frac{i\hbar}{m}\vec{p}\\
                &\braket{k|\vec{p}|s}=\frac{im}{\hbar}\braket{k|H_0\vec{r}-\vec{r}H_0|s}\\
                &=\frac{im}{\hbar}(E_k^{(0)}-E_s^{(0)})\braket{k|\vec{r}|s}=im\omega_{ks}\braket{k|\vec{r}|s}\\
                &\exv{|\braket{k|\vec{r}\cdot\hat{e}|s}|^2}=\frac{1}{3}|\braket{k|\vec{r}|s}|^2\tag{iso $\hat{r}$, $\hat{p}$}
            \end{align*}
        \end{column}
        \begin{column}{0.5\textwidth}
            \begin{align*}
                &\int_{\Delta\omega}\sigma(\omega)\,d\omega=4\pi^2\alpha\omega_{ks}|\braket{k|\vec{r}\cdot\hat{e}|s}|^2\\
                &f_{ks}=\frac{2m\omega_{ks}}{\hbar}|\braket{k|\vec{r}\cdot\hat{e}}|^2
            \end{align*}
        \end{column}
    \end{columns}
    For BB abs k,s are wave functions $\psi_k=R_{klm}Y_{lm}(\theta,\phi)$, for hydrogen-like $V=-\frac{Ze^2}{r}$ and $R_{klm}$ are Laguerre pol. Selection rules for el-dipole transition: $l-l'=\pm1$, $m-m'=0,\pm1$.
    \begin{columns}[T]
        \begin{column}{0.5\textwidth}
            \begin{align*}
                &\frac{B_{ll'}}{B_{l'l}}=\frac{2l'+1}{2l+1}\tag{abs rate per state l/induced-em. rate per state l'}
            \end{align*}
        \end{column}
        \begin{column}{0.5\textwidth}
            \begin{align*}
                &f_{ks}=\frac{2m\omega_{ks}}{3\hbar}|\braket{k|\vec{r}|s}|^2=\frac{2m\omega_{ks}}{3\hbar}(R_{10}^{21})^2\tag{Ly\alpha: $1s\to2p$}\\
                &\hbar\omega_{ks}=1-\frac{1}{2^2}\si{\ryd}=\frac{3}{4}\frac{e^2}{2a_0}, a_0=\frac{\hbar^2}{me^2}
            \end{align*}
        \end{column}
    \end{columns}
\end{frame}

\subsection{BF opacity}

\begin{frame}[allowframebreaks]{BF-absorption: Born Approx}
    \begin{columns}[T]
        \begin{column}{0.65\textwidth}
            \begin{align*}
                &\hbar\omega_{ks}=\chi+\frac{p^2}{2m}\\
                &[\sigma]=\frac{\si{\per\second}}{\si{\per\square\cm\per\second}}\tag{Prob. of \Pphoton abs per u.t./flux \Pphoton}\\
                &dI=-\sigma \frac{N_0\rho}{A}I\,dx, \kappa=\frac{N_0}{A}\sigma\\
                &\sigma\approx\left\{\begin{array}{l}C_KZ^4\lambda^3+b_K\ \lambda<\lambda_K\\C_{L_1}Z^4\lambda^3+b_{L_1}\ \lambda_K<\lambda<\lambda_{L_1}\\
            \end{array}\right.\tag{between shell edges abs increses as $\lambda^3$}\\
            &\frac{hc}{\lambda_K}=\chi_K, C_K=\SI{2.25e-2}{\per\cm}, C_{L_1}=\SI{0.33e-2}{\per\cm}\\
            &
            \end{align*}
            Saha eq.: at moderate density no bound level except if $\chi\geq kT$; since in Rosseland op. most important freq. corr. to few times $KT$ absorption dominate near th. - for photon energy ($E=h\nu=\frac{hc}{\lambda}$) slightly in eccess to threshold emitted electron has low velocity so is influenced by Coulomb potential of ion: for accurate calculations $\psi_k$ has to be coulomb functions (free wave functions describing a particle moving in a Coulomb field). For $\frac{hc}{\lambda}\gg\chi$, but non-rel, boundstate wavefunction is H-like:
            \begin{align*}
                &|C_k(+\infty)|^2\tag{continuum trans prob}\\
                &d\sigma(\omega)=\frac{4\pi^2\alpha}{m^2\omega}|\braket{k|\exp{i \frac{\omega}{c}\hat{n}\cdot\vec{r}}\vec{p}\cdot\hat{e}|s}|^2 \frac{\Delta n}{\Delta \omega}\tag{photoejection into $d\omega$}\\
                &\frac{\Delta n}{\Delta \omega}\tag{number of cont. eig.states in $d\Omega$ in energy band about $E_k=\hbar\omega-\chi=\hbar\omega+E_s$}
            \end{align*}
        \end{column}
        \begin{column}{0.35\textwidth}
            \begin{figure}[!ht]
                \includegraphics[trim={0.0cm 0cm 0.0cm 0},clip, keepaspectratio,width=0.95\textwidth]{BF-abs}\label{fig:BF-abs}
			\end{figure}
        \end{column}
    \end{columns}
    \begin{align*}
        &\psi_k=L^{-\frac{3}{2}}\Exp{i\scap{k}{r}}, E_k=\frac{\hbar^2k^2}{2m}: \Delta n= \frac{L^3\Delta p_x\Delta p_y\Delta p_z}{h^3}=\frac{L^34\pi p^2\,dp}{h^3}, \frac{\Delta n}{\Delta E}=\frac{m^{\frac{3}{2}}\sqrt{E}L^3}{\sqrt{2}\pi^2\hbar^2}
    \end{align*}
    
    \begin{align*}
        &\frac{\Delta n}{\Delta\omega}=\frac{d\Omega}{4\pi}\frac{m^{\frac{3}{2}}\sqrt{E}L^3}{\sqrt{2}\pi^2\hbar^2}\\
        &\psi_s=R_{1s}Y_{00}=\frac{1}{\sqrt{\pi}}(\frac{Z}{a})^{\frac{3}{2}}\exp {-\frac{Zr}{a}}\tag{photoejection from K-shell}\\
        &\TDy{\Omega}{\sigma(\omega)}=\frac{32\alpha\hbar k(\hat{e}\cdot\vec{k})^2}{m\omega}(\frac{Z}{a})^5(\frac{Z^2}{a^2}+q^2)^{-4}, q=\vec{k}-\frac{\omega}{c}\hat{n}\tag{$\hbar q$ momentum transfer - agree with obs for $\hbar\omega\gg\chi$}
    \end{align*}
    \begin{columns}[T]
        \begin{column}{0.7\textwidth}
                $\theta$: photon direction/ejected electron dir,$\psi$:polarization/plane photon and electron momenta
            \begin{align*}
                &(\frac{Z}{a})^2+q^2=(\frac{Z}{a})^2+k^2+(\frac{\omega}{c})^2-2k\frac{\omega}{c}\cos{\theta}=2 \frac{m\omega}{\hbar}(1-\frac{\hbar k}{mc}\cos{\theta}+\frac{\hbar\omega}{2mc^2})\\
                &\frac{\hbar^2k^2}{2m}=\hbar\omega-\chi=\hbar\omega-\frac{Z^2e^2}{2a}\tag{energy conservation}\\
                &\TDy{\Omega}{\sigma(\omega)}=2\alpha k(\hat{e}\cdot\vec{k})^2(\frac{\hbar}{m\omega})^5(\frac{Z}{a})^5(1-\beta\cos{\theta})^{-4}\tag{\Pelectron v is NR, ignore photon energy vs \Pelectron rest-mass en.}\\
                &=2\alpha k^3(\frac{\hbar}{m\omega})^5(\frac{Z}{a})^5 \frac{\sin^2{\theta}\cos^2{\phi}}{(1-\beta\cos{\theta})^4}: \sigma(\omega)\approx \frac{8\pi\alpha}{3}(\frac{Z}{a})^5(\frac{\hbar}{m\omega})^5k^3\\
                &\sigma(\hbar\omega\gg\chi)\approx \frac{2}{3}\frac{\alpha^{\frac{9}{2}}}{\pi^{\frac{5}{2}}a^{\frac{3}{2}}}Z^5\lambda^{\frac{7}{2}}\tag{High freq approx: $k^2\approx \frac{2m\omega}{\hbar}$}\\
                &\lambda^{\frac{7}{2}}\approx\lambda^3\lambda_k^{\frac{1}{2}},\frac{hc}{\lambda_k}=\frac{Z^2e^2}{2a}\Rightarrow Z^5\lambda^{\frac{7}{2}}\propto Z^4\lambda^3\tag{Near edge: wrong approx/right behaviour}
            \end{align*}
        \end{column}
        \begin{column}{0.3\textwidth}
            \begin{figure}[!ht]
                \includegraphics[trim={0.0cm 0cm 0.0cm 0},clip, keepaspectratio,width=0.95\textwidth]{photoejection-kin}\label{fig:photoejection-kin}
			\end{figure}
        \end{column}
    \end{columns}
    
\end{frame}

\begin{frame}{BF absorption: most important frequencies near edge $\hbar\omega\approx\chi$}
    Replacing free-electron wave function by plane wave (Born Approx) is inadequate near threshold where electron wavefunction is strongly perturbed by Coulomb interaction: these are the most important frequencies for stellar opacity problem we have to use exact Coulomb waves for $\psi_k$:
    \begin{align*}
        &\sigma_{BF}=\frac{64\pi^4me^{10}}{3\sqrt{3}ch^6}\frac{Z^4}{n^5}\frac{g(\nu,n,l,Z)}{\nu^3}=\num{2.82e29}\frac{Z^4}{n^5\nu^3}g(\nu,n,l,Z)\si{\square\cm}
    \end{align*}
\end{frame}

\subsection{FF opacity}

\begin{frame}{FF Absorption: Inverse bremsstrahlung}
    \begin{columns}[T]
        \begin{column}{0.65\textwidth}
    \begin{align*}
        &\frac{p_k^2}{2m}=\frac{p_s^2}{2m}+\hbar\omega, H_{tot}=H_{free}+V_c+H_{int}\tag{Heavy Z absorbs negl. kin ener.}
    \end{align*}
    \begin{itemize}
        \item If we take unpert. $H_0=H_f+V_c$, zeroth order wave function $\psi_k^{(0)}$ are coulomb wf - \Pelectron scattering by ion. Perturbation at first order in interaction with em field.
    \item Born Approx: $H_0=H_{free}$, 0th order wf are plane waves, perturbed H is $V_c+V_{int}$. Transition probability propto $|V_{ks}|^2$ if is null $V_{ks}\to\sum_m \frac{V_{km}V_{ms}}{E_s-E_m}$ - virtual transition $s\to m \to k$ (energy don't have to be conserved as m has short lifetime). Two possibility: 1) Electron absorbs photon ($H_{int}$) and its momentum becomes $\vec{k}_m=\vec{k}_s+\frac{\omega}{c}\hat{n}$ then $V_c$ causes scattering transition $\vec{k}_m\to\vec{k}_k$. 2) $V_c$ causes transition $\vec{k}_s\to\vec{k}_m$, such that $H_{int}$ has a momentum conserving matrix element for absorption of photon: $\vec{k}_k=\vec{k}_m+\frac{\omega}{c}\hat{n}$. The sum is on both intermediate states.
        \end{itemize}
        \end{column}
        \begin{column}{0.35\textwidth}
            \begin{figure}[!ht]
                \includegraphics[trim={0.0cm 0cm 0.0cm 0},clip, keepaspectratio,width=0.95\textwidth]{FF-feynman}\label{fig:FF-feynman}
			\end{figure}
        \end{column}
    \end{columns}
    \begin{align*}
        &r=\frac{2\pi}{\hbar}|\sum_m \frac{V_{km}V_{ms}}{E_s-E_m}|^2\rho(E_k)\tag{$\rho$ density of final states - GoldenRules: transition prob per unit time}\\
        &d\sigma_{ff}(Z,\nu,v)=\frac{4\pi Z^2e^6g_{ff}(v,\nu)}{3\sqrt{3}hcm^2v\nu^3}n_e(v)\,dv\tag{$n_e(v)$ \Pelectron density with $[v,v+dv]$ }\\
        &n_e(v)dv=4\pi n_e(\frac{m}{2\pi kT})^{\frac{3}{2}}\Exp{-\frac{mv^2}{2kT}}v^2\,dv, \bar{\sigma}_{ff}(Z,\nu)=\int_0^{\infty}d\sigma(Z,\nu,v):\\
        &\bar{\sigma}_{ff}(Z,\nu,T)=\num{3.69e8}\frac{Z^2n_e\bar{g}_{ff}}{T^{\frac{1}{2}}\nu^3}\si{\cm}\Rightarrow\kappa_{ff}(\nu)=\sum \frac{X_zN_0}{A_z}\bar{\sigma}_{ff}(Z,\nu)
    \end{align*}
\end{frame}

\begin{frame}{Rosseland Mean for FF opacity - Kramer's op.: $\propto\rho T^{-3.5}$}
    FF opacity dominate total opacity in important regions of $\rho$-T plane
    \begin{columns}[T]
        \begin{column}{0.65\textwidth}
            \begin{align*}
                &u=\frac{h\nu}{kT}: \frac{1}{\kappa}=\frac{2\pi k^4}{ac^3h^3}\int_0^{\infty}\frac{u^4\exp{2u}}{\kappa(u)[\exp{u}-1]^3}\,du\tag{all true abs}\\
                &\kappa_{ff}(\nu)\approx K\nu^{-3}=K(\frac{h}{kT})^3u^{-3}, \exv{g_{ff}}\approx\bar{g}_{ff}(u_{max}(=7))\tag{ignoring deps g on $\nu$}\\
                &\Rightarrow \frac{1}{\kappa}=\frac{1}{K\exv{\nu^{-3}}}, \exv{\nu^{-3}}\approx[197(\frac{kT}{h})^3]^{-1}:(h\nu)_{eff}\approx5.82kT\\
                &\exv{\sigma_{ff}}=\num{2.07e-25}\frac{Z^2n_e}{T^{3.5}}\bar{g}_{ff}(u=7)\si{\square\cm}=\num{1.25e-1}\frac{Z^2\bar{g}_{ff}}{\mu_e}\frac{\rho}{T^{3.5}}\si{\square\cm}\\
                &\approx\num{6.25e-2}(1+X)Z^2\bar{g}_{ff}\rho T^{-3.5}\si{\square\cm}\\
                &\Rightarrow\exv{\kappa_{ff}}=\num{7.53e22}\frac{\rho}{\mu_eT^{3.5}}\sum_Z \frac{Z^2\bar{g}_{ff}(Z,u=7)X_Z}{A_Z}\si{\square\cm\per\gram}
            \end{align*}
        \end{column}
        \begin{column}{0.35\textwidth}
            \begin{figure}[!ht]
                \includegraphics[trim={0.0cm 0cm 0.0cm 0},clip, keepaspectratio,width=0.95\textwidth]{rhoTplane-kappa}\label{fig:rhoTplane-kappa}
			\end{figure}
        \end{column}
    \end{columns}
    
\end{frame}

\subsection{Electron opacity}

\begin{frame}{Scattering of photons by free electrons}
    Scattering opacity by free lectrons so small that is dominated by BF opacity until ionization is complete and by FF opacity until T is suff. great: in high T range electron scattering is main opacity source. In quantum electrodyn. vector potential $\vec{A}$ is expanded linearly  in terms of creation/destruction operators for single photons - the term $\scap{p}{A}$ in BF, FF, BB gives rise to single photon absorption - but matrix element for scattering must involve destruction of initial photon/creation of scattered: must be quadratic in vector potential $\vec{A}$.
            \begin{align*}
                &H_{int}=\frac{e}{mc}\scap{p}{A}+\frac{e^2}{2mc^2}\scap{A}{A}=H_{int}^{(1)}+H_{int}^{(2)}: \braket{k|H_{int}|s}=\sum_m \frac{\braket{k|H^{(1)|m}\braket{m|H^{(1)}_{int}|s}}}{E_s-E_m}+\braket{k|H^{(2)}_{int}|s}\propto e^2
            \end{align*}
            \begin{columns}[T]
        \begin{column}{0.65\textwidth}
            second order part of $H^{(1)}_{int}$ and hitherto neglected $H_{int}^{(2)}$ - for photon's $E\ll m_ec^2$ answer con be obtained from classical EM. When free electron is invested by EM radiation accelerates so radiates in direction other than incident (scattering), for photon energy $\ll m_ec^2$ scattering radiation has same freq as incident. Power radiated into solid angle $d\Omega$ at angle $\psi$ to direction of accel $\vec{a}$ for NR particles: $dP=\frac{e^2}{4\pi c^3}a^2\sin^2{\psi}\,d\Omega$, scattered waves are pol in plane of $\vec{a}$ and in viewer dir. ($\sin^2{\psi}=1-\sin^2{\theta}\cos^2{\phi}$)
            \begin{align*}
                &\vec{E}(z,t)=\hat{\epsilon}E_0\cos{(kz-\omega t)}\Rightarrow\vec{a}=-\hat{\epsilon}\frac{e}{m}E_0\cos{(kz-\omega t)}\\
                &\bar{a^2}=\frac{1}{2}\frac{e^2}{m^2}E_0^2:\TDy{\Omega}{P}=\frac{e^2}{4\pi c^3}\frac{1}{2}\frac{e^2}{m^2}E_0^2\sin^2{\psi}=\frac{c}{8\pi}E_0^2(\frac{e^2}{mc^2})^2\sin^2{\psi}\\
                &\TDy{\Omega}{\sigma}=\frac{\text{power radiated/unit solid angle}}{\text{incident power/unit area}}, S=\frac{\text{power}}{\text{area}}=\frac{c}{4\pi}\bar{E^2}=\frac{c}{8\pi}E_0^2:\\
                &\TDy{\Omega}{\sigma}=(\frac{e^2}{mc^2})^2\sin^2{\psi}(\propto (H_{int}^{(2)}))^2\\
                &r_0=\frac{e^2}{mc^2}=\SI{2.818e-13}{\cm}\tag{$r_0$ radius of shell charge e of energy $m_ec^2$}
            \end{align*}
        \end{column}
        \begin{column}{0.35\textwidth}
            \begin{figure}[!ht]
                \includegraphics[trim={0.0cm 0cm 0.0cm 0},clip, keepaspectratio,width=0.75\textwidth]{kappa-electronscattering-kin}\label{fig:kappa-electronscattering-kin}
                \includegraphics[trim={0.0cm 0cm 0.0cm 0},clip, keepaspectratio,width=0.9\textwidth]{kappa-electronscattering-diag}\label{fig:kappa-electronscattering-diag}
			\end{figure}
        \end{column}
    \end{columns}
    
\end{frame}

\begin{frame}{Thomson and Klein-Nishina crosssection}
    \begin{columns}[T]
        \begin{column}{0.6\textwidth}
            \begin{align*}
                &\overline{\sin^2{\psi}}=1-\sin^2{\theta}\overline{cos^2{\phi}}=1-\frac{1}{2}\sin^2{\theta}\tag{unpol.}\\
                &\TDy{\Omega}{\sigma}=(\frac{e^2}{mc^2})^2 \frac{1}{2}(1+\cos^2{\theta})\tag{Thomson}\\
                &d\sigma=\frac{8\pi}{3}(\frac{e^2}{mc^2})^2p(\cos{\theta})\frac{d\Omega}{4\pi}\tag{Norm. scat. phase function}\\
                &=\sigma_T[p(\cos{\theta})\frac{d\Omega}{4\pi}],\sigma_T=\frac{8\pi}{3}(\frac{e^2}{mc^2})^2=\SI{0.665e-24}{\square\cm}\\
                &\kappa_{es}=\frac{0.4}{\mu_e}\approx0.2(1+X_H)\tag{complete ioniz.}\\
                &
            \end{align*}
        \end{column}
        \begin{column}{0.4\textwidth}
            Thomson: not valid for rel. energy or $E_{\gamma}\geq mc^2$ ($T>\SI{e9}{\kelvin}$) - $\TDy{\Omega}{\sigma}\propto \frac{1}{m^2}$ so electron scatterin far more important than scattering from nuclei.
            Other type of scattering occurs in cooler part of stars - Rayleigh scattering, by from ions (harmonically bounded electrons) and molecules where ionization is incomplete. In He-H gas ionized electron scattering opacity dominate FF only if $T>\num{4.5e6}\rho^{0.286}$. Scattering phase function propto $1+\cos^2{\theta}$ cancellation made in diffusion theory of rad-transf is satisfied.
        \end{column}
    \end{columns}
    Klein-Nishina formula: when photon energy becomes sign. fraction of $mc^2$, $\epsilon=\frac{h\nu}{mc^2}$
    \begin{align*}
        &d\sigma=\frac{d\Omega}{4\pi}\frac{\sigma_T \frac{3}{4}(1+\cos^2{\theta})}{(1+2\epsilon\sin^2{\frac{\theta}{2}})}[1+\frac{4\epsilon^2\sin^4{\frac{\theta}{2}}}{(1+\cos^2{\theta})(1+2\epsilon\sin^2{\frac{\theta}{2}})}]\\
        &\sigma=\frac{3}{4}\sigma_T\{\frac{1+\epsilon}{\epsilon^2}[\frac{2+2\epsilon}{1+2\epsilon}-\frac{\ln{(1+2\epsilon)}}{\epsilon}]+\frac{\ln{(1+2\epsilon)}}{2\epsilon}-\frac{1+3\epsilon}{(1+2\epsilon)^2}\}
    \end{align*}
    At high temperature $\sigma<\sigma_T$, scattering phase function contains odd power of $\cos{\theta}$ (scattering of photons into/out no longer cancels), energy of scattered photon is reduced in prop to $\frac{\Delta\nu}{\nu}\approx \frac{h\nu}{mc^2}$.
    High energy photons may create \Pelectron\APelectron pair in nucleus's field: $\gamma+Z\to Z+\APelectron+\Pelectron$ if $h\nu>2mc^2$, $\sigma_{pair}\approx Z^2\num{e-3}\sigma_T$ (not important opacity source, but created electron increase scatterers and its annihilation produce neutrinos).
\end{frame}

\section{Radiative transfer equation}

\begin{frame}{Derivation of transfer equation}
    \begin{columns}[T]
        \begin{column}{0.3\textwidth}
\begin{figure}[!ht]
	\includegraphics[trim={0cm 0cm 1cm 0cm},clip, keepaspectratio,width=0.99\textwidth]{radthroughcyl}
\end{figure}
        \end{column}
        \begin{column}{0.7\textwidth}
\begin{align*}
    &\Delta t=\frac{ds}{c}:\ I(\vec{x}+\Delta\vec{x},t+\Delta t;\vec{n},\nu)\\
    &=I(\vec{x},t;\vec{n},\nu)+[\frac{1}{c}\PDy{t}{I}]\,ds
\end{align*}
        \end{column}
    \end{columns}
  \begin{align*}
   &I(\vec{x}+\Delta\vec{x},t+\Delta t;\vec{n},\nu)-I(\vec{x},t;\vec{n},\nu)=[\eta(\vec{x},t;\vec{n},\nu)-\chi(\vec{x},t;\vec{n},\nu)I(\vec{x},t;\vec{n},\nu)]ds\,dS\,d\Omega\,d\nu\,dt&\\
   &[\frac{1}{c}\PDof{t}+\nabla_{\vec{s}}]I(\vec{x},t;\vec{n},\nu)=\eta(\vec{x},t;\vec{n},\nu)-\chi(\vec{x},t;\vec{n},\nu)I(\vec{x},t;\vec{n},\nu)&\tag{Trans. eq.}\\
   &[\frac{1}{c}\PDof{t}+\mu\PDof{z}]I(z,t;\mu,\nu)=\eta(z,t;\mu,\nu)-\chi(\ldots)I(\ldots)&\tag{1D planar}
  \end{align*}

  \begin{columns}[T]
      \begin{column}{0.15\textwidth}
          \begin{figure}[!ht]\includegraphics[trim={2cm 0cm 1cm 5mm},clip, keepaspectratio,width=0.99\textwidth]{radtransfspheric}\label{fig:}\end{figure}
      \end{column}
      \begin{column}{0.85\textwidth}
          \begin{align*}
          &dr=\cos{\Theta}\,ds=\mu\,ds,\ r\,d\Theta=-\sin{\Theta}\,ds=-\sqrt{1-\mu^2}\,ds&\tag{spherical sym}\\
          &\PDof{s}=\PDy{s}{r}\PDof{r}+\PDy{s}{\Theta}=\mu\PDof{r}+\frac{(1-\mu^2)}{r}\PDof{\mu}&\\
          &[\frac{1}{c}\PDof{t}+\mu\PDof{r}+\frac{(1-\mu^2)}{r}\PDof{\mu}]I(r,t;\mu,\nu)=\eta(r,t;\mu,\nu)-\chi(r,t;\mu,\nu)I(r,t;\mu,\nu)&
          \end{align*}
      \end{column}
  \end{columns}
  Classical, macroscopic, phenomenological: not included polarization, dispersion, coherence, interference and quantum effects.
\end{frame}

\begin{frame}{Einstein relations}
    \begin{columns}[T]
        \begin{column}{0.5\textwidth}
            Radiative transition $i\to j$: $h\nu_{ij}=\epsilon_j-\epsilon_i$, stat. weight $g_i, g_j$, \keyword{absorption probability $B_{ij}$}, \keyword{spontaneus emission probability $A_{ji}$}: isotropic
                        \[\dot{E}_{\nu}=\frac{A_{ji}h\nu_{ji}}{4\pi}n_i\phi_{\nu}\]
                \begin{align*}
                &(\frac{n_i}{n_j})^*B_{ij}I_{\nu}^*=A_{ji}+B_{ji}I_{\nu}^*\label{CR and TE}\\
                                &(\frac{n_j}{n_i})^*=\frac{g_j}{g_i}\exp{-h\nu/KT}\\
                            &\Rightarrow B_{\nu}=\frac{A_{ji}}{B_{ji}}\invers{[\frac{g_iB_{ij}}{g_jB_{ji}}\exp{h\nu_{ij}kT}-1]}\\
                        &g_iB_{ij}=g_jB_{ji}\tag{Atomic prop $\to$ E. coeff}\\
                    &A_{ji}=\frac{2h\nu_{ij}^3}{c^2}B_{ji}\tag{also for non TE}
            \end{align*}
        \end{column}
        \begin{column}{0.5\textwidth}
            Number of photons absorbed per unit volume per unit time:
            \[r_{ij}=n_i\phi_{\nu}B_{ij}I_{\nu}(\frac{d\Omega}{4\pi})d\nu\]
            \keyword{Induced emission probability $B_{ji}$}: rate of stimulated energy emission per unit volume (as $I_{\nu}$)
            \[\dot{E}_{\nu}=\frac{B_{ji}h\nu_{ij}}{4\pi}n_i\phi_{\nu}I_{\nu}\]
            \keyword{Line emission/absorption ceff} (co-moving frame):
            \begin{align*}
    &\eta_l(\nu)=n_j\frac{A_{ji}h\nu_{ij}}{4\pi}\phi_{\nu}\\
    &\chi_l(\nu)=\frac{n_iB_{ij}h\nu_{ij}}{4\pi}[1-\frac{g_in_j}{g_jn_i}]\phi_{\nu}
            \end{align*}
        \end{column}
    \end{columns}
    \begin{columns}[T]
        \begin{column}{0.4\textwidth}
             Source function for a line: ratio emissivity to opacity:
        \end{column}
        \begin{column}{0.6\textwidth}
            \[S_l=\frac{n_jA_{ji}}{n_iB_{ij}-n_jB_{ji}}=\frac{2h\nu_{ij}^3}{c^2}\invers{[\frac{g_jn_i}{g_in_j}-1]}\to B_{\nu}\]
        \end{column}
    \end{columns}
\end{frame}

\begin{frame}{Einstein-Milne relations}
    \begin{columns}[T]
        \begin{column}{0.55\textwidth}
            Fluid frame; $n_e(v)$ maxwellian - prob. ionization by rad $(\nu,\nu+d\nu)$: Energy absorption coeff. $\alpha_{\nu}=h\nu p_{\nu}$,
            photoionization rate: $n_0h\nu I_{\nu}\,d\nu$, $n_0$ atoms number density, $n_1$ density of ions - $F(v)$/$G(v)$ spontaneous/induced recombination prob for \Pelectron with $(v,v+dv)$, recombination rate: $n_1n_e(v)[F+GI]vdv$
        \end{column}
        \begin{column}{0.45\textwidth}
            Thermodynamic equilibrium: photoionizations equal recombinations
            \begin{align*}
                &n_0^*p_{\nu}B_{\nu}=n_1^*n_e(v)[F(v)+G(v)B_{\nu}]\\
                &\Rightarrow B_{\nu}=\frac{F(v)}{G(v)}\invers{(\frac{n_0^*p_{\nu}m_e}{n_1^*n_e(v)hG(v)}-1)}\\
                &\left\{\begin{array}{l}F(v)=\frac{2h\nu^3}{c^2}G(v)\\\frac{p_{\nu}}{G(v)}\frac{h}{m}[n_2(v)(\frac{n_1}{n_0})^*]\exp{h\nu/kT}\\
                \end{array}\right.\tag{1}
            \end{align*}
        \end{column}
    \end{columns}
    \begin{align}
        &n_e(v)\,dv=n_e[\frac{m_e}{2\pi kT}]^{3/2}\exp{-\frac{1}{2}\frac{mv^2}{kT}}4\pi v^2\,dv\tag{TE: $n_e$ maxwell.}\\
        &(\frac{n_1}{n_0})^*=(\frac{g_0}{2g_1})[\frac{h^2}{2\pi m_ekT}]^{3/2}\exp{\epsilon_{ion}/kT}=n_e\phi_0\tag{Saha eq.}\\
        &\Rightarrow p_{\nu}=\frac{8\pi m^2v^2g_1}{h^2g_0}G(v)=\frac{4\pi c^2m^2v^2g_1}{h^3g_0\nu^3}F(v)\tag{2}
    \end{align}
 (1)-(2) continuum analog of Einstein relations   
\end{frame}

\begin{frame}{Continuum emission/absorption coeff.}
    \begin{align*}
        &\kappa_{\nu}=h\nu[n_0p_{\nu}-\frac{h}{m}n_1n_e(v)G(v)]=[n_0-n_0^*\exp{-\frac{h\nu}{kT}}]\alpha_{\nu}\\
        &\kappa_{\nu}^*=n_0^*\underbrace{[1-\exp{-\frac{h\nu}{kT}}]}_{\text{correction stimulated emission}}\alpha_{\nu}\tag{LTE}
    \end{align*}
    Recombination (spontaneous/induced) results from ion-electrons collision: if they have Maxwellian distribution, recombination at LTE rate whereas for lines rate depends on upper level population - Spontaneous continuum emission coeff.:
\begin{align*}
    &\eta_{\nu}^t=h\nu n_1n_e(v)F(v)\frac{h}{m}=\frac{hn_1n_e(v)F(v)}{m_ep_{\nu}}\alpha_{\nu}\\
    &\to\frac{2h\nu^3}{c^2}n_0^*\alpha\exp{-\frac{h\nu}{kT}}=n_0^*(1-\exp{-\frac{h\nu}{kT}})\alpha_{\nu}B_{\nu}(T)=\kappa_{\nu}^*B_{\nu}(T)
\end{align*}
\end{frame}

\begin{frame}{Absorption, emission and scattering}
    \begin{itemize}
        \item \keyword{Total absorption coef $\chi(\vec{x},t;\hat{n},\nu)$}: material volume of cross section $dS$ and length $dl$ removes energy from beam in interval dt:
            \[\delta=\chi(\ldots)I(\ldots)dld\Omega d\nu dt\]
            Sum over all states that can absorb at $\nu$ of the product of occupation number of those states (\si{\per\cubic\cm}) times their atomic crosssection (\si{\square\cm}) at that freq.: $\lambda_{\nu}=\invers{\chi_{\nu}}$ is mean free path of photons
        \item \keyword{Doppler Shift}: Photons moving along $\hat{n}$ with freq. $\nu$ in Lab. frame, has in fluid frame of materials moving at $\vec{v}$ freq. $\nu_0=\nu[1-\frac{\vec{v}\cdot\hat{n}}{c}]$ ($\frac{\Delta\nu_D}{\nu_0}c=13(\frac{T}{\SI{e4}{\kelvin}})^{1/2}\si{\kilo\meter\per\second}$).
        \item \keyword{Emission coeff. (Emissivity)}: $\eta(\vec{x},t;\hat{n},\nu)$ such that an element $dS\,dl$ add energy to beam
            \[\delta E=\eta(\ldots)dldSd\Omega\,dVdt\ [\si{\erg\per\cubic\cm\per\second\per\hertz\per\ster}]\]
        \item \keyword{Scattering}: Photon excites atoms then atoms decay into fundamentals - Photons collides with free electrons (Thopson scattering), atom or molecule in which they excite a resonance (Rayleigh/Raman scattering) 
    \end{itemize}
\end{frame}

\begin{frame}{Kirchoff-Plank relation}
    In TE: $(\eta_{\nu}^t)^*=(\kappa_{\nu}I_{\nu})^*$, $*$ indica TE dove $I_{\nu}=B_{\nu}$ quindi $(\eta_{\nu}^t)^*=(\kappa_{\nu}B_{\nu})^*$ - when gradients of physical quantities are small over photon destruction lenght: LTE
\end{frame}

\begin{frame}{Thermodynamics of adiabat enclosed rad: Kirchoff-Plank rel.}
    \begin{block}{Einstein coeff.}
Radiative transition $i\to j$: $h\nu_{ij}=\epsilon_j-\epsilon_i$, stat. weight $g_i, g_j$, \keyword{absorption probability $B_{ij}$}, \keyword{spontaneus emission probability $A_{ji}$}: isotropic
            \[\dot{E}_{\nu}=\frac{A_{ji}h\nu_{ji}}{4\pi}n_i\phi_{\nu}\]
            \begin{align*}
                &(\frac{n_i}{n_j})^*B_{ij}I_{\nu}^*=A_{ji}+B_{ji}I_{\nu}^*\label{CR and TE}\\
                &(\frac{n_j}{n_i})^*=\frac{g_j}{g_i}\exp{-h\nu/KT}\\
                &\Rightarrow B_{\nu}=\frac{A_{ji}}{B_{ji}}\invers{[\frac{g_iB_{ij}}{g_jB_{ji}}\exp{h\nu_{ij}kT}-1]}\\
                &g_iB_{ij}=g_jB_{ji}\tag{Atomic prop $\to$ E. coeff}\\
                &A_{ji}=\frac{2h\nu_{ij}^3}{c^2}B_{ji}\tag{also for non TE}
            \end{align*}
   \end{block}
   \begin{block}{E. of transfer in star interior}
       \begin{align}
        &\TDy{s}{I_{\nu}}\frac{1}{\rho}=j_{\nu}-\kappa_{\nu}I_{\nu}\Leftrightarrow\TDy{s}{I_{\nu_{nm}}}=N_n(A_{nm}+B_{nm}I_{\nu_{nm}})h\nu_{nm}-N_mB_{mn}h\nu_{nm}I_{\nu_{nm}}\label{ERT}
    \end{align}
    \begin{columns}[T]
        \begin{column}{0.3\textwidth}
            $I_{\nu}$ in isotropic homogeneous medium adiabat. enclosed deps only on T:
        \end{column}
        \begin{column}{0.7\textwidth}
        \begin{align*}
        \frac{1}{\rho}\TDy{s}{I_{\nu}}=0\Rightarrow j_{\nu}=\kappa_{\nu}I_{\nu}\Rightarrow I_{\nu_{nm}}=\frac{N_nA_{nm}}{N_mB_{mn}-N_nB_{nm}}\label{Kirckoff-Plank}
\end{align*}
        \end{column}
    \end{columns} 
   \end{block}
\end{frame}
\begin{frame}{Scattering of radiation in spectral line}
    \begin{itemize}
    \item Broadened atomic level can be decomposed into sublevels and transition between these sublevels can be treated statistically: \keyword{redistribution function}: probability in atom's frame that incoming photon of freq $\nu'$ is scattered as outgoing $\nu$.
    \item Semiclassical picture
    \item Natural population of levels: probability emitting photon $(\hat{n},\nu)$ in transition to lower level averaged over ensemble of identical atoms is independent of previous history of ensemble (not naturally populated by radiative processes in which there is correlation with prop. prev. absorbed/emitted photon: stimulated emission)
\item Distro of individual sublevel of naturally popul. n at $E_n$: $L(\chi,\gamma)=\frac{\gamma}{\pi(\chi^2+\gamma^2)}$, $\chi=\frac{E-E_n}{h}$, $2\gamma_n=\sum_{m<n}A_{nm}+\sum_{m\neq n}B_{nm}\bar{J}_{nm}+R_{n\kappa}+\sum_{n\neq m}C_{nm}+C_{n\kappa}+Q_n$, A,B bb Einstein coeff., R photoion rate, C coll rate (bb,bf), $Q_n$ effective rate of elastic coll.in level n, $\bar{J}_{nm}=\int_0^{\infty}\oint \frac{d\Omega}{4\pi}\phi_{nm}(\nu,\hat{n})\,d\nu$ freq-averaged mean intensity for $n\to m$ (suppose spont em coeff much bigger than induced em coeff)
\item Atom's frame absorption profile for transition between broadened level $i\to e$: $\phi_{ie}(\nu')=\intsinf{}L(\chi_i,\gamma_i)L(\chi_e',\gamma_e)\,d\chi_i$; $L(\chi_i,\gamma_i)$ natural occupation probability of sublevels in $(\chi_i,\chi_i+\,d\chi_i)$, $L(\chi_e'=\nu'-\nu_{ie}+\chi_i,\gamma_e)$ is occupation prob of level e when photon $\nu'$ is absorbed, total prob of transition is product summed over all i states, Cauchy's residue theorem $\phi_{ij}(\nu')=L(\nu'-\nu_{ij},\gamma_{ij})=\frac{\gamma_{ij}}{\pi[(\nu'-\nu_{ij})^2+\gamma_{ij}^2]}$, $\gamma_{ij}=\gamma_i+\gamma_j$
\end{itemize}
\end{frame}

\begin{frame}{Semiclassical picture of photon scattering}
    
\end{frame}

\begin{frame}{True absorption and scattering coeff.}
    True absorption plus scattering $\chi=\kappa+\sigma$, thermal emission plus scattering emission $\eta=\eta^T+\eta^S$ - if $\sigma_l$ (line scattering cross section) is isotropic total energy removed from beam is
    \begin{align*}
        &\sigma(\ldots)\intzi\,d\nu_0'\phi(\vec{x},t;\nu_0')\oint\,d\Omega_0'I_0(\vec{x},t;\hat{n}_0',\nu_0')\\
        &=4\pi\sigma_l(\vec{x},t)\intzi\phi(\vec{x},t;\nu_0')J_0(\vec{x},t;\nu_0')d\nu_0'
    \end{align*}
    Prob. absorption of photon with $(\hat{n}',\nu')$ absorbed and re-emitted on $(\hat{n},\nu)$ is $R(\hat{n}',\nu',\hat{n},\nu)$.
    If photons are random in angle and along line profile (\keyword{complete redistribution}: good approx. in many cases, within Doppler core of a line, if excited atoms suffer many collisions before re-emission) the fluid frame/lab. frame emission(emissivity) by scattering is
    \begin{align*}
        &\eta_0^S(\vec{x},t;\nu_0)=\sigma(\ldots)\phi(\vec{x},t;\nu_0)\intzi\,d\nu_0'\phi(\vec{x},t;\nu_0')J_0(\vec{x},t;\hat{n}_0',\nu_0')\\
        &\eta^S(\vec{x},t;\hat{n},\nu_0)=\sigma(\ldots)\phi(\vec{x},t;\nu_0)\intzi\,d\nu_0'\oint\,d\Omega\phi(\vec{x},t;\nu_0')I_0(\vec{x},t;\hat{n}_0',\nu_0')
    \end{align*}
    When scattering is isotropic and coherent the emissivity is $\eta_0^S=\sigma_0(\vec{x},t)J_0(\vec{x},t;\nu_0)$: Thompson scattering of continuum photons by free electrons
\end{frame}

  \begin{frame}{Total Opacity and Total Emission coeff. and LTE}
      \begin{align*}
              &\chi_{\nu}=\kappa_{\nu}+\sigma_{\nu}=\underbrace{\sum_i\sum_{j>i}[n_i-\frac{g_i}{g_j}n_j]\alpha_{ij}(\nu)}_{\text{BB - line}}+\underbrace{\sum_i(n_i-n_i^*\exp{-\frac{h\nu}{kT}})\alpha_{ik}(\nu)}_{\text{BF}}\\
                      &+\underbrace{\sum_{\kappa}n_en_{\kappa}\alpha_{\kappa\kappa}(\nu,T)(1-\exp{-\frac{h\nu}{kT}})}_{\text{FF}}+\underbrace{n_e\sigma_e}_{\text{Thompson scatt. by free e}}\tag{Tot.Opa.}\\
          &\eta_{\nu}^{th}=\frac{2h\nu^3}{c^2}[\sum_i\sum_{j>i}n_j \frac{g_i}{g_j}\alpha_{ij}(\nu)+\sum_in_i^*\alpha_{i\kappa}(\nu)\exp{-\frac{h\nu}{kT}}+\sum_{\kappa}n_en_{\kappa}\alpha_{\kappa\kappa}(\nu,T)\exp{-\frac{h\nu}{kT}}]\\
                  &\xrightarrow{LTE}\frac{2h\nu^3}{c^2}\exp{-\frac{h\nu}{kT}}[\sum_in_i^* (\sum_{j>i}\alpha_{ij}(\nu)+\alpha_{i\kappa}(\nu))+\sum_{\kappa}n_en_{\kappa}\alpha_{\kappa\kappa}(\nu,T)\exp{-\frac{h\nu}{kT}}]\\
                          &\Rightarrow\chi_{\nu}^*=\kappa_{\nu}^*+n_e\sigma_e,\ (\eta_{\nu}^{th})^*=\kappa_{\nu}^*B_{\nu}
          \end{align*}
          \end{frame}

\section{Solution of RE in far interior}


\begin{frame}{Total Opacity and Total Emission coeff. and LTE}
    \begin{align*}
        &\chi_{\nu}=\kappa_{\nu}+\sigma_{\nu}=\underbrace{\sum_i\sum_{j>i}[n_i-\frac{g_i}{g_j}n_j]\alpha_{ij}(\nu)}_{\text{BB - line}}+\underbrace{\sum_i(n_i-n_i^*\exp{-\frac{h\nu}{kT}})\alpha_{ik}(\nu)}_{\text{BF}}\\
        &+\underbrace{\sum_{\kappa}n_en_{\kappa}\alpha_{\kappa\kappa}(\nu,T)(1-\exp{-\frac{h\nu}{kT}})}_{\text{FF}}+\underbrace{n_e\sigma_e}_{\text{Thompson scatt. by free e}}\tag{Tot.Opa.}\\
        &\eta_{\nu}^{th}=\frac{2h\nu^3}{c^2}[\sum_i\sum_{j>i}n_j \frac{g_i}{g_j}\alpha_{ij}(\nu)+\sum_in_i^*\alpha_{i\kappa}(\nu)\exp{-\frac{h\nu}{kT}}+\sum_{\kappa}n_en_{\kappa}\alpha_{\kappa\kappa}(\nu,T)\exp{-\frac{h\nu}{kT}}]\\
        &\xrightarrow{LTE}\frac{2h\nu^3}{c^2}\exp{-\frac{h\nu}{kT}}[\sum_in_i^* (\sum_{j>i}\alpha_{ij}(\nu)+\alpha_{i\kappa}(\nu))+\sum_{\kappa}n_en_{\kappa}\alpha_{\kappa\kappa}(\nu,T)\exp{-\frac{h\nu}{kT}}]\\
        &\Rightarrow\chi_{\nu}^*=\kappa_{\nu}^*+n_e\sigma_e,\ (\eta_{\nu}^{th})^*=\kappa_{\nu}^*B_{\nu}
    \end{align*}
\end{frame}


\section{Fusione nucleare}\linkdest{reactionrates}

\begin{frame}{Rate of nuclear reaction}
    \begin{columns}[T]
        \begin{column}{0.5\textwidth}
            \begin{align*}
                &\sigma=\frac{\#\text{ interaction per T}}{\# \text{ incident per A per T}*\#\text{ target within beam}}\\
                &=\frac{N_R/T}{N_{beam}/(AT)N_T}\Rightarrow \frac{N_R}{T}=\frac{N_{beam}}{TA}N_T\sigma\\
                &j_{beam}=\frac{N_{beam}}{TA},\ \SI{1}{\barn}=\SI{e-24}{\square\cm},\ \SI{1}{\square\femto\meter}=\SI{e-2}{\barn}\\
                &r_{01}=\frac{N_0N_1\exv{\sigma v}}{1+\delta_{01}},\ m_{01}=\frac{m_0m_1}{m_0+m_1}\\
                &P(v)\,dv=(\frac{m_{01}}{2\pi KT})^{\frac{3}{2}}\exp{-\frac{m_{01}v^2}{2KT}}4\pi v^2\,dv\\
                &=P(E)\,dE=\frac{2}{\sqrt{\pi}}\frac{\sqrt{E}}{(KT)^{\frac{3}{2}}}\exp{-\frac{E}{KT}}\,dE
            \end{align*}
            If velocity of 0,1 are sep. Maxwell. also $v_{01}$ is Maxwell.;$K\approx\SI{8.6e-5}{\ev\per\kelvin}$
        \end{column}
        \begin{column}{0.5\textwidth}
            \begin{align*}
                &\frac{N_R}{VT}=(\sigma N_T)(\frac{N_b}{VAT})\\
                &=\sigma \frac{N_T}{V}\frac{N_b}{AT}=\sigma \frac{N_T}{V}v \frac{N_b}{V}\\
                &r_{01}=N_0N_1\sigma v\xrightarrow{P(v)}N_0N_1\exv{\sigma v}_{01}\\
                &N_0N_1\tag{Pairs number density: non id.}\\
                &\frac{N_0(N_0-1)}{2}\to \frac{N_0^2}{2}\tag{Number density pairs: id}\\
                &v_T=\sqrt{\frac{2KT}{m_{01}}}\tag{Peak $P(v)$}\\
                &E_T=\frac{KT}{2}\tag{Peak $P(E)$}
            \end{align*}
            $N_0$, $N_1$ number density
        \end{column}
    \end{columns}
    \begin{block}{Reaction rate at high temperature: we have to consider excited states}
        \begin{align*}
        &P_{i\mu}=\frac{N_{i\mu}}{N_i}=\frac{g_{i\mu}\exp{-\frac{E_{i\mu}}{KT}}}{\sum_{\mu}g_{i\mu}\exp{-\frac{E_{i\mu}}{KT}}}=\frac{g_{i\mu}\exp{-\frac{E_{i\mu}}{KT}}}{G_i}\tag{$E_{i\mu}$ excitation energy}\\
        &N_A\exv{\sigma v}^*_{01\to23}=\sum_{\mu}P_{0\mu}\sum_{\nu}N_A\exv{\sigma v}^{\mu\to\nu}_{01\to23}\tag{$\mu$ states target 0, $\nu$ in 3}\\
        &=R_{tt}N_A\exv{\sigma v}_{01\to23}\tag{$R_{tt}$ stellar enhancement factor}
    \end{align*}
    \end{block}
\end{frame}

\begin{frame}{Photodisintegration ($\gamma+3\to0+1$)}
    Most astrophys. photodisin $\gamma+3\to0+1$ are derived from corresp. particle induced reaction rate applying reciproc. theorem but some are measured directly:
    \begin{align*}
        &j_b=\frac{N_b}{AT}=\frac{cN_b}{V}\Rightarrow r_{\gamma3}=\frac{N_R}{VT}=N_3N_{\gamma}c\sigma(E_{\gamma}),\ N_3=\frac{N_T}{V},\ N_{\gamma}=\frac{N_b}{V}\tag{$\gamma3\to01$}\\
        &r_{\gamma3}=N_3\int_0^{\infty}cN_{\gamma}(E_{\gamma})\sigma(E_{\gamma})\,dE_{\gamma}\tag{If $Q_{\gamma3\to01}<0$ integration lower limit is $E_t=Q_{01\to\gamma3}$}\\
        &N_{\gamma}(E_{\gamma})\,dE_{\gamma}=\frac{u(E_{\gamma})}{E_{\gamma}}\,dE_{\gamma}=\frac{8\pi}{(hc)^3}\frac{E_{\gamma}^2}{\exp{\frac{E_{\gamma}}{KT}}-1}\,dE_{\gamma},\ u(E_{\gamma})\,dE_{\gamma}=\frac{8\pi}{(hc)^3}\frac{E_{\gamma}}{\exp{\frac{E_{\gamma}}{KT}-1}},\ u(\nu)\,d\nu=\frac{8\pi}{(hc)^3}\frac{1}{\exp{\frac{E_{\gamma}}{KT}}-1}\,d\nu\\
        &\lambda_{\gamma}(3)=\frac{r_{\gamma3}}{N_3}=\int_0^{\infty}cN_{\gamma}(E_{\gamma})\sigma(E_{\gamma})\,dE_{\gamma}\Rightarrow\lambda_{\gamma}(3)=\frac{8\pi}{h^3c^2}\int_0^{\infty}\frac{E_{\gamma}^2\sigma(E_{\gamma})}{\exp{\frac{E_{\gamma}}{KT}}-1}\tag{decay const}\\
        &\frac{\sigma_{\gamma3\to01}}{\sigma_{01\to\gamma3}}=\frac{(2j_0+1)(2j_1+1)}{2(2j_2+1)}\frac{2m_{01}c^2E_{01}}{E_{\gamma}^2}\frac{1}{(1+\delta_{01})}\tag{forward/reverse}\\
        &\Rightarrow \frac{8\pi m_{01}}{h^3}\frac{(2j_0+1)(2j_1+1)}{(2j_3+1)}\int_0^{\infty}\overbrace{\frac{E_{\gamma}-Q_{01\to\gamma3}}{\exp{\frac{E_{\gamma}}{KT}-1}}}^{E_{01} \text{Kinetic energy in COM}}\sigma_{01\to\gamma3}\,dE_{\gamma}\\
    &\lambda_{\gamma}(3)\approx\int_0^{\infty}(E_{\gamma}-Q_{01\to\gamma3})\exp{-\frac{E_{\gamma}}{KT}}\frac{\exp{-2\pi\eta}}{E_{01}}S(E_{01})\,dE_{\gamma}\tag{charged particle em., $E_{\gamma}\gg KT$, $S\approx\const{}$}\\
    &\approx S(E_0)\exp{-\frac{Q_{01\to\gamma3}}{KT}}\int_0^{\infty}\exp{-2\pi\eta}\exp{-\frac{E_{01}}{KT}}\,dE_{01},\ E_{\gamma}^{eff}=E_0+Q_{01\to\gamma3}\tag{$E_0$ Gamow peak forward reaction}\\
    &\lambda_{\gamma}(3)\propto\int_0^{\infty}(E_{\gamma}-Q_{01\to\gamma3})\exp{-\frac{E_{\gamma}}{KT}}E_{01}^{l-1/2}\,dE_{\gamma}\tag{$(\gamma,n)$: small $E_n$: $\sigma_{n\gamma}\propto\sigma_l\propto E^{l-1/2}$, $E_{\gamma}^{eff}\approx E_{thr}$}\\
    &\approx\int_0^{\infty}\exp{-\frac{E_{\gamma}}{KT}}(E_{\gamma}-Q_{01\to\gamma3})^{l+1/2}\,dE_{\gamma},\ E^{eff}_{\gamma}=(l+\frac{1}{2})KT+Q_{n\gamma}\tag{n-capt.: $E_n^{eff}=(l+\frac{1}{2})KT$ so $E_{\gamma}^{eff}=(l+\frac{1}{2})KT+Q_{n\gamma}$}
    \end{align*}
\end{frame}

\begin{frame}{Resonance reaction rate}
    Narrow Res.: One level Breit-Wigner formula
            \begin{itemize}
                \item Isolated: Level density in compound nucleus is small so resonance do not overlap
                \item Narrow: Partial width cont. over total res. width ($\Gamma<\text{few kev}$)
            \end{itemize}
            \begin{align*}
                &\sigma_{BW}(E)=\frac{\lambda^2}{4\pi}\frac{(2J+1)(1+\delta_{01})}{(2j_0+1)(2j_1+1)}\frac{\Gamma_a\Gamma_b}{(E_r-E)^2+\Gamma^2/4}
            \end{align*}
            $j_i$ Spin of target/projectile, $J,E_R$ spin and energy of resonance, $\Gamma_i,\Gamma$ Partial width of entrance/exit channel, total res width.
            \begin{align*}
                &N_A\exv{\sigma v}=\sqrt{\frac{8}{\pi m_{01}}}\frac{N_A}{(KT)^{\frac{3}{2}}}\int_0^{\infty}E\sigma_{BW}(E)\exp{-\frac{E}{KT}}\,dE=N_A \frac{\sqrt{2\pi}\hbar^2}{(m_{01}KT)^{\frac{3}{2}}}\omega\int_0^{\infty}\frac{\Gamma_a\Gamma_b\exp{-\frac{E}{KT}}}{(E-E_R)^2+\Gamma^2/4}\,dE\\
                &\omega=\frac{(2J+1)(1+\delta_{01})}{(2j_0+1)(2j_0+1)}
            \end{align*}
            Narrow res.: MB factor $\exp{-\frac{E}{KT}}$ and $\Gamma_i$ approx const.:
            \begin{equation*}
            N_A\exv{\sigma v}=N_A(\frac{2\pi}{m_{01}KT})^{\frac{3}{2}}\hbar^2\exp{-\frac{E_r}{KT}}\omega\gamma 
            \end{equation*}
            $\omega\gamma=\omega \frac{\Gamma_a\Gamma_b}{\Gamma}$, proportional to area under resonance cross-section (ie product of max. cross-section $\sigma_{BW}(E_r)=\frac{\lambda_r^2}{\pi}\omega \frac{\Gamma_a\Gamma_b}{\Gamma}$ and total res. width): $\Gamma*\sigma_{BW}(E_r)=\frac{\lambda_r^2}{\pi}\omega\gamma$. Several narrow res. add incoherently.
            Temperature deps.: $N_A\exv{\sigma v}_T=N_A\exv{\sigma v}_{T_0}(\frac{T}{T_0})^{\num{11.605}\frac{E_r}{T_9}-\frac{3}{2}}$, $E_r$ in \si{\mega\ev}: T sensitive increases as \xdiminuisce{T} and \xaumenta{E_r}.
            Broad Res.: Explicit energy dependence of cross-section
            \begin{align*}
                &N_A\exv{\sigma v}=\sqrt{2\pi}\frac{N_A\omega\hbar^2}{(m_{01}KT)^{\frac{3}{2}}}\int_0^{\infty}\exp{-\frac{E}{KT}}\frac{\Gamma_a(E)\Gamma_b(E+Q-E_f)}{(E_r-E)^2+\Gamma(E)^2/4}
            \end{align*}
            $\Gamma_b$ has to be calculated at energy $E_{23}=E_{01}+Q_{01\to23}-E_f$.
\end{frame}

\begin{frame}{Reaction Rate Equilibrium}
    \begin{align*}
        &r=r_{01\to23}-r_{23\to01}=\frac{N_0N_1\exv{\sigma v}_{01to23}}{(1+\delta_{01})}-\frac{N_2N_2\exv{\sigma v}_{23\to01}}{(1+\delta_{23})}\tag{Equilibrium $r=0$}\\
        &\frac{N_2N_3}{N_0N_1}=\frac{(1+\delta_{23})\exv{\sigma v}_{01\to23}}{(1+\delta_{01})\exv{\sigma v}_{23\to01}}=\frac{(2j_2+1)(2j_3+1)}{(2j_0+1)(2j_1+1)}\frac{G_2^{norm}G_3^{norm}}{G_0^{norm}G_1^{norm}}(\frac{m_{23}}{m_{01}})^{\frac{3}{2}}\exp{\frac{Q_{01\to23}}{KT}}\\
        &G_i^{norm}=\frac{G_i}{g_i}=\frac{\sum_{\mu}g_{i\mu}\exp{-\frac{E_{i\mu}}{KT}}}{g_i}\tag{Normalized partition function, $g_i$ stat. weight ground state of nucleus i}\\
        &r=0=r_{01\to\gamma3}-r_{\gamma3\to01}=\frac{N_0N_1\exv{\sigma v}_{01\to\gamma3}}{(1+\delta_{01})}-\lambda_{\gamma}(3)N_3\tag{$01\to\gamma3$}\\
        &\frac{N_3}{N_0N_1}=\frac{1}{(1+\delta_{01})}\frac{\exv{\sigma v}_{01\to\gamma3}}{\lambda_{\gamma}(3)}=\frac{1}{1+\delta_{01}}\frac{1}{N_1}\frac{\lambda_1(0)}{\lambda_{\gamma}(3)}\\
        &=(\frac{h^2}{2\pi})^{\frac{3}{2}}\frac{1}{(m_{01}KT)^{\frac{3}{2}}}\frac{(2j_3+1)}{(2j_0+1)(2j_1+1)}\frac{G_3^{norm}}{G_0^{norm}G_1^{norm}}\exp{\frac{Q_{01\to\gamma3}}{KT}}\tag{Saha statistical equation}\\
            &01\to23\tag{$0$,$3$: Heavy, $1$, $2$: $\alpha$, \Pproton, \Pneutron}\\
            &\left.\TDy{t}{N_0}\right|_{01\to23}=-r_{01\to23},\ \left.\TDy{t}{N_3}\right|_{23\to01}=-r_{23\to01}\\
                    &f_{03}=|\TDy{t}{N_0}|_{01\to23}-\TDy{t}{N_3}|_{23\to01}|=|r_{01\to23}-r_{23\to01}|=|N_0N_1\exv{\sigma v}_{01\to23}-N_2N_3\exv{\sigma v}_{23\to01}|\tag{Net abundance flow between 0,3}\\
                    &(\TDy{t}{N_0})_{01\to23}\approx(\TDy{t}{N_3})_{23\to01}\gg f_{03}\approx0\ \phi_{03}=\frac{|r_{01\to23}-r_{23\to01}|}{\max{r_{01\to23},r_{23\to01}}}\approx0\tag{equilibrium condition}
                            \end{align*}
        A differenza di stato stazionario, la condizione di equilibrio non implica $N_0,N_3$ costanti: refers to near equality of forward/reverse abundance flow between pair nuclei; when a group of severals pairs comes into equilibrium the resulting solution of reaction network is of quasi-equilibrium (Arnett96).
\end{frame}

\begin{frame}{Sezione d'urto fusion nucleare $\sigma(E)= \pi\lambdabar^2*P_0(E)*S(E)$}
    \begin{itemize}
        \item $E$ energia cinetica nel CM dei nuclei. Prodotto sezione d'urto geometrica (nel riferimento CM: $\sigma\approx\sum_{l=0}^{\frac{R}{\lambdabar}}(2l+1)\pi\lambdabar^2=\pi(R+\lambdabar)^2$ - energie stellari: approx onda S), della prob. attraversamento della barriera coulombiana $P_0$ e del fattore astrofisico $S(E)$. \todo{prova sezione d'urto pg.200}
            $\lambdabar$ relativa delle particelle descrive l'indeterminazione sulla posizione nell'urto di due particelle con momento relativo p $\lambdabar=\frac{\hbar}{p}=\frac{\hbar}{\sqrt{2mE}}$: $\pi\lambdabar^2\propto \frac{1}{p^2}\propto \frac{1}{E}$; geometrical cross-section for \si{\kilo\ev} nucleons $\pi\lambdabar^2=657\invers{E}\si{\barn}$ where E is in \si{\kilo\ev} and reduced mass in amu.
        \item Prob. cross c-barrier: expansion of $\widehat{T}$ for $\frac{E}{V_C}\ll1$, $\eta=\frac{Z_1Z_2\alpha}{v}$ sommerfeld param. - $P_B=\frac{R_0^2|\psi_0(R_0)|^2}{R_E^2|\psi_0(R_E)|^2}$
            \begin{columns}[T]
                \begin{column}{0.45\textwidth}
                    \begin{align*}
                        &\widehat{T}=\widehat{T}_1\ldots\widehat{T}_n\approx\exp{[-\frac{2}{\hbar}\sqrt{2m(V_i-E)}(R_{i+1}-R_i)]}\\
                        &\to\exp{[-\frac{2}{\hbar}\int_{R_0}^{R_c}\sqrt{2m[V(r)-E]}\,dr]}\\
                        &\widehat{T}\approx\exp{(-\frac{2\pi}{\hbar}\sqrt{\frac{m}{2E}Z_0Z_1e^2})}=\Exp{-2\pi\eta}=P_0(E)\\
                        &2\pi\eta=\num{0.989534}Z_0Z_1\sqrt{\frac{1}{E}\frac{M_0M_1}{M_0+M_1}}
                    \end{align*}
                \end{column}
                \begin{column}{0.55\textwidth}
                    \begin{align*}
                    &\psi_0=\frac{\exp{\chi(r)}}{r}: \TtwoDy{r}{\chi}+(\TDy{r}{\chi})^2=\frac{2m}{\hbar^2}[U(r)-E]\\
                    &|\TtwoDy{r}{\chi}|\ll|\TDy{r}{\chi}|^2: \frac{\sqrt{2m}}{\hbar}\int_{R_0}^{R_E}\sqrt{\frac{Z_1Z_2e^2}{r}-E}\,dr\\
                    &=\frac{\sqrt{2mE}}{\hbar}R_E\int_{\frac{R_0}{R_E}}^1dx\sqrt{\frac{1}{x}-1}\approx\pi\sqrt{\frac{Z_1^2Z_2^2e^4m}{2E\hbar}}=\pi \Z_1Z_2\frac{e^2}{\hbar c}\frac{c}{v_{\infty}}
                    \end{align*}
                \end{column}
            \end{columns}
\item Astrophysical factor $S(E)$: $S(E)=\sigma(E)E\exp{\frac{2\pi Z_1Z_0e^2}{\hbar v}}$. Slow variation with E (non resonant case): expansion around $E=0$ $S(E)=S(0)+S'(0)E+\frac{1}{2}S''(0)E^2$ is suff. $N_A\exv{\sigma v}=(\frac{4}{3})^{\frac{3}{2}}\frac{\hbar}{\pi}\frac{N_A}{m_{01}Z_0Z_1e^2}S_{eff}\tau^2\exp{-\tau}$. $\tau=\frac{3E_0}{KT}$, $S_{eff}$ accounts for energy dependence of astrophysical factor and asymmetry of Gamow Peak
\item Gamow peak - Gauss Approx. $\PDof{E}[\frac{2\pi}{\hbar}\sqrt{\frac{m_{01}}{2E}Z_0Z_1e^2-\frac{E}{KT}}]_{E_0}=0=\PDof{E}(\frac{E}{KT}+bE^{-\frac{1}{2}})_{E_0}=\frac{1}{KT}-\frac{1}{2}bE_0^{-\frac{3}{2}}$:
    \begin{align*}
        &\exp{(-\frac{E}{KT}-bE^{-\frac{1}{2}})}\approx C\exp{-(\frac{E-E_0}{\Delta/2})^2},\ \Delta=\frac{4}{\sqrt{3}}\sqrt{E_0KT}=\num{0.75}(Z_0^2Z_1^2AT_6^5)^{\frac{1}{6}}\si{\kilo\ev}\tag{match 2nd der}\\
        &E_0=(\frac{bKT}{2})^{\frac{2}{3}}=[(\frac{\pi}{\hbar})^2(Z_0Z_1e^2)^2(\frac{m_{01}}{2})(KT)^2]^{\frac{1}{3}}=\num{0.1220}(Z_0^2Z_1^2\frac{M_0M_1}{M_0+M_1}T_9^2)
    \end{align*}
\end{itemize}
\end{frame}

\begin{frame}{Gamow Peak: most reactions happens in narrow cm-energy range - particles with $E_0$}
    \begin{columns}[T]
        \begin{column}{0.5\textwidth}
    \begin{align*}
        &\sigma(E)=\frac{S(E)}{E}\exp{-\frac{2\pi Z_1Z_2e^2}{\hbar v}}=\frac{S(E)}{E}\exp{-bE^{-\frac{1}{2}}}\\
        &\lambda=\exv{\sigma v}=\int_0^{\infty}\sigma(E)v(E)\psi(E)\,dE\\
        &=\int_0^{\infty}\frac{S(E)}{E}\exp{-bE^{-\frac{1}{2}}}\sqrt{\frac{2E}{\mu}}\frac{2}{\sqrt{\pi}}\frac{E}{KT}\exp{-\frac{E}{KT}}\frac{dE}{\sqrt{KTE}}
    \end{align*}    
        \end{column}
        \begin{column}{0.5\textwidth}
            Method of steepest descent: integral of the form $\int g(x)\exp{-f(x)}$, $g(x)$ slowly varying, $f(x)$ much larger than one with single minimum at $x_0$
            \begin{align*}
                &f(x)\approx f(x_0)+f''(x_0)\frac{(x-x_0)^2}{2}\tag{around minimum}\\
                &\Rightarrow\int g(x)\exp{-f(x)}\,dx\approx\sqrt{\frac{2\pi}{f''(x_0)}}g(x_0)\exp{-f(x_0)}
            \end{align*}
        \end{column}
    \end{columns}
    
    \begin{columns}[T]
        \begin{column}{0.65\textwidth}
            \begin{align*}
                &\TDof{E}(\frac{E}{KT}+bE^{-\frac{1}{2}})_{E=E_0}=\frac{1}{KT}-\frac{1}{2}bE_0^{-\frac{3}{2}}=0\Rightarrow E_0=(\frac{bKT}{2})^{\frac{2}{3}}\\
                &\Exp{-\frac{E}{KT}-bE^{-\frac{1}{2}}}\approx C\exp{-(\frac{E-E_0}{\Delta/2})^2}\tag{gaussian approx}\\
                &C=\Exp{-\frac{E_0}{KT}-bE_0^{-\frac{1}{2}}}=\exp{-\frac{3E_0}{KT}}\\
                &\Delta=\frac{4}{\sqrt{3}}\sqrt{E_0KT}=0.75(Z_1^2Z_2^2AT_6^5)^{\frac{1}{6}}\si{\kilo\ev}\tag{full width at $1/e$}\\
                &\lambda\approx(\frac{8}{\mu\pi})^{\frac{1}{2}}(\frac{1}{KT})^{\frac{3}{2}}\exp{-\tau}\int_0^{\infty}S(E)\exp{-(\frac{E-E_0}{\Delta/2})^2}\tag{$A=\frac{\mu}{m_u}$, $\tau=\frac{3E_0}{KT}$}\,dE\\
                &\lambda\approx \frac{\num{4.5e14}}{AZ_1Z_2}S_0(\si{\cgs})\tau^2\exp{-\tau}\\
                &=\frac{\num{7.2e-19}}{AZ_1Z_2}S_0\si{\kilo\ev\barn}\tau^2\exp{-\tau}\tag{\si{\cubic\cm\per\second}: rr per pair}\\
                &r_{12}=\invers{(1+\delta_{12})}N_1N_2\lambda\tag{rr/pair mult. by pairs per unit volume}
            \end{align*}
        \end{column}
        \begin{column}{0.35\textwidth}
\begin{figure}[!ht]
    \includegraphics[trim={0cm 0cm 0cm 0cm},clip, keepaspectratio,width=0.99\textwidth]{GamowPeak}\label{fig:GamowPeak}
\end{figure}
\begin{align*}
    &E_0=\num{1.220}(Z_1^2Z_2^2AT_6^2)^{\frac{1}{3}}\\
    &\tau=\num{42.48}(\frac{Z_1^2Z_2^2A}{T_6})^{\frac{1}{3}}
\end{align*}
        \end{column}
    \end{columns}
\end{frame}

\begin{frame}{Coulomb screening and continuum depression}
    In ionized plasma electron are attracted to nuclei while nuclei repel each others:each nucleus is surrounded by charged cloud, so Coulomb barrier in nuclear reaction becomes thinner increasing tunneling probability and reaction rate.
    L'energia potenziale dovuta all'interazione di due nuclei $Z_1$ e $Z_2$ a distanza r contiene un contributo delle altre cariche presenti nel plasma
    \begin{equation*}
    U=\frac{Z_1Z_2e^2}{r}+U_s(r_{12})
    \end{equation*}
    l'energia potenziale non schermata e contributo della nuvola elettronica: $U_s$ aumenta la probabilit\'a di attraversamento della barriera coulombiana.
    Fattore moltiplicativo: $f=\exp{-\midfrac{U_0}{KT}}$ dove $U_0=U_s(0)$ poich\'e $r\ll r_D$ e considerando solo la correzione al fattore di penetrazione ($E_G\gg U_0$).

Per determinare $U_0$ considero l'energia potenziale di $Z_1$ e $Z_2$ a distanza $r$
\begin{equation*}
U=Z_2e\int_{\infty}^r\PDy{r_1}{\phi_1}\,dr_1=\frac{Z_1Z_2e^2\exp{-\midfrac{r}{r_D}}}{r},\ U_s=U-\frac{Z_1Z_2e^2}{r}\approx\frac{Z_1Z_2e^2}{r_D}
\end{equation*}
Weak Screening regime: Condition of weak Coulomb energy compared to thermal.
\begin{align*}
    &R_D=\sqrt{\frac{KT}{4\pi e^2\rho N_A\zeta^2}}=\num{2.812e-7}\rho^{-\frac{1}{2}}T_9^{\frac{1}{2}}\invers{\zeta}\si{\cm}\\
    &\zeta=\sqrt{\sum_i \frac{(Z_i^2+Z_i\theta_e)X_i}{A_i}}\\
    &T\gg \num{e5}\rho^{\frac{1}{3}}\zeta^2\tag{weak screening}
\end{align*}
Continuum depression: $\chi_Z=\chi_Z-\frac{Ze^2}{R_D}$ (pressure ionization)
\end{frame}

\begin{frame}{Approfondimento fattore astrofisico: }
    \begin{itemize}
        \item S isn't constant (expansion in $\invers{\tau}=3\frac{E_0}{KT}$): $\exv{\sigma v}\approx S_{eff}\sqrt{\frac{32E_G}{3m}}(\frac{1}{KT})^{\frac{3}{2}}\exp{-\frac{3E_G}{KT}}$, neglecting terms of higher order than $\frac{KT}{E_G}$:
            \begin{equation*}
                S_{eff}=S(E_G)(1+\frac{5KT}{36E_G})+S'(E_G)(E_G+\frac{35KT}{36})+\frac{1}{2}S''(E_G)E_G(E_G+\frac{89KT}{36})
            \end{equation*}
\item Correction for gaussian approx
    \item partial shielding coulomb potential
        \item Strong E deps for resonant reactions
        \end{itemize}
$S_{eff}$ \'e il risultato dell'espansione dell'integrando per $\invers{\tau}\ll1$ ed estrapolato a $E_G$ a partire dal valore $S(0)$ determinato dalla fisica nucleare.
%\begin{equation*}
%\exv{\sigma v}\propto b\expy{1/3}T\expy{-2/3}\exp{-\frac{b\expy{2/3}}{t\expy{1/3}}}
%\end{equation*}
\end{frame}

\begin{frame}{Non-resonant reaction rate: Neutron induced reaction}
    \begin{itemize}
        \item Most likely at max MB: $E_T=KT$, $v_T=\sqrt{\frac{2KT}{m_{01}}}$.
        \item $\frac{1}{v}$-dependence of low energy $\sigma$: we have two approaches - $\sigma_r=\pi(R+\lambdabar)^2$ based on model of total absorption then include reflection of incident neutron wave at nuclear surface (transmission probability) resulting in $\sigma=\pi(R+\lambdabar)^2 \frac{4kK}{(k+K)^2}$ where $K=\sqrt{2m(E+V_0)/\hbar^2}$ for barrier depth $-V_0$ and $k=\sqrt{\frac{2mE}{\hbar^2}}$, for low energy neutrons $E\ll V_0$, $k\ll K$ and $\lambdabar=\invers{k}\gg R$ quindi $\sigma=\frac{4\pi}{kK}\propto \frac{1}{v}$ since $k=\frac{p}{\hbar}=\frac{mv}{\hbar}$, the other approach use one level resonance formula: decay by $\gamma$ emission and take $\Gamma$ indipendent on neutron energy, the neutron width $\Gamma_n$ (entrance channel) is dependent on density of states $\TDy{E}{n}\propto v$ available to captured neutron and far from resonance $E\ll E_R$: $\sigma=\frac{\pi}{k^2}\frac{\Gamma_n\Gamma}{E_R^2+\Gamma^2/4}\propto \frac{1}{v}$ (not only $(n,\gamma)$ but also $(n,p)$, $(n,\alpha)$).
        \item For $\sigma\propto \frac{1}{v}$:
            \begin{equation*}
                N_A\exv{\sigma v}=N_A\int_0^{\infty}vP(v)\sigma(v)\,dv=N_AS=\const{}
            \end{equation*}
        \item But Non-resonant neutron cross-section deviate from $\frac{1}{v}$:
            \begin{enumerate}
                \item S-wave neutron may have not small energy
                \item New reaction channel may becomes energetically accessible
                \item Higher partial waves may contribute to neutron cross-section.
            \end{enumerate}
    \end{itemize}
\end{frame}

\begin{frame}{Produzione energia reazioni nucleari}
La funzione $\epsilon(\rho,T,X_i)$ \'e determinata dalla somma di tutti i contributi
\begin{equation*}
\epsilon_{ij}=Q_{ij}\frac{n_in_j}{\rho(1+\delta_{ij})}\lambda_{ij}=\frac{1}{1+\delta_{ij}}Q_{ij}\frac{\rho N_A^2X_jX_k}{A_iA_j}\exv{\sigma v}_{ij}
\end{equation*}
dove $Q_{ij}$ \'e l'energia liberata per reazione tra nucleo di specie i e j e $\exv{\sigma v}_{ij}$ \'e il rate di reazione per coppia di particelle, mediata su MB- distro $f(E)dE\propto\frac{E\expy{\frac{1}{2}}}{(kT)\expy{\frac{3}{2}}}\exp{-\frac{E}{kT}}\,dE$: $S(E)\exp{-\frac{E}{kT}-\frac{b}{\sqrt{E}}}$ ha forma approssimativamente gaussiana il cui massimo $E_G$, energia pi\'u probabile di reazione, e FWHM sono: $E_G=\SI{5.665}{\kilo\ev} A\expy{\frac{1}{3}}T_7\expy{\frac{2}{3}}$ $\Delta E=4.249\si{\kilo\ev}W\expy{\frac{1}{6}}T_7\expy{\frac{5}{6}}$ posto $W=Z_i^2Z_j^2A=Z_i^2Z_j^2\frac{A_iA_j}{A_i+A_j}$.
Most of reactions occur at Gamow peak, in MB tail (except for $T>\SI{10}{\giga\kelvin}$, $E_0\ll V_c$); reaction with larger Coulomb barrier do not contribute significantly to energy production: \xaumenta{E_0}, \xaumenta{Z_i}, \xdiminuisce{\text{area picco}}.

    Neutrinos crosssection $\sigma_{\nu}\approx(\frac{E_{\nu}}{m_ec^2})^2\SI{e-44}{\square\cm}$: in matter of density $\rho=n\mu m_u$ mean-free path $l_{\nu}=\frac{1}{n\sigma_{\nu}}=\frac{\mu m_u}{\rho\sigma_{\nu}}\approx \frac{\SI{2e20}{\cm}}{\rho(\si{\gram\per\cubic\cm})}$, $l_{\nu}=\left\{\begin{array}{l}
            \SI{100}{\parsec} (\rho=1)\\
            3000\rsun{} (\rho=\num{e6})\\
            \SI{20}{\kilo\meter} (\rho=\num{e14})
    \end{array}\right.$.

    Electron $\nu$s produced via nuclear reactions: $\epsilon_n$ reduced by $Q_{\nu}$:
                \begin{align*}
                    &^1H+^1H\to^2H+\APelectron+\nu: Q_{\nu}=\SI{0.263}{\mega\ev}\tag{pp1,2,3}\\
                    &^7Be+\Pelectron\to ^7Li+\nu: Q_{\nu}=\SI{0.8}{\mega\ev}\tag{pp2}\\
                    &^8B\to ^8Be+\APelectron+\nu: Q_{\nu}=\SI{7.2}{\mega\ev}\tag{pp3}\\
                    &^{13}N\to ^{13}C+\APelectron+\nu: Q_{\nu}=\SI{0.71}{\mega\ev}\tag{CNO}\\
                    &^{15}O\to ^{15}N+\APelectron+\nu: Q_{\nu}=\SI{1}{\mega\ev}\tag{CNO}
                \end{align*}
\end{frame}

\begin{frame}{Neutrinos Losses}
    \begin{columns}[T]
        \begin{column}{0.45\textwidth}
    \begin{itemize}
            \item Electron capture: $\Pelectron+(Z,A)\to(Z-1,A)+\nu$
            \item URCA processes: \Pelectron-capture, followed by $\beta$-decay:
                \begin{align*}
                    &(Z,A)+\Pelectron\to(Z-1,A)+\nu\\
                    &(Z-1,A)\to(Z,A)+\Pelectron+\APnu
                \end{align*}
            \item Pair annihilation: $\Pelectron\APelectron\to\Pnu\APnu$ - At $T>\SI{1}{\giga\kelvin}$ photon produce \Pelectron\APelectron pair (not null equilibrium abundance of \APelectron): 1 over \num{e19} insted of annihilation into photons we have a pair \Pnu\APnu. Asymptotic expression $\epsilon_{\nu}^{pair}(\si{\cgs})=\left\{
                        \begin{array}{l}
                                \frac{\num{4.9e8}}{\rho}T_9^3\exp{-11.86T_9}\ T_9<1\\
                                \frac{\num{4.45e15}}{\rho}T_9^9\ T_9>3
                        \end{array}
                    \right.$
                    \item Photo-$\nu$s: $\gamma\Pelectron\to\Pelectron\Pnu\APnu$ (Compton scattering in which photon replaced by $\nu$-pair), interpolation formula (in cgs):
                        \begin{align*}
                        &\epsilon_{\nu}^{(phot)}\approx\epsilon_1+\epsilon_2\invers{(\mu_e+\bar{\rho})}\\
                        &\epsilon_1=\num{1.103e13}\invers{\rho}T_9^9\exp{-\frac{5.93}{T_9}}\\
                        &\epsilon_2=\num{0.976e8}T_9^8\invers{1+4.2T_9}\\
                        &\bar{\rho}=\num{6.446e-6}\rho\invers{T_9}\invers{(1+4.2T_9)}
                        \end{align*}
        \end{itemize}
        \end{column}
        \begin{column}{0.55\textwidth}
    \begin{itemize}
                    \item Plasma-$\nu$s: $\gamma_{pl}\to\Pnu\APnu$, plasma-freq $\omega_0$ given by
                        \[\omega_0^2 \frac{m_e}{4\pi e^2n_e}\left\{
                            \begin{array}{l}
                                1:\ \text{NDEG}\\
                                \{1+(\frac{\hbar}{m_ec})^2(3\pi^2n_e)^{\frac{2}{3}}\}^{-\frac{1}{2}}:\ \text{DEG}
                        \end{array}
                \right.\]
                Dispersion $\omega^2=K^2c^2+\omega_0^2$: propagating wave for $\omega>\omega_0$ - plasmon with rest mass $\hbar\omega_0$; for energy rate $\epsilon_{\nu}^{(plasm)}=\epsilon_{\nu}^{(l)}\epsilon_{\nu}^{(t)}$ (cgs):
                \begin{align*}
                    &\gamma=\frac{\hbar\omega_0}{kT},\ \lambda=\frac{kT}{m_ec^2}\\
                    &\epsilon_{\nu}^{plasm}=\num{3.356e19}\invers{\rho}\lambda^6(1+0.0158\gamma^2)T_9^3\tag{$\gamma\ll1$}\\
                    &=\num{5.252e20}\invers{\rho}\lambda^{7.5}T_9^{1.5}\exp{-\gamma}\tag{$\gamma\gg1$}
                \end{align*}
                        Exponential decrease for large $\gamma$, ie increasing $\omega_0\propto\rho^{\frac{1}{2}}$ at const $T$, due to fact that few plasmons can be excited if $KT<\hbar\omega_0$.
                    \item Bremsstrahlung-$\nu$s: Inelastic scattering (decell) of \Pelectron in Coul. field of nucleus emit \Pgamma (ff-emission), \Pgamma can be replaced by \Pnu\APnu: for very lorge $\rho$, $\epsilon_{\nu}^{(brems)}=\num{0.76}\frac{Z^2}{A}T_8^6$ (cgs); can dominates at low T and high $\rho$ as don't decreses with incresing deg (lack of free cells in PS compensated by incresed $\sigma$ for $\nu$-emission).
                    \item Synchrotron-$\nu$: Strng magnetic Field - photon emitted by \Pelectron moving in the field is replaced by \Pnu\APnu
        \end{itemize}
        \end{column}
    \end{columns}
\end{frame}

\begin{frame}{Periodic table}
    
\begin{figure}[!ht]
    \includegraphics[trim={0cm 0cm 0cm 0cm},clip, keepaspectratio,width=0.99\textwidth]{PeriodicTableAtomicMassColor}\label{fig:PeriodicTableAtomicMassColor}
\end{figure}
\end{frame}

\begin{frame}{Energy prodcution PP chains ($\SI{1}{\mega\ev}=\SI{1.602e-6}{\erg}$)}
    \begin{columns}[T]
        \begin{column}{0.45\textwidth}
\begin{figure}[!ht]
    \includegraphics[trim={0cm 0cm 0cm 0cm},clip, keepaspectratio,width=0.99\textwidth]{pp-Q-table}\label{fig:pp-Q-table}
\end{figure}
    \begin{itemize}
        \item PPII, PPIII - $He^3(He^4,\gamma)Be^7$ then electron capture $Be^7+\Pelectron\to Li^7+\nu$ (or proton captur: $Be^7+H^1\to B^8+\gamma$): rate of prodiuction of $\alpha$ may be up to twice that of PPI ($\frac{r_{pp}}{2}$) as only one PP reaction. Approx:
            \begin{itemize}
                \item Deuterium equilibrium(seconds-hours): $\TDy{t}{D}\approx0$ quindi $\lambda_{pp}\frac{H^2}{2}\approx\lambda_{pd}HD$
        \item $Li^7$, $Be^7$ equilibrium: $\TDy{t}{Li^7+Be^7}=\lambda_{34}He^3He^4-\lambda_{17}HBe^7-\lambda_{17}'HLi^7\approx0$, when equilibrium reached (years) Li/Be will follow buildup of $He^3$ - simplified equation for rate of production of $He^4$ and H: 
            \begin{align*}
                &\TDy{t}{He^4}=\lambda_{33}\frac{(He^3)^2}{2}+\lambda_{34}He^3He^4\\
                &\TDy{t}{H}=-3\lambda_{pp}\frac{H^2}{2}+2\lambda_{33}\frac{(He^3)^2}{2}-\lambda_{34}He^3He^4
            \end{align*}
            Competition for $He^3$ determines which PP dominates
            \end{itemize}
    \end{itemize}
        \end{column}
        \begin{column}{0.55\textwidth}
        PPI - Split in two parts: pp is followed rapidly by $D(p,\gamma)He^3$ (net effect $3H\to He^3+\nu$) at rate $r_{pp}$, energy liberated $Q=\SI{6.936}{\mega\ev}-\SI{0.263}{\mega\ev}$ so $\rho\epsilon(3H\to He^3)=\num{1.069e-5}r_{pp}\si{\erg\per\cubic\cm\per\second}$ - then for $He^3(He^3,2p)He^4$ has $Q=\SI{12.858}{\mega\ev}$ and summing $\rho\epsilon_{PPI}=\num{1.069e-5}r_{pp}+\num{2.060e-5}r_{33}\si{\erg\per\cubic\cm\per\second}$. Equilibrium $\TDy{t}{He^3}\approx0$: $2r_{33}=r_{pp}$, $\TDy{t}{He^4}=r_{33}=\frac{r_{pp}}{2}$ and energy production $\rho\epsilon_{PPI}=\num{2.099e-5}r_{pp}\si{\erg\per\cubic\cm\per\second}$.
        \begin{itemize}
            \item Third approx when $He^3$ reaches equilibrium:
                \begin{align*}
                    &\TDy{t}{He^3}\approx0\Rightarrow\lambda_{pp}-\lambda_{33}(He^3_e)^2-\lambda_{34}He_e^3He^4\approx0\\
                    &\TDy{t}{He^4}=\frac{1}{4}\lambda_{pp}H^2+\frac{1}{2}\lambda_{34}He_e^3He^4\\
                    &\TDy{t}{H}=-\lambda_{pp}^2-2\lambda_{34}He_e^3He^4\\
                    &\TDy{t}{He^4}=-\frac{1}{4}\TDy{t}{H}\tag{All reactions in equil.}\\
                    &\left(\frac{PP1}{PPII+PPIII}\right)_e=\frac{1}{2}\frac{\lambda_{33}}{\lambda_{34}}\frac{He_e^3}{He^4}\\
                    &PPI: \frac{2*0.263}{26.73}\approx2.0\%\ PPII: \frac{0.263+0.80}{26.73}\approx4.0\%\\
                    &PPIII: \frac{0.263+7.2}{26.73}\approx27.9\%\tag{$\nu$ losses}
                \end{align*}
        \end{itemize}
        \end{column}
    \end{columns}
\end{frame}

\begin{frame}{Approfondimenti PP}
    \begin{columns}[T]
        \begin{column}{0.4\textwidth}
            \begin{figure}[!ht]
    \includegraphics[trim={0cm 0cm 0cm 0cm},clip, keepaspectratio,height=0.9\textheight]{PP-lifetime}\label{fig:PP-lifetime}
\end{figure}
        \end{column}
        \begin{column}{0.6\textwidth}
            \begin{align*}
                &Q=(4(ME)_{H}-(ME)_{He^4})c^2=\SI{26.731}{\mega\ev}\\
                &\rho\epsilon=\TDy{t}{He^4}(4M_H-M_{He^4})[0.980F_{PPI}+0.960F_{PPII}+0.721F_{PPIII}]\\
                &\rho\epsilon_{PPI}=\SI{6.671}{\mega\ev}r_{pp}+\SI{12.861}{\mega\ev}r_{He^3He^3}\\
                &\epsilon_{pp}^e=\num{6.551}N_A\exv{\sigma v}_{pp}(\frac{X_H}{M_H}^2\rho N_A\si{\mega\ev\per\gram\per\second}\\
                    &\epsilon_{PPI}^e=\epsilon_{PPI}^e(T_0)(\frac{T}{T_0})^{3.9}
            \end{align*}
            \begin{figure}[!ht]
    \includegraphics[trim={0cm 0cm 0cm 0cm},clip, keepaspectratio,width=0.75\textwidth]{PP-energy-fraction}\label{fig:PP-energy-fraction}
\end{figure}
        \end{column}
    \end{columns}
\end{frame}

\begin{frame}{Ciclo CN-NO}
    \begin{columns}[T]
        \begin{column}{0.4\textwidth}
\begin{figure}[!ht]
    \includegraphics[trim={0cm 0cm 0cm 0cm},clip, keepaspectratio,width=0.99\textwidth]{cnno-scheme}\label{fig:cnno-scheme}
    \includegraphics[trim={0cm 0cm 0cm 0cm},clip, keepaspectratio,width=0.99\textwidth]{CNOs}\label{fig:cnos}
\end{figure}
        \end{column}
        \begin{column}{0.6\textwidth}
            Energetic. allowed: proton induced $(p,\alpha)$ reactions on $^{15}N$, $^{17}O$, $^{18}O$, $(p,\gamma)$ and $(p,\alpha)$ on $^{19}F$, and only $(p,\gamma)$ on $^{12}C$, $^{13}C$, $^{14}N$, $^{16}O$.
            \begin{align*}
                &C^{12}(p,\gamma)N^{13}(\beta^+\nu)C^{13}(p\gamma)N^{14}\tag{CN-CNO1}\\
                &N^{14}(p,\gamma)O^{15}(\beta^+\nu)N^{15}\\
                &N^{15}(p,\alpha)C^{12}\\
                &N^{15}(p,\gamma)O^{16}(p,\gamma)F^{17}(\beta^+\nu)O^{17}\tag{NO-CNO2}\\
                &O^{17}(p,\alpha)N^{14}\\
                &O^{17}(p,\gamma)F^{18}(\beta^+\nu)O^{18}(p,\alpha)N^{15}\tag{CNO3}\\
                &O^{18}(p,\gamma)F^{19}(p,\alpha)O^{16}\tag{CNO4}
            \end{align*}
\begin{figure}[!ht]
    \includegraphics[trim={0cm 0cm 0cm 0cm},clip, keepaspectratio,width=0.65\textwidth]{CNObranching}\label{fig:CNObranching}
\end{figure}
        \end{column}
    \end{columns}
\end{frame}

\begin{frame}{Steady state CNO: Abundances}
    \begin{columns}[T]
        \begin{column}{0.5\textwidth}
            For $T<\SI{0.1}{\giga\ev}$ $\beta$-decay lifetime of $^{13}N$ $\tau_{^{13}N}\ll\tau_{^{12}C}$ for $(p,\gamma)$ destruction, and after few minutes $^{13}N$, $^{15}O$ are at equilibrium
            \begin{align*}
                &\TDy{t}{^{12}C}=H ^{15}N\exv{\sigma v}_{^{15}N(p,\alpha)}-H ^{12}C\exv{\sigma v}_{^{12}C(p,\gamma)}\\
                &\TDy{t}{^{13}C}=H ^{12}C\exv{\sigma v}_{^{12}C(p,\gamma)}-H ^{13}C\exv{\sigma v}_{^{13}C(p,\gamma)}\\
                &\TDy{t}{^{14}N}=H ^{13}C\exv{\sigma v}_{^{13}C(p,\gamma)}-H ^{14}N\exv{\sigma v}_{^{14}N(p,\gamma)}\\
                &\TDy{t}{^{15}N}=H ^{14}N\exv{\sigma v}_{^{14}N(p,\alpha)}-H ^{15}N\exv{\sigma v}_{^{15}N(p,\alpha)}\\
                &\left(\frac{^{14}N}{^{12}C}\right)_e=\frac{\exv{\sigma v}_{^{12}C(p,\gamma)}}{\exv{\sigma v}_{^{14}N(p,\gamma)}}=\frac{\tau_p(^{14}N)}{\tau_p(^{12}C)}
            \end{align*}
        \end{column}
        \begin{column}{0.5\textwidth}
            \begin{align*}
                &\Rightarrow\TDy{t}{^{12}C}+\TDy{t}{^{13}C}+\TDy{t}{^{14}N}+\TDy{t}{^{15}N}=0\\
                &\Rightarrow\sum CNO1=\const{}
            \end{align*}
            Ratio of any nuclidic abb. is inverse ratio of reaction rate (or lifetime): ie $(\frac{^{14}N}{^{12}C})=\frac{\exv{\sigma v}_{^{12}C(p,\gamma)}}{\exv{\sigma v}_{^{14}N(p,\gamma)}}$. CNO abundance evolution: conversion of $^{12}C$, $^{16}O$ to $^{14}N$.
        \end{column}
    \end{columns}
    \begin{figure}[!ht]
    \includegraphics[trim={0cm 0cm 0cm 0cm},clip, keepaspectratio,width=0.65\textwidth]{CNO1-equil-abund}\label{fig:CNO1-equil-abund}
\end{figure}
\end{frame}

\begin{frame}{Steady state CNO: energy production}
            \begin{align*}
                &\epsilon_{CNO}=\frac{1}{\rho}\sum_{i\to j}(Q_{i\to j}-\bar{E}^{i\to j}_{\nu})r_{i\to j}\\
                &Q_{^{12}C(p,\gamma)^{13}N(\beta^+\nu)}-\bar{E}^{^{13}N(\beta^+\nu)}=(\num{1.944}+\num{2.22}-\num{0.706})\si{\mega\ev}=\SI{3.458}{\mega\ev}\\
                &Q_{^{13}C(p,\gamma)}=\SI{7.551}{\mega\ev}\\
                &Q_{^{14}N(p,\gamma)^{15}O(\beta^+\nu)}-\bar{E}_{\nu}^{^{15}(\beta^+\nu)}=(\num{7.297}+\num{2.754}-\num{0.996})\si{\mega\ev}=\SI{9.055}{\mega\ev}\\
                &Q_{^{15}N(p,\alpha)}=\SI{4.966}{\mega\ev}\\
                &\rho\epsilon^e_{CNO1}=\SI{3.458}{\mega\ev}H(^{12}C)_e\exv{\sigma v}_{^{12}C(p,\gamma)}+\SI{7.551}{\mega\ev}H(^{13}C)_e\exv{\sigma v}_{^{13}C(p,\gamma)}+\SI{9.055}{\mega\ev}H(^{14}N)_e\exv{\sigma v}_{^{14}N(p,\gamma)}\\
                &+\SI{4.966}{\mega\ev}H(^{15}N)_e\exv{\sigma v}_{^{15}N(p,\alpha)}\\
                &=\SI{3.458}{\mega\ev}\frac{(^{12}C)_e}{\tau_p(^{12}C)}+\SI{7.551}{\mega\ev}\frac{(^{13}C)_e}{\tau_p(^{13}C)}+\SI{9.055}{\mega\ev}\frac{(^{14}N)_e}{\tau_p(^{14}N)}+\SI{4.966}{\mega\ev}\frac{(^{15}N)_e}{\tau_p(^{15}N)}\\
                &\epsilon^e_{CNO1}=\frac{\SI{25.030}{\mega\ev}}{\rho}\frac{\sum CNO1}{\tau_p(^{12}C)+\tau_p(^{13}C)+\tau_p(^{14}N)+\tau_p(^{15}N)}\approx\frac{\SI{25.030}{\mega\ev}}{\rho}\frac{\sum CNO1}{\tau_p(^{14}N)}\\
                &=\frac{\SI{25.030}{\mega\ev}}{\rho}(\sum CNO1)H\exv{\sigma v}_{^{14}N(p,\gamma)}=\num{25.030}N_A\exv{\sigma v}_{^{14}N(p,\gamma)}(\sum_i \frac{X_i}{M_i})\frac{X_H}{M_H}\rho N_A\si{\mega\ev\per\gram\per\second}\\
                &\epsilon^e_{CNO1}(T)=\epsilon^e_{CNO1}(T_0)(\frac{T}{T_0})^{16.7}\tag{$T=\SI{25}{\mega\kelvin}$}\\
                &\epsilon_{pp}\propto\rho X_H^2,\ \epsilon_{CN}\propto\rho X_HX_{CN}
            \end{align*}
\end{frame}

\begin{frame}{He burning: energy production}
    He-burn in massive* contribute $^{16}O$, $^{18}O$ to universe while massive* and AGB of $^{12}C$.
    \begin{columns}[T]
        \begin{column}{0.7\textwidth}
            \begin{align*}
                &^4He(\alpha\alpha,\gamma)^{12}C\tag{$Q=\SI{7274.7}{\kilo\ev}$}\\
                &\alpha+\alpha\to^8Be\tag{$Q=\SI{-91.84\pm0.04}{\kilo\ev}$}\\
                &E_r=\SI{287.6\pm0.2}{\kilo\ev}\tag{Res in $^8Be(\alpha,\gamma)^{12}C$}\\
                &^{12}(\alpha,\gamma)^{16}O\tag{$Q=\SI{7161.9}{\kilo\ev}$}\\
                &^{16}O(\alpha,\gamma)^{20}Ne\tag{$Q=\SI{4729.8}{\kilo\ev}$}\\
                &^{20}Ne(\alpha,\gamma)^{24}Mg\tag{$Q=\SI{9316.6}{\kilo\ev}$}\\
                &\lambda_{3\alpha\to^{12}C}=\lambda_{3\alpha}=3N_{\alpha}(\frac{h^2}{2\pi})^{\frac{3}{2}}\frac{1}{(m_{\alpha\alpha}KT)^{\frac{3}{2}}}\exp{\frac{Q_{\alpha\alpha\to^8Be}}{KT}}\lambda_{^8Be(\alpha,\gamma)}\\
                &\lambda_{^8Be(\alpha,\gamma)}=\lambda_{\alpha}(^8Be)=N_{\alpha}\exv{\sigma v}_{^8Be(\alpha,\gamma)}\\
                &\sigma_{^8Be(\alpha,\gamma)}=(\frac{2\pi}{m_{\alpha^8Be}KT})^{\frac{3}{2}}\hbar^2\exp{\frac{E_r}{KT}}\omega\gamma_{^8Be(\alpha,\gamma)}\tag{Narrow res.}\\
                &\omega\gamma_{^8Be(\alpha,\gamma)}=\frac{2J+1}{(2j_0+1)(2j_1+1)}\frac{\Gamma_{\alpha}\Gamma_{rad}}{\Gamma}\approx\Gamma_{rad}\approx\SI{3.7\pm0.5}{\milli\ev}\\
                &\Gamma_{\alpha}=\SI{8.3\pm0.1}{\ev},\ \Gamma_{rad}=\Gamma_{\gamma}+\Gamma_{pair}=\SI{3.7\pm0.5}{\milli\ev},\\
                &J(^{12}C)=j_0(\alpha)=j_1(^8Be)=0\\
                &\lambda_{3\alpha}=\num{8.7590e-10}\frac{(\rho X_{\alpha})^2}{T_9^3}\exp{\frac{-\num{4.4040}}{T_9}}\si{\per\second}\tag{$E'=\SI{287.6}{\kilo\ev}-\SI{-91.84}{\kilo\ev}=\SI{379.4}{\kilo\ev}$}\\
                &\epsilon_{3\alpha}=\frac{Q_{3\alpha}}{\rho}r_{3\alpha}=\frac{\SI{7.275}{\mega\ev}}{\rho}\frac{1}{3}\num{8.7590e-10}(\rho N_A \frac{X_{\alpha}}{M_{\alpha}})\frac{(\rho X_{\alpha})}{T_9^3}\exp{-\frac{\num{4.4040}}{T_9}}\\
                &=\num{3.1771e14}\frac{\rho^2X_{\alpha}^3}{T_9^3}\exp{-\frac{4.4040}{T_9}}\si{\mega\ev\per\gram\per\second}
            \end{align*}
        \end{column}
        \begin{column}{0.3\textwidth}
            $\rho=\SIrange{e2}{e5}{\gram\per\cubic\cm}$, $T=\SIrange{0.1}{0.4}{\giga\kelvin}$
\begin{figure}[!ht]
    \includegraphics[trim={0cm 0cm 0cm 0cm},clip, keepaspectratio,height=0.40\textheight]{Heburning-network}\label{fig:Heburning-network}
\end{figure}
    \begin{align*}
        &\epsilon_{3\alpha}(T)=\epsilon_{3\alpha}(T_0)(\frac{T}{T_0})^{41.0}\\
        &\frac{N(^{12}C)}{N(^{16}O)}\tag{universe}
    \end{align*}
    \todo{Iliadis: He burning beyonf C}
        \end{column}
    \end{columns}
\end{frame}

\begin{frame}{Advanced Burning Stages ($T>\SI{6e8}{\kelvin}$): C-Burning}
    \begin{columns}[T]
        \begin{column}{0.45\textwidth}
            Coulomb barrier $CC$: \SI{8}{\mega\ev}
            \begin{align*}
                &C^{12}+C^{12}\to Mg^{24}+\gamma\tag{$Q=\SI{13.930}{\mega\ev}$}\\
                &\to Na^{23}+p\tag{$Q=\SI{2.238}{\mega\ev}$}\\
                &\to Ne^{20}+\alpha\tag{$Q=\SI{4.616}{\mega\ev}$}\\
                &\to Mg^{23}+n\tag{\xaumenta{T}:shell $Q=\SI{-2.6}{\mega\ev}$}\\
                &\to O^{16}+2\alpha\tag{$Q=\SI{-0.11}{\mega\ev}$}\\
                &O^{16}+O^{16}\to S^{32}+\gamma\tag{$Q=\SI{16.539}{\mega\ev}$}\\
                &\to P^{31}+p\tag{$Q=\SI{7.676}{\mega\ev}$}\\
                &\to S^{31}+n\tag{$Q=\SI{1.459}{\mega\ev}$}\\
                &\to Si^{28}+\alpha\tag{$Q=\SI{9.593}{\mega\ev}$}\\
                &\to Mg^{24}+2\alpha\tag{$Q=\SI{-0.4}{\mega\ev}$}
            \end{align*}
\begin{figure}[!ht]
    \includegraphics[trim={0cm 0cm 0cm 0cm},clip, keepaspectratio,height=0.32\textheight]{CBurning-abund}\label{fig:CBurning-abund}
\end{figure}
        \end{column}
        \begin{column}{0.55\textwidth}
        Difference mass between compound nucleus $^{12}C+^{12}C$ and $^{24}Mg$: excess energy carried away from light massive particles (Branching ratio $B_p\approx B_{\alpha}\approx \frac{1-B_n}{2}$, $B_n\approx2-10\%$ at $E_{cm}=\SIrange{3.5}{5.0}{\mega\ev}$): $^{12}C(^{12}C,\alpha)$ and $^{12}C(^{12}C,p)$ rates approx equal while $^{12}C(^{12}C,n)$ is far smaller. Important electron screening correction: approx factor 3. Secondary reactions contribute sign. to nuclear energy release by primary C-burning: each $^{12}C+^{12}C$ releases $\bar{Q}_C\approx\SI{10}{\mega\ev}$
            \begin{align*}
                &\epsilon_C=\frac{\bar{Q}_C}{\rho}r_{^{12}C^{12}C}=\frac{\bar{Q}_C}{\rho}\frac{(N_{^{12}C})^2\exv{\sigma v}_{^{12}^{12}C}}{2}\\
                &=\frac{N_A\bar{Q}_C}{288}X_{^{12}C}^2\rho N_A\exv{\sigma v}_{^{12}C^{12}C}=\num{2.09e22}X_{^{12}C}^2\rho N_A\exv{\sigma v}_{^{12}C^{12}C}\\
            \end{align*}
\begin{figure}[!ht]
    \includegraphics[trim={0cm 0cm 0cm 0cm},clip, keepaspectratio,height=0.32\textheight]{CBurning-S}\label{fig:CBurning-S}
\end{figure}
        \end{column}
    \end{columns}
    \begin{equation*}
    \epsilon_C=\epsilon_C(T_0)(\frac{T}{T_0})^{28} \int\epsilon_C(t)\,dt=\frac{N_A\bar{Q}_C}{2M_{^{12}C}}\Delta X_{^{12}C}=\num{2.51e23}\Delta X_{^{12}C}\si{\mega\ev\per\gram}
    \end{equation*}
\end{frame}

\begin{frame}{C-burning: weak interactions and neutron excess param.}
    \begin{itemize}
        \item AGB enriched in s-elements (Sr,Y,Zr, Ba, La, Ce, Pr, Nd) produced by slow (compared to beta decay)
        \item First recognized neutron sources activating s processes are $^{22}Ne(\alpha,n)^{25}Mg$ and $^{13}C(\alpha,n)^{16}O$
        \item  $\eta=\sum_i(N_i-Z_i)Y_i=\sum_i \frac{N_i-Z_i}{M_i}X_i$ $Y_i$ is the mole fraction, number of excess neutron per nucleon in the plasma: can change due to weak interactions.
    \item Most important source of neutrons is $^{22}Ne(\alpha,n)^{25}Mg$ (and $^{21}Ne(\alpha,n)^{24}Mg$). Neutron induced processes on $^{12}C$, $^{20}Ne$, $^{23}Na$, $^{24}Mg$, $^{25}Mg$, neutron excess parameter increases slightly as $^{20}Ne(n,\gamma)^{21}Ne(p,\gamma)^{22}Na(n,p)^{22}Ne(\alpha,n)^{25}Mg(p,\gamma)^{26}Al(\beta^+\nu)^{26}Mg$.
    \item As light nuclei are small number radioactive nuclei undergoes $\beta$-decay; for $T-\rho$ condition inside $25\msun{}$ photodisintegration of $^{13}N$ prevent $^{13}C$ production ($^{12}C(p,\gamma)^{13}N(\beta^+\nu)^{13}C$, $Q_{^{12}Cp}=\SI{1944}{\kilo\ev}$), equilibrium between $^{12}C$ and $^{13}N$ is quickly estab.: equil. abundance ratio and decay const. $\lambda_{^{12}C\to^{13}N\to^{13}C}$ are proportional to mass fraction of proton so flow through $^{12}C(p,\gamma)^{13}N(\beta^+\nu)^{13}C$ is neglig., for lower T typical of lower mass stare $^{13}N$ photodis. is less important and $^{12}C(p,\gamma)^{13}N(\beta^+\nu)^{13}C(\alpha,n)$ may becomes dominant neutron source and increases $\eta$.
    \end{itemize}
\end{frame}

\begin{frame}{Ne-burning ($T\approx\SI{1.5}{\giga\kelvin}$, $\rho\approx\SI{5e6}{\gram\per\cubic\cm}$, $\tau_{Ne}\approx\SI{280}{\day}$)}
    \begin{columns}[T]
        \begin{column}{0.60\textwidth}
            \begin{align*}
                &^{20}Ne(\gamma,\alpha)^{16}O\tag{$Q=\SI{-4.7}{\mega\ev}$}\\
                &^{20}Ne(\alpha,\gamma)^{24}Mg(\alpha,\gamma)^{28}Si&\tag{$Q=\SI{9.3}{\mega\ev}$}\\
                &&\tag{$Q=\SI{10}{\mega\ev}$}\\
                &^{23}Na(\alpha,p)^{26}Mg(\alpha,n)^{29}Si&\tag{$Q=\SI{1.8}{\mega\ev}$}\\
                &&\tag{$Q=\SI{34}{\kilo\ev}$}
            \end{align*}
            Most important energy generating reactions: $^{20}Ne(\gamma,\alpha)^{16}Ne$, $^{20}Ne(\alpha,\gamma)^{24}Mg$ ($^{20}Ne+^{20}Ne\to^{16}O+^{24}Mg+\SI{4.6}{\mega\ev}$), each $Ne+Ne$ liberates $\bar{Q}_{Ne}\approx\SI{6.2}{\mega\ev}$ near $T\approx\SI{1.5}{\giga\kelvin}$.
            \begin{align*}
                &\epsilon_{Ne}\approx\num{6.24e33}\frac{(X_{^{20}Ne})^2}{X_{^{16}O}}T_9^{\frac{3}{2}}\exp{-\frac{54.89}{T_9}}N_A\exv{\sigma v}_{^{20}Ne(\alpha,\gamma)}\si{\mega\ev\per\gram\per\second}\\
                &\epsilon_{Ne}(T)=\epsilon_{Ne}(T_0)(\frac{T}{T_0})^{49}\tag{$T_0\approx\SI{1.5}{\giga\kelvin}$}\\
            \end{align*}
            Approx factor 3 less energy produced for same amount of consumed fuel comp. to C-Burning
        \end{column}
        \begin{column}{0.40\textwidth}
            At end of C-burning core is made of $^{16}O$, $^{20}Ne$, $^{23}Na$, $^{24}Mg$: p,n,$\alpha$ separation energy is \SIrange{7}{17}{\mega\ev}, while for $^{20}Ne$ is \SI{4.7}{\mega\ev} - $\lambda_{\gamma}(^{20}Ne)\approx\SI{1.5e-6}{\per\second}$: $^{20}Ne$ will photodisintegrate and liberated $\alpha$ will induce secondary reactions.
\begin{figure}[!ht]
    \includegraphics[trim={0cm 0cm 0cm 0cm},clip, keepaspectratio,width=0.9\textwidth]{NeBurning-RR}\label{fig:NeBurning-RR}
\end{figure}
        \end{column}
    \end{columns}
    
            \begin{align*}
                &\int\epsilon_{Ne}(t)\,dt=\frac{N_A\bar{Q}_{Ne}}{2M_{^{20}Ne}}\Delta X_{^{20}Ne}=\num{9.32e22}\Delta X_{^{20}Ne}\si{\mega\ev\per\gram}
            \end{align*}
            Approx factor 3 less energy produced for same amount of consumed fuel comp. to C-Burning
\end{frame}

\begin{frame}{Oxigen Burning ($25\msun{}$, $T\approx\SI{2.2}{\giga\kelvin}$, $\rho\approx\SI{3e6}{\gram\per\cubic\cm}$, $\tau_O\approx\SI{162}{\day}$)}
    \begin{columns}[T]
        \begin{column}{0.45\textwidth}
            Neon exhausted: in stellar core we have $^{16}O$, $^{24}Mg$, $^{28}Si$ - $^{16}O+^{16}O\to ^{32}S$ has lowest coulomb barrier: the compound nucleus is highly excited with mass difference approx \SI{16.5}{\mega\ev}. Primary reactions:
            \begin{align*}
                &^{16}O(^{16}O,p)^{31}P\tag{$Q=\SI{7.7}{\mega\ev}$}\\
                &^{16}O(^{16}O,2p)^{30}Si\tag{$Q=\SI{0.4}{\mega\ev}$}\\
                &^{16}O(^{16}O,\alpha)^{28}Si\tag{$Q=\SI{9.6}{\mega\ev}$}\\
                &^{16}O(^{16}O,2\alpha)^{24}Mg\tag{$Q=\SI{-0.4}{\mega\ev}$}\\
                &^{16}O(^{16}O,d)^{30}P\tag{$Q=\SI{-2.4}{\mega\ev}$}\\
                &^{16}O(^{16}O,n)^{31}S\tag{$Q=\SI{1.5}{\mega\ev}$}
            \end{align*}
        \end{column}
        \begin{column}{0.55\textwidth}
\begin{figure}[!ht]
    \includegraphics[trim={0cm 0cm 0cm 0cm},clip, keepaspectratio,width=0.5\textwidth]{OBurning-S}\label{fig:OBurning-S}~
    \includegraphics[trim={0cm 0cm 0cm 0cm},clip, keepaspectratio,width=0.5\textwidth]{OBurning-Odepl}\label{fig:OBurning-Odepl}
    
\end{figure}
        \end{column}
    \end{columns}
    Energy sep. for n,p, $\alpha$ are approx. \SI{9}{\mega\ev} except for $^{16}O(\gamma,\alpha)^{12}C$ which is \SI{7.2}{\mega\ev}: $^{16}O$ fusion is major O depletion agent for $T<\SI{4}{\giga\kelvin}$. Proton exit channel dominates at all T: at $T=\SI{2.2}{\giga\kelvin}$ reaction rate contrib. are for p ($62\%$), $\alpha(21\%)$, n($17\%$); secondary reactions contrib. sign. to energy release, $\bar{Q}_O\approx\SI{17.2}{\mega\ev}$ for each $^{16}O+^{16}O$. Electron screening factor $\approx1.3$.
    \begin{align*}
        &\epsilon_O=\frac{\bar{Q}_O}{\rho}r_{^{16}O^{16}O}=\frac{N_A\bar{Q}_O}{512}X_{^{16}O}^2\rho N_A\exv{\sigma v}_{^{16}O^{16}O}=\num{2.03e22}X^2_{^{16}O}\rho N_A\exv{\sigma v}_{^{16}O^{16}O}\si{\mega\ev\per\gram\per\second}\\
        &\epsilon_O(T)=\epsilon_O(T_0)(\frac{T}{T_0})^{34},\ \int\epsilon_O(t)\,dt=\frac{N_A\bar{Q}_O}{2M_{^{16}O}}\Delta X_{^{16}O}=\num{3.24e23}\Delta X_{^{16}O}\si{\mega\ev\per\gram}
    \end{align*}
\end{frame}

\begin{frame}{Nucleosynthesis during O-Burning}
    \begin{columns}[T]
        \begin{column}{0.35\textwidth}
\begin{figure}[!ht]
    \includegraphics[trim={0cm 0cm 0cm 0cm},clip, keepaspectratio,width=0.95\textwidth]{OBurning-nucl}\label{fig:OBurning-nucl}
\end{figure}
Quasi-equilibrium clusters: several pairs at equil. between forw./reverse photodiss. rate: $A=24-46$, iron peak.
        \end{column}
        \begin{column}{0.65\textwidth}
            At end of Ne burning most abb. species are $^{16}O(X_i=0.77)$, $^{24}Mg(X_i=0.11)$, $^{28}Si(X_i=0.083)$. As O-burn $^{28}Si$, $^{32}S$ increases, also $^{34}S$, $^{35}Cl$, $^{36}Ar$, $^{38}Ar$, $^{39}K$, $^{40}Ca$, $^{42}Ca$ also increase.
            \begin{itemize}
                \item Production of $^{28}Si$, $^{32}S$
                    \begin{align*}
                        &^{16}O(^{16}O,p)^{31}P(p,\gamma)^{32}S\\
                        &^{16}O(^{16}O,p)^{31}P(p,\alpha)^{28}Si\\
                        &^{16}O(^{16}O,\alpha)^{28}Si\\
                        &^{16}O(^{16}O,n)^{31}S(\gamma,p)^{30}P(\gamma,p)^{29}Si(\alpha,n)^{32}S
                    \end{align*}
                \item $^{31}S$, $^{30}P$ have rel. low sep energy ($S_p\approx\SI{6.1}{\mega\ev},\SI{5.6}{\mega\ev}$): photodisintegration domin. over $^{31}S(\beta^+\nu)^{31}P$ still relevant.
                \item Sulfur-32: $^{28}S(\alpha,\gamma)^{32}S$ and converted back via $^{32}S(n,\gamma)^{33}S(n,\alpha)^{30}Si(p,\gamma)^{31}P$ or converted to heavier $^{32}S(\alpha,p)^{35}Cl(p,\gamma)^{36}Ar$ and so on.
                \item Depletion of $^{24}Mg$ via liberated $\alpha$: $^{24}Mg(\alpha,\gamma)^{28}Si$ and $^{24}Mg(\alpha,p)^{27}Al$.
                \item Neutron excess increases by factor 5: $^{31}S(\APelectron\nu)^{31}P$, $^{30}P(\APelectron\nu)^{30}Si$, and \Pelectron-capture $^{33}S(\Pelectron,\nu)^{33}P$, $^{35}Cl(\Pelectron,\nu)^{35}S$, $^{37}Ar(\Pelectron,\nu)^{37}Cl$.
                \item Explosive O-burning is bel. major source of $^{28}Si$, $^{32,33,34}S$, $^{35}Cl$, $^{36,38}Ar$, $^{39,41}K$, $^{40,42}Ca$
                \item Reduced lifetime against $\beta$-decay
                \item At end of O-burning T is suff. high to produce light particles via photodiss.
            \end{itemize}
        \end{column}
    \end{columns}
\end{frame}

\begin{frame}{Si-Burning ($25\msun{}$, $T\approx\SI{3.6}{\giga\kelvin}$, $\rho\approx\SI{3e7}{\gram\per\cubic\cm}$, $\tau_{Si}\approx\SI{4000}{\second}$)}
    \begin{columns}[T]
        \begin{column}{0.35\textwidth}
\begin{figure}[!ht]
    \includegraphics[trim={0cm 0cm 0cm 0cm},clip, keepaspectratio,width=0.95\textwidth]{SiBurning-chains}\label{fig:SiBurning-chains}
    \includegraphics[trim={0cm 0cm 0cm 0cm},clip, keepaspectratio,width=0.95\textwidth]{SiBurning-lambda}\label{fig:SiBurning-lambda}
\end{figure}
            
        \end{column}
        \begin{column}{0.65\textwidth}
            Photodis. rearrangement similar to Ne-burning but on larger scale: $^{28}Si+^{28}Si$ and $^{28}Si+^{32}S$ have high Coulomb barrier, so reactions go via photodis. of less bound nuclei and captures of light part.
            \begin{align*}
                &S_p(^{32}S)\approx\SI{8.9}{\mega\ev}, S_n(^{32}S)\approx\SI{15}{\mega\ev}, S_{\alpha}(^{32}S)\approx\SI{10}{\mega\ev}\\
                &S_p(^{28}Si)\approx\SI{11.6}{\mega\ev}, S_n(^{28}Si)\approx\SI{17.2}{\mega\ev}, S_{\alpha}(^{28}Si)\approx\SI{10}{\mega\ev}
            \end{align*}
            Above $T=\SI{2}{\giga\kelvin}$ $^{32}S$ will be consumed via: $^{32}S(\gamma,\alpha)^{28}P$, $^{32}S(\gamma,p)^{31}P(\gamma,p)^{30}Si(\gamma,n)^{29}Si(\gamma,n)^{28}Si$; T increases further to becomes disintegration of $^{28}Si$ ($^{28}Si(\gamma,p)^{27}Al$, $^{28}Si(\gamma,\alpha)^{24}Mg$: energy sep. for latter is much smaller, Coulomb barrier for ejected particles, reduced particle width of res).
            \begin{align*}
                &\epsilon_{Si}=\frac{Q_{2^{28}Si\to^{56}Ni}}{\rho}r_{2^{28}Si\to^{56}Ni}\approx \frac{Q_{2^{28}Si\to^{56}Ni}}{\rho}r_{^{24}Mg(\gamma,\alpha)^{20}Ne}\\
                &=\frac{Q_{2^{28}\to^{56}Ni}}{\rho}N_{^{24}Mg}\lambda_{\gamma\alpha}(^{24}Mg)\\
                &\epsilon_{Si}\approx\num{1.2985e34}X_{^{28}Si}(\frac{2X_{^{28}Si}}{1-X_{^{28}Si}})^{\frac{1}{7}}\exp{-\frac{142.12}{T_9}}T_9^{\frac{3}{2}}N_A\exv{\sigma v}_{^{20}Ne}\si{\mega\ev\per\gram\per\second}\\
                &\epsilon_{Si}(T)=\epsilon_{Si}(T_0)(\frac{T}{T_0})^{47}\tag{$T_0=\SI{3.6}{\giga\ev}$}\\
                &\bar{Q}_{Si}\approx Q_{2^{28}Si\to ^{56}Fe}=\SI{17.62}{\mega\ev}\\
                &\int\epsilon_{Si}(t)\,dt=\frac{N_A\bar{Q}_{Si}}{2M{^{28}Si}}\Delta X_{^{28}}=\num{1.9e23}\Delta X_{^{28}Si}
            \end{align*}
        \end{column}
    \end{columns}
\end{frame}

\begin{frame}{Si-Burning: Abundance flows}
    \begin{columns}[T]
        \begin{column}{0.4\textwidth}
\begin{figure}[!ht]
    \includegraphics[trim={0cm 0cm 0cm 0cm},clip, keepaspectratio,width=0.95\textwidth]{SiBurning-abbflows }\label{fig:SiBurning-abbflows }
\end{figure}
        \end{column}
        \begin{column}{0.6\textwidth}
            \begin{itemize}
                \item Thermally excited levels have big effect on decay const. for weak interactions and rates of many for./rever. reactions.
                \item Quasiequilibrium clusters: $A=25-40$, $A=46-64$
                \item At end of Si-burning stellar matter is $^{56}Fe$ ($X_f=0.56$), $^{52}Cr(X_f=0.19)$, $^{54}Fe(X_f=0.11)$, $^{55}Fe(X_f=0.05)$, $^{53}Mn(X_f=0.034)$
                \item Neutron excess remains almost constant until iron peak is reached then increases sign.: $^{53}Mn(\Pelectron,\nu)^{53}Cr$, $^{54}Fe(\Pelectron,\nu)^{54}Mn$, $^{55}Fe(\Pelectron,\nu)^{55}Mn$, $^{55}Co(\Pelectron,\nu)^{55}Fe$. At end $\eta_f\approx0.067$
            \end{itemize}
        \end{column}
    \end{columns}
\end{frame}

\begin{frame}{Nuclear Statistical equilibrium and $\alpha$ freeze-out}
    \begin{columns}[T]
        \begin{column}{0.4\textwidth}
            
        \end{column}
        \begin{column}{0.6\textwidth}
            \begin{itemize}
                \item Every nuclide in the network is in equilibrium: NSE.
                \item While shock wave after massive stars explode move outward stellar matter in NSE cools and expands: denote $T_{\alpha}$ temperature at which $\alpha$ particles are not at equilibrium.
                \item If $T_{\alpha}$ high such that few $\alpha$ presents:  When T falls below $T_{\alpha}$ photodisintegration fails to produce enough $\alpha$ particles. Particles-poor freeze-out.
                \item If $T_{\alpha}$ low such that abb. of $\alpha$ is large: as T falls toward $T_{\alpha}$ free $\alpha$ tends to merge into iron peak $3\alpha\to^{12}C(\alpha,\gamma)^{16}O\ldots^{52}Fe(\alpha,\gamma)^{56}Ni$; when $T<T_{\alpha}$ suff. rapidly $\alpha$ can't be converted rapidly enough: excess of $\alpha$ particles. $\alpha$-rich freeze-out.
            \end{itemize}
        \end{column}
    \end{columns}
    
\end{frame}


\part{Modelli stellari: struttura ed evoluzione}\linkdest{evolutivemodels}
\begin{frame}{this part toc}
%\begin{columns}[T]
%\begin{column}{0.5\textwidth}
\begin{itemize}
\item Eqs struttura stellare
\item Meccanismi di trasporto e stabilit\'a
\item Fenomenologia del plasma stellare
\item Luminosity production. Nuclear reactions
\item Metodi soluzione modello stellare
\end{itemize}
%\end{column}
%\begin{column}{0.5\textwidth}
%\tableofcontents
%\end{column}
%\end{columns}
\end{frame}
\section{Equazioni struttura stellare}\linkdest{stellarmodel}

\begin{wordonframe}{da fare: kippenhahn wiegert}
\begin{itemize}
\item strutture autogravitanti 1-62' (43). EQuilibrio idrostatico, vento stellare, stabilit\'a e pulsazioni
\item metodi numerici 77'-84(44-48)
\item esistenza e unicit\'a 85'-99'(48-56)
\item Properties of stellar matter: ideal gas with radiation, ionization, degenerate electron gas, equazione di stato, opacit\'a 102'-144'(57-78)
\item produzione energia reazioni nucleari:  146'-172'(79-92)
\item politrope 174'-190' (93-102)
\end{itemize}
\end{wordonframe}

\subsection{Struttura di equilibrio}\linkdest{stellarstructure}

\begin{frame}{Equazioni struttura di equilibrio}

\begin{align*}
&\TDy{m}{r}=\frac{1}{4\pi r^2\rho}\\
&\TDy{m}{P}=-\frac{Gm}{4\pi r^4}\overbrace{[-\frac{1}{4\pi r^2}\PtwoDy{t}{r}]}^{\tau_{hyd}=\frac{1}{2}(G\exv{\rho})\expy{-1/2}}\\
&\TDy{m}{T}=-\nabla\frac{T}{p}\frac{Gm}{4\pi r^4}\\
&\TDy{m}{L}=\epsilon-\epsilon_{\nu} \underbrace{-c_P[\TDy{t}{T}-\nad\frac{T}{P}\TDy{t}{P}]}_{-c_P\PDy{t}{T}+\frac{\delta}{\rho}\PDy{t}{P}}\\
&\TDy{t}{X_s}\frac{1}{A_s}=\sum_{production}\rho^{n_h+n_k-1}n_p\frac{X_h^{n_h}X_k^{n_k}}{A_h^{n_h}A_k^{n_k}}\frac{\exv{\sigma v}_{hk}}{m_H^{n_h+n_k-1}n_h!n_k!}\\
&-\sum_{distruction}\rho^{n_d+n_j-1}n_d\frac{X_s^{n_d}X_j^{n_j}}{A_s^{n_d}A_j^{n_j}}\frac{\exv{\sigma v}_{sj}}{m_H^{n_d+n_j-1}n_d!n_j!}
\end{align*}
\end{frame}

\begin{frame}{Conservazione massa e HE}
\begin{columns}[T]
	\begin{column}{0.5\textwidth}
\begin{align*}
&\TDy{m}{r}=\frac{1}{4\pi r^2\rho}\\
&\TDy{m}{P}=-\frac{Gm}{4\pi r^4}\overbrace{[-\frac{1}{4\pi r^2}\PtwoDy{t}{r}]}^{\tau_{hyd}=\frac{1}{2}(G\exv{\rho})\expy{-1/2}}\\
&\tau_{hyd}=\sqrt{\frac{R^3}{GM}}
\end{align*}
\end{column}\begin{column}{0.5\textwidth}
\begin{align*}
&\TDy{r}{m}=4\pi r^2\rho\\
&\TDy{r}{P}=-\frac{Gm}{r^2}\rho\overbrace{[-\rho\PtwoDy{t}{r}]}^{\tau_{hyd}=\frac{1}{2}(G\exv{\rho})\expy{-1/2}}\\
&\tau_{ff}=\sqrt{\frac{R}{g}}\\
&\tau_{exp}=R\sqrt{\frac{\rho}{P}}
\end{align*}
\end{column}
\end{columns}
Red giant: $\tau_H\approx\SI{5}{\day}$ (time for the shock to cross a supergiant star making a SNII, Cepheid); Sun: $\tau_H\approx\SI{38}{\minute}$; WD: $\tau_H\approx\SI{2}{\second}$
\end{frame}

\begin{frame}{Trasporto energia verso la superficie e gradiente termico}
\begin{columns}[T]
	\begin{column}{0.6\textwidth}
		\begin{align*}
		&\TDy{m}{T}=-\nabla\frac{T}{p}\frac{Gm}{4\pi r^4}\\
		&\nabla=\nrad=\frac{3\kappa_R}{16\pi acG}\frac{LP}{mT^4}\tag*{$\nrad\leq\nad$}\\
		&\nabla=\nad+\Delta\nabla\tag*{$\nrad>\nad-\frac{\chi_{\mu}}{\chi_T}\nmu$}
		\end{align*}
	\end{column}\begin{column}{0.4\textwidth}
		\begin{align*}
		&\TDy{r}{T}=\nabla\frac{T}{p}\TDy{r}{p}\\
		&\nabla=\TDly{P}{T}
		\end{align*}
	\end{column}
\end{columns}
\begin{align*}
&dP_{rad}=-dp=-\frac{dF_{Rad}}{c}=-\frac{F_{rad}}{c}\frac{dr}{l}=-\frac{F_{rad}}{c}\kappa_R\rho\,dr=\frac{4}{3}aT^3dT\\
&dP_{rad}(\nu)=-\frac{F_{rad}(\nu)}{c}\kappa_{\nu}\rho\,dr=\frac{4\pi}{3c}\TDy{r}{B_{\nu}(T)}\,dr\\
&B_{\nu}(T)=\frac{2h\nu^3}{c^2}\frac{1}{\exp{\frac{h\nu}{kT}}-1}
\end{align*}
\end{frame}

\begin{frame}{Equazioni struttura stellare: conservazione energia - Luminosit\'a}
\begin{columns}[T]
\begin{column}{0.4\textwidth}
\begin{align*}
&\TDy{m}{L}=\epsilon-\epsilon_{\nu} \underbrace{-c_P[\TDy{t}{T}-\nad\frac{T}{P}\TDy{t}{P}]}_{-c_P\PDy{t}{T}+\frac{\delta}{\rho}\PDy{t}{P}0-T\PDy{t}{s}}\\
&\epsilon_{gr}=-T\PDy{t}{s}=-\TDof{t}u+\frac{P}{\rho^2}\TDy{t}{\rho}\\
&\epsilon_{gr}>0\tag*{contrazione}
\end{align*}
\end{column}\begin{column}{0.6\textwidth}
\begin{align*}
&\TDy{r}{L}=4\pi r^2[\rho(\epsilon-\epsilon_{\nu})-\rho\TDof{t}u+\frac{P}{\rho}\TDy{t}{\rho}]\\
&
\end{align*}
\end{column}
\end{columns}
\end{frame}

\begin{wordonframe}{Chemical evolution: nuclear burning, diffusion and convective mixing}\linkdest{diffusion}\

\end{wordonframe}

\begin{frame}{Trasporto radiativo e convettivo: \'e valida approx idrostatica}

\end{frame}

\subsection{Relazioni approssimate per grandezze stellari fondamentali}\linkdest{omrel}

\begin{frame}{Teorema del viriale}
\begin{columns}[T]
	\begin{column}{0.5\textwidth}
		\begin{align*}
&\Omega=-\int_0^M\frac{Gm(r)}{r}\,dm\\
&\frac{1}{2}\TtwoDy{t}{I}=2E_i+\Omega\tag{T. viriale}\\
&0=\int_M\frac{3P}{\rho}\,dm(r)+\Omega\tag{stationary}\\
&E_i=\frac{1}{\gamma-1}\frac{P}{\rho}
		\end{align*}
	\end{column}\begin{column}{0.5\textwidth}
		\begin{align*}
&W=E_i+\Omega \tag{total E}\\
&\TDy{t}{W}+L=0\tag{E conservation}\\
&L=-\frac{1}{2}\dot{\Omega}=\dot{E}_i\\
&E_T=E_i+\Omega=\frac{3\gamma-4}{3(\gamma-1)}\Omega\\
&\gamma>4/3\tag{stability}
		\end{align*}
	\end{column}
\end{columns}
Nel caso in cui la contrazione gravitazionale sia l'unica fonte di energia per una massa gassosa in equilibrio idrostatico, il suo tempo di evoluzione caratteristico \'e il tempo di \kh{} $\tkh{}=\frac{\Omega}{L}\approx\frac{GM^2}{2RL}$
\end{frame}

\begin{frame}{Relazione massa, densit\'a, temperatura/ massa, peso molecolare, opacit\'a}

\end{frame}

\section{Trasporto}\linkdest{transport}

\subsection{Trasporto radiativo}\linkdest{trarad}

\begin{frame}{Trasporto radiativo}
media di rosseland
\end{frame}

\subsection{Conduzione}

\begin{frame}{Diffusione per conduzione}
stima opacit\'a per conduzione gas degenere NR
\end{frame}

\begin{frame}{Conduzione elettronica}
\begin{columns}[T]
\begin{column}{0.5\textwidth}
	\begin{align*}
	&F_e=-N_evl\TDy{r}{E}=-N_ekvl\TDy{r}{T}\\
	&E_T\approx\frac{3}{2}kT,\ v_T\approx\sqrt{\frac{2E_T}{m}}
	\end{align*}
\end{column}
\begin{column}{0.5\textwidth}
	\Pelectron degenerate are forced in higher momentum state - $P_F=2\pi\hbar(\frac{3}{4\pi g})\expy{1/3}n\expy{1/3}$
\end{column}
\end{columns}
\end{frame}

\subsection{Opacit\'a}\linkdest{kapparad}

\begin{frame}{Opacit\'a radiativa (analitico)}
Scattering elettronico, processi ff, fotoionizzazione; opacit\'a atmosfera: ione H-
\end{frame}

\begin{frame}{Andamento opacit\'a: electron scattering e Kramer's opacity (FF)}
\begin{itemize}
\item Electron scattering $\rho\kappa_{\nu}=n_e\frac{8\pi}{3}(\frac{e^2}{m_ec^2})^2=0.2(1+X)\si{\square\cm\per\gram}$ - $\sigma_T=\SI{0.66e-24}{\square\cm}$ - for $T$ maggiore di few milion K is dominant source - Compton scattering $h\nu|_M\gtrsim0.1 m_ec^2$, $h\nu|_M\approx4.96kT$: Compton scattering reduces opacity about $20\%$ for $T>\SI{e8}{\kelvin}$
\item Kramers opacity: $T<\SI{e7}{\kelvin}$ -when FF, BF dominates: $\kappa_R\propto\rho T\expy{-7/2}$
\end{itemize}
\begin{block}{Free-free: radiation boost free-\Pelectron from lower to higher state}
Fully ionized-$Z_i$ mixture: 
\begin{align*}
&\rho\kappa_{\nu}(ff)=\sum_in_{Z_i}ne\sqrt{\frac{2m_e}{3\pi kT}}[\frac{4\pi Z_i^2e^6}{3m_e^2ch\nu^3}]g_{ff}(\nu)(1-\exp{-h\nu/kT})\\
&=\num{3.8e22}\overbrace{(X+1)}^{\propto n_e}\overbrace{[X+Y+B]}^{\propto\frac{X}{m_u}+\frac{4Y}{4m_u}+\sum_i\frac{X_iZ_i^2}{A_i}}\ [\si{cgs}]
\end{align*}
$\tau_{\Pelectron-int}\propto\frac{1}{v}\propto T\expy{-1/2}$: \Pelectron experience sharp acceleration $\approx\delta$ resulting in constant emission frequency, two body ion-\Pelectron encounter $\propto n_in_e$, \Pelectron have thermal distro $\propto\exp{-\epsilon/kT}$: $\rho j_{\nu}=n_en_iT\expy{-1/2}\exp{-\frac{h\nu}{kT}}$ - dalla legge di Kirchhoff, $j_{\nu}=4\pi\kappa_{\nu}^{abs}B_{\nu}(T)$, we have $\kappa_{\nu}^{abs}\propto\rho T\expy{-1/2}\nu\expy{-3}[1-\expy{-h\nu/kT}]$
\end{block}
\end{frame}

\begin{frame}{Kramer's opacity (BF, BB) and $H^-$ ions opacity}
\begin{block}{BF}
Hydrogenic atom with charge Z and one \Pelectron in bound state n:
\begin{align*}
&\sigma_{\nu}(Z)=n\expy{-5}\frac{8\pi}{3\sqrt{3}}\frac{Z^4m_ee\expy{10}}{c\hbar^3(h\nu)^3}:\ h\nu>\chi=\frac{Ze^2}{2a_Z}\\
&\rho\kappa_{\nu}(bf)=\sum_in_{Z_i}\sigma_{\nu}(Z_i)(1-\exp{-h\nu/kT})\\
&\kappa_R(bf)=\num{3e25}\overbrace{1-X-Y}^{n_{\exv{Z}}}\overbrace{1+X+\frac{3}{4}Y}^{n_e}\rho T\expy{-7/2}
\end{align*}
\end{block}
\begin{block}{BB: usually neglected in stellar interior}
\end{block}
\begin{block}{$H^-$ opacity $T<\SIrange{e4}{e5}{\kelvin}$}
BB/BF transition for $H^-$ don't follow Kramer as ion abundance in solar photosphere is sensitive to other considerations: law of mass action for $H^-=H+\Pelectron$ is $\frac{n_{H^-}}{n_Hn_e}=K_1(T)$, and sources of free electrons are metals (Na, K) $M^++\Pelectron=M$ so $\frac{n_{M^+}n_e}{n_M}=K_2(T)$, supposing $n_e=n_{M^+}$:
\begin{align*}
&\rho\kappa_{H^-}^{ff}\propto n_{H^-}n_eT\expy{-7/2}=n_Hn_MT\expy{-7/2}K_1(T)K_2(T)
\end{align*}
\end{block}
\end{frame}

\subsection{Convezione}\linkdest{traconv}

\begin{wordonframe}{Forza di archimede}
Una regione stellare \'e convettivamente stabile se una perturbazione di densit\'a infinitesima non cresce ad ampiezza finita.
\begin{equation*}\label{eq:buoyancyEOM}
\rho\PtwoDy{t}{(\Delta r)}=-g\Delta\rho=-g[\Dcvar{\TDy{r}{\rho}}{e}-\Dcvar{\TDy{r}{\rho}}{amb}]\Delta r
\end{equation*}
La forza di Archimede ha verso opposta alla perturbazione se
\begin{equation*}
[\Dcvar{\TDy{r}{\rho}}{e}-\Dcvar{\TDy{r}{\rho}}{amb}]>0
\end{equation*}
Riscrivo prima legge della termodinamica come $dq=c_P\,dT-\frac{\delta}{\rho}\,dP$
\begin{equation*}
N^2=g(\frac{1}{\Gamma_1P}\TDy{r}{P}-\frac{1}{\rho}\TDy{r}{\rho})=g(\frac{1}{\densityscale{}}-\frac{g}{c_s^2})\label{eq:bvfs}
\end{equation*}
$N^2$ rappresenta la massima frequenza sotto cui pu\'o oscillare una particella di fluido sottoposta a onde di gravit\'a mantenendo l'equilibrio di pressione con l'ambiente.
\begin{equation*}
\PtwoDy{t}{(\Delta r)}=-N^2\Delta r
\end{equation*}
che descrive un comportamento oscillatorio per $N^2>0$
\end{wordonframe}

\begin{frame}{Convective regions and temperature gradient}
\begin{columns}[T]
	\begin{column}{0.6\textwidth}
\begin{block}{Criterio di \sch e Ledoux: regioni convettive}
	\begin{align*}
	&\rho\PtwoDy{t}{(\Delta r)}=-g\Delta\rho=-g[\Dcvar{\TDy{r}{\rho}}{e}-\Dcvar{\TDy{r}{\rho}}{amb}]\Delta r\\
	&\nrad{}>\nad+\frac{\phi}{\delta}\nmu{}=\nad-\frac{\chi_{\mu}}{\chi_T}\TDly{(P)}{(\mu)}\\
	&\nrad{}>\nad\\
	&\frac{d\rho}{\rho}=\alpha\frac{dP}{P}-\delta\frac{dT}{T}+\phi\frac{d\mu}{\mu}\\
	&=-\frac{\chi_T}{\chi_{\rho}}\frac{\Delta T}{T}-\frac{\chi_{\mu}}{\chi_{\rho}}\frac{\Delta\mu(\Lambda)}{\mu}\\
	&\Delta P=0=\chi_{\rho}\frac{\Delta\rho}{\rho}+\chi_T\frac{\Delta T}{T}+\chi_{\mu}\frac{\Delta\mu}{\mu}
	\end{align*}
\end{block}
	\end{column}
	\begin{column}{0.4\textwidth}
\begin{figure}[!ht]
	\includegraphics[trim={0cm 0cm 1cm 0cm},clip, keepaspectratio,width=0.99\textwidth]{convectivestability}\label{fig:convectivestability}
\end{figure}
\end{column}\end{columns}
\end{frame}

\begin{frame}{Mixing length: gradiente ambientale e velocit\'a elementi convettivi}
Le stelle con massa $M\leq1.1\msun{}$ hanno una regione radiativa interna mentre la parte esterna \'e convettiva. Convezione esterni $\nabla>\nad$, $v\approx1-10\si{\kilo\meter\per\second}\approx c_s$- $\nabla\to\nad{}$ in interni stellari convettivi: $\TDy{r}{T}=(1-\frac{1}{\Gamma_2})\frac{T}{P}\TDy{r}{P}$, $v\approx\SI{100}{\meter\per\second}\ll c_s$
	\begin{align*}
&F=\frac{L}{4\pi r^2}=F_{rad}+F_{con}=-\frac{4acT^3}{3\kappa\rho}\TDy{r}{T}|_{amb}+\frac{1}{2}\rho vc_p[\TDy{r}{T}|_{Ad}-\TDy{r}{T}|_{amb}]\Lambda\\
&v^2=-\frac{1}{8}g\frac{\Delta\rho}{\rho}\Lambda=\frac{1}{8}g\frac{\Lambda}{H_P}Q(\nabla-\nad),\ Q=1-\TDly{T}{\mu},\ \Lambda=\alpha H_P\\
&F_{con}^{up}=\frac{1}{2}\rho vc_PT\frac{\lambda}{H_P}(\nabla-\nad{})
\end{align*}
\begin{columns}[T]
\begin{column}{0.2\textwidth}
\begin{align*}
&f_r&=-g\Delta\rho(r)=0\tag*{r}\\
&&\propto\Delta r
\end{align*}
\end{column}
\begin{column}{0.8\textwidth}
Work done per unit volume moving bubble of $\Delta r$
\begin{align*}
&W(\Delta r)=-g\int_0^{\Delta r}\Delta\rho(\Delta r')d(\Delta r')=-\frac{1}{2}g\Delta\rho(\Delta r)\Delta r\\
&\exv{W(\Delta r)}_{\Delta r}=\frac{1}{4}W(\Lambda)=\frac{1}{2}\rho v^2
\end{align*}
\end{column}\end{columns}
%Calcolo altezza scala di pressione, flusso convettivo e gradiente ambientale in funzione della velocit\'a media degli elementi
\end{frame}

\begin{wordonframe}{EOS e $\nad{}$}
Considero un'equazione di stato generica $\rho(P,T,\mu)$ e definita tramite:
\begin{align*}
&\frac{d\rho}{\rho}=\alpha\frac{dP}{P}-\delta\frac{dT}{T}+\phi\frac{d\mu}{\mu}\\
&P=\frac{\rho\gasconstant{}T}{\mu}\quad\Rightarrow\quad\alpha=\delta=\phi=1
\end{align*}
Definisco le lunghezze caratteristiche per variazione di densit\'a e pressione:
\begin{equation*}
\densityscale{}=-\frac{dr}{d\ln{\rho}},\ H_P=-\frac{dr}{d\ln{P}}
\end{equation*}
e i gradienti termici per il blob, l'ambiente e il gradiente di composizione chimica ambientale
\begin{equation*}
\nabla=\Dcvar{\TDly{P}{T}}{amb},\ \nabla_e=\Dcvar{\TDly{P}{T}}{e},\ \nmu{}=\Dcvar{\TDly{P}{\mu}}{amb}
\end{equation*}
\end{wordonframe}

\begin{wordonframe}{EOS e EOM}
Riscrivo l'equazione del moto utilizzando l'equazione di stato per scrivere la differenza di densit\'a in termini dei gradienti termici e di composizione chimica; inoltre supponendo il moto dell'elemento in equilibrio di pressione con l'ambiente e assumendo $\nmu{}_{blob}\approx0$ risulta:
\begin{equation*}
\PtwoDy{t}{(\Delta r)}=-g\frac{\delta}{H_P}[\nabla_e-\nabla-\frac{\phi}{\delta}\nmu{}]\Delta r
\end{equation*}
\end{wordonframe}

\begin{wordonframe}{Criterio di \sch/Ledoux}
Infine per ricavare il criterio di stabilit\'a per convezione suppongo  il moto del blob adiabatico:
\begin{equation*}
dq=c_P\,dT-\frac{\delta}{\rho}\,dP
\end{equation*}
da cui risulta:
\begin{equation*}
\nabla_e=\nabla_{ad}=\frac{P\delta}{T\rho c_P}
\end{equation*}
cio\'e una regione solare \'e stabile per convezione se
\begin{equation*}
\nrad{}<\nad+\frac{\phi}{\delta}\nmu{}\label{eq:ledoux}
\end{equation*}
dove ho usato $\nabla_{amb}=\nrad{}$, cio\'e il gradiente che si ha nel caso l'energia sia trasportata dai fotoni.
\end{wordonframe}

\subsection{Teoria della mixing-length.}

\begin{wordonframe}{Convezione in esterni stellari}

In presenza di convezione il flusso di energia verso l'esterno ha una componente radiativa, determinata dal gradiente di temperatura, e una componente dominante convettiva 
\begin{equation*}\label{eq:radconvflux}
F=F_{con}+F_{rad}=\frac{\lsun{}}{4\pi r^2}
\end{equation*}

Una maggiore efficienza del trasporto convettivo di energia si riflette in una minore differenza tra il gradiente di temperature adiabatico ed effettivo.

\begin{figure}[!h]
%   \includegraphics[ width=0.99\textwidth,keepaspectratio]{proportionflux}
%   \subcaption{Profilo radiale (profondit\'a in \si{\kilo\meter}) del flusso convettivo $F_c$ rispetto al flusso totale $F$, della super-adiabaticit\'a $\nabla-\nad{}$ e regioni di ionizzazione idrogeno e $\cel{He}{4}{}{}$. Da \cite{christensen1997effects}.}\label{fluxproportion}

%\includegraphics[keepaspectratio,width=0.9\textwidth]{specificheatnablaa}
%\subcaption{Profilo radiale di $c_P$ e $\nabla_a$: si ha cambiamento di comportamento nelle regioni di ionizzazione parziale di idrogeno ed elio. Da \cite{stix91sun}.}\label{specificheatnablaa}
\end{figure}

Per determinare il gradiente di temperatura effettivo $\nabla$ uso la teoria della mixing-length:
si considera l'eccesso di calore trasportato dai blob di gas nel moto convettivo $c_P\Delta T$ rispetto all'ambiente, il cui cammino libero medio \'e la mixing-length $l_m=\alpha H_P$, che da luogo al flusso di energia
\begin{equation*}
F_{con}=\exv{\rho vc_P\Delta T}\label{eq:convectiveflux}
\end{equation*}
dove $\exv{}$ indica una media opportuna sulla sfera di raggio r. Determino il valor medio della differenza di temperatura prendendo come valore caratteristico dello spostamento del blob $\Delta r\approx\frac{l_m}{2}$:
%, considerando moti in entrambi i versi,
\begin{equation*}
\frac{\Delta T}{T}\approx\frac{1}{T}\PDy{r}{(\Delta T)}\frac{l_m}{2}=(\nabla-\nabla_e)\frac{l_m}{2}\frac{1}{H_P}\label{eq:blobambdiff}
\end{equation*}

Assumo il lavoro medio fatto dalla forza di galleggiamento per unit\'a di massa $-g\frac{\Delta\rho}{\rho}$ uguale al valore medio della forza, cio\'e la met\'a di quello alla superficie sferica data, moltiplicato lo spostamento medio $\frac{l_m}{2}$ quindi, assumendo in oltre che in media met\'a del lavoro fatto dalla forza di galleggiamento sia trasformato in energia cinetica del blob si ottiene
\begin{equation*}
v^2=g\delta(\nabla-\nabla_e)\frac{l_m^2}{8H_P}\label{eq:blobvelocity}
\end{equation*}

Infine determino gli scambi radiative del blob: il modulo del flusso radiativo \'e proporzionale al gradiente termico in direzione normale alla superficie del blob
\begin{equation*}
f=\frac{4acT^3}{3\kappa\rho}|\PDy{n}{T}|
\end{equation*}
quindi l'energia scambiata dall'intera superficie S del blob \'e $\lambda=Sf$ che determina, per la prima legge della termodinamica, una variazione di temperatura per unit\'a di tempo:
\begin{equation*}
\PDy{t}{T_e}=-\frac{\lambda}{\rho Vc_P}
\end{equation*}
indicato con $V$ il volume del blob.

La variazione della temperatura del blob per unit\'a distanza percorsa \'e quindi
\begin{equation*}
\Dcvar{\TDy{r}{T}}{e}=\Dcvar{\TDy{r}{T}}{ad}-\frac{\lambda}{\rho Vc_Pv}\label{eq:Tchangelength}
\end{equation*}
e approssimando il gradiente normale alla superficie con $\exv{\Delta T}$ ed usando le definizioni \eqref{eq:nablavitense} si ottiene:
\begin{equation*}
\frac{\nabla_e-\nad{}}{\nabla-\nabla_e}=\frac{6acT^3}{\kappa\rho^2c_Pl_mv}
\end{equation*}
Il gradiente termico ambientale $\nabla$ e del blob $\nabla_e$ sono determinati da \eqref{eq:radconvflux} e \eqref{eq:Tchangelength} inserendo le espressioni per il flusso radiativo \eqref{eq:radiativeflux} e il flusso convettivo \eqref{eq:convectiveflux}.

In figura (\subref{fluxproportion}) si mostrano l'andamento di $\nabla-\nad{}$, il profilo termico e la frazione di flusso totale trasportato dalla convezione; in figure (\subref{specificheatnablaa}) si mostrano il profilo del calore specifico per unit\'a di massa e del gradiente adiabatico.

Le 5 equazioni del flusso convettivo

Le 5 equazioni determinano completamente le variabili $F_{rad}, F_{con}, v, \nabla_e, \nabla$ in funzione di $P,T,l(r),m(r),c_P,\nad{},\nrad{},g$.

Come determino il gradiente effettivo ??

Determino $\nabla-\nabla_e$
cubic equation for $(\nabla-\nabla_e)$

\end{wordonframe}

\subsection{Approssimazione politropa}\linkdest{poly}

\begin{frame}{Trasformazioni politropiche}
\begin{block}{Gener. T. adiabatica}
	Il rapporto $\gamma=\frac{c_P}{c_V}$ costante per gas perfetto di sole particelle totalmente ionizzato.
	T. adiabatica:
	\[TV\expy{\gamma-1}=\const,\ PV\expy{\gamma}=\const,\ P\expy{1-\gamma}T\expy{\gamma}=\const\]
$0=dE-\frac{P}{\rho^2}d\rho$
Caso pi\'u generale delle trasformazioni adiabatiche: \keyword{trasformazione politropa} trasformazione quasi-statica in maniera che $c=\TDy{Q}{T}$ (calore specifico) vari in maniera assegnata. (adiabatica: $c=0$, isoterma: $c=\infty$, isometrica: $c=c_V$, ...)
\end{block}
\begin{block}{Relazione politropa - EOS}
\begin{columns}[T]
	\begin{column}{0.4\textwidth}
\begin{align*}
&\TDy{r}{P}=-\TDy{r}{\phi}\rho\\
&\frac{1}{r^2}\TDof{r}(r^2\TDy{r}{\phi})=4\pi G\rho\\
&\TDy{r}{\phi}=-\gamma K\rho\expy{\gamma-2}\TDy{r}{\rho}
\end{align*}
\end{column}
\begin{column}{0.6\textwidth}
\begin{align*}
&P=K\rho\expy{\gamma}=K\rho\expy{1+1/n}\\
&\gamma=5/3, n=3/2\tag{NR deg}\\
&\gamma=4/3, n=3\tag{R deg}\\
&\gamma=1, n=\infty\tag{isot.}\\
&\nad\approx\frac{2}{5}\tag{conv}\\
&P\propto\rho\expy{5/3}
\end{align*}
For convective exponent K varies from star to star
\end{column}\end{columns}
\end{block}
\end{frame}

\begin{frame}{Equazione Lane-Emden}
\begin{columns}[T]
	\begin{column}{0.5\textwidth}
\begin{align*}
&\rho=(\frac{-\phi}{(n+1)K})^n\tag{HE}\\
&\TtwoDy{r}{\phi}+\frac{2}{r}\TDy{r}{\phi}=4\pi G(\frac{-\phi}{(n+1)K})^n\tag{Poiss}\\
&\TtwoDy{z}{w}+\frac{2}{z}\TDy{z}{w}+w^n=0\\
&\frac{1}{z^2}\TDof{z}(z^2\TDy{z}{w})+w^n=0\tag{Lane-Emden}
\end{align*}
	\end{column}
	\begin{column}{0.5\textwidth}
		\begin{align*}
	&z=Ar\tag{new vars}\\
	&A^2=\frac{4\pi G}{(n+1)^nK^n}(-\phi_c)\expy{n-1}\\
	&=\frac{4\pi G}{(n+1)K}\rho_c\expy{\frac{n-1}{n}}\\
	&w=\frac{\phi}{\phi_c}=(\frac{\rho}{\rho_c})\expy{1/n}
		\end{align*}
\end{column}\end{columns}
\end{frame}

\begin{frame}{Strutture isoterme e modello solare}

\end{frame}

\subsection{Equazione di stato}\linkdest{eos}

\begin{wordonframe}{Schema fisico/chimico: ingredienti EOS realistiche}
\begin{itemize}
\item schema chimico: il primo considera atomi e molecole, la cui popolazione per stati eccitati e diversi gradi di ionizzazione \'e ottenuto minimizzando l'energia libera da cui sono ricavate le altre grandezze termodinamiche; utilizzando questo approccio \'e stata ricavata l'equazione di stato MHD (\cite{hummer1988equation})
\item schema fisico: nuclei ed elettroni come costituenti fondamentali interagenti tramite potenziale Coulombiano e trova le soluzione dell'equazione di Schr\"oedinger per un problema a molti corpi, questo approccio, usato per ricavare l'equazione di stato OPAL (\cite{rogers1986occupation})
\end{itemize}


Come illustrato in figura, per entrambe $\Gamma_1\approx\midfrac{5}{3}$ nell'interno solare e maggiori deviazioni si hanno nelle regioni di ionizzazione parziale degli elementi in particolare di idrogeno ed elio.
\end{wordonframe}

\begin{wordonframe}{Relazione tra $P(\rho, T)$: approx zero gas perfetto di atomi completamente ionizzati}
\begin{columns}[T]
\begin{column}{0.55\textwidth}
EOS gas perfetto di ioni ed elettroni 
\begin{equation*}
P_G=P_I+P_e=\frac{\rho}{\mu}\gasconstant{}T
\end{equation*}
Energia interna per unit\'a di massa: somma delle energie traslazionali delle particelle pesate secondo la distribuzione di equilibrio di Maxwell-Boltzmann per grammo di materia
\begin{align*}
&u=\frac{1}{\rho}\sum_i\int f^{(0)}(\vec{p}_i)\frac{p^2_i}{2m_i}\,d^3p_i\\
&=\frac{3}{2}\frac{P}{\rho}=\frac{3}{2}\frac{\gasconstant T}{\mu}
\end{align*}
\end{column}
\begin{column}{0.45\textwidth}
Peso molecolare medio: massa media in amu per particella libera
\begin{align*}
&\mu=\frac{1}{\bar{n}_HX+\bar{n}_{He}Y+\bar{n}_{Z}Z}\\
&\mu_0=\frac{1}{X+\midfrac{Y}{4}+\midfrac{Z}{\bar{A}}},\ \mu_e\approx\frac{2}{1+X}
\end{align*}
con $\bar{n}_i=\frac{1+f_i}{A_i}$ numero medio di particelle libere per unit\'a di massa atomica dovute alla specie i di peso atomico $A_i$ e $f_i$ numero medio di elettroni liberati da ione della specie i; peso atomico medio per ione $\mu_0$ ed elettrone libero (ionizzato) $\mu_e$.
\end{column}
\end{columns}
$f^{(0)}(\vec{p}_i)$: numero di particelle della specie i per unit\'a di volume con impulso in $[\vec{p}_i,\vec{p}_i+d\vec{p}_i]$
\end{wordonframe}

\begin{wordonframe}{Deviazioni dalla legge dei gas perfetti: radiazione e degenerazione elettronica}
\begin{itemize}
	\item Radiazione. Il contributo alla pressione ed energia interna per unit\'a di volume dei fotoni $P_R=\frac{a}{3}T^4$, $u_R=aT^4$, e $P-P_R=\beta P$.
	\item Degenerazione elettronica - Principio di Pauli: non pi\'u di 2 elettroni in volume di spazio delle fasi $h^3$. $n_e$ la densit\'a numerica di \Pelectron, $\psi(P,T)$ il parametro di degenerazione, tale che per $\psi\to-\infty$ si abbia la distribuzione di Boltzmann e per $\psi\to+\infty$ completa degenerazione
	\begin{align*}
	&n_e=\rho N_A\frac{1+X}{2}=\intzi{}\frac{8\pi p^2\,dp}{h^3(\exp{\frac{u_k}{KT}-\psi}+1)}=\frac{8\pi}{h^3}(2m_ekT)\expy{3/2}a(\psi)\\
	&P_e=\beta P-\rho\gasconstant{}(X+\frac{Y}{4}+\frac{Z}{\exv{A_Z}})=\frac{8\pi}{3h^3}\intzi{}p^3v(p)\frac{dp}{1+\exp{\epsilon/kT-\psi}}\\
	&U_e=\frac{8\pi}{h^3}\int_0^{\infty}\frac{p^2\epsilon(p)\,dp}{\exp{(-\psi+\midfrac{\epsilon}{KT})}+1}
	\end{align*}
\end{itemize}
\end{wordonframe}

\begin{frame}{degenerazione completa: casi limite NR e R}
\begin{align*}
&P_e=\int_{2\pi}\frac{d\Omega}{4\pi}\intzi{}dpf(p)v(p)p^2\cos{\theta}=\frac{8\pi}{3h^3}\int_0^{P_F}p^3v(p)\,dp\\
&=\frac{8\pi c}{3h^3}\int_0^{P_F}\frac{p/(m_ec)}{\sqrt{1+p^2/(m_ec)^2}}=\frac{\pi m_e^4c^5}{3h^2}f(x)\\
&U_e=\int_0^{P_F}f(p)E(p)\,dp=\frac{8\pi}{h^3}\int_0^{P_F}E(p)p^2\,dp=\frac{\pi m_e^4c^5}{3h^3}g(x)\
\end{align*}
\begin{align*}
&n_e=\frac{\rho}{\mu_em_H}=\frac{8\pi m_ec^3}{3h^3}x^3\\
&x=\frac{P_F}{m_ec}\ll1:\quad&x\gg1:\\
&P_e=\frac{8\pi m_e^4c^5}{15h^3}x^5=\frac{2}{3}U_e&P_e=\frac{2\pi m_e^4c^5}{3h^3}x^4=\frac{1}{3}U_e\\
&=\num{1.0036e13}(\frac{\rho}{\mu_e})\expy{5/3}&=\num{1.2435e15}(\frac{\rho}{\mu_e})\expy{4/3}
\end{align*}
\end{frame}

\begin{wordonframe}{Elettrostatic screening of ions: weak screening}
La principale correzione che tiene conto dell'interazioni tra particelle \'e dovuta alle interazioni coulombiane: influence EOS and nuclear rection rates.

Screening of ion i with charge $Z_ie$ ar $\vec{r_i}$ in NR dilute plasma, motion of screened ions is slow compared to screening particles: continuum static equilibrium charge distribution (\Pelectron, light ions of mean charge $Z_p$)
\begin{equation*}
\nabla^2\phi=4\pi n_ee[\exp{(\frac{e\phi}{kT})}-\exp{-\frac{Z_pe\phi}{kT}}]-4\pi\sum_iZ_ie\delta(\vec{r}-\vec{r_i})
\end{equation*}
Regime di schermaggio debole, $e\phi\ll KT$: $\phi=\sum_i\phi_i$ potenziale attorno a ione pesante isolato:
\begin{align*}
%&\nabla^2\phi=-4\pi e\sum_Z Zn_Z-4\pi e\sum_i Z_i\delta(\vec{r}-\vec{r}_i)\\
&\phi_i=\frac{Z_ie}{r_i}\exp{-\frac{r_i}{r_D}}
&\frac{1}{r_D^2}=\frac{4\pi e^2}{kT}\sum Z^2\overline{n}_Z=\frac{4\pi e^2}{kT}N_A\zeta,\ \zeta=\sum_{i}(Z_i^2+Z_i)\frac{\rho X_i}{A_i}
\end{align*}
\end{wordonframe}

\begin{wordonframe}{Elettrostatic screening of ions: energy pressure correction}
Energy required to assemble uniform shere with charge $Ze$: $U_{ee}=\int_0^{R_Z}\frac{q_r}{r}\,dq=\frac{3}{5}\frac{(Ze)^2}{R_Z}$.
Energy required to assemble uniform cloud of charge $Ze$ around Z-nucleus: $U_{eZ}=-Ze\int_0^{R_Z}\frac{dq}{r}=-\frac{3}{2}\frac{(Ze)^2}{R_Z}$
Le correzioni dovute alle interazioni coulombiane sono dovute a numero sfere ioniche per unit\'a di volume $n_Z=\frac{\rho X_Z}{A_Z}N_0$ with average potential energy per electron $\exv{-e\phi}_Z=-\frac{9}{10}\frac{(Ze)^2}{R_Z}$ that contain Z \Pelectron.
\begin{align*}
&\rho u_c=(\frac{U}{V})_e=\frac{1}{2}\phi(\vec{r})\rho_c(\vec{r})\to \frac{1}{2}\sum_ZeZ\overline{n}_Z\phi_Z=-e^3\sqrt{\frac{\pi\rho}{kT}}(N_A\zeta)\expy{\frac{3}{2}},\ P_c=\frac{1}{3}\rho u_c\\
&E_0=\frac{(U/V)_e}{n_e}=\mu_e\sum_Z\exv{-e\phi}_Z\frac{ZX_Z}{A_Z}\approx-1.3(\mu_e^2\rho)\expy{1/3}[X+0.79Y+\sum_{Z>2}\frac{Z\expy{5/3}X_Z}{A_Z}]
\end{align*}
\end{wordonframe}

\begin{frame}{Equazione di Saha e continuum depression. Ioniozzazione da pressione}
L'equazione di Saha descrive la frazione relativa di ionizzazione
\begin{align*}
&\frac{n_{r+1}}{n_r}n_e=\frac{g_{r+1}}{g_r}f_r(T)\\
&f_r(T)=2\frac{(2\pi m_ekT)\expy{3/2}}{h^3}\exp{-\chi_r/(kT)}
\end{align*}
Saha limitation:
\begin{itemize}
\item LTE: is the case when collision dominate over radiative processes
\item Decreases ionization energy with increasing density: what is called pressure ionizzation is produced by coulomb interaction of bound electron with other electron in the plasma $\chi'_Z=\chi_Z-\frac{Ze^2}{R_D}$
\end{itemize}
\end{frame}

\begin{frame}{Crystallization and Neutronization}
\begin{block}{Crystallization}
Per $\rho$ alta e $T$ bassa (WD interior: $\Gamma_c=\frac{(Ze)^2}{r_{ion}kT}\approx180$) gli ioni formano reticolo quando energia termica uguale interazione coulombiana $\frac{3}{2}kT\approx E_c$ - melting $T_m\approx\frac{Z^2e^2}{\Gamma_ck}(\frac{4\pi\rho}{3\mu_0m_H})\expy{1/3}=\num{2.3e3}Z^2\mu_0\expy{-1/3}\rho\expy{1/3}$
\end{block}
\begin{block}{Neutronization}
If \Pelectron have $E_e>E^*=c^2(m_n-m_p)\approx\SI{1.3}{\mega\ev}$ they combine with protons and form neutrons: if we put $E=E_{kin}+m_ec^2=E_F+m_ec^2=c^2(m_n-m_p)\approx\SI{1.3}{\mega\ev}$ if $E_{kin}<E_F$ l'elettrone non trova stati liberi e il neutrone non decade ($x=\frac{p_F}{m_ec}\approx2.2$).
For $\rho>\SI{4e11}{\gram\per\cubic\cm}$ we have inverse $beta$-decay:nuclei capture electron and become neutron rich - neutron drip
\end{block}
\end{frame}

\section{Energy production}\linkdest{luminositysourcessinks}

\subsection{Work against gravity}\linkdest{epsilong}

\begin{frame}{Work done in expansion/contraction: }

\end{frame}

\subsection{Fusione nucleare}\linkdest{epsilonn}

\begin{frame}{Sezione d'urto fusion nucleare}
$E$ l'energia cinetica nel centro di massa dei nuclei: $\sigma(E)=\pi\lambdabar^2*P_0(E)*S(E)$
prodotto della sezione d'urto geometrica (nel riferimento del CM: $\sigma\approx\sum_{l=0}^{\frac{R}{\lambdabar}}(2l+1)\pi\lambdabar^2=\pi(R+\lambdabar)^2$
per energie tipiche degli interni stellari approx onda S), della probabilit\'a di attraversamento della barriera coulombiana e del fattore astrofisico. La lunghezza d'onda di de Broglie relativa delle particelle descrive l'indeterminazione sulla posizione nell'urto di due particelle con momento relativo p $\lambdabar=\frac{\hbar}{p}=\frac{\hbar}{\sqrt{2mE}}$.

L'energia potenziale dovuta all'interazione di due nuclei $Z_1$ e $Z_2$ a distanza r contiene un contributo delle altre cariche presenti nel plasma $U=\frac{Z_1Z_2e^2}{r}+U_s(r_{12})$
l'energia potenziale non schermata e contributo della nuvola elettronica: $U_s$ aumenta la probabilit\'a di attraversamento della barriera coulombiana. Fattore moltiplicativo: $f=\exp{-\midfrac{U_0}{KT}}$ dove $U_0=U_s(0)$ poich\'e $r\ll r_D$ e considerando solo la correzione al fattore di penetrazione ($E_G\gg U_0$).

Per determinare $U_0$ considero l'energia potenziale di $Z_1$ e $Z_2$ a distanza $r$
\begin{equation*}
U=Z_2e\int_{\infty}^r\PDy{r_1}{\phi_1}\,dr_1=\frac{Z_1Z_2e^2\exp{-\midfrac{r}{r_D}}}{r},\ U_s=U-\frac{Z_1Z_2e^2}{r}\approx\frac{Z_1Z_2e^2}{r_D}
\end{equation*}
\end{frame}

\begin{frame}{Energia prodotta in reazioni di fusione???}
$S(E)$ descrive l'interazione a livello nucleare: debolmente dipendente dall'energia in assenza di risonanze.
La probabilit\'a di attraversamento della barriera coulombiana: $P_0(E)=\exp{-2\pi\eta},\ \eta=\sqrt{\frac{m}{2}}\frac{Z_1Z_2e^2}{\hbar E\expy{\frac{1}{2}}}$
Per i nuclei di carica $Z_1$, $Z_2$ e m massa ridotta: $\sigma(E)=\frac{S(E)}{E}\exp{-2\pi\eta}$
Il rate per coppia di particelle
\begin{equation*}
\exv{\sigma v}=\num{1.3005e-15}[\frac{Z_1Z_2}{AT_6^2}]\expy{\frac{1}{3}}fS_{eff}\exp{-\tau}\si{\cubic\cm\per\second},\ \tau=\frac{3E_G}{kT}\approx\num{42.487}(Z_1^2Z_2^2AT_6\expy{-1})\expy{\frac{1}{3}}
\end{equation*}
$S_{eff}$ \'e il risultato dell'espansione dell'integrando per $\invers{\tau}\ll1$ ed estrapolato a $E_G$ a partire dal valore $S(0)$ determinato dalla fisica nucleare.
La funzione $\epsilon(\rho,T,X_i)$ \'e determinata dalla somma di tutti i contributi
\begin{equation*}
\epsilon_{ij}=Q_{ij}\frac{n_in_j}{\rho(1+\delta_{ij})}\lambda_{ij}=\frac{1}{1+\delta_{ij}}Q_{ij}\frac{\rho N_A^2X_jX_k}{{A_iA_j}}\exv{\sigma v}_{ij}\label{eq:energyrate}
\end{equation*}
dove $Q_{ij}$ \'e l'energia liberata per reazione tra nucleo di specie i e j e $\exv{\sigma v}_{ij}$ \'e il rate di reazione per coppia di particelle, mediata su MB- distro
$f(E)dE\propto\frac{E\expy{\frac{1}{2}}}{(kT)\expy{\frac{3}{2}}}\exp{-\frac{E}{kT}}\,dE$:
$S(E)\exp{-\frac{E}{kT}-\frac{b}{\sqrt{E}}}$
ha forma approssimativamente gaussiana il cui massimo $E_G$, energia pi\'u probabile di reazione, e FWHM sono: $E_G=\SI{5.665}{\kilo\ev} A\expy{\frac{1}{3}}T_7\expy{\frac{2}{3}}$ $\Delta E=4.249\si{\kilo\ev}W\expy{\frac{1}{6}}T_7\expy{\frac{5}{6}}$
posto $W=Z_i^2Z_j^2A=Z_i^2Z_j^2\frac{A_iA_j}{A_i+A_j}$.
%\begin{equation*}
%\exv{\sigma v}\propto b\expy{1/3}T\expy{-2/3}\exp{-\frac{b\expy{2/3}}{t\expy{1/3}}}
%\end{equation*}
\end{frame}

\subsection{Catena PP}\linkdest{epsilonpp}

\begin{frame}{PP1: twice $\Pproton(\Pproton,\Pnue\APelectron)d(p,\gamma)^3He$ and $^3He(^3He,21^H)^4He$}
\begin{columns}[T]
	\begin{column}{0.55\textwidth}
\begin{itemize}
	\item $\tau_p(d)\ll[\tau_{^3He}(d)]_e\ll[\tau_d(d)]_e$: $\frac{\dot{D}}{H}=\frac{H}{2}\exv{\sigma v}_{pp}-H(\frac{D}{H})\exv{\sigma v}_{dp}$ - \keyword{D evolution} $(\frac{D}{H})=\frac{\exv{\sigma v}_{pp}}{2\exv{\sigma v}_{dp}}$, evolution: $(\frac{D}{H})_t=(\frac{D}{H})_e-[(\frac{D}{H})_e-(\frac{D}{H})_0]\exp{-\frac{t}{\tau_p(d)}}$
	\item \keyword{$^3He$ evolution} - $\frac{\dot{^3He}}{H}=\frac{H}{2}\exv{\sigma v}_{pp}-H(\frac{^3He}{H})^2\exv{\sigma v}_{^3He^3He}$ - 
\scalebox{0.8}{	\[\frac{^3He}{H}|_t=0+\sqrt{\frac{\exv{\sigma v}_{pp}}{\exv{\sigma v}_{^3He^3He}}}\tanh(t\sqrt{\frac{H}{2}\exv{\sigma v}_{PP}H\exv{\sigma v}_{^3He^3He}})\]}
\item Produzione energia - $\epsilon_{PPI}^e(T_0=\SI{15}{\mega\ev})(\frac{T}{T_0})\expy{3.9}$
\begin{align*}
&\epsilon_{PPI}=\frac{(6.936-0.265)\si{\mega\ev}H^2\exv{\sigma v}_{pp}}{2\rho}\\
&+\frac{\SI{12.861}{\mega\ev}H^2\exv{\sigma v}_{^3He^3He}}{2\rho}(\frac{^3He}{H})^2\\
&\epsilon_{PPI}^e(T)=6.551N_A\exv{\sigma v}_{PP}(\frac{X_H}{m_H})^2\rho N_A\frac{\si{\mega\ev}}{\si{\second\gram}}
\end{align*}
\end{itemize}
	\end{column}
	\begin{column}{0.45\textwidth}
	\begin{figure}[!ht]
\includegraphics[trim={0cm 0cm 1cm 0cm},clip, keepaspectratio,height=0.28\textheight]{He3eq}
\includegraphics[trim={0cm 0cm 1cm 0cm},clip, keepaspectratio,height=0.28\textheight]{PP1DHet}
\includegraphics[trim={0cm 0cm 1cm 0cm},clip, keepaspectratio,height=0.28\textheight]{pplifetime}
	\end{figure}
	\end{column}
\end{columns}
\end{frame}

\begin{frame}{Network reazioni PP completo}

\begin{columns}[T]
\begin{column}{0.55\textwidth}
Rapid $^8B/^8Be$ decays: $^7Be(p,\gamma)^8B(\APelectron,\nu)^8Be(\alpha)\alpha$ as $^7Be+p\to2\alpha+\gamma$, $\TDof{t}(^7Li+^7Be)\approx0$
\begin{align*}
&Q_{PPI}=27.73-2\exv{E}_{\nu}^{pp}=\SI{26.19}{\mega\ev}\\
&Q_{PPII}=26.73-\exv{E}_{\nu}^{^7Be}-\exv{E}_{\nu}^{pp}=\SI{25.65}{\mega\ev}\\
&Q_{PPIII}=26.73-\exv{E}_{\nu}^{^8B}-\exv{E}_{\nu}^{pp}=\SI{19.75}{\mega\ev}\\
&\epsilon_{PP}=\frac{Q_{4H\to^4He}}{\rho}\dot{^4He}[0.98F_{PPI}+0.96F_{PPII}\\
&+0.74F_{PPIII}],\ f_{PPI}=\frac{r_{^3He^3He}}{r_{^3He^3He}+r_{\alpha^3He}}\\
&F_{PPII}=(1-F_ {PPI})\frac{r_{e^7Be}}{r_{e^7Be}+r_{p^7Be}}
\end{align*}
\end{column}
\begin{column}{0.45\textwidth}
\begin{figure}[!ht]
	\includegraphics[trim={0cm 0cm 1cm 0cm},clip, keepaspectratio,width=0.72\textwidth]{ppchainsequi}
	\includegraphics[trim={0cm 0cm 1cm 0cm},clip, keepaspectratio,width=0.72\textwidth]{ppfraction}
\end{figure}
\end{column}
\end{columns}
\begin{align*}
&\dot{(^3He)}=\frac{H^2}{2}\exv{\sigma v}_{pp}-2(^3He)^2\frac{\exv{\sigma v}_{^3He^3He}}{2}-(^3He)(^4He)\exv{\sigma v}_{\alpha^3He}\\
&(^3He)_e=\frac{-(^4He)\exv{\sigma v}_{\alpha^3He}+\sqrt{(^4He)^2\exv{\sigma v}_{\alpha^3He}^2+2H^2\exv{\sigma v}_{\alpha^3He}\exv_{\sigma v}_{^3He^3He}}}{2\exv{\sigma v}_{^3He^3He}}
\end{align*}
\end{frame}


\subsection{Ciclo CN-NO:}\linkdest{epsilonCNO}

\begin{frame}{Biciclo CNO}
dipendenza da T
flusso neutrini
Modalit\'a combustione H in He: sequenza principale inferiore/superiore
\end{frame}

\begin{frame}{CNOF Network reactions: $4^1H\to^4He+2\APelectron+2\Pnue$}
\begin{columns}[T]
\begin{column}{0.5\textwidth}
\begin{itemize}
\item CNOF elements acts as catalyst: relative initial CNOF abundance are important - produced in previous gen stars at He-burning stages ($^{12}C$, $^{16}O$, less $^{14}N$: solar $^{12}C:^{14}N:^{16}O=10:3:24$)
\item 4 cycle: active cycle influences heavy element abundance - $(p,\gamma)$ compete with $(p,\alpha)$ on $^{15}N$, $^{17}O$, $^{18}$, $^{19}F$: $(p,\alpha)$ faster over entire T-range except $^{17}O$/$^{18}O$ at $T<\SI{20}{\mega\kelvin}$
\item At hydro-burning ($T<\SI{55}{\mega\kelvin}$) much faster than p-induced reactions, at $T>\SI{100}{\mega\kelvin}$ also other reactions are important (HCNO)
\end{itemize}
\end{column}
\begin{column}{0.5\textwidth}
	\begin{figure}[!ht]
	\includegraphics[trim={0cm 0cm 1cm 0cm},clip, keepaspectratio,height=0.28\textheight]{CNO}
	\includegraphics[trim={0cm 0cm 1cm 0cm},clip, keepaspectratio,height=0.28\textheight]{HCNO}
\end{figure}
\end{column}
\end{columns}
\end{frame}

\begin{frame}{CNO1: equilibrium properties}
\begin{columns}[T]
	\begin{column}{0.5\textwidth}
		\begin{itemize}
			\item Elements involving beta-decay reaches equilibrium in few minutes: $\dot{^{13}N}=0$ - $(^{13}N)_t=\frac{\tau_{\beta}(^{13}N)}{\tau_p{^{12}C}}^{12}C[1-\exp{-\frac{t}{\tau_{\beta}(^{13}N)}}]$ - $(\frac{^{13}N}{^{12}C})_e=\frac{\tau_{\beta}(^{13}N)}{\tau_p(^{12}C)}$
			\item At equilibrium ratio of abundance of $^{12}C$, $^{13}C$, $^{14}N$, $^{15}N$ are given by inverse ratio of reaction time (ie $(\frac{^{14}N}{^{12}C})_e=\frac{\exv{\sigma v}_{^{12}C(p,\gamma)}}{\exv{\sigma v}_{^{14}N(p,\gamma)}}=\frac{\tau_p(^{14}N)}{\tau_p(^{12}C)}$), fractional abundance $\frac{(^{12}C)_e}{\sum CNO1}=\frac{\tau_p(^{12}C)}{\tau_p(^{12}C)+\tau_p(^{13}C)+\tau_p(^{14}N)+\tau_p(^{15}N)}$
		\end{itemize}
	\end{column}
	\begin{column}{0.5\textwidth}
		\begin{figure}[!ht]
			\includegraphics[trim={0cm 0cm 1cm 0cm},clip, keepaspectratio,height=0.28\textheight]{CNO1ab}
			\includegraphics[trim={0cm 0cm 1cm 0cm},clip, keepaspectratio,height=0.28\textheight]{CNOXivsT}
		\end{figure}
	\end{column}
\end{columns}
\end{frame}

\begin{frame}{CNO1: energy production rate}


\end{frame}

\subsection{He-C-Ne-O-Si-Burning}\linkdest{epsilonheavy}

\begin{frame}{Combustion He}

\end{frame}


\begin{frame}{Produzione nuclei fino al Fe56}
3$\alpha$: $He4+\alpha\to C12$, $C12+\alpha\to O16$
Produzione neutroni liberi
Fusione $C12$, fotodisintegrzione $Ne20$, fusione $O16$, fotodisintegrzione $Si28$, catture $\alpha$ su nuclei fino a produzione $Fe56$
\end{frame}

\begin{frame}{Cattura neutronica: processi r e s}
picchi r e s nella nella curva universale delle abbondanze
\end{frame}

\begin{frame}{Nuclear burning efficiency}
\begin{equation*}
\epsilon_{ij}(\rho,T,X_i)=Q_{ij}\frac{n_in_j}{\rho(1+\delta_{ij})}\lambda_{ij}=\frac{1}{1+\delta_{ij}}Q_{ij}\frac{\rho N_A^2X_jX_k}{{A_iA_j}}\exv{\sigma v}_{ij}
\end{equation*}
dove $Q_{ij}$ \'e l'energia liberata per reazione tra nucleo di specie i e j e $\exv{\sigma v}_{ij}$ \'e il rate di reazione per coppia di particelle; $X_i$ indica la frazione in  massa della specie i
\begin{columns}[T]
	\begin{column}{0.5\textwidth}
		\begin{align*}
		&PP\ \exv{\sigma v}\propto T\expy{3.9}\ E_C=\SI{0.55}{\mega\ev}\\
		&P^{14}N\ \exv{\sigma v}\propto T\expy{20}\ E_C=\SI{2.27}{\mega\ev}\\
		&\alpha+^{12}C\ \exv{\sigma v}\propto T\expy{42}\ E_C=\SI{3.43}{\mega\ev}\\
		&^{16}O+^{16}O\ \exv{\sigma v}\propto T\expy{182}\ E_C=\SI{14.07}{\mega\ev}
		\end{align*}
	\end{column}
	\begin{column}{0.5\textwidth}
		\begin{align*}
		&E_C\approx Z_1Z_2\si{\mega\ev}
		\end{align*}
	\end{column}
\end{columns}
\end{frame}

\section{Ordini di grandezza}

\begin{frame}{Energia interna}
\begin{equation*}
E_i=\int_0^Mu\,dm=\frac{3}{2}\int_M\frac{P}{\rho}\,dm\label{eq:traslintenergy}
\end{equation*}
\end{frame}

\section{Metodi per integrazione equazioni di struttura e raccordo con modelli dell'atmosfera stellare}\linkdest{nummod}

\subsection{4 Structure ODE with boundary conditions}\linkdest{fourODE}

\begin{frame}{Equazioni struttura di equilibrio}
\begin{align*}
&\TDy{m}{r}=\frac{1}{4\pi r^2\rho}\\
&\TDy{m}{P}=-\frac{Gm}{4\pi r^4}\overbrace{[-\frac{1}{4\pi r^2}\PtwoDy{t}{r}]}^{\tau_{hyd}}\\
&\TDy{m}{T}=-\nabla\frac{T}{p}\frac{Gm}{4\pi r^4}\\
&\TDy{m}{L}=\epsilon-\epsilon_{\nu} \underbrace{-c_P[\TDy{t}{T}-\nad\frac{T}{P}\TDy{t}{P}]}_{-c_P\PDy{t}{T}+\frac{\delta}{\rho}\PDy{t}{P}: \tkh}\\
&\TDy{t}{X_s}\frac{1}{A_s}=\sum_{production}\rho^{n_h+n_k-1}n_p\frac{X_h^{n_h}X_k^{n_k}}{A_h^{n_h}A_k^{n_k}}\frac{\exv{\sigma v}_{hk}}{m_H^{n_h+n_k-1}n_h!n_k!}\\
&-\sum_{distruction}\rho^{n_d+n_j-1}n_d\frac{X_s^{n_d}X_j^{n_j}}{A_s^{n_d}A_j^{n_j}}\frac{\exv{\sigma v}_{sj}}{m_H^{n_d+n_j-1}n_d!n_j!}
\end{align*}
\end{frame}

\begin{frame}{Equazioni struttura di equilibrio: condizioni al bordo}
\begin{itemize}
\item Le 4+I equazioni determinano $r$, $P$, $T$, $L$, $X_s$ specificata la massa e composizione iniziale (omogenea)
\item $\tau_n\gg\tkh\gg\tau_{dyn}$: solve 4 structure equations at time t - do time step $\Delta t$ and determine new composition - solve structure at $t+\Delta t$ with new composition
\item Solution of 4 structure equations require 4 boundary condition: 2 at surface (atmospere model without diffusion approx, PP geometry), parametri $\rho_c,T_c$; 2 at center (via Taylor expansion for $m=m'$)
\end{itemize}
\begin{columns}[T]
\begin{column}{0.65\textwidth}
\begin{align*}
&r=(\frac{3}{4\pi\rho_c})\expy{1/3}{m'}\expy{1/3}\\
&P=P_c-\frac{3G}{8\pi}(\frac{4\pi\rho_c}{3})\expy{4/3}{m'}\expy{2/3}\\
&L=\epsilon_cm'\\
&T^4=T_c^4-\frac{1}{2ac}(\frac{3}{4\pi})\expy{2/3}\kappa_c\epsilon_c\rho_c\expy{4/3}{m'}\expy{2/3}\tag*{rad}\\
&\ln{T}=\ln{T_c}-(\frac{\pi}{6})\expy{1/3}G\frac{{\nad}_c\rho_c\expy{4/3}}{P_c}{m'}\expy{2/3}\tag*{con}
\end{align*}
\end{column}
\begin{column}{0.35\textwidth}
At surface $m=M$, $L=L_s$, atmospheric model for $P$, $T$ - atmosphere defined by $g=\frac{GM}{R^2}$, $T_e$, composition: provides $P_s$ at $\tau$ where diffusion approx. starts to be valid
\end{column}
\end{columns}
\end{frame}

\begin{frame}{Simplified atmosferic model: grey atmosphere}
Atmosphere model: usually PP geometry and solve HE equation using non grey radiative transport and EOS and convection if needed
\begin{align*}
&\TDy{\tau}{P}=\frac{g}{\kappa}\tag*{HE using $\tau$ as indip var}\\
&T^4=\frac{3}{4}T_e^4(\tau+\frac{2}{3})\tag*{or solar $T(\tau)$ empirical relation}
\end{align*}
integration from $\tau\approx0$ where $T\approx0$, $P\approx0$ down to $\tau=\frac{2}{3}$ where $T=T_e$ - using shooting method.
\end{frame}

\begin{frame}{Chemical mixing: diffusion and convection}

\end{frame}

\subsection{Metodo di Henyey: modelli evolutivi}

\begin{frame}{Metodo di Henyey $[96]$}
N mass-shell with boundaries at $m_j$ $m_1=0,m_2=m',\ldots,m_N=M$
\begin{columns}[T]
\begin{column}{0.5\textwidth}
\begin{align*}
&\frac{r_{j+1}-r_j}{m_{j+1}-m_j}=\frac{1}{4\pi r_{j+1/2}^2\rho_{j+1/2}}\\
&\frac{P_{j+1}-P_j}{m_{j+1}-m_j}=-\frac{Gm_{j+1/2}}{4\pi r_{j+1/2}}\\
&\frac{L_{j+1}-L_j}{m_{j+1}-m_j}=\epsilon_n(T_{j+1/2},\rho_{j+1/2},X_{s,j+1/2})+!!!\\
&\frac{T_{j+1}-T_j}{m_{j+1}-m_j}=-\frac{T_{j+1/2}}{P_{j+1/2}}\nabla_{j+1/2}\frac{Gm_{j+1/2}}{4\pi r^4_{j+1/2}}
\end{align*}
\end{column}
\begin{column}{0.5\textwidth}
\begin{align*}
&(Y^1=r, y^2=P, y^3=T, y^4=L)\\
&E_j^{i=1,\ldots,4}=\frac{y_{j+1}^i-y^i_j}{m_{j+1}^i-m^i_j}\\
&-f_i(y_{j+1/2}^1,\ldots,y_{j+1/2}^4)=0
\end{align*}
$j=2,\ldots,N-2$: 4N-8 vincoli e le condizioni al bordo
\begin{align*}
&J=N: S_1=y_N^2-f_S(y_N^1,y_N^4)=0\\
&S_2:y^3_N-g_S(y_N^1,y_N^4)=0\\
&j=1: C_i(y^1_2,y_2^2,y_2^3,y_2^4,y_1^2,y_1^3)=0\\
&y_1^1=y_1^4=0
\end{align*}
\end{column}
\end{columns}
$4N-2$ equazioni in $4N-2$ incognite
\end{frame}

\begin{frame}{Metodo di Henyey: solve system of algebraic equations}
\'A la Newton-Rapshon - Trial solution (solution at step $t-\Delta t$): $(S_i)_1\neq0$, $(C_i)_1\neq0$, $(E_i^j)_1\neq0$ so we have to find correction to trial solution $(y_i^j)_2=(y_i^j)_1+\delta y_i^j$ - by Taylor expansion we can express $\delta S_i$, $\delta C_i$, $\delta E^i_j$ as function of the unknown small correction:
\begin{align*}
&(S_i)_1+\delta S_i=0, (C_i)_1+\delta C_i=0, (E_i^j)_1+\delta E_i^j=0\\
&\left\{\begin{array}{l}\TDy{y_N^1}{S_i}\delta y_N^1+\TDy{y_N^2}{S_i}\delta y_N^2+\TDy{y_N^3}{S_i}\delta y_N^3+\TDy{y_N^4}{S_i}\delta y_N^4=-(S_i)_1\\
\TDy{y_2^1}{C_i}\delta y_2^1+\ldots+\TDy{y_2^4}{C_i}\delta y_2^4+\TDy{y_1^2}{C_i}\delta y_1^2+\TDy{y_1^3}{C_i}\delta y_1^3=-(C_i)_1\\
\TDy{y_j^1}{E_j^i}\delta y_j^1+\ldots+\TDy{y_j^4}{E_j^i}\delta y_j^4+\TDy{y_{j+1}^1}{E_j^i}\delta y_{j+1}^1\ldots+\TDy{y_{j+1}^4}{E_j^i}\delta y_{j+1}^4=-(E_j^i)_1\\
\end{array}\right.
\end{align*}
con $i=1,\ldots,4$, $j=2,\ldots,N$: sistema algebrico di $4N-2$ equazioni in $4N-2$ $\delta y_j^i$ incognite ha matrice dei coefficienti (\keyword{Henyey matrix}) non-zero only near the diagonal e determinante non nullo - soluzione con metodi algebrici standard - fisso $\epsilon>0$ accuratezza con cui voglio risolvere le equazioni della struttura stellare e ripeto la procedura finch\'e $S_i<\epsilon$, $C_i<\epsilon$ - local errors doesn't propagate to other mesh
\end{frame}

\subsection{Metodo di shooting: soluzione primo modello stellare con condizioni al bordo}\linkdest{shooting}

\begin{frame}{Step temporale e problema modello iniziale}
Step temporale $\Delta t$: uso metodo di Henyey per determinare le nuove abbondanze with I equations as I elements accounted for.
\keyword{Problem of first model}: i) Turn on $\epsilon_g$ within few models ii) Initial model of evolutionary sequence is evaluated with \keyword{shooting method}
starting from outer mesh with 4 boundary conditions for $T_e$, $L_s$, $P_s$, $R$ and using first order approx. $y_j^i=y_{j\pm1}^i+\TDy{m}{y_{j\pm1}}\,dm$: we determine $(y_f^i)_{surf}$ until some mesh f midway between center and surface and then integrate from centerup to $j=f$ using central trial conditions - we have $(y_f^i)_{center}\neq(y_f^i)_{surf}$ and have to correct trial central boundary conditions
\begin{columns}[T]
	\begin{column}{0.5\textwidth}
\begin{align*}
&\Delta(y_f^i)_{surf}=\TDy{y_N^1}{(y^i_f)_{surf}}\delta y_N^1+\TDy{y_N^4}{(y^i_f)_{surf}}\delta y_N^4\\
&\Delta(y_f^i)_{center}=\TDy{y_1^2}{(y^i_f)_{center}}\delta y_1^2+\TDy{y_1^3}{(y^i_f)_{center}}\delta y_1^3
\end{align*}
	\end{column}
	\begin{column}{0.5\textwidth}
correzione ai 4 valori al contorno $\delta y_N^1=\delta R$, $\delta y_N^4=\delta L_s$ e $\delta y_1^2=P_c$, $\delta y_1^3=T_c$ found solving \[-\Delta_f=\Delta(y_f^i)_{center}-\Delta(y_f^i)_{surf}\]
	\end{column}
\end{columns}
Fails in advanced stages when structure has strong gradient - In massive stars late phases we can't ignore acceleration term
\end{frame}

 \part{Formazione ed evoluzione stellare}
%\begin{frame}{TOC formation evolution}
%\tableofcontents[sections]
%\end{frame}
\section{Star formation and pre MS}\linkdest{starformation}

\begin{wordonframe}{da fare: kippenhahn wiegert}
\begin{itemize}
\item Onset of star formation 248'-255' (131-134)
\item Formation of protostars 256'-265' (135-139)
\end{itemize}
\end{wordonframe}


\begin{frame}[allowframebreaks]{List of things}
%\printbibliography[keyword={inference},heading=beamer]
%\printbibliography[keyword={\mybibcat},heading=beamer]
\listofkeywords
\listoftodos
\end{frame}

\subsection{From clouds to proto-stars}\linkdest{protostellar}

\begin{frame}{Clouds collapse phases}
\begin{columns}[T]\begin{column}{0.5\textwidth}
\begin{itemize}
\item clouds equilibrium shape: Observed cloud mass greater than Jeans mass; Rotation and magnetic field support increases Jeans mass
\end{itemize}
	\end{column}\begin{column}{0.5\textwidth}

\end{column}\end{columns}
\end{frame}

\subsection{Pre main sequence ed approccio a ZAMS per stelle di sequenza superiori/inferiori}\linkdest{preMS}

\begin{frame}{Traccia di Hayashi}
Primo/secondo core di Larson; Evoluzione di PMS sulla traccia di Hayashi; ruolo di opacit\'a di H- nella verticalit\'a della traccia di Hayashi; fusione deuterio; stelle completamente convettive o con nucleo radiativo. Abbondanza elementi leggeri in stelle di pre-sequenza
\end{frame}


\begin{frame}{Jeans mass ($\num{e5}\msun$)}
\todo{Instability} $\omega^2=k^2c_s^2-4\pi G\rho<0$:
\[\lambda>\lambda_H=\sqrt{\frac{\pi c_s^2}{G\rho}}=\SI{0.19}{\parsec}\sqrt{\frac{T}{\SI{10}{\kelvin}}}(\frac{n_{H^2}}{\SI{e4}{\per\cubic\cm}})\expy{-1/2}\]
\end{frame}

\begin{frame}{Protostar: first core and main accretion}
Virial T. $2E_i+\Omega=0$: $3kT\frac{M}{\mu m_H}=\int_0^M\frac{Gm}{r}\,dm$: collapse $\frac{3kTM}{\mu m_H}<\frac{3}{5}\frac{GM^2}{R}$ - $M>M_J=(\frac{3}{4\pi\rho})\expy{1/2}(\frac{5kT}{G\mu m_H})\expy{3/2}\propto T\expy{\frac{3}{2}}\rho\expy{-\frac{1}{2}}$
$t_{ff}\approx(G\rho)\expy{-\frac{1}{2}}\ll t_{cool}$: collasso adiabatico $P\propto T\expy{5/2}$ ($PT\expy{\frac{\gamma}{1-\gamma}}$) - $t_{ff}\gg t_{cool}$: collasso isotermo - \keyword{fragmentation}: Hydrostatic core surrounded by ff gas ($0.01\msun$) \todo{star formation hydrostatic core}
\end{frame}
\section{Evoluzione stellare}

\begin{wordonframe}{da fare: kippenhahn wiegert}
\begin{itemize}
\item main sequence 207'-214' (110-114)
\item Hayashi line 224'-232' (119-123)
\item Stability 234'-246 (124-130)
\item Onset of star formation 248'-255' (131-134)
\item Formation of protostars 256'-265' (135-139)
\item pre-main sequence contraction 266'-270' (140-142)
\item from initial to present sun 271'-276' (142-144)
\item chemical evolution in MS 277'-291' (145-152)
\item He-burning: massive stars 292'-307' (153-160)
\item He-burning:low-mass stars 308'-327' (161-170)
\item Later phases:  328'-343' (171-178)
\item Explosion and collapse 344'-364' (179-189)
\end{itemize}
\end{wordonframe}

\subsection{Pre main sequence ed approccio a ZAMS per stelle di sequenza superiori/inferiori}

\begin{frame}{Traccia di Hayashi}
Primo/secondo core di Larson; Evoluzione di PMS sulla traccia di Hayashi; ruolo di opacit\'a di H- nella verticalit\'a della traccia di Hayashi; fusione deuterio; stelle completamente convettive o con nucleo radiativo. Abbondanza elementi leggeri in stelle di pre-sequenza
\end{frame}

\begin{frame}{Approccio alla ZAMS per stelle di sequenza superiori/inferiori}
dipendenza della ZAMS dall'abbondanza originale di He e metalli; metodo determinazione $DY/DZ$ dal confronto teoria-osservazione per stelle di disco locale parallassate; dipendenza massa minima di transizione dall'abbondanza di He e metalli; influenza sulla ZAMS dell'incertezza degli input fisici e dell'efficienza della convezione
\end{frame}

\subsection{Evoluzione di sequenza principale}

\begin{frame}{Yield of H-burning star analysis}
\begin{itemize}
\item Longest evolutionary phase: larger number of observed stars
\item central/shell-H-burning determine successive phases
\item most important clock is central H-burning termination
\item Final shell H-burning phase in low mass, low-Z star provide distance indicator for old stellar pop
\item count of stars evolving through central H-burning give insight on IMF
\end{itemize}
\end{frame}

\begin{frame}{Major H burning reaction: PP chains}
\begin{columns}[T]\begin{column}{0.5\textwidth}
\begin{align*}
&^1H+^1H\to^2D+\APelectron+\Pnue\\
&^2D+^1H\to^3He+\gamma\\
&T\geq\SI{8e6}{\kelvin}: ^3He+^3He\to^4He+2^1H\tag*{PPI}\\
&T\geq\SI{15e6}{\kelvin}: ^3He+^4He\to^7Be+\gamma\\
&^7Be+\Pelectron\to^7Li+\Pnue\\
&^7Li+^1H\to^4He+^4He\tag*{PPII}\\
&^7Be+^1H\to^8B+\gamma\\
&^8B\to^8Be+\APelectron+\gamma\\
&^8Be\to2^4He
\end{align*}
\end{column}\begin{column}{0.5\textwidth}
\begin{align*}
&r_{pp}=\num{11.5e10}\rho^2X_H^2T_6\expy{-2/3}\exp{-33.81T_6\expy{-1/3}}(1+0.0123T_6\expy{1/3}+0.0109T_6\expy{2/3}+0.00095T_6)\\
&\rho\epsilon(3H\to^3He)=(\SI{6.936}{\mega\ev}-\SI{0.263}{\mega\ev})*\SI{1.602e-6}{\erg}*r_{pp}\\
&\rho\epsilon(^3He(^3He,2p)^4He)=(\SI{6.936}{\mega\ev}-\SI{0.263}{\mega\ev})*\SI{1.602e-6}{\erg}*r_{pp}\\
&\frac{PPI}{PPII+PPIII}=\frac{r_{33}}{r_{34}}=\frac{\lambda_{33}(^3He)^2/2}{\lambda_{34}^3He^4He}
\end{align*}
\end{column}\end{columns}
\end{frame}

\begin{frame}{H burning: CN-NO cycle}
\begin{columns}[T]\begin{column}{0.5\textwidth}
Ciclo CN-NO
\begin{align*}
&^{12}C+^1H\to^{13}N+\gamma\\
&^{13}N\to^{13}C\APelectron+\Pnue\\
&^{13}C+^1H\to^{14}N+\gamma\\
&^{14}N+^1H\to^{15}O+\gamma\\
&^{15}O\to^{15}N+\APelectron+\Pnue\\
&^{15}+^1H\to^{12}+^4He\tag*{CN}\\
&T\geq\SI{20e6}{\kelvin}\tag*{\num{e-4} volte}\\
&^{15}N+^1H\to^{16}O+\gamma\\
&^{16}O+^1H\to^{17}F+\gamma\\
&^{17}F\to^{17}O+\APelectron+\Pnue\\
&^{17}O+^1H\to^{14}N+^4He
\end{align*}
\end{column}\begin{column}{0.5\textwidth}
\begin{align*}
&\epsilon_{CN}(T_6)=\epsilon_{CN}(25)(\frac{T_6}{25})^{16.7}
\end{align*}
\end{column}\end{columns}
\end{frame}


\subsection{Lower main sequence ($M^*\leq1.3\msun{}$)}

\begin{frame}{Da ZAMS a TO for $M^*\leq1.3\msun{}$}
\begin{columns}[T]\begin{column}{0.5\textwidth}
\begin{figure}[!ht]\includegraphics[trim={0cm 0cm 0 0},clip, keepaspectratio,width=0.99\textwidth]{HDR-LMS}\label{fig:HDR-LMS}\end{figure}
\end{column}\begin{column}{0.5\textwidth}
ZAMS: first MS model fully supported by H-burning in which secondary elements are in equilibrium:
\begin{itemize}
\item $^3He$ production in central zones: si forma piccolo core convettivo
\[\TDy{t}{N_3}=N_1N_2\exv{\sigma v}_{12}-2\frac{(N_3)^2}{2}\exv{\sigma v}_{33}-N_3N_4\exv{\sigma v}_{34}\]
\item La stella continua a contrarsi fino alla partenza della reazione $^3He(^3He,2^1H)^4He$ e il core convettivo svanisce con l'espandersi della regione in cui \'e prodotto $^3He$
\end{itemize}
Struttura: H-burn in central radiative core (Small T-dep of $\epsilon_{PP}$), convective core (large opacity associated to partial ionized H, He)
Evoluzione: $\#$ free particles $\downarrow$, \xaumenta{\mu} per HE \xaumenta{T}
, \xdiminuisce{R} quindi $L^*$ aumenta lentamente.
TO is hottest point in evolutionary track: H exhausted 
\end{column}\end{columns}
\end{frame}


\subsection{Upper main sequence ($M^*\geq1.2-1.3\msun{}$)}

\begin{frame}{Da ZAMS a Overall contraction}
\begin{columns}[T]\begin{column}{0.5\textwidth}
\begin{figure}[!ht]\includegraphics[trim={0cm 0cm 0 0},clip, keepaspectratio,width=0.99\textwidth]{HDR-UMS}\label{fig:HDR-UMS}\end{figure}
\end{column}\begin{column}{0.5\textwidth}
Higher T: CNO dominant: $\epsilon_{CNO}$ steeper:convective core (\xaumenta{M^*}, \xaumenta{M_{con}}, \xaumenta{P_{rad}}, \xdiminuisce{\nad{}}.
\end{column}\end{columns}
\end{frame}

\begin{frame}{Problem of convective core boundary: Overshooting}
\begin{block}{What increases convective core}
\begin{itemize}
\item Changes in physical input
\item Stellar rotation
\item Physical overshooting
\end{itemize}
\end{block}
\begin{block}{Effects of increased convective core}
\begin{itemize}
\item \xaumenta{M_c}, \xaumenta{\mu} (involve more mass), \xaumenta{L}
\item Longer central H-burning
\item Larger He core at end of MS: brighter He-burning phase star, shorted lifetime
\end{itemize}
\end{block}
\begin{block}{\sch vs Ledoux in-stability criterion}
In radiative region the retracting convective core leave chem composition gradient; but \xaumenta{He}, \xdiminuisce{\kappa}
\end{block}
\end{frame}

\subsection{Easurimento H centrale}

\begin{frame}{Esaurimento H per stelle di S superiore/inferiore}

\end{frame}

\subsection{Combustione He}

\begin{frame}{Fase subgigante rossa (SGB): H-burning shell}
Turnoff; overall contraction;Gap di hertzsprung;
\end{frame}

\begin{frame}{Fase di gigante rossa (innesco He)}
primo dredge-up; RGB transition mass; morfologia RGB per stelle piccola/intermedia;
Dipendenza RGB-transition mass da composizione;
Massa del core di elio all'innesco in funzione della massa stellare
Luminosit\'a tip rgb vs massa, massa nucleo He a innesco, composizione
bump rgb;
Innesco He a flash per piccole masse
\end{frame}

\begin{frame}{Dipendenza da He/composizione iniziale del SG-RG branch}

\end{frame}

\begin{frame}{Evoluzione da ZAHB}
combustion di He per stelle medio-grandi; clump He; loop He; Esaurimento He centrale: autotrascinamento del nucleo, semiconvezione e pulsi convettivi
\end{frame}

\begin{frame}{Discussioni parametri che influenzano HB}
Parametro R per determinazione He
\end{frame}

\subsection{Ramo asintotico (AGB)}

\begin{frame}{Evoluzione in AGB}
Ingresso in agb: clump. AGB manqu\'e. Secondo Dredge-up; pulsi termici; terzo dredge-up
\end{frame}

\begin{frame}{Nucleosintesi in AGB}
Elementi s:tasca C13. Produzione di Li
\end{frame}

\subsection{Destino finale di stelle massicce}

\begin{frame}{Destino finale di stelle di varia massa}
nane bianche di He, C/O, O/Ne; supernovae di tipo II da deflagrazione del carbonio e cattura elettronica su nuclei; supernovae di tipo II da fotodisintegrazione del ferro; Caratteristiche di pre-SN; Neutronizzazione esplosiva e esplosione ritardata; classificazione SN in base a spettro e morfologia curva di luce
\end{frame}

\begin{frame}{Caratterizzazione SNII}
$Fe60$ come indicatore esplosioni nelle vicinanze della terra negli ultimi milioni di anni; Interazione neutrini-nucleo denso: intrappolamento e tempi scala emissione; stima energia emessa: neutrini, fotoni, fronte di shock; SN1987A: flusso di neutrini e osservazione in bande EM
\end{frame}


\begin{frame}{Evoluzione di nane bianche}

\end{frame}

\subsection{Modello Solare standar}

\begin{frame}{Caratteristiche e metodo di calcolo}
eliosismologia e neutrini
\end{frame}

\subsection{Stelle pulsanti}

\begin{frame}{Striscia di instabilit\'a e tipi di stelle pulsanti}
RR LyrAE: diagramma di Bailey, curve di luce; relazioni periodo-luminosit\'a, massa-temperatura effettiva.
Parametro A come indicatore di He.
Stelle cefeidi; dicotomia di Oosterhoff
\end{frame}
 
 \part{Stellar population in the milky way}\linkdest{constraints}
\begin{frame}{this part toc}
\begin{columns}[T]
\begin{column}{0.5\textwidth}
\begin{itemize}
	\item Solar neighborhood
\end{itemize}
\end{column}
\begin{column}{0.5\textwidth}
\tableofcontents
\end{column}
\end{columns}
\end{frame}
\section{Galaxy}

\begin{frame}{Refs}
The Gaia-ESO Survey: the Galactic Thick to Thin Disc transition
\end{frame}

\subsection{Via Lattea e gruppo locale. Teorie di formazione galattica}

\subsection{Popolazioni stellari}

\subsection{cluster stellari}

\begin{frame}{Distanza ammassi}
main sequence fitting
tip rgb
ZAHB
clump elio
RR lyrae
cefeidi
\end{frame}

\begin{frame}{indicatori di et\'a}
Turn-off/overall contraction: Isocrone di ammassi giovani/vecchi; metodo orizzontale e vertical per ammassi antichi; Lithium depletion boundary per datazione di ammassi giovani
\end{frame}

\begin{frame}{Indicatori di He (e Z??)}
??
\end{frame}

\section{Star formation - IMF}

\begin{frame}{Popolazioni in galassie esterne}
star formation history; popo non risolte semplici e complesse
\end{frame}
 
\end{document}

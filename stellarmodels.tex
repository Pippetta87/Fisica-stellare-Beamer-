\section{Equazioni struttura di equilibrio}

\begin{wordonframe}{da fare: kippenhahn wiegert}
\begin{itemize}
\item strutture autogravitanti 1-62' (43)
\item metodi numerici 77'-84(44-48)
\item esistenza e unicit\'a 85'-99'(48-56)
\item Properties of stellar matter: ideal gas with radiation, ionization, degenerate electron gas, equazione di stato, opacit\'a 102'-144'(57-78)
\item produzione energia reazioni nucleari:  146'-172'(79-92)
\item politrope 174'-190' (93-102)
\end{itemize}
\end{wordonframe}

Determino la struttura solare integrando numericamente le equazioni fondamentali della struttura stellare
\begin{subequations}\label{subeqn:stellarstructure}
\begin{align}
&\TDy{r}{m}=4\pi r^2\rho\\
&\TDy{r}{P}=-\frac{Gm(r)\rho(r)}{r^2}\\
&\TDy{r}{T}=\nabla\frac{T}{p}\TDy{r}{p}\\
&\TDy{r}{L}=4\pi r^2[\rho(\epsilon-\epsilon_{\nu})-\rho\TDof{t}u+\frac{P}{\rho}\TDy{t}{\rho}]
\end{align}

\begin{equation}
\PDy{t}{n_i}+\frac{1}{r^2}\PDof{r}(r^2n_iv_i)=\Dcvar{\PDy{t}{n_i}}{Nucl}\label{eq:difffusionchange}
\end{equation}
\end{subequations}
con $v_i$ velocit\'a di diffusione della specie i. Ottengo il profilo radiale delle grandezze $\{P,m,T,L,X_i\}$, note la metallicit\'a iniziale Z, l'equazione di stato $P(\rho,T,X_i)$, l'opacit\'a $\kappa(P,T,X_i)$, il rate di produzione di energia nucleare per grammo $\epsilon(P,T,X_i)$.

\section{Evoluzione stellare}
\begin{wordonframe}{da fare: kippenhahn wiegert}
\begin{itemize}
\item main sequence 207'-214' (110-114)
\item Hayashi line 224'-232' (119-123)
\item Stability 234'-246 (124-130)
\item Onset of star formation 248'-255' (131-134)
\item Formation of protostars 256'-265' (135-139)
\item pre-main sequence contraction 266'-270' (140-142)
\item from initial to present sun 271'-276' (142-144)
\item chemical evolution in MS 277'-291' (145-152)
\item He-burning: massive stars 292'-307' (153-160)
\item He-burning:low-mass stars 308'-327' (161-170)
\item Later phases:  328'-343' (171-178)
\item Explosion and collapse 344'-364' (179-189)
\end{itemize}
\end{wordonframe}
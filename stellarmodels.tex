\section{Equazioni struttura di equilibrio}

\begin{wordonframe}{da fare: kippenhahn wiegert}
\begin{itemize}
\item strutture autogravitanti 1-62' (43)
\item metodi numerici 77'-84(44-48)
\item esistenza e unicit\'a 85'-99'(48-56)
\item Properties of stellar matter: ideal gas with radiation, ionization, degenerate electron gas, equazione di stato, opacit\'a 102'-144'(57-78)
\item produzione energia reazioni nucleari:  146'-172'(79-92)
\item politrope 174'-190' (93-102)
\end{itemize}
\end{wordonframe}

\begin{frame}{Trasporto radiativo e convettivo: \'e valida approx idrostatica}

\end{frame}

\begin{frame}{Equazioni struttura di equilibrio}
Determino la struttura solare integrando numericamente le equazioni fondamentali della struttura stellare
\begin{subequations}\label{subeqn:stellarstructure}
\begin{align}
&\TDy{r}{m}=4\pi r^2\rho\\
&\TDy{r}{P}=-\frac{Gm(r)\rho(r)}{r^2}\\
&\TDy{r}{T}=\nabla\frac{T}{p}\TDy{r}{p}\\
&\TDy{r}{L}=4\pi r^2[\rho(\epsilon-\epsilon_{\nu})-\rho\TDof{t}u+\frac{P}{\rho}\TDy{t}{\rho}]
\end{align}

\begin{equation}
\PDy{t}{n_i}+\frac{1}{r^2}\PDof{r}(r^2n_iv_i)=\Dcvar{\PDy{t}{n_i}}{Nucl}\label{eq:difffusionchange}
\end{equation}
\end{subequations}
con $v_i$ velocit\'a di diffusione della specie i. Ottengo il profilo radiale delle grandezze $\{P,m,T,L,X_i\}$, note la metallicit\'a iniziale Z, l'equazione di stato $P(\rho,T,X_i)$, l'opacit\'a $\kappa(P,T,X_i)$, il rate di produzione di energia nucleare per grammo $\epsilon(P,T,X_i)$.
\end{frame}

\begin{frame}{Energia gravitazionale}

\end{frame}

\subsection{Metodi per integrazione equazioni di struttura e raccordo con modelli dell'atmosfera stellare}

\begin{frame}{Atmosfera e subatmosfera}

\end{frame}

\begin{frame}{Metodo del fitting}

\end{frame}

\begin{frame}{Metodo di Henyey}

\end{frame}

\subsection{Relazioni approssimate per grandezze stellari fondamentali}

\begin{frame}{Teorema del viriale}

\end{frame}

\begin{frame}{Relazione massa, densit\'a, temperatura/ massa, peso molecolare, opacit\'a}

\end{frame}

\subsection{Approssimazione politropa}

\begin{frame}{Equazione Lane-Emden}

\end{frame}

\begin{frame}{Strutture isoterme e modello solare}

\end{frame}

\section{Trasporto (di energia)}

\subsection{Radiativo}

\begin{frame}{Trasporto radiativo}
media di rosseland
\end{frame}

\begin{frame}{Opacit\'a radiativa (analitico)}
Scattering elettronico, processi ff, fotoionizzazione; opacit\'a atmosfera: ione H-
\end{frame}

\begin{frame}{Criterio di stabilit\'a per convezione}

\end{frame}

\subsection{Conduzione}

\begin{frame}{Diffusione per conduzione}
stima opacit\'a per conduzione gas degenere NR
\end{frame}

\subsection{Convezione}

\begin{frame}{Criterio di \sch e Ledoux}

\end{frame}

\begin{frame}{Mixing length per esterni stellari}
Calcolo altezza scala di pressione, flusso convettivo e gradiente ambientale in funzione della velocit\'a media degli elementi
\end{frame}

\section{Equazione di stato}

\begin{frame}{Approssimazione gas perfetto monoatomico}

\end{frame}

\begin{frame}{Degenerazione e- ed effetti coulombiani. Equazione di saha}

\end{frame}

\section{Fonti di energia}

\begin{frame}{Sezione d'urto nucleare}
schermaggio elettronico nelle stelle
\end{frame}

\begin{frame}{Catena PP}
dipendenza da T
flusso neutrini
\end{frame}

\begin{frame}{Biciclo CNO}
dipendenza da T
flusso neutrini
Modalit\'a combustione H in He: sequenza principale inferiore/superiore
\end{frame}

\begin{frame}{Combustion He}

\end{frame}

\begin{frame}{Produzione nuclei fino al Fe56}
3$\alpha$: $He4+\alpha\to C12$, $C12+\alpha\to O16$
Produzione neutroni liberi
Fusione $C12$, fotodisintegrzione $Ne20$, fusione $O16$, fotodisintegrzione $Si28$, catture $\alpha$ su nuclei fino a produzione $Fe56$
\end{frame}

\begin{frame}{Cattura neutronica: processi r e s}
picchi r e s nella nella curva universale delle abbondanze
\end{frame}

\section{Evoluzione stellare}

\begin{wordonframe}{da fare: kippenhahn wiegert}
\begin{itemize}
\item main sequence 207'-214' (110-114)
\item Hayashi line 224'-232' (119-123)
\item Stability 234'-246 (124-130)
\item Onset of star formation 248'-255' (131-134)
\item Formation of protostars 256'-265' (135-139)
\item pre-main sequence contraction 266'-270' (140-142)
\item from initial to present sun 271'-276' (142-144)
\item chemical evolution in MS 277'-291' (145-152)
\item He-burning: massive stars 292'-307' (153-160)
\item He-burning:low-mass stars 308'-327' (161-170)
\item Later phases:  328'-343' (171-178)
\item Explosion and collapse 344'-364' (179-189)
\end{itemize}
\end{wordonframe}

\subsection{Pre main sequence ed approccio a ZAMS per stelle di sequenza superiori/inferiori}

\begin{frame}{Traccia di Hayashi}
Primo/secondo core di Larson; Evoluzione di PMS sulla traccia di Hayashi; ruolo di opacit\'a di H- nella verticalit\'a della traccia di Hayashi; fusione deuterio; stelle completamente convettive o con nucleo radiativo
\end{frame}

\begin{frame}{Approccio alla ZAMS per stelle di sequenza superiori/inferiori}
dipendenza della ZAMS dall'abbondanza originale di He e metalli; metodo determinazione $DY/DZ$ dal confronto teoria-osservazione per stelle di disco locale parallassate; dipendenza massa minima di transizione dall'abbondanza di He e metalli; influenza sulla ZAMS dell'incertezza degli input fisici e dell'efficienza della convezione
\end{frame}

\subsection{Evoluzione di sequenza principale}

\begin{frame}{Esaurimento H per stelle di S superiore/inferiore}

\end{frame}

\begin{frame}{Fase subgigante rossa (SGB): H-burning shell}
Turnoff; overall contraction;Gap di hertzsprung;
\end{frame}

\begin{frame}{Fase di gigante rossa (innesco He)}
primo dredge-up; RGB transition mass; morfologia RGB per stelle piccola/intermedia;
Dipendenza RGB-transition mass da composizione;
Massa del core di elio all'innesco in funzione della massa stellare
Luminosit\'a tip rgb vs massa, massa nucleo He a innesco, composizione
bump rgb;
Innesco He a flash per piccole masse
\end{frame}

\begin{frame}{Dipendenza da He/composizione iniziale del SG-RG branch}

\end{frame}

\begin{frame}{Evoluzione da ZAHB}
combustion di He per stelle medio-grandi; clump He; loop He;
\end{frame}

\begin{frame}{Discussioni parametri che influenzano HB}
Parametro R per determinazione He
\end{frame}

\subsection{Ramo asintotica (AGB)}


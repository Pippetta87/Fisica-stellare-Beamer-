\section{A cosa mi serve?}

\section{RegLez 19}
\begin{frame}[allowframebreaks]{Reg Lez 19}
\begin{itemize}
    \item 18/02/2019 Descrizione del corso. Descrizione generale della via Lattea. Definizione di ammasso stellare. Ammassi aperti ed ammassi globulari. Caratteristiche generali delle stelle di disco e di alone. Richiami alle definizioni di magnitudine, indice di colore, modulo di distanza. Cenni alla curva universale delle abbondanze (nel disco galattico). Alpha enhancement.  
    \item 19/02/2019 Caratteristiche generali del bulge e del thick disk. Discussione dell'origine dell'alpha enhancement. Scenario qualitativo generale della formazione della nostra Galassia. Alone interno ed alone esterno e relazione con la cattura di galassie nane durante la formazione della Galassia. Cenni alla formazione inside out del disco. Generalità su diagrammi colore- magnitudine di disco e di campo. Present day mass function ed initial mass function. Relazione massa-luminosit\'a. 
    \item 22/02/2019 Cenni alla derivazione dello Star formation rate e della relazione massa-luminosit\'a per il disco della nostra Galassia. Cenni alla produzione di elementi nella nucleosintesi primordiale. Generalit\'a sui metodi di determinazione delle abbondanze chimiche nel sistema solare. 
    \item 25/02/2019 Discussione del problema della determinazione dell'elio primordiale e dell'elio in stelle di disco. Cenni alla struttura del gruppo locale ed alle caratteristiche delle popolazioni stellari all'interno delle galassie del Gruppo Locale. Discussione del problema della multipopolazione negli ammassi globulari. 
    \item 26/02/2019 Generalit\'a su strutture autigravitanti. Stima del tempo scala dinamico. Richiamo al tempo di free fall. Equazioni di equilibrio stellare: equilibrio idrostatico ed equazione di continuit\'a. Peso molecolare medio. Esempio: calcolo del peso molecolare medio per materia completamente ionizzata. Peso molecolare medio degli ioni. Peso molecolare medio per elettrone. Generalit\'a sull'equazione del trasporto radiativo e sull'opacit\'a. Dimostrazione per ordini di grandezza che la presenza di un flusso negli interni stellari non contraddice l'assunzione di equilibrio termodinamico. 
    \item 01/03/2019 Equilibrio termico per le stelle. Stima del tempo scala termico. Equazione dell'equilibrio termico (di conservazione dell'energia). Energia persa per neutrini; cenni ai processi principali. Calcolo della produzione di energia gravitazionale; problema della stima dell'energia gravitazionale del "primo modello" calcolato. 
    \item 04/03/2019 Integrazione delle equazioni di equilibrio stellare: interno, subatmosfera ed atmosfera. Cenni ai metodi di integrazione dell'atmosfera (variabile indipendente profondit\'a ottica). Metodo iterativo per il calcolo delle variabili fisiche esterne in atmosfera. Cenni al problema della determinazione della pressione di turbolenza e del gradiente ambientale nelle zone convettive esterne. Metodi numerici di integrazione delle equazioni di equilibrio stellare: il metodo del fitting. 
    \item 05/03/2019 Metodo di Henyey per l'integrazione delle equazioni di equilibrio stellare. Relazione tra energia termica ed energia gravitazionale in una struttura stellare all'equilibrio idrostatico. Il teorema del viriale per le strutture stellari: gas perfetto monoatomico e non monoatomico. Criterio di stabilit\'a per le strutture stellari in base al valore della costante adiabatica del gas. Relazioni per ordini di grandezza tra quantit\'a fisiche in strutture stellari ricavate utilizzando le equazioni di equilibrio stellare ed il teorema del viriale: relazione tra massa, densita' e temperatura, relazione tra luminosit\'a, massa, peso molecolare medio ed opacit\'a. 
    \item 08/03/2019 Approssimazione di strutture stellari a politropiche. Equazione di Lane-Emden e modalit\'a di risoluzione. Cenni a strutture isoterme. Esempio. il modello solare approssimato con una politropica. 
    \item 11/03/2019 Equazione del trasporto radiativo. Generalit\'a sul trasporto di energia tramite conduzione elettronica. Coefficiente di diffusione per conduzione. Definizione dell'"opacità di conduzione" e sua stima approssimata nel gaso di gas degenere non relativistico. Generalita' su calcolo dell'opacita' radiativa. Calcolo dell'opacit\'a (in forma analitica) per: scattering elettronico, processi free free, fotoionizzazione. Opacit\'a nelle atmosfere stellari: opacit\'a di interazione fotoni-ione H- . 
    \item 12/03/2019 Seminario del dott. Tognelli su: l'equazione di stato delle strutture stellari. Gas degenere elettronicamente. Effetti coulombiani nelle strutture stellari. Equazione di Saha.
\end{itemize}\end{frame}

\section{RegLez 17}

\begin{frame}[allowframebreaks]{Reg Lez 17}
\begin{itemize}
\item testo: Popolazioni stellari: formazione stellare (fenomenologia)
\item lezione: Descrizione generale della struttura della nostra Galassia. Concetti base: parallasse, magnitudine assoluta ed apparente, modulo di distanza, estinzione ed arrossamento. Caratteristiche generali di ammassi aperti e globulari. Alpha enhancement.
\item lezione: Caratteristiche fotometriche, dinamiche e chimiche delle popolazioni stellari della nostra Galassia. Caratteristiche generali delle stelle di thick disk. Relazione massa-luminosita' per le stelle di sequenza principale. Relazione generale tra massa e tempo di vita di una stella. Initial Mass Function and Present Day Mass Function.
\item lezione: Differenze generali tra stelle di ammasso e stelle di campo. Esempi di diagrammi Colore-Magnitudine di stelle di ammasso e stelle di campo. Discussione generale sullo studio delle caratetristiche delle stelle di campo. Cenni alla determinazione dello ''Star Formation Rate'' per il disco della nostra Galassia ed alla determinazione della relazione et\'a-metallicit\'a. 
\item lezione: Scenario generale di formazione della Via Lattea. Cenni alle galassie del Gruppo Locale. Cenni alle popolazioni stellari nella Via Lattea e nella galassie del Gruppo Locale. Metodi di determinazione delle abbondanze stellari. La curva ''universale'' delle abbondanze. Cenni alla nucleosintesi primordiale.
\item lezione: Discussione generale sulla multipopolazione negli ammassi globulari della nostra Galassia. 
\item Testo: Equazioni struttura stellare: fenomenologia e metodi numerici
\item lezione: Equilibrio idrostatico nelle stelle. Equazioni di equilibrio stellare: equilibrio idrostatico ed equazione di continuita'.
\item lezione: Peso molecolare medio. Esempio: calcolo del peso molecolare medio per materia completamente ionizzata. Peso molecolare medio degli ioni. Peso molecolare medio per elettrone. Generalit\'a sull'equazione del trasporto radiativo. Equilibrio termico. Le equazioni di equilibrio stellare. Calcolo dell'energia "gravitazionale".
\item lezione: Calcolo approssimato del tempo scala termico nell'interno del Sole. L'equazione del trasporto nelle atmosfere stellari. Profondit\'a ottica. Equazioni di equilibrio stellare in atmosfera. Generalit\'a sui metodi numerici di integrazione delle equazioni di equilibrio stellare. Il metodo del fitting.
\item lezione: Seminario del dott. Tognelli su: l'equazione di stato delle strutture stellari. Gas degenere elettronicamente. Effetti coulombiani nelle strutture stellari. Equazione di Saha.
\item lezione: Il metodo di Henyey per la risoluzione delle equazioni di equilibrio stellare. Integrazione dell'atmosfera. L'equazione del trasposto radiativo. Il teorema del viriale per le strutture stellari: caso di gas perfetto monoatomico. Tempo scala di Kelvin-Helmholtz.
\item lezione: Il teorema del viriale: gas perfetto non monoatomico. Criterio di stabilit\'a delle strutture stellari. Utilizzo delle equazioni di equilibrio stellare e del teorema del viriale per ottenere relazioni per ordini di grandezza tra: 1) massa, densit\'a e temperatura delle stelle 2) massa-luminosita'.
\item lezione: Calcolo approssimato dell'opacit\'a da scattering Thompson nel caso di ionizzazione totale. Formula di Kramer per opacit\'a free-free e bound free. Opacit\'a legata agli ioni H-.
\item lezione: La conduzione elettronica. Opacita' conduttiva. Equazione del trasporto in presenza di opacita' conduttiva. Criterio di Schwarzschild e di Ledoux per l'innesco della convezione in ambiente stellare. Cenni al fenomeno dell'overshooting.
\item lezione: Il metodo della mixing lenght per il trattamento della convezione negli inviluppi esterni stellari. Calcolo approssimato dell'altezza di scala della pressione. 
\item lezione: La teoria della mixing lenght per il trattamento della convezione negli esterni stellari: calcolo del flusso convettivo, della velocita' media delle bolle di convezione e del gradiente ambientale negli esterni stellari. Modelli politropici di strutture stellari: equazione di Lane-Emden e calcolo dell'andamento di pressione e densita' per modelli stellari politropici. Esempi: il modello solare.
\item Testo: Produzione enrgia: reazioni nucleari (evoluzione)
\item lezione: Calcolo dei rates di reazioni di fusione nucleare tra particelle cariche. La probabilita' di penetrazione della barriera coulombiana ed il fattore astrofisico. Il picco di Gamow. Espressione approssimata dei rates di fusione nucleare. Dipendenza approssimata delle reazioni dalla temperatura.
\item lezione: Lo schermaggio elettronico in laboratorio. Lo schermaggio elettronico nel plasma stellare: schermaggio debole, intermedio e forte. Trattamento dello schermaggio debole negli interni stellari secondo il metodo di Salpeter. 
\item lezione: Reazioni nucleari di combustione di elementi leggeri. Elementi primari ed elementi secondari. Concentrazione di equilibrio per gli elementi secondari. Reazioni di fusione di H in He: la catena protone-protone ed il bi-clo CN-NO. Neutrini solari.
\item lezione: Reazioni del ciclo CNO veloce. Reazioni di fusione di elio in C ed O. Catene di produzione di neutrini liberi in fasi evolutive avanzate.
\item lezione: Reazioni di fusione del C e dell'O. Reazioni nucleari successive fino alla produzione degli elementi del picco del ferro. Struttura a cipolla di pre-supernova e nucleosintesi esplosiva. Catture neutroniche su nuclei: processi s e processi r. Sezione d'urto per cattura neutronica ed andamento con l'energia del rate delle reazioni di cattura neutronica. Stima del flusso di neutroni caratteristico di processi s ed r. Stima dell'abbondanza di equilibrio per processi s. Spiegazione qualitativa dell'andamento dei picchi "s" ed "r" nella curva universale delle abbondanze.
\item Testo: MS-PMS
\item lezione: Seminario del dott. Emanuele Tognelli su caratteristiche delle fasi di protostella e di Pre-Sequenza Principale (PMS). Traccia di Hayashi. Effetto di variazione di massa e composizione chimica in PMS. Combustione del deuterio ed effetti sulle strutture di PMS.
\item lezione: Caratteristiche e metodo di calcolo del solare standard. Generalita' sull'eliosismologia e sui modelli solari eliosismologici.
\item lezione: Seminario del dott. Tognelli su evoluzione di Pre-sequenza principale, evoluzione temporale dell'abbondanza superficiale di elementi leggeri. Ingresso in ZAMS per stelle di sequenza principale inferiore e superiore.
\item lezione: Dipendenza della posizione di PMS nel diagramma HR dalla composizione chimica e dall'efficienza della convezione esterna. Zero Age Main Sequence. Dipendenza della posizione di ZAMS nel diagramma HR dalla composizione chimica e dall'efficienza della convezione esterna. Stelle di sequenza principale inferiore e superiore. Approccio alla ZAMS: sviluppo di un piccolo nucleo convettivo per stelle di SPI durante il raggiungimento dell'abbondanza di equilibrio dell'3He. Profilo dell'abbondanza di equilibrio dell'3He per stelle di SPI.Very low mass. Calcolo approssimativo della massa minima per l'innesco della fusione di H in He. 
\item lezione: Evoluzione di sequenza principale. Modalit\'a di esaurimento dell'idrogeno centrale per stelle di sequenza principale inferiore e superiore. Fase di subgigante rossa. Isocrona di ammasso. La luminosita' all'esaurimento dell'idrogeno centrale come indicatore di et\'a di una popolazioen stellare semplice. 
\item lezione: Incertezze nelle previsioni teoriche di MS. Generalita' sulle isocrone di ammasso. Il metodo del fitting della MS per la determinazione della disatnza degli ammassi globulari. La luminosita' del TO/OC come indicatore di eta'. Evoluzione di sub-gigante rossa. Dipendenza delle tracce di MS/SGB dalla composizione chimica. Influenza della diffusione sulla traccia di stelle di data massa nel diagramma HR.
\item Testo: Post Hydrogen
\item lezione: Evoluzione di gigante rossa: il primo dredge up ed il bump dell'RGB. Dipendenza della luminosita' del bump dalla composizione chimica. Perdite di massa in RGB. Il flash dell'elio. Morfologia dell'RGB per ammassi stellari giovani ed antichi.
\item lezione: Massa dell'elio all'innesco dell'elio centrale in funzione della massa totale della stella. Massa di RGB transition e sua dipendenza dalla composizione chimica. Dipendenza della luminosita' del vertice del ramo della giganti rosse dalla composizione chimica. Il tip dell'RGB come indicatore di distanza.
\item lezione: Dipendenza dell'isocrona dalla composizione chimica e dalla diffusione. Fase di combustione di elio per stelle di ammasso antico: il ramo orizzontale, HB. Zero Age Horizontal Branch. Il ramo orizzontale come candela campione, metodo verticale per la determinazione dell'et\'a di ammassi antichi. Dipendenza della ZAHB dalla composizione chimica.
\item lezione: Influenza della diffusione sulla determinazione dell'eta' degli ammassi tramite il metodo verticale. Evoluzione di ramo orizzontale. Morfologia del ramo orizzontale: dipendenza da metallicita' ed et\'a dell'ammasso. Hot flashers. Il clump dell'elio ed il loop dell'elio. Il clump dell'elio come indicatore di distanza.
\item lezione: Effetto della presenza di overshooting sulla determinazione di et\'a di ammassi giovani. Discussione dell'incertezza sulla detrminazione di et\'a in ammassi stellari. Evoluzione in combustione di elio centrale: autotrascinamento e semiconvezione. Il parametro R per la stima dell'abbondanza di elio in ammassi antichi.
\item lezione: Cenni su stelle variabili come indicatori di distanza: RR Lyrae e cefeidi classiche. Ingresso in ramo asintotico. Il clump dell'AGB in stelle di piccola massa. Caratteristiche generali per stelle di AGB. Il secondo dredge up. Nucleosintesi in AGB. Pulsi termici.
\item lezione: Evoluzione della luminosit\'a durante i pulsi termici. Terzo dredge up, hot bottom burning. Catene di produzione di neutroni liberi e processi s in fase di ramo asintotico.
\item lezione: Evoluzione finale di stelle di varia massa. Nane bianche di C/O e O/Ne, supernovae di tipo II da deflagrazione del carbonio e da cattura elettronica su nuclei. Supernovae di tipo II da fotodisintegrazioen del ferro.
\item Lezione seminariale del prof. Prada Moroni su caratteristiche strutturali delle nane bianche, curva di raffreddamento di nana bianca. Misura di distanza ed et\'a in ammasso tramite la curva di raffreddamento delle nane bianche.
\item lezione: Lezione seminariale del prof. Cignoni su popolazioni stellari complesse nelle vicinanze del Sole e nelle galassie nane. recupero della star formation rate in popolazioni complesse.
\end{itemize}
\end{frame}
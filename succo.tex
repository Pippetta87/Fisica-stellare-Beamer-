%! TEX root = main.tex
\section{Succo}\linkdest{succo}

\begin{frame}{Pressione di radiazione}
Tutte le volte che un atomo emette/assorbe un fotone perde/guadagna quantit\'a di moto e dato che un atomo emette in maniera isotropa il momento netto \'e nullo una volta mediato su molte emissioni.
I processi di assorbimento non sono isotropicamente distribuiti dato il flusso uscente di energia per $cm^2$ per sec F: solo una frazione $\kappa$ del flusso di momento $\frac{F}{c}$ \'e assorbita dalla materia. Il trasferimento da parte della radiazione di momento alla materia per $cm^3$ per sec, cio\'e la forza esercitata dalla radiazione \'e $\kappa H \frac{1}{c}$.
Un elemento di volume $dS\,dr$ subisce per effetto dell'assorbimento della radiazione una variazione d'impulso $dq$, nel caso un fotone venga assorbito la variazione del flusso uscente \'e $dF<0$.
The distribution of photons over over different quantum states with energies $\epsilon_k=\hbar\omega_k$ (large volume $\omega_k\to\omega$)
\begin{align*}
\overline{n_k}=\frac{1}{\exp{\frac{\hbar\omega}{KT}}-1}
\end{align*}
Moltiplicando il numero di stati nel dato range di frequenze per la distribuzione di Plank (numero di occupazione) ottengo il numero di fotoni e l'energia radiativa nel range di frequenza
\begin{align*}
&dN_{\omega}=\frac{V}{\pi^2c^3}\frac{\omega^2\,d\omega}{\exp{\frac{\hbar\omega}{KT}}-1}\\
&dE_{\omega}=\frac{V\hbar}{\pi^2c^3}\frac{\omega^3\,d\omega}{\exp{\frac{\hbar\omega}{KT}}-1}
\end{align*}
\end{frame}

\begin{frame}{Gradiente per trasporto radiativo nell'interno stellare}
\begin{align*}
&dq=-(n_{\nu}\,dSc\,dt)*(\kappa\rho\,dr)*\frac{h\nu}{c}&\intertext{Il primo termine \'e il numero di fotoni pasanti per superficie $dS$ in tempo $dt$, il secondo \'e la probabilit\'a d'assorbimento attraverso spessore $dr$, il terzo \'e la quantit\'a di moto di ogni fotone.}\\
&dP_r=\TDy{S}{F}=\TDof{S}\TDy{t}{q}\\
&=-\int \,d\nu n_{\nu}c\kappa_{\nu}\rho\,dr\frac{h\nu}{c}\\
&F_{\nu}=n_{\nu}ch\nu,\\
&\TDy{r}{P(Rad)}=-\int\,d\nu\frac{F(Rad)}{c}\kappa_{\nu}\rho&\intertext{In condizioni di LTE posso confrontare $\uparrow$ con}\\
&P_{\nu}=\frac{1}{3}u_{\nu},\ P(Rad)=\frac{1}{3}aT^4&\intertext{e ricavare il gradiente di temperatura necessario per il flusso di energia $F(Rad)$:}\\
&\TDy{r}{T}=-\frac{3\kappa\rho l(r)}{16\pi acT^3r^2}
\end{align*}
\end{frame}

\begin{frame}{* formation}
	contenu...
\end{frame}

\begin{frame}[fragile]{Struttura di equilibrio}\frameintoc
\begin{itemize}
\item Equilibrio idrostatico: pressione in un mesh \'e il peso della materia sopra per unit\'a di superficie. Stabilit\'a e tempi reazione a perturbazione
\item Pressione radiativa. Una frazione $\kappa$ del flusso di momento $\frac{F}{c}$ \'e assorbita dalla materia (momentum transfer per $cm^3$ per sec): $dq=-(n_{\nu}\,dSc\,dt)*(\kappa\rho\,dr)*\frac{h\nu}{c}$, il primo termine \'e il numero di fotoni pasanti per superficie $dS$ in tempo $dt$, il secondo \'e la probabilit\'a d'assorbimento attraverso spessore $dr$, il terzo \'e la quantit\'a di moto di ogni fotone.
\begin{align*}
&dP_r=\TDy{S}{F}=\TDof{S}\TDy{t}{q}=-\int \,d\nu n_{\nu}c\kappa_{\nu}\rho\,dr\frac{h\nu}{c}\\
&F_{\nu}=n_{\nu}ch\nu,\ \TDy{r}{P(Rad)}=-\int\,d\nu\frac{F(Rad)}{c}\kappa_{\nu}\rho
\end{align*}
\begin{comment}
Un elemento di volume $dS\,dr$ subisce per effetto dell'assorbimento della radiazione una variazione d'impulso $dq$, nel caso un fotone venga assorbito la variazione del flusso uscente \'e $dF<0$
The distribution of photons over over different quantum states with energies $\epsilon_k=\hbar\omega_k$ (large volume $\omega_k\to\omega$) 
\begin{align*}
\overline{n_k}=\frac{1}{\exp{\frac{\hbar\omega}{KT}}-1}
\end{align*}
Moltiplicando il numero di stati nel dato range di frequenze per la distribuzione di Plank (numero di occupazione) ottengo il numero di fotoni e l'energia radiativa nel range di frequenza
\begin{align*}
&dN_{\omega}=\frac{V}{\pi^2c^3}\frac{\omega^2\,d\omega}{\exp{\frac{\hbar\omega}{KT}}-1}\\
&dE_{\omega}=\frac{V\hbar}{\pi^2c^3}\frac{\omega^3\,d\omega}{\exp{\frac{\hbar\omega}{KT}}-1}
\end{align*}
\end{comment}
\end{itemize}
\end{frame}

\begin{frame}{Rotazione, mass loss}
\begin{itemize}
\item Rotazione. \[\frac{\nabla P}{\rho}=-\nabla\phi+\vec{a}=-\nabla\phi+\Omega^2r_{\perp}=\vec{g}_{eff}\]
If $\nabla\wedge(\Omega^2r_{\perp})=0$: $\phi\to\phi-V$, $V=\int_0^{r_{\perp}}\Omega^2r_{\perp}\,dr_{\perp}$ (OK if $P(\rho)$, politrope, regioni convettive)
\end{itemize}
\end{frame}

\begin{frame}{Stability}

\end{frame}

\begin{frame}{Hayashi Line (HL)}

\end{frame}

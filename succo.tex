%! TEX root = main.tex
\section{Definizioni}

\begin{itemize}
    \item  Homologus star:they have the same $\frac{T}{T_c}$, $\frac{P}{P_c}$, $\frac{\rho}{\rho_c}$ when expressed in terms of $\frac{r}{R}$.
\end{itemize}

\section{Grandezze Fondamentali}\linkdest{succo}

\begin{frame}{Stime Euristiche: Tempo scala, Pressione e Temperatura centrali}
    \begin{columns}[T]
        \begin{column}{0.5\textwidth}
            \begin{align*}
                &\PDy{m}{P}=-\frac{Gm}{4\pi r^4}\Rightarrow \frac{P_0-P_c}{M}\approx \frac{2G(M/2)^2}{4\pi(R/2)^4}\\
                &\Rightarrow P_c\approx \frac{2GM^2}{\pi R^4}\\
                &\rho\xrightarrow{\text{ideal g.}}\frac{\mu}{R}\frac{P}{T}\Rightarrow T_c\approx \frac{P_c}{\rho_c}\frac{\mu}{R}\\
                &=P_c\frac{\mu}{R}\underbrace{\frac{\bar{\rho}}{\rho_c}}_{\approx0.01-0.03}\Rightarrow T_c<\frac{8}{3}\frac{G\mu}{R}\frac{M}{R}
            \end{align*}
        \end{column}
        \begin{column}{0.5\textwidth}
            \begin{align*}
                &f_P=-\PDy{m}{P}\,dm\\
                &f_g=-\frac{g\,dm}{4\pi r^2}=-\frac{Gm}{r^2}\frac{dm}{4\pi r^2}\\
                &\frac{dm}{4\pi r^2}\PtwoDy{t}{r}=f_P+f_g\Rightarrow \frac{1}{4\pi r^2}\PtwoDy{t}{r}=-\PDy{m}{P}-\frac{Gm}{4\pi r^4}\\
                &|\PtwoDy{t}{r}|\to \frac{R}{\tau_{ff}}\Rightarrow \frac{R}{\tau_{ff}}\approx g\Rightarrow\tau_{ff}\approx\sqrt{\frac{R}{g}}\\
                &\to \frac{R}{\tau_{expl}^2}\Rightarrow \frac{R}{\tau_{expl}^2}=\frac{P}{\rho R}=4\pi r^2\PDy{m}{P}=\PDy{m}{P}/\rho\\
                &\Rightarrow\tau_{expl}\approx R\sqrt{\frac{\rho}{P}}
            \end{align*}
        \end{column}
    \end{columns}
\end{frame}

\begin{frame}{Masse limite}
    \begin{itemize}
        \item $M_{up}$: Highest *-mass at which \Pelectron-degeneracy prevent C-ignition in CO core. Depends strongly on chem. composition - $M_{up}\approx8\msun{}$ at solar metallicity.
    \end{itemize}<++>
\end{frame}

\section{Equazioni struttura: Energy transport}

\begin{frame}{Radiative transport}
    \begin{columns}[T]
        \begin{column}{0.5\textwidth}
            \begin{align*}
                &I(\theta)\,d\Omega=cu(\theta)\,d\Omega\\
                &u=\int^{4\pi}u(\theta)\,d\Omega=\frac{1}{c}\int I(\theta)\,d\Omega\tag{Energy density}\\
                &J(\vec{r},\nu,t)=\invers{(4\pi)}\int I(\vec{r},\hat{n},\nu,t)\,d\Omega\tag{Mean Intensity}\\
                &P_r=\frac{1}{3}\int_0^{\infty}\frac{h\nu}{c}cn(\nu)\,d\nu=\frac{1}{3}u\tag{rad Press}\\
                &P_r=\frac{1}{c}\int I(\theta)\cos^2{\theta}\,d\Omega\tag{Rad Press}\\
                &=\int^{4\pi}\frac{I(\theta)\cos{\theta}}{c}\cos{\theta}\,d\Omega\\
                &=\frac{2\pi}{c}\int_0^{2\pi}I(\theta)\cos{\theta}\sin{\theta}\,d\theta\\
                &H=\int I(\theta)\cos{\theta}\,d\Omega\\
                &=2\pi\int_0^{\pi}I(\theta)\cos{\theta}\sin{\theta}\,d\theta\tag{Net En. Flux polar dir}
            \end{align*}
        \end{column}
        \begin{column}{0.4\textwidth}
            Pressione di radiazione: radiazione contenuta in parallelepipedo di superficie unitaria lunghezza c in direzione $\theta$ (l'asse polare forma con la radiazione un angolo $\theta$) quindi la sezione d'urto geometrica in direzione polare contiene fattore $\cos{\theta}$, la proiezione del momento rispetto alla direzione polare ha un altro $\cos{\theta}$.
        \end{column}
    \end{columns}
\end{frame}

\begin{frame}{Flusso di energia proporzionale al gradiente termico}
\begin{block}{Diffusion Approx: Stellar Interior Near TE}
    \begin{columns}[T]
        \begin{column}{0.5\textwidth}
    \begin{align*}
                &U=aT^4\\
                &''\vec{j}=-D\nabla n''\tag{diffusion}\\
                &D=\frac{1}{3}vl_p\\
                &\vec{F}_{\nu}=-D_{\nu}\nabla U_{\nu}\\
                &D_{\nu}=\frac{1}{3}cl_{\nu}=\frac{c}{3\kappa_{\nu}}\rho
            \end{align*}
        \end{column}
        \begin{column}{0.5\textwidth}
            \begin{align*}
                &I(\theta)=I_0+I_1\cos{\theta}+\ldots\\
                &u=\frac{4\pi}{c}I_0\\
                &H=\frac{4\pi}{3}I_1\\
                &P_r=\frac{4\pi}{3c}I_0
            \end{align*}
        \end{column}
    \end{columns}
    
\end{block}

\begin{block}{Momentum transfer Rad-Mat}
    \begin{columns}[T]
        \begin{column}{0.6\textwidth}
    \begin{align*}
        &dp=\frac{dF_{Rad}}{c}=\frac{F_{Rad}}{c}\frac{dr}{l}\\
        &\TDy{r}{P_{Rad}}=-\frac{\kappa\rho}{c}F_{Rad}\\
        &-\frac{F_{rad}(\nu)}{c}\kappa_{\nu}\rho\,dr=\frac{4\pi}{3c}\TDy{r}{B_{\nu}(T)}\,dr\tag{L-grad T}
    \end{align*}
        \end{column}
        \begin{column}{0.4\textwidth}
            Flux of photons through volume matter at r, flux of energy $F_{rad}$. $dp$ momentum transfered from photons to volume element, $l$ photon mean free path: $\invers{l}=\kappa\rho$, $dp$ opposite of change of $dP_{Rad}$ of pressure exerted by photons over dr    
        \end{column}
    \end{columns}
    
\end{block}
\end{frame}

\section{evolution}

\begin{frame}{Pre-MS}
    \begin{itemize}
        \item Lithium but not berylium burn s at bottom of convective zone for less massive yhen $1.3\msun{}$.
    \end{itemize}
\end{frame}

\section{Equazioni da ricordare}

\begin{frame}{Pressione di radiazione}
Tutte le volte che un atomo emette/assorbe un fotone perde/guadagna quantit\'a di moto e dato che un atomo emette in maniera isotropa il momento netto \'e nullo una volta mediato su molte emissioni.
I processi di assorbimento non sono isotropicamente distribuiti dato il flusso uscente di energia per $cm^2$ per sec F: solo una frazione $\kappa$ del flusso di momento $\frac{F}{c}$ \'e assorbita dalla materia. Il trasferimento da parte della radiazione di momento alla materia per $cm^3$ per sec, cio\'e la forza esercitata dalla radiazione \'e $\kappa H \frac{1}{c}$.
Un elemento di volume $dS\,dr$ subisce per effetto dell'assorbimento della radiazione una variazione d'impulso $dq$, nel caso un fotone venga assorbito la variazione del flusso uscente \'e $dF<0$.
The distribution of photons over over different quantum states with energies $\epsilon_k=\hbar\omega_k$ (large volume $\omega_k\to\omega$)
\begin{align*}
\overline{n_k}=\frac{1}{\exp{\frac{\hbar\omega}{KT}}-1}
\end{align*}
Moltiplicando il numero di stati nel dato range di frequenze per la distribuzione di Plank (numero di occupazione) ottengo il numero di fotoni e l'energia radiativa nel range di frequenza
\begin{align*}
&dN_{\omega}=\frac{V}{\pi^2c^3}\frac{\omega^2\,d\omega}{\exp{\frac{\hbar\omega}{KT}}-1}\\
&dE_{\omega}=\frac{V\hbar}{\pi^2c^3}\frac{\omega^3\,d\omega}{\exp{\frac{\hbar\omega}{KT}}-1}
\end{align*}
\end{frame}

\begin{frame}{Gradiente per trasporto radiativo nell'interno stellare}
\begin{align*}
&dq=-(n_{\nu}\,dSc\,dt)*(\kappa\rho\,dr)*\frac{h\nu}{c}&\intertext{Il primo termine \'e il numero di fotoni pasanti per superficie $dS$ in tempo $dt$, il secondo \'e la probabilit\'a d'assorbimento attraverso spessore $dr$, il terzo \'e la quantit\'a di moto di ogni fotone.}\\
&dP_r=\TDy{S}{F}=\TDof{S}\TDy{t}{q}\\
&=-\int \,d\nu n_{\nu}c\kappa_{\nu}\rho\,dr\frac{h\nu}{c}\\
&F_{\nu}=n_{\nu}ch\nu,\\
&\TDy{r}{P(Rad)}=-\int\,d\nu\frac{F(Rad)}{c}\kappa_{\nu}\rho&\intertext{In condizioni di LTE posso confrontare $\uparrow$ con}\\
&P_{\nu}=\frac{1}{3}u_{\nu},\ P(Rad)=\frac{1}{3}aT^4&\intertext{e ricavare il gradiente di temperatura necessario per il flusso di energia $F(Rad)$:}\\
&\TDy{r}{T}=-\frac{3\kappa\rho l(r)}{16\pi acT^3r^2}
\end{align*}
\end{frame}

\begin{frame}{* formation}
	contenu...
\end{frame}

\frameinlbftrue
\begin{frame}[fragile]{Struttura di equilibrio}

\begin{itemize}
\item Equilibrio idrostatico: pressione in un mesh \'e il peso della materia sopra per unit\'a di superficie. Stabilit\'a e tempi reazione a perturbazione
\item Pressione radiativa. Una frazione $\kappa$ del flusso di momento $\frac{F}{c}$ \'e assorbita dalla materia (momentum transfer per $cm^3$ per sec): $dq=-(n_{\nu}\,dSc\,dt)*(\kappa\rho\,dr)*\frac{h\nu}{c}$, il primo termine \'e il numero di fotoni pasanti per superficie $dS$ in tempo $dt$, il secondo \'e la probabilit\'a d'assorbimento attraverso spessore $dr$, il terzo \'e la quantit\'a di moto di ogni fotone.
\begin{align*}
&dP_r=\TDy{S}{F}=\TDof{S}\TDy{t}{q}=-\int \,d\nu n_{\nu}c\kappa_{\nu}\rho\,dr\frac{h\nu}{c}\\
&F_{\nu}=n_{\nu}ch\nu,\ \TDy{r}{P(Rad)}=-\int\,d\nu\frac{F(Rad)}{c}\kappa_{\nu}\rho
\end{align*}
\begin{comment}
Un elemento di volume $dS\,dr$ subisce per effetto dell'assorbimento della radiazione una variazione d'impulso $dq$, nel caso un fotone venga assorbito la variazione del flusso uscente \'e $dF<0$
The distribution of photons over over different quantum states with energies $\epsilon_k=\hbar\omega_k$ (large volume $\omega_k\to\omega$) 
\begin{align*}
\overline{n_k}=\frac{1}{\exp{\frac{\hbar\omega}{KT}}-1}
\end{align*}
Moltiplicando il numero di stati nel dato range di frequenze per la distribuzione di Plank (numero di occupazione) ottengo il numero di fotoni e l'energia radiativa nel range di frequenza
\begin{align*}
&dN_{\omega}=\frac{V}{\pi^2c^3}\frac{\omega^2\,d\omega}{\exp{\frac{\hbar\omega}{KT}}-1}\\
&dE_{\omega}=\frac{V\hbar}{\pi^2c^3}\frac{\omega^3\,d\omega}{\exp{\frac{\hbar\omega}{KT}}-1}
\end{align*}
\end{comment}
\end{itemize}

\end{frame}
\frameinlbffalse

\begin{frame}{Rotazione, mass loss}
\begin{itemize}
\item Rotazione. \[\frac{\nabla P}{\rho}=-\nabla\phi+\vec{a}=-\nabla\phi+\Omega^2r_{\perp}=\vec{g}_{eff}\]
If $\nabla\wedge(\Omega^2r_{\perp})=0$: $\phi\to\phi-V$, $V=\int_0^{r_{\perp}}\Omega^2r_{\perp}\,dr_{\perp}$ (OK if $P(\rho)$, politrope, regioni convettive)
\end{itemize}
\end{frame}

\begin{frame}{Stability}

\end{frame}

\begin{frame}{Hayashi Line (HL)}

\end{frame}
